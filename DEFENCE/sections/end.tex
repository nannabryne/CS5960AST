








\subsection{Way forward}



\begin{frame}[t]{Possible ways forward}
    
    \only<1>{%
    \begin{center}
        \medskip
        \textcolor{B2}{\emph{Testing the limits of the framework} $\leadsto$}
    
        \textcolor{B2}{{Comparing thin-wall description to full-theory simulations\dots}}
    \end{center}
    }


    
    \medskip
    \begin{itemize}
        \item<2-> \textcolor<3->{uiogrey}{Perform higher-resolution experiments.}
        % {\only<3-8>{\color}}
        \only<2>{
            \begin{itemize}
                \item Dissociate discrepancies due to numerical limitations. 
                \item Investigate the effect of changing the wall thickness.
            \end{itemize}
        }
        \item<3-> \textcolor<4->{uiogrey}{Investigate the effect of a Yukawa-like force term, \uncover<4->{\(\ddot{\epsilon}\supset A\exp(-\mu \mathfrak{d})\).}}
        \only<3>{
            \[\ddot{\epsilon}\supset A\exp(-\mu \mathfrak{d}), \quad \mathfrak{d}\triangleq \abs{\mathfrak{D}-\epsilon(\tau,y)}.\]
        
        }
        %\footnote{$d= \mathfrak{D}-\epsilon(\tau,y) = \text{wall separation}$}%.in the e.o.m. for $\epsilon$.  %($\ddot{\epsilon}\supset\exp(-\mu d)$)
        \item<4-> \textcolor<5->{uiogrey}{Find suitable statistical measures for the GWs. }
        \only<4>{
            \begin{itemize}
                \item Perhaps asymptotic analyses are the ways to go.
            \end{itemize}
        }
    \end{itemize}

    \medskip
    \only<5>{
        \medskip
        \begin{center}
            \textcolor{B2}{\emph{Further development of the framework} $\leadsto$} 
            
            \textcolor{B2}{Assume limitations are well understood\dots}
        \end{center}
    }

    \begin{itemize}
        \item<6-> \textcolor<7->{uiogrey}{Vary parameters of the theory.}
        \only<6>{
            \begin{itemize}
                \item Establish relationship between symmetron parameters $\mu$, $\lambda$ and $M$, and GW signature.
            \end{itemize}
        }
        \item<7-> \textcolor<8->{uiogrey}{Apply more complicated displacement $\epsilon$.}
        \only<7>{
            \begin{itemize}
                \item $\epsilon = \varepsilon(\tau) \sin(py)$ is the simplest non-trivial solution, and a very unlikely configuration.
                % \item Can help \comment{\dots }
            \end{itemize}
        }
        \item<8-> \textcolor<9->{uiogrey}{Introduce energy bias $v=v_+-v_-\neq 0$.}
        \only<8>{
            \begin{itemize}
                \item Does it change the motion significantly?
                \item Is there analytical solutions for this?
            \end{itemize}
        }
    \end{itemize}

    \uncover<9>{}


% *************************************
% NOTES *******************************
\begin{notes}[9][way forward]
    \itnote{1}{
        \item With more time\dots {VALIDATION} + INSIGHT
        \item In the thesis, I describe specifics of how to set up new experiments in a consistent and robust way.
    }
    \nnote{2}{Or in general, \emph{more} experiments.}
    \nnote{6}{Can help constrain this model with GW observations.}
    \nnote{8}{By calculation or simulation}
    \nnote{9}{This is just a sample; there are many interesting strategies going forward}
\end{notes}
% *************************************
\end{frame}






\subsection{Conclusion}



\begin{frame}{Concluding remarks}
    \begin{itemize}
        \item Analytical solution $\epsA[\epsilon]$ is not found in other literature (to the best of our knowledge). 
        \item Formula for gravitational waves needs further validation. 
        \item Solid foundation is laid for further similar analyses with \asgrd. 
    \end{itemize}

% *************************************
% NOTES *******************************
\begin{notes}[1][conclusion]
    \itnote{1}{
        \item MD, symmetron
        \item One step closer to analytical estimations of GW spectra
        \item Groundwork laid for further similar analyses.
    }
\end{notes}
% *************************************
\end{frame}



% \begin{frame}{Closing thoughts}
    
% \end{frame}


% \subsection{Way forward}

