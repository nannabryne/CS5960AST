\documentclass[UKenglish,aspectratio=169]{beamer}
\usepackage[utf8]{inputenc}
\usepackage{babel, fancyvrb, textcomp}

\usetheme[toc, sectionsep=uioyellow, summary, uiostandard]{UiO}

\newcommand{\Dots}{\ensuremath{\ldots}}
\newcommand{\p}[1]{\texttt{#1}}
\newcommand{\pb}[1]{\textbf{\p{#1}}}
\newcommand{\pcmd}[1]{\p{\textbackslash #1}}
\newcommand{\penv}[1]{\pcmd{begin}\ppar{#1} \Dots{} \pcmd{end}\ppar{#1}}
\newcommand{\ppar}[1]{\p{\{#1\}}}
\newcommand{\zip}{\textsc{zip}}

\author{Dag Langmyhr}
\uioemail{dag@ifi.uio.no}
\title{A Beamer theme for UiO}
\subtitle{according to the UiO graphics standard of~2022}

\begin{document}
\uiofrontpage[dept={Department of Informatics},
  info={Combined user guide and doc},
  image={Villa_Eika}, inverted, smaller]

\section{Introduction}
\subsection{Background}

\begin{frame}{Background}
  \textbf{Beamer} is a \LaTeX{} document class for making presentations.

  \medskip
  This presentation describes a Beamer theme called \textbf{UiO}
  designed according to the Oslo University
  specifications (see
    {\small\url{https://www.uio.no/om/designmanual/}})
  and is itself an example of the UiO theme.

  \medskip
  This Beamer theme was implemented by \emph{Dag Langmyhr} at the
  Department of Informatics and \emph{Karoline Moe} from the Oslo
  University Library.

  \bigskip
  \scriptsize
  All photos are from the University of Oslo image archive.
\end{frame}

\subsection{Installation}
\begin{frame}{Installation}
If you are processing your \LaTeX{} document on a stationary Linux
computer at the University of Oslo, you need not worry about
installing the UiO Beamer theme; it is already there.

\medskip
\textbf{On your personal computer}\\
To use this package on your own computer you must do the following:
\begin{enumerate}
\item Download
  {\footnotesize\url{https://www.mn.uio.no/ifi/tjenester/it/hjelp/latex/uiobeamer.zip}}.

\item Unpack the \zip{} file.

\item You may then place all the
  unzipped files in the same folder
  as your \LaTeX{} source files. (If you know where \LaTeX{}
    packages are kept on your computer, you 
    can save them there to make them generally available. Remember to
    refresh your file name database afterwards.)
\end{enumerate}
And that should be all.
\end{frame}

\subsection{Using Overleaf}

\begin{frame}
  \frametitle{Using Overleaf}

  If you are using Overleaf, you must do the
  following:

\begin{enumerate}
\item Select ``New Project''.

\item Under ``Institution Templates'', select ``Univeristy of Oslo''.

\item Select ``UiO Beamer theme''.

\item Select ``Open as Template''.
\end{enumerate}
Then, you can start writing your presentation.
\end{frame}

\section{User guide}
\subsection{Starting a Beamer file}

\begin{frame}{Starting a Beamer file}
  All \LaTeX{} Beamer files should start like this:

  \medskip
  ~~~~\pcmd{documentclass}\p{[aspectratio=169,}\textit{class
        options}\p{]}\ppar{beamer}

  \bigskip
  \begin{alert}{\textbf{Important!} }
    We recommend always using the 16:9 aspect ratio since most
    projectors at UiO use that format.
  \end{alert}
\end{frame}

\begin{frame}[fragile,label=author+title]{Author and title}
  You specify
  \begin{itemize}
  \item your name, 
  \item the title and 
  \item the subtitle (if any)
  \end{itemize}
  of your presentation using the standard \LaTeX{} commands

  \medskip
  \begin{Verbatim}[fontsize=\footnotesize]
\title{Title of my talk}
\subtitle{My chosen subtitle}    
\author{My name}
  \end{Verbatim}

  \bigskip
  Additionally, you can specify your e-mail address with a command
  provided by the UiO theme:

  \medskip
  \begin{Verbatim}[fontsize=\footnotesize]
\uioemail{me@uio.no}
  \end{Verbatim}
\end{frame}

\subsection{Loading the UiO Beamer theme}

\newcommand{\falseTRUE}{=false|\underline{true}}
\newcommand{\FALSEtrue}{=\underline{false}|true}

\begin{frame}{Loading the UiO Beamer theme}
  You load the UiO theme like this:

  \medskip
  ~~~~\pcmd{usetheme}\p{[}\textit{theme options}\p{]}\ppar{UiO}

  \medskip
  The main theme options are:
  \begin{description}[\pb{toc}]
  \item[\pb{sectionheaders\falseTRUE} ] lists the section and
    subsection names in the upper right-hand corner, as shown in this
    user guide.
  \item[\pb{summary\FALSEtrue} ] adds a summary page at the end, as shown in this
    user guide.
  \item[\pb{toc\FALSEtrue} ] automatically inserts a table of content for every
    section; you can see two examples in this guide.
  \item[\pb{uiostandard\FALSEtrue} ] follows the UiO standard more
    strictly, for instance using square bullets in the
    \penv{itemize} environment, as shown in frame~\ref{author+title}.
  \end{description}
  \medskip
  \footnotesize (The default choice is underlined.)
\end{frame}

\newcommand{\colordemo}[1]{\small \p{#1}& \color{#1}\rule{2em}{1ex}}

\begin{frame}{Section separation frames}
  You may ask Beamer to add very distinct separation frames between
  the sections, as has been done in this document. This is achieved by
  using this theme option:
  \begin{description}[\pb{toc}]
  \item[\pb{sectionsep=\emph{colour}}] adds section separation sheets
    in the specified colour. 
  \end{description}
  
  \smallskip
  You may choose any of the official UiO colours:
  
  \begin{tabular}{@{}lclclclc@{}}
    \colordemo{uioblue1}& \colordemo{uiogreen1}&
    \colordemo{uioorange1}& \colordemo{uiopink1}\\
    \colordemo{uioblue2}& \colordemo{uiogreen2}&
    \colordemo{uioorange2}& \colordemo{uiopink2}\\
    \colordemo{uioblue3}& \colordemo{uiogreen3}&
    \colordemo{uioorange3}& \colordemo{uiopink3}\\
    \colordemo{uioyellow}& \colordemo{uiogrey}\\
  \end{tabular}

  \medskip
  or any colour you have defined yourself (using \pcmd{definecolor}).
\end{frame}

\begin{frame}[label=fonts]{UiO Beamer theme fonts}
  You may specify which font to use by giving one of the following
  theme options:

  \begin{description}[\pb{font=arial}]
  \item[\pb{font=arial}] selects Arial which the University recommends
    as the standard font.

  \item[\pb{font=noto}] selects Noto. Use this font if you need
    Unicode characters other than the basic ones. This font is the
    default choice.

  \item[\pb{font=arev}] selects Arev. Use this font if you need advanced
    mathematics in your presentation.

  \item[\pb{font=none}] does not select any particular font. Use this
    option only if you intend to specify the font selection yourself.
  \end{description}
\end{frame}

\begin{frame}[fragile,label=ex]{An example}
  A \LaTeX{} file using the UiO Beamer theme typically starts like
  this:

  \medskip
  \begin{Verbatim}[fontsize=\scriptsize]
\documentclass[norsk,aspectratio=169]{beamer}
\usepackage[utf8]{inputenc}
\usepackage{babel, textcomp}
\usetheme[toc, summary]{UiO}

\title{Title of my talk}
\subtitle{A suitable subtitle}
\author{My name}   \uioemail{me@uio.no}

\begin{document}
\uiofrontpage[...]

\section{...}
\subsection{...}
\begin{frame}
  \frametitle{...}
    :
  \end{Verbatim}
\end{frame}

\subsection{The front page}

\begin{frame}{The front page}
  A front page is created using the \pcmd{uiofrontpage} command:

  \medskip
  ~~~~\pcmd{uiofrontpage}\p{[}\emph{front page options}\p{]}

  \bigskip
  \begin{alert}{\textbf{Important!} }
    This command should \textbf{\emph{not}} be used inside a
    \penv{frame} environment! 
  \end{alert}

  \medskip
  See the \LaTeX{} code on frame~\ref{ex} for an example of proper
  use.
\end{frame}
  
\begin{frame}[label=fp-options]
  The front page options are:

  \medskip
  \begin{description}[\pb{academictitle=\ppar{\Dots}}]
  \item[\pb{academictitle=\ppar{\Dots}}] gives your academic title.
  %% \item[\pb{author=\ppar{\Dots}}] provides your name (if other than in
  %%   \pcmd{author}).
  \item[\pb{date=\ppar{\Dots}}] supplies the date (if other than
    today).
  \item[\pb{dept=\ppar{\Dots}}] names your department.
  %% \item[\pb{email=\ppar{\Dots}}] gives your e-mail address
  %% (if no \pcmd{uioemail} used).
  \item[\pb{image=\ppar{\Dots}}] is the file name of the image to
    use. (Ideally, this image should be almost quadratic, or 8:9 to be
    precise.)
  \item[\pb{info=\ppar{\Dots}}] provides an additional info line for
    the supervisor's name, grant support or whatever.
  \item[\pb{inverted}] will use a white seal rather than the black
    one. (This choice depends, of course, on the image.)
  \item[\pb{smaller}] will use a smaller font than the standard one.
  %% \item[\pb{subtitle=\ppar{\Dots}}] gives the subtitle.
  %% \item[\pb{title=\ppar{\Dots}}] gives the title of your presentation.
  \end{description}
\end{frame}

\begin{frame}[fragile]{An example}
  The front page of this user guide was made using the following commands:

  \medskip
  \VerbatimInput[fontsize=\footnotesize, firstline=15, lastline=18]{uiobeamer-guide.tex}
  ~~~~$\vdots$
  \VerbatimInput[fontsize=\footnotesize, firstline=20, lastline=23]{uiobeamer-guide.tex}
\end{frame}

\subsection{Text and image frames}

\begin{frame}{Making frames with both text and image}
  If you want a frame with half-and-half of text and image, you can use the
  \p{uioimageframe} environment as demonstrated below. The image
  (\emph{SRs\_hus} in this example) should have an aspect ratio close
  to~8:9. The result is shown in the next frame.

  \medspace
  \VerbatimInput[fontsize=\footnotesize]{rosselands-hus.tex}
\end{frame}

\begin{uioimageframe}[info={© Universitetet i Oslo}]{SRs_hus}
  \frametitle{Svein Rosselands hus}

  Svein Rosselands hus er oppkalt etter astrofysikeren \emph{Svein
  Rosseland}. Arkitektene bak bygningen er Finn Bryn og Johan
  Ellefsen. Bygningen har høy arkitektur- og kulturhistorisk verdi.
\end{uioimageframe}


\subsection{Full-page images}

\begin{frame}{Full-page image}
  If you want a full-page image, you can use either the
  \pcmd{uiobigimage} (which has a thin frame for title and copyright
  information)
  or the \pcmd{uiofullpageimage} command:

  \medskip
  ~~~~\pcmd{uiobigimage}\p{\{}\emph{title}\p{\}\{}\emph{file
        name}\p{\}\{}\emph{copyright info}\p{\}}\\
  ~~~~\pcmd{uiofullpageimage}\p{\{}\emph{file name}{\}}

  \medskip
  Full page images should have an aspect ratio close to~16:9.

  \bigskip
  \begin{alert}{\textbf{Important!} }
    These commands should \textbf{\emph{not}} be used inside a
    \penv{frame} environment! 
  \end{alert}

  \medskip
  \textbf{Examples}\\
  See the next two frames.
\end{frame}

\uiobigimage{Ole-Johan Dahl building}{OJDs_hus}{Foto: Francesco Saggio/UiO} 
\uiofullpageimage{UiO_sentrum}

\end{document}
