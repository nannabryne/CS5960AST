\documentclass[UKenglish]{texmex/uiomasterthesis}

\usepackage{import}
\usepackage{texmex/overhead}

% \addbibresource{refs.bib}


% Writing tools (remove before submission):
% \import{texmex/}{writingtools.sty}

% !TEX root = ../thesis.tex


\makeatletter

% -------
% Colours:

\definecolor{n@pink}{rgb}{0.93, 0.51, 0.93}
\definecolor{n@yellow}{rgb}{0.87, 1.0, 0.0}
\definecolor{n@darkblue}{rgb}{0.0, 0.14, 0.4}
\definecolor{n@redish}{rgb}{1.0, 0.0, 0.16}
\definecolor{n@green}{rgb}{0.31, 0.49, 0.16}
\definecolor{n@lilac}{rgb}{0.87, 0.0, 1.0}
\definecolor{n@babyblue}{rgb}{0.63, 0.79, 0.95}
\definecolor{n@blue}{rgb}{0.0, 0.5, 1.0}
\definecolor{n@darkred}{RGB}{165, 63, 38}
\definecolor{NANNATEST}{RGB}{240,77,108}
\definecolor{n@grey}{rgb}{0.7, 0.75, 0.71}



%TODO : "spelling" / "grammar"


% in-sentence comment:
% \newcommand{\comment}[1]{\textcolor{n@redish}{$\lceil$~\small{#1}~$\rfloor$}}
\newcommand{\comment}[1]{\textcolor{n@redish}{%
    {\large{$\boldsymbol{\ulcorner}$}}%
    % {\fontfamily{phv}\selectfont
    {\small{#1}}%
    {\large{$\boldsymbol{\lrcorner}$}}}}

% something important?
\newcommand{\important}[1]{\colorbox{black}{\textcolor{white}{\textbf{#1}}}}

% highlight
\newcommand{\hl}[1]{\colorbox{n@yellow}{#1}}    

% need citation:
% \makeatletter
% \newcommand{\nc}[1][\@nil]{%
% \def\tmp{#1}%
%     \ifx\tmp\@nnil
%         ~\textcolor{n@darkblue}{$\vdots $ \copyright ~$\vdots $}~
%     \else
%         \textcolor{n@darkblue}{#1$^{\text{\copyright}}$}
%     \fi}

\newcommandx*{\nc}[2][2=]{%
    \textcolor{n@darkblue}{#1$^{\text{\copyright}}_{\text{{\tiny{(#2)}}}}$}
}
% \makeatother


% rephrase:
\newcommand{\rephrase}[1]{%
    {\large{\textcolor{n@green}{$\boldsymbol{\lceil}$}}}%
    {\normalsize{#1}}%
    {\large{\textcolor{n@green}{$\boldsymbol{\rfloor}_{\circlearrowleft}$}}}}
    
% awkward wording:
\newcommand{\cringe}[1]{%
    {\large{\textcolor{n@green}{$\boldsymbol{\lceil}$}}}%
    {\normalsize{#1}}%
    {\large{\textcolor{n@green}{$\boldsymbol{\rfloor}_{\circlearrowleft}$}}}}


% needs double-checking:
\newcommand{\checkthis}[1]{%
    {\large{\textcolor{n@lilac}{$\boldsymbol{\lceil}$}}}%
    {\normalsize{#1}}%
    {\large{\textcolor{n@lilac}{$\boldsymbol{\rfloor}_{?}$}}}}

% Bullet points:
\newenvironment{bullets}
    {\par\medskip
    \begin{itemize}[itemsep=0em, leftmargin=2cm, rightmargin=4cm, labelsep=3pt]
     \begingroup\color{n@pink}%
     \ignorespaces}
   {\endgroup\end{itemize}
    \medskip}

% Phantom paragraph:
% \makeatletter
\newcommand{\phpar}[1][\@nil]{%
\def\tmp{#1}%
    \ifx\tmp\@nnil
        \par\colorbox{n@darkblue}{\textcolor{white}{\textsc{phantom paragraph}}}
    \else
        \par\colorbox{n@darkblue}{\textcolor{white}{\textsc{phantom paragraph:}}}~\textcolor{n@darkblue}{\textsc{#1}}
    \fi}
% \makeatother

% \newcommand{\phpar}[1][]{\colorbox{n@darkblue}{\textsc{phantom paragraph}}}



% \newcommand*{\grammar}[2][]{\textcolor{n@green}{#2}\colorbox{n@green}{\textcolor{white}{#1}}}
% Check spelling
% \makeatletter
\newcommand{\grammar}[2][\@nil]{%
\textcolor{n@blue}{#2\(^{?}\)}%
\def\tmp{#1}%
    \ifx\tmp\@nnil
        %
    \else
        ~$^\text{\colorbox{n@blue}{\tiny{\textcolor{white}{#1}}}}$
    \fi}
% \makeatother



% sometext [...]
\newcommand*{\blahblah}{\textcolor{n@darkred}{blah blah [\dots]\,}}

% Temporary section
\newenvironment{draft}{% 
    \noindent\textcolor{n@blue}{
    \rule{\linewidth}{1.2pt}
    }
    
    \textcolor{n@blue}{
    \section*{DRAFT}
    \begin{flushleft}
        {\huge{$\boldsymbol{\ulcorner}$}}
    \end{flushleft}
    }%
    \noindent%
    }{
    \textcolor{n@blue}{
    \section*{}
    \begin{flushright}
        {\huge{$\boldsymbol{\lrcorner}$}}
    \end{flushright}
    \noindent\rule{\linewidth}{1.2pt}
    }%
    }


% need reference/showing
\newcommand*{\provethis}[1]{\textcolor{n@darkblue}{%
    {\large{[}}$\leftarrow${\small{#1}}{\large{]}$_\blacksquare$}%
    }}



% temporary title
\newcommand*{\tmptitle}[1]{\textcolor{n@babyblue}{TITLE~(}{#1}\textcolor{n@babyblue}{)}}







%% define \begin{Nannasnotes} 
\newcommand*\n@Nannasnotesname{Nanna's notes}
\newcommand*\n@Nannasnotescolour{n@pink}
\newenvironment{Nannasnotes}{%
    \chapter*{\textcolor{\n@Nannasnotescolour}{$\blacksquare \mathcal{NOTES}$}}
    \markboth{\textcolor{\n@Nannasnotescolour}{\n@Nannasnotesname: \today}}{}
    % \newcommand*{\newnote}[1]{\section*{\textcolor{\n@Nannasnotescolour}{#1}} }
    }
    {
    \textcolor{\n@Nannasnotescolour}{
    \begin{flushright}
        {\huge{$\blacksquare$}}
    \end{flushright}
    }%
    }

\newcommandx*\notesdate[3]{%
    \noindent\textsf{\textcolor{\n@Nannasnotescolour}{\ordinaldate{#1}~\monthname[#2]~{#3}}
    }%
}

% communicate through note
\newcommand*\speak[1]{\textcolor{uioblue}{\textbf{#1}}}




% question
\newcommand*{\question}[2][]{%
{\colorbox{black}{\textcolor{n@yellow}{\textbf{?`}}}}%
\def\tmp{#1}%
    \ifx\tmp\@nnil
        %
    \else
        $^{~#1}$~%
    \fi
{\normalsize{#2}}%
{~\colorbox{black}{\textcolor{n@yellow}{\textbf{?}}}}%
}



% separation line
\newcommandx*\hlineSep[2][1=0.8, 2=n@pink]{%
\noindent\textcolor{#2}{%
\rule{\linewidth}{{#1}pt}}%
\noindent
}







% jokes, thoughts etc.
\newcommand*\pensive[1]{\noindent\textcolor{n@green}{\textit{#1}}}


% verb tenses
\newcommand*\verbtense[2]{\textcolor{n@green}{$\langle$\texttt{#1}$\rangle$$\lceil$}{#2}\textcolor{n@green}{$\rfloor$}}


\newcommand*\deleteme[1]{{\small{%
    \textcolor{n@babyblue}{\texttt{DELETE ME:}$\langle$}%
    {\textcolor{n@grey}{#1}}%
    \textcolor{n@babyblue}{$\rangle$}
    }}}%



\newcommand{\iftime}[1]{%
    \textcolor{n@yellow}{\textsc{ \colorbox{n@grey}{\textbf{if time:}}~{\small{#1}}}}%
    }

\newcommand*{\rcomment}[1]{%
    \textcolor{n@lilac}{$\star$}
    \marginpar{\raggedright\textcolor{n@lilac}{{\footnotesize{#1}}}}%
    }
\newcommand*{\lcomment}[1]{%    
    \textcolor{n@lilac}{$\star$}
    \marginpar[\raggedleft\textcolor{n@lilac}{{\footnotesize{#1}}}]{}%
    }



\makeatother

\title{Gravitational waves from topological defects} 
% \title{If Domain Walls Could Talk}
\subtitle{Signatures of late-time first-order phase transitions} 
% \subtitle{They would say ...} 
\author{Nanna Bryne}                    

\begin{document}

% \bibliographystyle{plain}


%%%%%%%%%%%%%%% FRONT PAGE %%%%%%%%%%%%%%%

\uiomasterfp[
    dept={Institute of Theoretical Astrophysics\and Department of Physics},
    program={Computational Science:~Astrophysics}, 
    supervisor={David Fonseca Mota}, % or supervisors={A Name\and B Name},  
    color={blue},
    fgimage={starry-night.png},
    long] 


%%%%%%%%%%%%%%%%%%%%%%%%%%%%%%%%%%%%%%%%%%


%%%%%%%%%%%%%%% FRONT MATTER %%%%%%%%%%%%%%%

\frontmatter{}
\begin{abstract}
    Topological defects predicted by extensions of general relativity can manifest in gravitational-wave observations on Earth. As a working example, we present symmetron domain walls formed at redshift $\sim 2$ with structural ripples and investigate how gravitational radiation carries information about the symmetry-breaking theory. Future broadband gravitational-wave observations  \speak{COMPLETE THIS!}
\end{abstract}


\tableofcontents{}
\listoffigures{}     
% \listoftables{}       

% \part*{Prologue}
\begin{preface}
    Here comes your preface, including acknowledgments and thanks.


    % I would like to give thanks to all my friends and family.

    % To my mother and father, I am especially grateful. 
    % \comment{Thank m+p; Lars, the coffee maker; stjerne20 and -22; computer resources}
\end{preface}

% NOMENCLATURE etc
\begin{nomen}
    \import{chapters/0_Notation/}{notation.tex}
\end{nomen}

%%%%%%%%%%%%%%%%%%%%%%%%%%%%%%%%%%%%%%%%%%%%



%%%%%%%%%%%%%%% MAIN MATTER %%%%%%%%%%%%%%%

\mainmatter{}

% >>> 

% \part*{}%

% \chapter*{(chapter) Title Case}
% Lorem ipsum\dots
% \section*{(section) Sentence case}
% Lorem ipsum\dots
% \subsection*{(subsection) Sentence case}
% Lorem ipsum\dots

% \paragraph*{(paragraph) Sentence case with punctuation.} %
% Lorem ipsum\dots

% Below, we describe some phenomena or whatever.
% \begin{description}
%     \item[Gravitational waves] are called that or GWs, tensor perturbations, \dots pdsfovnsoz avoszjasvo aoc awvn anvo pewnfvao noav
%     \item[Stress--energy tensor] is called that or SE tensor(?) \comment{OBS: Hilbert SE tensor = HSE tensor?}.
%     \item[Domain walls] are called that or walls, \underline{never} DWs.
%     \item[General relativity] is called that or GR, Einstein's theory of gravity.
%     \item[Einstein field equation(s)] are called that or EFE(s).
%     \item[Equation(s) of motion] are called that or eom(s).  
% \end{description}







% INTRODUCTION
\chapter{Introduction}\label{chap:intro}
    \import{chapters/1_Introduction/}{intro.tex}



% øøøøøøøøøøøøøøøøøøøøøøøøøøøøøøøøøøøøøøøøøøøøøøøøøøøøøøøøøøøøøøøøøøøøøøøøøøøøøø 
% øøøøøøøøøøøøøøøøøøøøøøøøøøøøøøøøøøøøøøøøøøøøøøøøøøøøøøøøøøøøøøøøøøøøøøøøøøøøøø
% > BACKGROUND (math+context)
% ______________________________________________________________________________



% >>> THEORETICAL BACKGROUND

\part{Background}\label{part:bckg}


% GR
\chapter{General Relativity}\label{chap:GR}
    {\import{chapters/2_General_Relativity/}{GR.tex}}

% DWs
%Kinks in Cosmology
\chapter{Symmetry-Breaking Dark Energy}\label{chap:cosmo}
    {\import{chapters/3_TMP/}{cosmo.tex}}




% øøøøøøøøøøøøøøøøøøøøøøøøøøøøøøøøøøøøøøøøøøøøøøøøøøøøøøøøøøøøøøøøøøøøøøøøøøøøøø






% øøøøøøøøøøøøøøøøøøøøøøøøøøøøøøøøøøøøøøøøøøøøøøøøøøøøøøøøøøøøøøøøøøøøøøøøøøøøøø 
% øøøøøøøøøøøøøøøøøøøøøøøøøøøøøøøøøøøøøøøøøøøøøøøøøøøøøøøøøøøøøøøøøøøøøøøøøøøøøø
% > METHOD (calculations+context+equations+assumptions)
% ______________________________________________________________________________


% >>> THE PROJECT

\part{Methodology}\label{part:method}




% ANALYTICAL CALCULATIONS
\chapter{Imperfect Defects}\label{chap:pertwalls}
    {\import{chapters/4_Imperfect_Defects/}{pertwalls.tex}}


% Get simulation typesetting
{\import{texmex/}{simulation_typeset.tex}


% COMPUTATIONAL CALCULATIONS
\chapter{Cosmic Phase Transitions}\label{chap:PT}
    {\import{chapters/5_Cosmic_Phase_Transition/}{PT.tex}}

% øøøøøøøøøøøøøøøøøøøøøøøøøøøøøøøøøøøøøøøøøøøøøøøøøøøøøøøøøøøøøøøøøøøøøøøøøøøøøø 














% øøøøøøøøøøøøøøøøøøøøøøøøøøøøøøøøøøøøøøøøøøøøøøøøøøøøøøøøøøøøøøøøøøøøøøøøøøøøøø 
% øøøøøøøøøøøøøøøøøøøøøøøøøøøøøøøøøøøøøøøøøøøøøøøøøøøøøøøøøøøøøøøøøøøøøøøøøøøøøø
% > FINDINGS (results+analysis+discussion+conclusion)
% ______________________________________________________________________________


\part{Findings}\label{part:findings}



\chapter{\tmptitle{Toy Model Trials}}\label{chap:results}
    {\import{chapters/6_Toy_Model_Trials/}{results.tex}}


\chapter{\tmptitle{Ifs, buts and maybes}}\label{chap:whatif}
    {\import{chapters/7_TMP_Whatif/}{whatif.tex}}

}




% >>> 


\part*{Summary} % Epilogue?

% CONCLUSION

\chapter{Conclusion and Outlook}
    {\import{chapters/8_Conclusion_and_Outlook/}{finalchap.tex}}

% øøøøøøøøøøøøøøøøøøøøøøøøøøøøøøøøøøøøøøøøøøøøøøøøøøøøøøøøøøøøøøøøøøøøøøøøøøøøøø





%%%%%%%%%%%%%%%%%%%%%%%%%%%%%%%%%%%%%%%%%%%










%%%%%%%%%%%%%%% BACK MATTER %%%%%%%%%%%%%%%

\backmatter{}




\bibliography{refs}
\addcontentsline{toc}{chapter}{Bibliography}

\part*{Appendices}

\appendix


% >>>>>>>>>>>>>>>>>>>>>>>>>>>>>>>>>>>>>>
% Notes that are not to be included in the thesis!
% \begin{Nannasnotes}
%     {\import{appendices/misc/}{misc.tex}}
% \end{Nannasnotes}
% <<<<<<<<<<<<<<<<<<<<<<<<<<<<<<<<<<<<<<



\chapter{Cylinder Functions}\label{app:special}
    {\import{appendices/Special_tmp/}{cylinder.tex}}


\chapter{Stable Symmetron}\label{app:stablesym}
    {\import{appendices/Untitled2_tmp/}{untitled2.tex}}

\chapter{Symmetron Walls in Expanding Spacetime}\label{app:walls}
    {\import{appendices/TMP_DW/}{walls.tex}}


% \chapter{Derivations}\label{app:derivations}
    % {\import{appendices/Derivations_tmp/}{derivations.tex}}







    
% \chapter{I do not have an appendix}


% \chapter*{Old drafts}
%     {\import{appendices/}{oldtexts.tex}}



%%%%%%%%%%%%%%%%%%%%%%%%%%%%%%%%%%%%%%%%%%%



\end{document}

















% |||||||||||||||||||||||||||||||||||||||||||||||| 

% \chapter*{My commands}

% \important{This section is so that my commands are saved in the cache.} 

% \hl{Delete before submission!}

% \begin{bullets}
%     \item \dots
%     \item bullet points
% \end{bullets}

% \nc{need citation}

% \checkthis{check this}

% \rephrase{rephrase}

% \cringe{awkward wording}

% \comment{comment}

% \phpar[\dots]

% \noindent
% \begin{tabular*}{\linewidth}{ r l }
%     \toprule    
% \end{tabular*}

% \Group{SO}{3} is a group 

% \textsf{SO(3)} is a group \nc

% $\Group{SO}{3}$ is a group

% The order \BigO{1} is large

% The order $\BigO{1}$ is large



% The Planck unit \Planck{M}

% \begin{draft}
%     \dots
% \end{draft}

% \begin{align}
%     \pclosed{\dots}\,  \bclosed{\dots}\,  \cclosed{\dots} \\
%     \integ{x} f 
% \end{align}






% % |||||||||||||||||||||||||||||||||||||||||||||||| 


% % \end{document}