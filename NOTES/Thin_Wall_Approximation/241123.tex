
\section{Dynamics of Domain Walls in The Thin Wall Approximation}
%%%%%%%%%%%%%%%%%%%%%%%%%%%%%%%%%%%%%%%%%%%%%%%%%%%%%%%%%%%%%%%%%


% \newcommand*{\gthree}{g^{(3)}} % det g^3

\newcommand*{\mfa}{\mathfrak{a}}
\newcommand*{\mfb}{\mathfrak{b}}
\newcommand*{\mfc}{\mathfrak{c}}
\newcommand*{\QQ}[1][\mu\nu\kappa\lambda\alpha\beta]{\mathbf{Q}\^{#1}}
\newcommand*{\xdw}{{x\ped{dw}}}


%%%%%%%%%%%%%%%%%%%%%%%%%%%%%%%%%%%%%%%%%%%%%%

Preceding Julian's notes (`Dynamics of Domain Walls in the Thin Wall approximation'). I could not get a symmetric stress-energy tensor from equations (18) and (4) in said notes. I then calculated the determinant ($g\ap{(3)}$) for myself, and by using that expression (\cref{eq:thinwall:det_g3} and~\cref{eq:thinwall:define_QQ} below) the functional derivative $\delta g\ap{(3)}/\delta g\_{\rho\sigma}$ becomes symmetric, hence $T\^{\rho\sigma}$ is symmetric.




\paragraph{Covariant action.} %
Consider symmetron potential, thin wall limit. Surface tension is
\begin{equation}
    \sigma \simeq \integ{\phi}[\phi_-][\phi_+] \sqrt{2V\ped{eff}(\phi) - 2 V\ped{eff}(\phi_{\pm})},
\end{equation}
where $\phi_\pm= \phi(z\! \to\! \pm \infty)$. We write the covariant action as $S\ped{dw}=-\sigma \integ[3]{\xi \sqrt{-g\ap{(3)}}} $.
% \subparagraph{World volume metric.} %
The induced metric is
\begin{equation}
    g\indices*{^{(3)}_{AB}} = g\_{\mu\nu} \dv{{x\ped{dw}}^\mu}{\xi\^A} \dv{{x\ped{dw}}\^\nu}{{\xi}^{B}};\quad A, B = 0,1,2,
\end{equation}
where ${x\ped{dw}}\^\mu (\xi\^{A})$ is the embedding function.
The determinant of the world volume metric is
\begin{equation}\label{eq:thinwall:det_g3}
    g\ap{(3)} = \LeviCivita{ABC} g\indices*{^{(3)}_{0A}} g\indices*{^{(3)}_{1B}} g\indices*{^{(3)}_{2C}}  = g\_{\mu\nu} g\_{\kappa\lambda} g\_{\alpha \beta} \QQ,
\end{equation}
where
{%%%%%%%%%%%%%
\newcommandx*\DD[2]{\Delta\indices{^{#1}_{#2}}}%
\newcommandx*{\func}[6]{\DD{\mu}{#1} \DD{\nu}{#2} \DD{\kappa}{#3} \DD{\lambda}{#4}  \DD{\alpha}{#5} \DD{\beta}{#6}}%
\newcommand*{\col}[1]{\textcolor{uiopink}{#1}}%
%%%%%%%%%%%%
$\QQ = \LeviCivita{ABC} \func{0}{A}{1}{B}{2}{C}$; $\DD{\mu}{A} \equiv \dv*{\xdw\^\mu}{\xi\^{A}}$. In particular,
\begin{multline}\label{eq:thinwall:define_QQ}
    \QQ= \func{0}{\col{0}}{1}{\col{1}}{2}{\col{2}}  + \func{0}{\col{1}}{1}{\col{2}}{2}{\col{0}} +\func{0}{\col{2}}{1}{\col{0}}{2}{\col{1}} \\
    - \func{0}{\col{0}}{1}{\col{2}}{2}{\col{1}} - \func{0}{\col{1}}{1}{\col{0}}{2}{\col{2}} - \func{0}{\col{2}}{1}{\col{1}}{2}{\col{0}}.
\end{multline}
}

\paragraph{Stress-energy tensor.} %
We consider a planar wall lying in the $xy$-plane with a small perturbation in the $z$-direction.
The stress-energy tensor is given by 
\begin{equation}
    T\^{\rho\sigma} = \frac{2}{\sqrt{-g}}\frac{\delta \mathcal{L}}{\delta g\_{\rho\sigma}}
    = \frac{\sigma \Diracdelta{z-z\ped{dw}}}{ \sqrt{-g}  \sqrt{-g^{(3)}} } \frac{\delta g^{(3)} }{\delta g\_{\rho\sigma}}.
\end{equation}
We need the functional derivative of $g\ap{(3)}$ and the quantity $\sigma \Diracdelta{z-z\ped{dw}}$.


\subsection{My Calculation}

We vary $g\ap{(3)}$ w.r.t. $g\_{\rho\sigma}$, ignoring \BigO{(\delta g\_{\rho \sigma})^2}-terms:
\begin{equation}
    \begin{split}
        g\ap{(3)}+\delta g\ap{(3)} &= \pclosed{ g\_{\mu\nu} + \delta g\_{\mu\nu} } \pclosed{ g\_{\kappa\lambda} + \delta g\_{\kappa\lambda} } \pclosed{ g\_{\alpha\beta} + \delta g\_{\alpha\beta} } \QQ \\
        &= g\ap{(3)} + \pclosed{ \delta g\_{\mu\nu} g\_{\kappa\lambda} g\_{\alpha\beta} +  g\_{\mu\nu} \delta g\_{\kappa\lambda} g\_{\alpha\beta} + g\_{\mu\nu} g\_{\kappa\lambda} \delta g\_{\alpha\beta} } \QQ \\
        &= g\ap{(3)} + \pclosed{ \pdv{g\_{\mu\nu}}{g\_{\rho\sigma}} \delta g\_{\rho\sigma} g\_{\kappa\lambda} g\_{\alpha\beta} +  g\_{\mu\nu} \pdv{g\_{\kappa \lambda}}{g\_{\rho\sigma}} \delta g\_{\rho\sigma} g\_{\alpha\beta} + g\_{\mu\nu} g\_{\kappa\lambda} \pdv{g\_{\alpha\beta}}{g\_{\rho\sigma}}  \delta g\_{\rho\sigma}} \QQ  \\
        &= g\ap{(3)} + \pclosed{ \Krondelta{^\rho_\mu} \Krondelta{^\sigma_\nu} g\_{\kappa\lambda} g\_{\alpha\beta} +  g\_{\mu\nu} \Krondelta{^\rho_\kappa} \Krondelta{^\sigma_\lambda} g\_{\alpha\beta} + g\_{\mu\nu} g\_{\kappa\lambda} \Krondelta{^\rho_\alpha} \Krondelta{^\sigma_\beta} }\QQ  \cdot \delta g\_{\rho \sigma}  \\
        &= g\ap{(3)} + \pclosed{ g\_{\kappa\lambda} g\_{\alpha\beta}\QQ[\rho\sigma \kappa \lambda \alpha\beta] +g\_{\mu\nu} g\_{\alpha\beta}\QQ[\mu\nu \rho\sigma  \alpha\beta] + +g\_{\mu\nu} g\_{\kappa\lambda}\QQ[\mu\nu \kappa \lambda \rho\sigma ]  } \cdot \delta g\_{\rho \sigma}
    \end{split}
\end{equation}
Thus,
\begin{equation}
    \frac{\delta g\ap{(3)}}{\delta g\_{\rho\sigma}} = g\_{\kappa\lambda} g\_{\alpha\beta} \cclosed{ \QQ[\rho\sigma \kappa \lambda \alpha\beta] + \QQ[\alpha\beta \rho\sigma \kappa \lambda ] +  \QQ[\kappa \lambda \alpha\beta\rho\sigma ] }.
\end{equation}


\paragraph{Flat FRW universe.} With $g\_{\mu\nu}\diff x\^\mu \diff x\^\nu = - \diff t^2 + a(t)^2 \Krondelta{_{ij}} \diff x\^i \diff x\^j$ and $\xdw\^{A}=\xi\^A, \xdw^3= z\ped{dw}=\epsilon(\xi\^A)$, we may insert
\begin{equation}
    \Delta\indices{^{\mu}_{A}} = \begin{cases}
        \Krondelta{^{\mu}_{A}}, &\mu \neq 3 \\
        \pdv*{\epsilon}{\xi\^A}, &\mu= 3
    \end{cases}
\end{equation}
into~\cref{eq:thinwall:define_QQ} to compute $g\ap{(3)}$ and $\delta g\ap{(3)} / \delta g\_{\rho\sigma}$. The result of the latter is a symmetric tensor of type (2,0); 
\begin{multline}\label{eq:thinwall:func_der_of_g3}
    \left[\frac{\delta g\ap{(3)}}{\delta g\_{\rho\sigma}}\right] = \\
    \left(\begin{array}{rrrr}
        {\left({\iota_1}^{2} + {\iota_2}^{2} + 1\right)} a^{4} & -{\iota_0} {\iota_1} a^{4} & -{\iota_0} {\iota_2} a^{4} & {\iota_0} a^{4} \\
        -{\iota_0} {\iota_1} a^{4} & {\iota_0}^{2} a^{4} - {\left({\iota_2}^{2} + 1\right)} a^{2} & {\iota_1} {\iota_2} a^{2} & -{\iota_1} a^{2} \\
        -{\iota_0} {\iota_2} a^{4} & {\iota_1} {\iota_2} a^{2} & {\iota_0}^{2} a^{4} - {\left({\iota_1}^{2} + 1\right)} a^{2} & -{\iota_2} a^{2} \\
        {\iota_0} a^{4} & -{\iota_1} a^{2} & -{\iota_2} a^{2} & {\iota_0}^{2} a^{4} - {\left({\iota_1}^{2} + {\iota_2}^{2}\right)} a^{2}
        \end{array}\right),
\end{multline}
where we defined $\iota_A \equiv \pdv*{\epsilon}{\xi\^A}$. 

\paragraph{Symmetron potential.} %

We let $V\ped{eff}(\phi)= V(\phi)=\frac{\lambda}{4}\pclosed{\phi^2 -\phi_0^2}^2$. As such, $\phi_\pm = \pm \phi_0$ and $V(\phi_\pm)=0$. Now,
\begin{equation}\label{eq:thinwall:sigma_delta_wrong}
    \begin{split}
        \sigma \Diracdelta{z-z\ped{dw}} &= \integ{z}[-\infty][\infty] \dv{\phi}{z} \sqrt{2V\ped{eff}(\phi(z)) - 2 V\ped{eff}(\phi_{\pm})} \Diracdelta{z-z\ped{dw}} \\
        &= \sqrt{2} \integ{z}[-\infty][\infty] \dv{\phi}{z} \sqrt{V\ped{eff}(\phi(z))}\Diracdelta{z-z\ped{dw}} \\
        &= \sqrt{2} \sqrt{V\ped{eff}(\phi(z\ped{dw}))}  \dv{\phi}{z} \bigg|_{z=z\ped{dw}} \\
        % &= \sqrt{2}  \bclosed{\dv{\phi}{z} \sqrt{V\ped{eff}(\phi)}}_{z=z\ped{dw}} \\
        &= \sqrt{\frac{\lambda}{2}} \pclosed{\phi(z\ped{dw})^2-\phi_0^2} \dv{\phi}{z} \bigg|_{z=z\ped{dw}} .
    \end{split}
\end{equation}
\textcolor{uioblue}{\textbf{Have I completely misunderstood something here?}} 

% \question{Have I completely misunderstood something here?}




