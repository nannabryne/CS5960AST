

\section{Dynamics of Domain Walls in The Thin Wall Approximation}
%%%%%%%%%%%%%%%%%%%%%%%%%%%%%%%%%%%%%%%%%%%%%%%%%%%%%%%%%%%%%%%%%


We have the spacetime metric $g\_{\mu\nu}$ and the induced metric 
\begin{align}
    \gamma\_{ab} = g\_{\mu \nu}\dv{x\ped{dw}^\mu}{\xi^a} \dv{x\ped{dw}^\nu}{\xi^b};\quad \left[x\ped{dw}^\mu\right] = (\xi\^0, \xi^1, \xi^2, \epsilon(\xi\^a)),
\end{align}
where we let $a,b=0,1,2$. Consider $g\_{\mu\nu}\diff x\^\mu \diff x\^\nu = -{\diff t}^2 + {a^2(t)}\Krondelta{_{ij}}{\diff x\^i}{\diff x\^j}$. We define $\iota_a \equiv \pdv*{\epsilon}{\xi\^a}$ and $\iota_3 \equiv -1$ for notational ease. The determinant of the induced metric is 
\begin{align}
    \gamma = -a^4 [- (a{\iota_0})^2   + \underbrace{{\iota_1}^2  +{\iota_2}^2  + {\iota_3}^2}_{\equiv I}  ].
\end{align}

In the thin wall approximation, the surface tension
\begin{align}
    \sigma = \integ{z}[-\infty][+\infty] T\_{00} \simeq \textcolor{uiopink}{-} \integ{\phi}[\phi_-][\phi_+] \sqrt{2V\ped{eff}(\phi) - 2 V\ped{eff}(\phi_{\pm})}.
\end{align}
The covariant action 
\begin{align}
    S\ped{dw} = \integ[4]{x} \mathcal{L}\ped{dw} = -\sigma \integ[3]{\xi \sqrt{-\gamma}} = -\sigma \integ[4]{x\sqrt{-\gamma}}  \Diracdelta{z-z\ped{dw}}
\end{align}
and thus, the stress--energy tensor
\begin{align}
    T\^{\mu\nu} = \frac{-2}{\sqrt{-g}} \frac{\delta \mathcal{L}\ped{dw}}{\delta g\_{\mu\nu}} = \frac{2 \sigma \Diracdelta{z-\epsilon}}{\sqrt{-g}\sqrt{-\gamma}} \frac{\delta \gamma}{\delta g\_{\mu\nu}} = \frac{2 \sigma \Diracdelta{z-\epsilon}}{a^5\sqrt{I-(a\iota_0)^2}} \frac{\delta \gamma}{\delta g\_{\mu\nu}}.
\end{align}
We have calculated the functional derivative $\delta \gamma \/ \delta g\_{\mu\nu}$ before:
% \begin{align}
%     \frac{\delta \gamma}{\delta g\_{00}} &= a^4 I  &  \frac{\delta \gamma}{\delta g\_{ii}} &= a^2 \pclosed{ (a\iota_0)^2 +{\iota_i}^2-I } \nonumber \\
%     \frac{\delta \gamma}{\delta g\_{0i}} &= -a^4\iota_0 \iota_i & \frac{\delta \gamma}{\delta g\_{ij}} &= a^2{\iota_i} {\iota_j}, i\neq j
% \end{align}
\begin{align}
    \frac{\delta \gamma}{\delta g\_{00}} &= a^4 I  &
    \frac{\delta \gamma}{\delta g\_{0i}} &= -a^4\iota_0 \iota_i &
    \frac{\delta \gamma}{\delta g\_{ij}} &=  a^2 \bclosed{{\iota_i} {\iota_j} + \Krondelta{_{ij}} \pclosed{ (a\iota_0)^2 -I } }
\end{align}

With the ansatz $\epsilon (\xi\^a)= \epsilon_t(\xi\^0)\eu[\im u\_1 \xi\^1]\eu[\im u\_2 \xi\^2]$, solutions for the equations of motion for $\epsilon_t$ are known for $a \propto t^\beta$. In that case, $\iota_0 = \dot{\epsilon}$, $\iota_1 = \im u\_1 \epsilon$, $\iota_2 = \im u\_2 \epsilon$ and, of course, $\iota_3 = -1$. 





\subsection{The Symmetron Potential}
    It is easily shown that for $V\ped{eff}(\phi) = V\ped{Sym}(\phi) \equiv \frac{\lambda}{4} \pclosed{\phi^2-{\phi_0}^2}^2$, the surface tension reduces to $\sigma = \sigma_0 \equiv\frac{4}{3}  {\phi_0}^3 \sqrt{\lambda/2}$. We consider matter domination ($a\propto t^{2/3} \propto \eta^2$) and assume a solution $\epsilon_{\eta} = \epsilon_0  \eta^{-5/2} \Bessel[\sfrac{5}{2}](u \eta) $, where $\eta$ is conformal time. Note that $\epsilon' = a\dot{\epsilon}$. We have
    \begin{align}
        T\^{\mu\nu}(\eta,\vec{x}) = \frac{2 \sigma_0 \Diracdelta{z-\epsilon}}{a^5\sqrt{I-(a\dot{\epsilon})^2}} \frac{\delta \gamma}{\delta g\_{\mu\nu}} = \frac{2 \sigma_0 \Diracdelta{z-\epsilon}}{a^5\sqrt{1 - (u\epsilon)^2 - {\epsilon'}^2}} \frac{\delta \gamma}{\delta g\_{\mu\nu}}.
    \end{align}
    Neglecting all $\BigO{\epsilon^2}$-terms, we get that the only non-vanishing contributions are: 
    \begin{align}
        \delta \gamma \/ \delta g\_{00} &= a^4 & 
        \delta \gamma \/ \delta g\_{11} &= -a^2  &  
        \delta \gamma \/ \delta g\_{22} &= -a^2  &  \\
        \delta \gamma \/ \delta g\_{03} &= a^3 \epsilon' & 
        \delta \gamma \/ \delta g\_{13} &= -\im u\_{1}a^2 \epsilon &
        \delta \gamma \/ \delta g\_{23} &= -\im u\_{2}a^2 \epsilon
    \end{align}
    
    % \begin{multline} 
    %     \left[T\^{\mu\nu}\right] (\eta, \vec{k})= \frac{8\sigma_0\ppi^2}{a^3}
    %     \Bigg\{ 
    %     \bclosed{\Diracdelta{k\_1} \Diracdelta{k\_2} + \im k\_3 \epsilon_\eta \Diracdelta{k\_1+u\_1}  \Diracdelta{k\_2 + u\_2} } \cdot
    %     \left(\begin{array}{rrrr}
    %         a^2 & & & \\ %
    %          & 1 & & \\ %
    %          & & 1& \\ %
    %          & & & 0%
    %         \end{array}\right)  \\
    %     + \Diracdelta{k\_1+u\_1}  \Diracdelta{k\_2 + u\_2} \cdot
    %     \left(\begin{array}{rrrr}
    %         0& & & a\epsilon_\eta' \\ %
    %         & 0& & -\im u\_1 \epsilon_\eta \\ %
    %         & & 0& -\im u\_2 \epsilon_\eta \\ %
    %        a\epsilon_\eta' & -\im u\_1 \epsilon_\eta  & -\im u\_2 \epsilon_\eta &0 %
    %        \end{array}\right) \Bigg\}
    % \end{multline}

    \begin{multline} 
        \left[T\^{\mu\nu}\right] (\eta, \vec{k})= \frac{8\sigma_0\ppi^2}{a^3}
        \left(\begin{array}{cccc}
            a^2 A &0 &0 & a\epsilon_\eta'B \\ %
             0& -A & 0& -\im u\_1 \epsilon_\eta B\\ %
             0& 0& -A& -\im u\_2 \epsilon_\eta B\\ %
             a\epsilon_\eta'B & -\im u\_1 \epsilon_\eta B  & -\im u\_2 \epsilon_\eta B &0%
            \end{array}\right) ; \\
        A = \Diracdelta{k\_1} \Diracdelta{k\_2} + \im k\_3 \epsilon_\eta B;\quad B= \Diracdelta{k\_1+u\_1}  \Diracdelta{k\_2 + u\_2}
    \end{multline}
    \comment{Maybe not the best representation of this ...}

    
    
    % An explicit calculation of the argument in the square root and the functional derivatives \comment{using ${\epsilon_\eta'} = {\epsilon_0} u\eta^{5/2} {\Bessel[7/2]}(u\eta)$; fairly easy to show this}, we have an analytical \speak{(?)} expression for $T\^{\mu\nu}$. \speak{Maybe expand the root? Then remove $\BigO{\epsilon^3}$-terms?}

    % \paragraph{Gravitational Waves.} %
    % We solve $\sq h\_{\mu\nu} = 16\ppi G\ped{N} T\_{\mu\nu}$ 
    {
    %%%%%%%
    \newcommand*{\Lam}{\ProjectionLambda}
    %%%%%%%
    \paragraph{Gravitational Waves.} %
    The transverse, traceless tensor perturbation $h\_{ij}$, cropping up in the perturbed line element ${ds}^2 = a^2 \cclosed{- {\diff \eta}^2 + (\Krondelta{_{ij}}  + h\_{ij}){\diff x\^i}{\diff x\^j}}$, has the e.o.m.
    \begin{align}
        \bclosed{\dv[2]{}{\eta} + 2 \frac{a'}{a} \dv{}{\eta} + k^2 }h\_{ij}(\eta, \vec{k}) = 16\ppi G\ped{N} \Lam{ij}{kl}(\vec{n}) T\_{kl} (\eta, \vec{k}); \quad \vec{k} = k\vec{n}, \abs{\vec{n}}=1.
    \end{align}
    We extracted the transverse, traceless (TT) part of the symmetric stress--energy tensor by use of the ``Lambda tensor'' $\Lam{ij}{kl}$.
    
    }