% ABSTRACT


%
Topological defects predicted by extensions of general relativity can manifest in gravitational-wave observations on Earth. Future broadband gravitational-wave observations show promise of constraining such models. Analytical models can in the ultimate consequence access certain parameter spaces and dynamical ranges that are prohibitively expensive to simulate.
% simulatively unavailable. 

%
As a toy example, we present symmetron domain walls formed at redshift $\sim 2$ (around ten billion years ago) 
% at redshift $\sim 2$ \speak{(lookback time $8.1\times 10^9$ yrs)} 
with structural ripples and investigate how gravitational radiation carries information about the symmetry-breaking theory. 
We derive the dynamics in the thin-wall limit from the Nambu--Goto action, and analytically solve the equation of motion for a small perturbation to planar walls \emph{during} phase transition, when the surface energy density is time-dependent. These results are compared to the full field theory with deterministic simulations.

%
We find good qualitative correspondence between the thin-wall approximation and cosmological simulations. 
% The symmetron field is not surprisingly quantitatively 
The wall positions from the two theories have average difference $\sim 7 \%$ of initial perturbation magnitude when this is $\lesssim 10\%$ of the simulation box size. 
Lack of analytical summary statistics for the prototype makes comparative gravitational-wave analyses complicated, but we provide a superficial presentation of the results and discuss specifics of potential ways forward. 
% The gravitational-wave results are not compared quantitatively, but show \speak{noe bra}

%
We present a protocol for reducing scalar-field oscillations to perform controlled toy-model simulations.
% \speak{Noe om symmetron stabil}