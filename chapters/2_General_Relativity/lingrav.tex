% ||||||||||||||||||||||||||||||||||||
% |||||| 2.3 Linearised gravity ||||||
% ||||||||||||||||||||||||||||||||||||


% ----------------------------------------
% labels: \label{[type]:GR:lingrav:[name]}
% ----------------------------------------



% 
% \newcommand{\pert}[1]{\accentset{\circ}{#1}}    % perturbed quantities
% \newcommand*{\ah}{\bar{h}}
\newcommand{\pert}[1]{\ALIASpert{#1}}           % perturbed quantities
\newcommand*{\ah}{\ALIASah}                     % a h
%


Applying perturbation theory to the metric gives rise to a new set of equations, often referred to as the ``linearised Einstein field equations.'' The general starting point is to expand the metric to the order $o$ in question $\pert{o}$, 
\begin{equation}
    \pert{g}\_{\mu\nu} = g\_{\mu\nu}+ \sum_{i=1}^{o} \delta^{(i)}g\_{\mu\nu}.
\end{equation}
% $g\_{\mu\nu}+ \sum_{i=0}^{\text{order}} \delta^{(i)}g\_{\mu\nu}$
Consequently, the perturbed Einstein reads 
\begin{equation}
    \pert{\mathcal{G}}\_{\mu\nu} = 8\ppi G\nped{N} \pert{T}\_{\mu\nu},
\end{equation}
where $\pert{\mathcal{G}}\_{\mu\nu}=\sum_{i=0}^{o} {}^{(0)}\mathcal{G}\_{\mu\nu}$ and $\pert{T}\_{\mu\nu}=\sum_{i=0}^{o} {}^{(0)}T\_{\mu\nu}$, in which ${}^{(i)}Q$ implies the perturbed quantity to order $o$. It is then solved order by order.
% which is solved order by order.
% \begin{equation}
%     \mathcal{G}\_{\mu\nu} + \delta \mathcal{G}\_{\mu\nu} = 8\ppi G\nped{N}  \pclosed{T\_{\mu\nu} + \delta T\_{\mu\nu}},
% \end{equation}
\comment{Either put details in appendix or refer to e.g.~\citet{jokelaGravitationalWaveMemory2022}}


% Four of the ten degrees of freedom 

\phpar[About gauge freedom, TT gauge etc.---why ``waves''] 


Metric perturbations on a homogeneous and isotropic background leaves two tensor degrees of freedom. We extract these by use of the spin-2 projection tensor On the Transverse-Traceless Projection in Lattice
Simulations of Gravitational Wave Production



\hlineSep


\subsection{Gravitational waves in an expanding universe}\label{sec:GR:lingrav:gws_FLRW}
    We are interested in a flat FRLW background plus first order in perturbations, and define the perturbed metric as \( \pert{g}\_{\mu\nu} = a^2 (\eta\_{\mu\nu} + h\_{\mu\nu} )\). Furthermore, we focus on the tensorial part of the perturbations %and consider the gauge-invariant $h\_{ij}$ that fulfills the TT criteria.\footnote{We omit the $^{\mathrm{TT}}$ superscript for cleaner expressions.}
    \begin{equation}
        \pert{ds}^2 = a^2(\tau) \pclosed{ -{\diff \tau}^2 + \bclosed{\Krondelta{_{ij}} + h\_{ij}(\tau, \vec{x}) } {\diff x\^i}{\diff x\^j}  }
    \end{equation}
    \blahblah
    \begin{equation}
        \sq h\_{ij}(\tau, \vec{x}) = - 16 \ppi G\nped{N} a^2  T\_{kl}(\tau, \vec{x})
    \end{equation}
    For notational ease, we define \(S\_{ij}\equiv 16 \ppi G\nped{N} T\_{ij} \) and omit the lower indices. In real space, the equation reads
    % \begin{equation}
    %     \ddot{h}(\tau, \vec{x}) + 2 \mathcal{H} \dot{h}(\tau, \vec{x}) -\vec{\nabla}^2 h(\tau, \vec{x}) = S(\tau, \vec{x}),
    % \end{equation}
    \begin{equation}
        \ddot{h} + 2 \mathcal{H} \dot{h} -\vec{\nabla}^2 h = S,
    \end{equation}
    which in its homogeneous form is we recognise as a damped harmonic oscillator. %
    It is convenient to transform to Fourier space where $k\_{i} \leftrightarrow \im \partial\_{i}$, and even more so to introduce $\ah \equiv ah$, leaving us with
    % \begin{equation}
    %     \ddot{h} + 2 \mathcal{H} \dot{h} + k^2 h = S.
    % \end{equation}
    \begin{equation}
        \ddot{\ah} + \pclosed{ k^2 - \frac{\ddot{a}}{a} } \ah = aS.
    \end{equation}
    We see that for large modes $k \gg \ddot{a}/a \sim \tau^{-2}$, the linear operator is approximately $\partial_\tau^2 + k^2$, i.e.~the harmonic oscillator, with plane wave solutions for $S=0$. That is to say, small-scale gravitational waves, in the absence of a source, propagate free waves in an FRW spacetime (divided by the scale factor). On larger scales, the propagation is damped in accordance with the expansion of the universe. This damping term generally depends on expansion history.


    \paragraph{Homogeneous solution.} %
    We see that for large modes $k^2 \gg \ddot{a}/a \sim \tau^{-2}$, the general solution


    \paragraph{Inhomogeneous solution.} %
    The method of Green's functions presents a suitable recipe for determining the dynamics of tensor perturbations on an expanding background. There are some limitations, however, as to the analytical solvability of the system. In the small-scale limit, the damping is neglected, and we only need the retarded Green's function associated with the harmonic oscillator, $\sin{\pclosed{k(\tau-\tau')}}/k$. Otherwise, an equation of the form $\mathop{\mathrm{L}_{u=k\tau}}h=aS/k^2$, where
    \begin{equation}
        \mathop{\mathrm{L}_{u}} = \dv[2]{}{u} + \pclosed{1 - \frac{(\alpha-1)\alpha}{u^2}}
    \end{equation}
    has a Green's function in terms of Bessel functions $\sqrt{u}\Cylindrical[\alpha-1/2](u)$, which is the case in a single-substance universe with $a \propto \tau^\alpha$. On an even more compact form, if $n\equiv\alpha-1 \in \Integer$, we can use the Green's function
    \begin{equation}\label{eq:GR:lingrav:Greens_function_n}
        G(u,v) = \RiccatiBessel[n](u)\RiccatiNeumann[n](v) - \RiccatiNeumann[n](u)\RiccatiBessel[n](v) \mathsf{\chi} \Lambda
    \end{equation}
    where $\RiccatiBessel[n](x)$ and $\RiccatiNeumann[n](x)$ are the Riccati--Bessel and --Neumann functions (given in \cref{app:special}).
        
    % \end{equation}
    % % Otherwise, for a single-substance universe, 
    % \begin{equation}
        
    % \end{equation}
    Assume homogeneous initial conditions, $\ah(\tau\ped{init}, \vec{k})= \dot{\ah}(\tau\ped{init}, \vec{k})=0$. 
    The full solution is as follows:
    \begin{equation}\label{eq:GR:lingrav:hij_expression_of_Greens_and_source}
        h\_{ij}(\tau, \vec{k}) = \frac{16\ppi G\nped{N}}{k^2} \integ{\eta}[\tau\ped{init}][\tau] G(k\tau, k\eta) \frac{a(\eta)}{a(\tau)} T\_{ij}(\eta, \vec{k}).
    \end{equation}







\subsection{Polarisation of tensor modes / \tmptitle{Effect on test particles}}\label{sec:GR:lingrav:polbasis}
    {% -----------------------
    \newcommand*\mhat{\hat{\vec{m}}}
    \newcommand*\nhat{\hat{\vec{n}}}
    \newcommand*\Ohat{\hat{\vec{\varOmega}}}

    \newcommand*{\polplus}{\rotatebox[origin=c]{0}{+}}
    \newcommand*{\polcross}{\rotatebox[origin=c]{45}{+}}


    %%%%%%%%%%%%%%%
    \phpar[a longer discussion---specific to GR]





    % A popular choice is the linear polarisation basis
    From the right-handed orthonormal basis $\{\mhat, \nhat, \Ohat \}$---for which $\Ohat \parallel \vec{k}$---we may construct a linear polarisation basis from the polarisation tensors
    \begin{equation}
        \begin{split}
            e^+     (\Ohat) &= \mhat \otimes \mhat - \nhat \otimes \nhat \quad \text{and} \\
            e^\times(\Ohat) &= \mhat \otimes \nhat - \nhat \otimes \mhat.
        \end{split}
    \end{equation}
    This is a popular choice, and it \blahblah

    Now we retrieve
    \begin{equation}
        h\_{ij}(\tau, \vec{k}) = \sum_{P=\polplus, \polcross} h_P(\tau,k \Ohat) e^P_{ij} (\Ohat)
    \end{equation}
    and observe that
    \begin{equation}
        \abs{h^2(\tau, \vec{k}) } = \sum_{ij} h\^{ij}(\tau, \vec{k}) h\_{ij}(\tau, \vec{k})  = 2 \sum_{P}h_P^2(\tau, \vec{k}).
    \end{equation}

    \pensive{monochromatic\polplus: O-I-O-Å, \dots monochromatic \polcross: Michael Jackson dance?}
    % -----------------------
    }%





% \subsection{Transverse and traceless projection}
%     The 
%     \phpar[Address this]



% \subsection{Gravitational waves}
%     The term ``gravitational waves'' refers to the \nc{tensor perturbations to the background metric}. These ``waves'' are spacetime distortions whose name comes from the fact that \checkthis{they obey the wave equation}.



