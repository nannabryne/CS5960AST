% |||||||||||||||||||||||||||||||||||||||||||||
% |||||| 2.3 Standard model of cosmology ||||||
% |||||||||||||||||||||||||||||||||||||||||||||



% -------------------------------------
% labels: \label{[type]:GR:lcdm:[name]}
% -------------------------------------


% ¨¨¨¨¨¨¨¨¨¨¨¨¨¨¨¨¨¨¨¨¨¨¨¨¨¨¨¨¨¨¨¨¨¨¨¨¨¨¨¨¨¨¨¨¨
% LOCAL MACROS:
\newcommand*\pert{\ALIASpert}
% ¨¨¨¨¨¨¨¨¨¨¨¨¨¨¨¨¨¨¨¨¨¨¨¨¨¨¨¨¨¨¨¨¨¨¨¨¨¨¨¨¨¨¨¨¨







Employing the cosmological principle (CP) to the Einstein equation, simplifies them drastically. The cosmological principle states that the universe is spatially homogeneous and isotropic%
% \comment{---some ``folkelig'' comment---}%
, %
or equivalently that the line element of the universe takes the form 
${ds}^2 = -{\diff t}^2 + a^2(t){d\varSigma}^2$ where ${d\varSigma}^2$
is a three-dimensional metric with a specified Gaussian curvature. The scale factor $a$ describing the universe's expansion is %
\begin{equation}
    a = \frac{1}{1+\redshift},
\end{equation}
where $\redshift$ is the cosmic redshift. %
This was the starting point for Alexander Friedmann (1888--1925) when he developed the Friedmann equations. If CP holds---which is widely accepted as a good approximation at large scales, typically orders larger than $\mathscr{O}(100~\mathrm{Mpc})$---any constituent $s$ behaves as a perfect fluid, and the Hubble parameter $H$ reads %\comment{decide which here}
% \begin{equation}
%     \Omega\ped{tot} \equiv \frac{\rho}{\rho\ped{cr0}} = \sum_{s} \Omega_s
% \end{equation}
\begin{equation}\label{eq:GR:lcdm:first_Friedmann}
    H^2 = H_0^2 \sum_s \Omega_{s0} a^{-3(1+w_s)}.
\end{equation}
The parameter $\Omega_{s0}\equiv \rho_{s0}/\rho\ped{cr0}$ is a dimensionless present-day energy density measure, normalised to the critical density today, $\rho\ped{cr0}= 3H_0^2 M\nped{Pl}^2$. %
The equation-of-state parameter of $s$, $w_s$, is given by
% $w_s$ is the equation-of-state parameter of $s$, i.e. 
\[\text{isotropic pressure} = w \cdot \text{energy density}.\] %
\Cref{tab:GR:lcdm:fluids} provides an overview of the various phenomena (potentially) contributing to the total energy in the universe.
% \begin{itemize}
%     \item Relativistic matter (radiation, neutrinos): $w\ped{r}=1/3$
%     \item Non-relativistic matter (baryons, CDM): $w\ped{m}=0$ % "dust"
%     \item Effective curvature: $w\ped{K}=-1/3$
%     \item Domain wall network: $w\ped{dw}=-2/3$
%     \item Vacuum energy (\textLambda): $w_{\text{\textLambda}} =-1$
% \end{itemize}

% Note that we define Big Bang as the time when $a=0$ and today as the time when $a=a_0=1$.
% Note that $a|_{\text{today}}= a(t_0) = a_0 = 1$, and $a|_{}$ 

% \begin{description}
%     \item[relativistic matter] has $w\ped{r}=1/3$ and slows the expansion
%     \item[non-relativistic matter] has $w\ped{r}=1/3$, i.e. is pressureless
%     \item[effective curvature] has $w\ped{K}=-1/3$
%     \item[cosmological constant] has $w_\text{{\textLambda}}=-1$, i.e. negative pressure and therefore drives the expansion rate up
% \end{description}

% \begin{
% This is widely accepted as a good approximation at large scales $>\mathscr{O}(100~\mathrm{Mpc})$.

% \pensive{CP + EFEs = FI \& FII}

% \pensive{CP holds for scales $>\mathscr{O}(100~\mathrm{Mpc})$}


% \comment{Get to Hubble tension.}




We will work with conformal time $\tau$ that relates to cosmic time $t$ such that the metric is%$a \diff \tau = \diff t$
\begin{equation}
    {ds}^2 = a^2 (-{\diff \tau}^2+ {\diff \varSigma}^2).
\end{equation}
Subsequently, we make use of the conformal Hubble factor $\mathcal{H}\equiv a H$, given by
\begin{equation}\label{eq:GR:lcdm:first_Friedmann_conformal}
    \mathcal{H}^2 = H_0^2 \sum_s \Omega_{s0} a^{-(1+3w_s)},
\end{equation}
To get simplified---or simply analytically solvable---equations, we often assume $a\sim \tau^\alpha$ that corresponds to a universe dominated by a single substance. Now, $\alpha$ is ultimately determined by the equation-of-state parameter $w_s$ associated with the substance $s$ in question, 
\begin{equation}
    \alpha = \frac{2}{1+3w_s}.
\end{equation}
This means that $\alpha=1$ and $\alpha=2$ for radiation (RD) and matter domination (MD), respectively. %
% Put in cosmic context, \comment{maybe have a figure? Or write something about the different eras?}

\begin{table}[h]
    {\import{tables/Background}{fluids.tex}
    \caption{Perfect-fluid description of various phenomena. The last coloumn shows the corresponding topological defect, which will be introduced in~\cref{sec:cosmo:defects}.}
    \label{tab:GR:lcdm:fluids}}
\end{table}

% With the present-day fiducial values $\Omega\ped{m0}\simeq 0.67$
With fiducial observed values for $\Omega\ped{r0}$, $\Omega\ped{m0}$ and $\Omega_{\text{\textLambda}0}$, it is understood that our universe was dominated by radiation in its early stages, after inflation and until before $a=\Omega\ped{r0}/\Omega\ped{m0}$ (when $\rho\ped{m}=\rho\ped{r}$, about 50 thousand years after Big Bang). The universe entered a matter-dominated era after this, until $a=(\Omega\ped{m0}/\Omega_{\text{\textLambda}0})^{1/3}$ (when $\rho_{\text{\textLambda}}=\rho\ped{m}$, about 10 billion years after Big Bang). The universe started accelerating, and is currently on steady course to a vacuum-dominated era. 


\subsubsection{Hubble trouble}
    The~\newconcept{Hubble tension} refers to a discrepancy between measurements of the Hubble constant describing the expansion rate of the universe. Different methods yield conflicting values of $H_0 = 100~ \mathrm{h} \unit{km} \unit{s^{-1}} \unit{Mpc^{-1}}$, where $\mathrm{h}$ is the reduced Hubble constant. Local measurements involving for example distant supernovae suggest $\mathrm{h} \approx 0.73$~\citep[e.g.][]{maggioreGravitationalWavesVol2018}. In contrast, the other approach, derived from CMB observations by the Planck satellite, indicates $\mathrm{h}\approx 0.67$~\citep[e.g.][]{maggioreGravitationalWavesVol2018}. Their confidence intervals do not overlap, thus necessitating new measurements~\citep[e.g.][]{maggioreGravitationalWavesVol2018}. %\speak{This citing seems excessive\dots}


    Many argue that revision of \textLambda{}CDM is necessary. The accelerated expansion is a late-universe phenomenon conservatively made clear by the dark-energy/cosmological-constant duality, with $w\ped{de}=w_{\Lambda}=-1$. %$w_{\Lambda}=-1$. 
    Introducing instead a time-dependent ``phantom'' equation-of-state parameter $w\ped{de}(a)$ can harmonise the two measurements~\citep{maggioreGravitationalWavesVol2018}. %
    % Among the proposed solutions to the Hubble tension is to allow the dark-energy equation-of-state parameter to vary with time. 
    %A scalar 
    % Allowing an effective dark-energy equation-of-state parameter to vary with time, \blahblah
    % \blahblah
    % \rcomment{More details?}







\subsection{Cosmological perturbation theory}\label{sec:GR:lcdm:cosmopert}
    Just look around you---the universe is definitely \emph{not} homogeneous and isotropic. %CP generally fholds for scales $>\mathscr{O}(100~\Mpch)$, but 
    The story of structure formation as we know it is told through cosmological perturbation theory. The leading-order perturbed metric $\pert{g}\_{\mu\nu}= g\_{\mu\nu} + \delta g\_{\mu\nu}$ can be written in terms of functions $A$, $B\_i$ and $C\_{ij}$,
    \begin{equation}\label{eq:GR:lcdm:perturbed_metric_Bardeen}
        \pert{ds}^2 = a^2 \bclosed{ -(1+2A){\diff\tau}^2  + 2B\_{i}{\diff x\^i}{\diff \tau} + (\Krondelta{_{ij}} + C\_{ij}){\diff x\^i}{\diff x\^j} }.
    \end{equation}
    % \begin{equation}\label{eq:GR:lcdm:perturbed_metric_Bardeen}
    %     \pert{ds}^2 = {\diff \tau}^2 \bclosed{ -(1+2\Phi){\diff\tau}^2  + 2w\_{i}{\diff x\^i}{\diff \tau} + ((1-2\Psi)\Krondelta{_{ij}} + 2s\_{ij}){\diff x\^i}{\diff x\^j} }.
    % \end{equation}
    It is convenient to adopt the convention that spatial vectors and tensors are raised and lowered with $\Krondelta{}$, e.g.~$C\^{ij}=\Krondelta{^{ik}}\Krondelta{^{jl}} C\_{kl}$. The symmetric metric has ten degrees of freedom, and a scalar-vector-tensor (SVT) decomposition separates these into scalar perturbations $A$, $B$, $C$, $D$, vector perturbations $F\_i$, $G\_i$, and tensor perturbations $E\_{ij}$:
    % four scalar , four vector and two tensor (unphysical) degrees of freedom:
    \begin{subequations}
        \begin{align}
            A    &\to  A, \\
            B\_i &\to \partial\_i B + F\_i,  \\
            C\_{ij} &\to ( 2 \Krondelta{_{ij}}C + 2\partial\_{i} \partial\_j D  - (2/3)\Krondelta{_{ij}}\vec{\nabla}^2 D ) + 2\partial\_{(i}G\_{j)}  + 2E\_{ij}.
        \end{align}
    \end{subequations}
    % We have the scalar perturbations $A$, $B$, $C$, $D$, the vector perturbations $F\_i$, $G\_i$, and the tensor perturbations $E\_{ij}$. 
    % \comment{IMPROVE.} %
    This is extremely useful since in the first-order linearised Einstein equation for scalars, vectors and tensors do not mix. We treat these separately and assume they originate from inflation.
    % \begin{description}
    %     \item[Scalar] perturbations are density perturbations and describe structure formation.
    %     \item[Vector] perturbations  
    % \end{description}
    % \begin{itemize}
    %     \item Scalar perturbations ($A$, $B$, $C$, $D$) are \comment{density perturbations} and describe structure formation.
    %     \item Vector perturbations ($F\_i$, $G\_i$) are not predicted by inflation, and would in any case only have decaying solutions, and are thus cosmologically irrelevant.
    %     \item Tensor perturbations ($E\_{ij}$)---gravitational waves---are predicted by inflation.
    % \end{itemize}
    This thesis focuses solely on the tensorial part of the metric perturbation. Therefore, we will consider the divergence- and traceless $h\_{ij}=2E\_{ij}$. This particular choice is called the \newconcept{transverse-traceless} (TT) \newconcept{gauge}, and will be addressed in the next section. %
    % $\partial\^i h\_{ij} =0$ and $\Krondelta{^{ij}}h\_{ij}=0$. Any symmetric tensor $T\_{ij}$ can be projected onto the TT gauge by use of the projection tensor in~\cref{eq:notation:projection_tensor},
    % \begin{equation}
    %     T\ap{TT}_{ij}(\vec{k}) = \Lambda\indices*{^{ij}_{kl}}(\vec{k}) T\_{kl}(\vec{k}).
    % \end{equation}
    Plugging this into the linearised Einstein equation~\cref{eq:GR:einstein:linearised_Einstein_eq}, we get the equation of motion for the tensor perturbations $h\_{ij}$, which (spoiler alert) are in fact the gravitational waves.

    % \comment{Plug this into Einstein's equation}

    % \begin{bullets}
    %     \item Hodge decomposition $\accentset{\circ}{g}\_{\mu\nu} = g\_{\mu\nu} + \delta g\_{\mu\nu} $ where $g\_{\mu\nu}$ is FRLW
    % \end{bullets}








% \subsection{Hubble trouble}\label{sec:GR:lcdm:problems}
%     The~\newconcept{Hubble tension} refers to a discrepancy between measurements of the Hubble constant describing the expansion rate of the universe. Different methods yield conflicting values of $h_0 = 100 H_0 \unit{km} \unit{s^{-1}} \unit{Mpc^{-1}}$. Local measurements involving for example distant supernovae suggest $h_0 \approx 0.73$. In contrast, the other approach, derived from CMB observations by the Planck satellite, indicates $h_0\approx 0.67$. Their confidence intervals do not overlap, thus necessitating new measurements. 


%     Many argue that revision of \textLambda{}CDM is necessary. The accelerated expansion is a late-universe phenomenon conservatively made clear by dark energy with $w\ped{de}=w_{\Lambda}=-1$. %$w_{\Lambda}=-1$. 
%     Among the proposed solutions to the Hubble tension is to allow the dark-energy equation-of-state parameter to vary with time. %A scalar 
%     % Allowing an effective dark-energy equation-of-state parameter to vary with time, \blahblah
%     \blahblah


    % \begin{bullets}
    %     \item Hubble tension
    %     \item Flatness problem
    %     \item Extended models of gravity
    %     \item \textLambda{}CDM still very good!
    % \end{bullets}

    % The cosmic microwave background (CMB) 
    % That the universe is expanding, there is little doubt of.




    % In combining the distance to a nearby galaxy and the rate at which it moves away from us, we find the expansion rate of the universe
    % Observations of stellar objects measures the cosmic redshift $\redshift = 1/a - 1$ 


    % In short, local measurements suggests $h_0 \simeq 0.73$, but CMB observations will have it $h_0\simeq 0.67$. The measurement errors cannot explain this. 



