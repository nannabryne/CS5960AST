%%%%%%%%%%%%%%%%%%%%%%%%%%%%%%%%%%%%%%%%%%
%%%%%%% Ch. 2: General Relativity  %%%%%%%
%%%%%%%%%%%%%%%%%%%%%%%%%%%%%%%%%%%%%%%%%%



% --------------------------------
% labels: \label{[type]:GR:[name]}
% --------------------------------





\begin{bullets}
    \item SECTIONS: %
        \begin{description}
            \item[Differential geometry:] The basics, hypersurfaces++, conformal geometry 
            \item[Einstein's field equations:] How-to, SE tensor 
            \item[Gravitational waves:] Basics
            \item[Standard model of cosmology:] Basics, problems 
        \end{description}
    \item AIM: Basis for standard GR 
\end{bullets}


% ////////////////// intro //////////////////




Alongside quantum mechanics, Einstein's theory of gravity---general relativity (GR)---is widely accepted as the most accurate description of our surroundings. GR can be formulated from a geometrical point of view, or it can be viewed as a classical field theory. In the former approach we meet geometrical tools such as the geodesic equation, whereas the latter allows the application of field-theoretical methods. 
%This chapter lays emphasis on the field interpretation of GR. 

\phpar[Two perspectives insightful; better overall understanding of aspects of concepts in GR]




% ///////////////////////////////////////////




% ****************** SECTIONS ******************


% Differential geometry
\section{Differential geometry}\label{sec:GR:diffgeo}
{\subimport{./}{diffgeo.tex}}


% Einstein's equation
\section{Einstein's equation}\label{sec:GR:einstein}
{\subimport{./}{einstein.tex}}


% Standard model of cosmology
\section{Standard model of cosmology}\label{sec:GR:lcdm}
{\subimport{./}{lcdm.tex}}


% Linearised gravity
\section{Linearised gravity; tensor sector}\label{sec:GR:lingrav}
{\subimport{./}{lingrav.tex}}



% **********************************************








\clearpage
\newpage
