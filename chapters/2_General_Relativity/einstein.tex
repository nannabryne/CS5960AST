% ||||||||||||||||||||||||||||||||||||||
% |||||| 2.2 Einstein's equation  ||||||
% ||||||||||||||||||||||||||||||||||||||


% -----------------------------------------
% labels: \label{[type]:GR:einstein:[name]}
% -----------------------------------------









How does the gravitational field affect how matter behaves, and in what way is matter controlling the gravitational field? Newtonian gravity proposes very good answers to these questions: The acceleration of an object in a gravitational potential $\Phi$ is
\begin{equation}
    \vec{a} = -\vec{\nabla} \Phi,
\end{equation}
and said field is governed by the matter density $\rho$ through the Poisson equation
\begin{equation}
    \vec{\nabla}^2\Phi =  4 \ppi G\nped{N} \,\rho.
\end{equation}
% This framework 
% In physics, the answer to a question is highly dependent on \emph{how the question was asked}. %
% In physics, the answer you get is hugely dependent on \emph{how the question was asked}.
In physics, the answer to a question is highly dependent on \emph{how the question was asked}.
A common misconception is that Newtonian gravity was disproven by Einstein.  
% In retrospect, we can interpret Newton's 
Newton was simply telling a different story; a story about dynamics in non-relativistic systems.\footnote{Which, to be fair, are most common on Earth.} %Einstein's theory of general relativity proposes both different questions and answers to gravitational physics. 
The discovery of the limitations to Newtonian dynamics (cf.~precision of Mercury's orbit) called for novel descriptions of gravitation. 
Einstein confronted gravitational physics with different but analogous questions, and subsequently more complex answers than Newton. General relativity explains how curvature of spacetime influences matter, manifesting as gravity, and in what way energy and momentum affects spacetime to create curvature. %
In mathematical terms, these are the \newconcept{geodesic equation}
\begin{equation}
    \dv[2]{x\^\mu}{\lambda} = - \Gamma\indices*{^\mu_{\rho\sigma}} \dv{x\^\rho}{\lambda} \dv{x\^\sigma}{\lambda}
\end{equation}
and \newconcept{Einstein's equation}
\begin{equation}\label{eq:GR:einstein:Einstein_eq_G}
    \mathcal{G}\_{\mu\nu} = 8 \ppi G\nped{N} \, T\_{\mu\nu}.
\end{equation}
One can arrive at this equation by use of the minimal coupling principle and Bianchi identities (\cref{eq:GR:diffgeo:Bianchi_identities}), as explained in~\citet[Ch.~4]{carrollSpacetimeGeometryIntroduction2019}. %
% \checkthis{Through the arguments like the minimal coupling principle Bianchi identities (\cref{eq:GR:diffgeo:Bianchi_identities}), it is possible to arrive at this expression, as explained in~\citet[Ch.~4]{carrollSpacetimeGeometryIntroduction2019}.}
% These can be obtained through arguments like the Bianchi identities (\cref{eq:GR:diffgeo:Bianchi_identities}) and
% These can be obtained by
% \comment{Not too long chapter, but want to mention the ``naive'' way of thinking from which these can be obtained (minimal coupling etc.).}
% \citep[Ch.~4]{carrollSpacetimeGeometryIntroduction2019}
A more tangible way to obtain the same equation is to vary the combined matter and Einstein--Hilbert actions $S =S\nped{EH} + S\ped{m}$, where~\citep{carrollSpacetimeGeometryIntroduction2019}
\begin{equation}
    S\nped{EH} =  \frac{M\nped{Pl}^2}{2} \integ[4]{x\sqrt{-g} } \mathcal{R}.
\end{equation}
The result is~\citep{carrollSpacetimeGeometryIntroduction2019}
\begin{equation}\label{eq:GR:einstein:Einstein_eq_R}
    \mathcal{R}\_{\mu\nu} - \frac{1}{2}g\_{\mu\nu} \mathcal{R}= - \frac{1}{M\nped{Pl}^2} \frac{2}{\sqrt{-g}}\Fdv{S\ped{m}}{g\^{\mu\nu}},
\end{equation}
where the right-hand side is identified as proportional the energy--momentum tensor, $8\piGN T\_{\mu\nu}$, and the left-hand side is the Einstein tensor $\mathcal{G}\_{\mu\nu}$ (\cref{eq:GR:diffgeo:Einstein_tensor}). We see that~\cref{eq:GR:einstein:Einstein_eq_G,,eq:GR:einstein:Einstein_eq_R} are equivalent. %
% \comment{$(d+1)$ dimensions?} 
\iftime{rewrite this}


\subsection{Energy--momentum tensor}\label{sec:GR:einstein:SE_tensor}
    The (Hilbert) \newconcept{stress--energy} (SE) \newconcept{tensor}, also known as the energy--momentum tensor, constitute the non-geometrical part of the Einstein equation---the right-hand side of~\cref{eq:GR:einstein:Einstein_eq_G}. It encodes information about the energy and momentum contained in the system.

    For a perfect fluid, the SE tensor in the fluid rest frame is $T\indices{^\mu_\nu}= \text{diag}(-\rho, p, p, p)$ where $\rho=-T\iud{0}{0}$ is the homogeneous energy density and $p= (1/3) T\iud{i}{i}$ is the isotropic pressure. We will address this in a cosmological context in the next section.


    \subsubsection{Conformal transformations}
        {
        % ¨¨¨¨¨¨¨¨¨¨¨¨¨¨¨¨¨¨¨¨¨¨¨¨¨¨¨¨¨¨¨¨¨¨¨¨¨¨¨
        % LOCAL MACROS:
        % \newcommand\Chr{\ChristoffelSym}
        % \newcommand*\Kd{\Krondelta}
        % \newcommand*\Ric{\RicciScalar}
        \newcommand\cfac{\Upsilon}
        \newcommand\wt{\widetilde}
        % \newcommand\hypacc{\ALIAShypacc}
        % ¨¨¨¨¨¨¨¨¨¨¨¨¨¨¨¨¨¨¨¨¨¨¨¨¨¨¨¨¨¨¨¨¨¨¨¨¨¨¨
        Let us comment on the energy--momentum tensor under conformal transformations~\cref{eq:GR:diffgeo:conformal_trafo}. Consider the matter action in $d$ spacetime dimensions
        \begin{equation}\label{eq:GR:einstein:matter_action_conformal}
            \widetilde{S}\ped{(m)} = \integ[d]{x\sqrt{-\wt{g}}} \wt{\mathcal{L}}\ped{(m)} = \integ[d]{x\sqrt{-g}} \mathcal{L}\ped{(m)} = S\ped{(m)},
        \end{equation}
        where the Lagrangian $\wt{\mathcal{L}}\ped{(m)}=\Upsilon^{-d}\mathcal{L}\ped{(m)}$~\citep{dabrowskiConformalTransformationsConformal2009}. %
        % Under~\cref{eq:GR:diffgeo:conformal_trafo}, the SE tensor transforms as
        % \begin{equation}
        %     % {[\wt{T}\ped{(m)}]}\indices{^\mu_\nu} =
        %     {\wt{T}\indices{^\mu_\nu} }\rvert\ped{(m)} = \cfac^{-d}{T\indices{^\mu_\nu}}\rvert\ped{(m)},
        % \end{equation}
        % and thus the trace $T\ped{m} \equiv g\^{\mu\nu}{T\_{\mu\nu}}\rvert\ped{(m)}$ transforms as $ \wt{T}\ped{m} = \cfac^{-d} T\ped{m}$.
        % We omit matter superscripts, i.e. ${}\ap{(m)}T \cor T$ and ${}\ap{(m)}T \cor T$ . %
        We let ${}\ap{(m)}T\triangleq T$ and ${}\ap{(m)}\wt{T}\tmpequiv \wt{T}$ denote the matter energy and momentum in the two frames for notational ease. %
        Under~\cref{eq:GR:diffgeo:conformal_trafo}, the \emph{matter} SE tensor transforms as
        \begin{equation}
            % {[\wt{T}\ped{(m)}]}\indices{^\mu_\nu} =
            {\wt{T}\indices{^\mu_\nu} }= \cfac^{-d}{T\indices{^\mu_\nu}},
        \end{equation}
        and thus the trace $T \equiv g\^{\mu\nu}{T\_{\mu\nu}}$ transforms as $ \wt{T} = \cfac^{-d} T$. %
        %

        As a result, we get different conservation laws in the two frames:
        \begin{equation}
            T\indices{^{\mu\nu}_{;\nu}} = 0 \quad \Rightarrow\quad  \wt{T}\indices{^{\mu\nu}_{\tilde{;}\nu}} = -\cfac^{-1}\cfac\^{,\mu} \wt{T},
        \end{equation}
        or vice versa:
        \begin{equation}
            \wt{T}\indices{^{\mu\nu}_{\tilde{;}\nu}} = 0 \quad \Rightarrow\quad T\indices{^{\mu\nu}_{;\nu}} =\cfac^{-1}\cfac\^{,\mu} T.
        \end{equation}
        %
        }



\subsection{Linearised gravity}\label{sec:GR:einstein:lingrav}
{

% ¨¨¨¨¨¨¨¨¨¨¨¨¨¨¨¨¨¨¨¨¨¨¨¨¨¨¨¨¨¨¨¨¨¨¨¨
% LOCAL MACROS:
\newcommand{\pert}{\ALIASpert}           % perturbed quantities
% \newcommand*{\ah}{\ALIASah}                     % a h
% \newcommand{\Ft}{\ALIASFt}
% \newcommand{\dummy}{\ALIASdummy}
% ¨¨¨¨¨¨¨¨¨¨¨¨¨¨¨¨¨¨¨¨¨¨¨¨¨¨¨¨¨¨¨¨¨¨¨¨




Applying perturbation theory to the metric gives rise to a new set of equations, often referred to as the ``linearised Einstein field equations.'' The general starting point is to expand the metric to the order $o$ in question,
\begin{equation}
    \pert{g}\_{\mu\nu} = g\_{\mu\nu}+ \sum_{i=1}^{o} (\delta^{i}g)\_{\mu\nu}.
\end{equation}
% $g\_{\mu\nu}+ \sum_{i=0}^{\text{order}} \delta^{(i)}g\_{\mu\nu}$
Consequently, the perturbed Einstein reads
\begin{equation}\label{eq:GR:einstein:linearised_Einstein_eq}
    \pert{\mathcal{G}}\_{\mu\nu} = 8\ppi G\nped{N} \pert{T}\_{\mu\nu},
\end{equation}
where $\pert{\mathcal{G}}\_{\mu\nu}$ and $\pert{T}\_{\mu\nu}$ are the $o$th order perturbed Einstein and SE tensor, respectively. \Cref{eq:GR:einstein:linearised_Einstein_eq} is then solved order by order. We refer to~\citet{jokelaGravitationalWaveMemory2022,maggioreGravitationalWavesVol2007,carrollSpacetimeGeometryIntroduction2019} for calculations. 

\deleteme{For first-order perturbations to the metric, we get~\citep{jokelaGravitationalWaveMemory2022}
\begin{subequations}
    \begin{align}
        {}^{(0)}\mathcal{G}\indices{^\mu_\nu} &= 8\piGN {}^{(0)}T\indices{^\mu_\nu} \label{eq:GR:einstein:linearised_Einstein_eq_zeroth},\\
        {}^{(1)}\mathcal{G}\indices{^\mu_\nu} &= 8\piGN {}^{(1)}T\indices{^\mu_\nu} \label{eq:GR:einstein:linearised_Einstein_eq_first},
    \end{align}
\end{subequations}
where superscript ${}^{(i)}$ refers to perturbative order $i$. \Cref{eq:GR:einstein:linearised_Einstein_eq_zeroth} is the background equation. %
{\newcommand{\delg}{q}%{{\underaccent{\bar}{\delta g}}}
With $\nabla\^{\mu}\delg\_{\mu\nu}=0$, \cref{eq:GR:einstein:linearised_Einstein_eq_first} now reads~\citep{jokelaGravitationalWaveMemory2022}
\begin{equation}
    \sq \delg\_{\mu\nu} - 2\mathcal{R}\_{\rho\mu\nu\sigma}\delg\^{\rho\sigma} - 2 \mathcal{R}\indices{^\rho_{(\mu}}\delg\_{\nu)\rho}
    - g\_{\mu\nu} \delg\^{\rho\sigma}\mathcal{R}\_{\rho\sigma} + \delg\_{\mu\nu} \mathcal{R} = - 16\piGN {}^{(1)}T\_{\mu\nu}
\end{equation}
where
\begin{equation}
    \delg\_{\mu\nu} \triangleq (\delta^{1} g)\_{\mu\nu} - \frac{1}{2}  g\_{\mu\nu} (\delta^{1} g)\indices{^\rho_\rho} 
\end{equation}
and
\begin{equation}
    {}^{(1)}T\_{\mu\nu} = {}^{(1)}\pclosed{g\_{\mu\rho}T\^{\rho\nu}} = (\delta^1 g)\_{\mu\rho}T\indices{^\rho_\nu}+ g\_{\mu\rho} (\delta^1 T)\indices{^\rho_\nu}.
\end{equation}
}
\speak{Possibly unnecessary\dots}}
% where $\pert{\mathcal{G}}\_{\mu\nu}=\sum_{i=0}^{o} {}^{(i)}\mathcal{G}\_{\mu\nu}$ and $\pert{T}\_{\mu\nu}=\sum_{i=0}^{o} {}^{(i)}T\_{\mu\nu}$, in which ${}^{(i)}Q$ implies the perturbed quantity to order $i$. It is then solved order by order.
% which is solved order by order.
% \begin{equation}
%     \mathcal{G}\_{\mu\nu} + \delta \mathcal{G}\_{\mu\nu} = 8\ppi G\nped{N}  \pclosed{T\_{\mu\nu} + \delta T\_{\mu\nu}},
% \end{equation}
% \comment{Either put details in appendix or refer to e.g.~\citet{jokelaGravitationalWaveMemory2022}}


% Four of the ten degrees of freedom


% Gauge freedom in the context of gravitational waves refers to the invariance under coordinate transformations that do not change the physical content of the theory. By choosing an appropriate gauge, such as the transverse-traceless (TT) gauge, we can eliminate redundancies and focus on the physical aspects of gravitational waves, which are essential for understanding their properties and detecting them.


% Gauge freedom, in a general context, refers to the invariance of a physical theory under certain transformations that do not alter the observable quantities or physical content of the theory.

% Gauge freedom, in a general context, refers to the invariance of a physical theory under certain transformations that do not alter the observable quantities or physical content of the theory. In GR, gauge freedom manifests as invariance under coordinate transformations, which is used to simplify the metric or solve Einstein's equation more easily. By choosing an appropriate gauge we can eliminate redundancies.

% A physical theory that



% \phpar[About gauge freedom, TT gauge etc.---why ``waves'']


}




% \begin{draft}

% A more tangible way to obtain the same equation is to vary the combined matter and Einstein--Hilbert actions \blahblah
% % \boxedeq{eq:GR:einstein:Einstein_eq}{
% %     \mathcal{R}\_{\mu\nu} - \frac{1}{2}\mathcal{R} g\_{\mu\nu} = 8\ppi G\nped{N} T\_{\mu\nu}.
% % }



%     The Einstein--Hilbert action in vacuum is \comment{check Planck mass def.}
%     \begin{equation}
%         S\ped{EH} = \shalf M\nped{Pl}^2 \int \! \diff[4]x \,\sqrt{-\abs{g}}\,  \mathcal{R},
%     \end{equation}
%     where $\mathcal{R} = g\^{\mu\nu}\mathcal{R}\_{\mu\nu}$. By varying $S\ped{EH}$ with respect to $g\_{\mu\nu}$ one obtains the equation of motion
%     \begin{equation}
%         \mathcal{G}\_{\mu\nu} \equiv \mathcal{R}\_{\mu\nu} + \frac{1}{2} g\_{\mu\nu} \mathcal{R} = 0.
%     \end{equation}
%     Thus, we interpret GR as a \emph{classical} field theory where the tensor field $g\_{\mu\nu}$ is the gravitational field, \cringe{with the particle realisation named ``graviton''}.



%     % \begin{bullets}
%     %     \item Einstein's field equations
%     %     \item scalar field (ST theories)
%     %     \item energy momentum tensor
%     % \end{bullets}



%     Einstein's equation for general relativity
%     % \begin{equation}\label[eq]{eq:CFTgrav:GR:Einsteins_eq_GR}
%     %     \mathcal{R}\_{\mu\nu} - \frac{1}{2}\mathcal{R} g\_{\mu\nu} = 8\ppi G\nped{N} T\_{\mu\nu}
%     % \end{equation}
% \boxedeq{eq:GR:einstein:Einstein_eq}{
%     \mathcal{R}\_{\mu\nu} - \frac{1}{2}\mathcal{R} g\_{\mu\nu} = 8\ppi G\nped{N} T\_{\mu\nu}.
% }


% \end{draft}

