
\newcommand\Chr{\ChristoffelSym}




\subsubsection{Constants and units}
% \paragraph{Constants and units.} %
We use natural units where the speed of light in vacuum is $c=1$ and the reduced Planck mass is $M\nped{Pl}=1/\sqrt{8\ppi G\nped{N}}=1$. Here, $G\nped{N}$ is the Newtonian constant of gravitation. Comoving lengths are usually given in units of megaparsec per Hubble constant $\Mpch$, where $1\unit{pc}=3.26156$~light-years.
% We use \checkthis{`natural units'} where $\hbar = c = 1$, where $\hbar$ is the reduced Planck constant and $c$ is the speed of light in vacuum. %\comment{Planck units? Set $k\nped{B}=G\pned{N}=1$?} 
% The Newtonian constant of gravitation $G\nped{N}$ is referenced explicitly, and we use Planck units such as the (reduced) Planck mass $M\nped{Pl}=1/\sqrt{ 8\ppi G\nped{N} }$. 
%$\pclosed{\hbar c /G\nped{N}}^{\shalf} = G\nped{N}^{-\shalf} \sim 10^{-8} \unit{kg}$. 
%

\subsubsection{Tensors}

We use the convention that tensors in the text are identified by their components in the normal coordinate basis, e.g.~$T\iud{\mu}{\nu}$ refers to the tensor $T=T\iud{\mu}{\nu}\partial\_{\mu} \otimes {\diff x}\^{\nu}$. %
For convenience, we will often use equality between a tensor and its corresponding matrix or vector representation, e.g.~$x\^\mu = (t,x, y,z)$.% and keep the

\paragraph{Indices.} %
Greek indices run from $0$ to $N$, and Latin $i,j,k, \dots$ from $1$ to $N$, in an $(N+1)$-dimensional spacetime. Indices $a,b,c$ are reserved for auxiliary coordinate systems. We adopt Einstein's summing convention in which repeated indices are 
summed over, e.g.~$x\^\mu x\_\mu= \sum_{\mu=0}^N x\^{\mu} x\_{\mu}$.

% \paragraph{Tensors.} %
\paragraph{Metric.} %
The metric signature $(-,+,\dots,+)$ is considered. %i.e.~$\det(g\_{\mu\nu})\equiv g < 0 $. 
The Minkowski metric is denoted $\eta\indices{_{\mu\nu}}\equiv\text{diag}(-1, +1,\dots,+1)$, 
% \begin{equation}
%     \eta\_{\mu\nu} \equiv\text{diag}(-1, +1,\dots,+1) = \left(\begin{array}{ccccc}
%         -1 & 0 & \cdots & & 0 \\
%         0 & +1 & 0  & &  \vdots \\
%         \vdots & 0 & \ddots & & 0 \\
%         0 & \cdots & & & +1
%     \end{array}\right),
% \end{equation}
whereas a general metric is denoted $g\indices{_{\mu\nu}}$. 
The metric raises and lowers indices of tensors, e.g.~$x\^\mu = g\^{\mu\nu}x\_\nu$. We let $g$ denote the determinant of the metric, $g\equiv \det~\!(g\_{\mu\nu})$.

% A four-vector $p\indices{^{\mu}} = $

% $[\eta\_{\mu\nu}] = \text{diag}(-1, 1, 1, 1)$




\paragraph{Derivatives.} 
We sometimes adopt the comma-notation for partial derivatives $\partial\_\mu$ and covariant derivatives $\nabla\_\mu$. For a scalar $\phi$, we have:
% $f\_{,\mu}\equiv \partial\_\mu f = \pdv{f}{x\^\mu}$
\begin{subequations}\label{eq:notation:comma_notation_darivatives}
    \begin{align}
        \phi\_{,\mu} &\equiv \partial\_\mu \phi = \pdv{\phi}{x\^\mu} =\pclosed{\partial\_0, \partial\_i }, \\
        \phi\_{;\mu} &\equiv \nabla\_\mu \phi = \partial\_\mu \phi . 
        % {\sq}\phi &\equiv \nabla\^\mu \nabla\_\mu \phi
    \end{align}
\end{subequations}
The covariant derivative of a general rank-$(k,l)$ tensor $T\iud{\rho_1 \dots \rho_k}{\sigma_1 \dots \sigma_l}$ is
\begin{equation}\label{eq:notation:covariant_derivative_rank_kl}
    \begin{split}
        \nabla\_\mu T\iud{\rho_1 \dots \rho_k}{\sigma_1 \dots \sigma_l} = \partial\_\mu T\iud{\rho_1 \dots \rho_k}{\sigma_1 \dots \sigma_l} &+ \Chr{\rho_1}{\mu\nu} T\iud{\nu \rho_2 \dots \rho_k}{\sigma_1 \dots \sigma_l} + \dots \\
        &- \Chr{\nu}{\mu\sigma_1} T\iud{\rho_1 \dots \rho_k}{\nu\sigma_2 \dots \sigma_l} + \dots,
    \end{split}
\end{equation}
where $\Chr{\rho}{\mu\nu}$ are the Christoffel symbols in~\cref{eq:notation:Christoffel_symbols}.
The d'Alembertian reads
\begin{equation}\label{eq:notation:dAlembertian}
    \sq   = \nabla\^\mu \nabla\_\mu  = \frac{1}{\sqrt{-g}} \partial\_\mu \pclosed{ \sqrt{-g} \partial\^\mu},
\end{equation}
and we use $\sq\nped{M} \equiv \partial\^\mu\partial\_\mu$ to specify the Minkowski box operator.


\paragraph{Miscellanea.} %
We make use of the notation
\begin{subequations}\label{eq:notation:sym_and_antisym_tensor_notation}
    \begin{align}
        T\_{(\mu\nu)} &\equiv \frac{1}{2} \pclosed{ T\_{\mu\nu} + T\_{\nu\mu}  }, \\
        T\_{[\mu\nu]} &\equiv \frac{1}{2} \pclosed{ T\_{\mu\nu} - T\_{\nu\mu}  }.
    \end{align} 
\end{subequations}
We write $T\_{\mu\nu}= T\_{(\mu\nu)} + T\_{[\mu\nu]}$. %A tensor with 
\comment{Maybe define diag?}



\paragraph{Special symbols and tensors.} %
The Kronecker delta is defined as
\begin{equation}\label{eq:notation:Kronecker_delta}
    \Krondelta{_{ij}} = \begin{cases}
        1 &\text{if}\, i=j, \\ 
        0 &\text{else,}
    \end{cases}
\end{equation}
and the Levi-Civita symbol is
\begin{equation}\label{eq:notation:Levi_Civita}
    \LeviCivita_{{i_1 i_2 \dots i_n}} = \begin{cases}
        +1 &\text{if $(i_1 i_2 \dots i_n)$ is an even permutation of $(12\dots n)$} \\ 
        -1 &\text{if $(i_1 i_2 \dots i_n)$ is an odd permutation of $(12\dots n)$} \\ 
        0 &\text{else,}
    \end{cases}
\end{equation}
in $n$ dimensions. %
The Christoffel symbols (also called connection coefficients) are given by
\begin{equation}\label{eq:notation:Christoffel_symbols}
    \ChristoffelSym{\rho}{\mu\nu} = \frac{1}{2} g\^{\rho\sigma} \pclosed{ g\_{\mu\sigma,\nu} + g\_{\mu\sigma,\nu} - g\_{\mu\nu,\sigma}  },
\end{equation}
where $g\_{\mu\nu}$ is the spacetime metric. %
% Note that this is not true tensor.
% \subparagraph[Gamma]{Christoffel symbols.} %
% The Christoffel symbols or ``connections'' are written
% \begin{equation}\label{eq:notation:Christoffel_symbols}
%     \ChristoffelSym{\rho}{\mu\nu} = \frac{1}{2} g\^{\rho\sigma} \pclosed{ g\_{\mu\sigma,\nu} + g\_{\mu\sigma,\nu} - g\_{\mu\nu,\sigma}  },
% \end{equation}
% and are not true tensors.
The Riemann curvature tensor is
% \subparagraph{Riemann curvature tensor.} %
\begin{equation}\label{eq:notation:Riemann_tensor}
    \mathcal{R}\indices{^\rho_{\sigma \mu \nu}} = 
    % 2\partial\_{[\mu} \Chr{{|\rho|}}{\nu]\sigma} + 2\Chr{\rho}{[\mu |\lambda }\Chr{\lambda|}{\nu]\sigma } =  
    \partial\_{\mu} \Chr{\rho}{\nu\sigma}- \partial\_{\nu} \Chr{\rho}{\mu\sigma} + \Chr{\rho}{\mu \lambda }\Chr{\lambda}{\nu \sigma} -  \Chr{\rho}{\nu \lambda }\Chr{\lambda}{\mu\sigma }.
\end{equation}

% \subparagraph[Lambda]{``Lambda tensor.''} %
We express the \textit{Lambda tensor}---sometimes called the ``projection operator'' or ``spin-2 operator''---that projects a symmetric tensor onto the TT gauge as
\begin{equation}\label{eq:notation:projection_tensor}
% \begin{split}
    \ProjectionLambda{ij}{kl}(\vec{k}) = P\indices*{_i^k}(\vec{k})P\indices*{_j^l}(\vec{k}) - \frac{1}{2}P\_{ij}(\vec{k})P\^{kl}(\vec{k});
    \quad P\_{ij}(\vec{k}) =  \Krondelta{_{ij}}-k\_i k\_j/ k^2, %\land \vec{n}^2 = n\_1\!^2 + n\_2\!^2+ n\_3\!^2 = 1
% \end{split}
\end{equation}
where $\vec{k}$ is the propagation direction (see~\cref{sec:GR:gws}).
% $\forall \vec{n}^2 = n\_1\!^2 + n\_2\!^2+ n\_3\!^2 = 1$
%$\forall\, \vec{n}$ of unit length; $\vec{n}^2 = n\_1\!^2 + n\_2\!^2+ n\_3\!^2 = 1$. %Note that the notation is somewhat different from the convention 
% Other literature may use a comma (`,') instead of a 
% We use a dot (`$.$') instead of the more conventional comma (`$,$') to distinguish from the Minkowskian partial derivative.
% $[\tens{\eta}] = \text{diag}(-1, 1, 1, 1)$

\subsubsection{Fourier transforms} %
% We use the following convention for the Fourier transform of $f(x)$, $\tilde{f}(k)$, and its inverse, where $x$ and $k$ are Lorentz four-vectors:
% \begin{equation}
%     \begin{split}
%         % f(x) &= \int \del[2 3]{k}  \eu[-\im k\cdot x] \tilde{f}(k) \\
%         % f(x) &= \int \del[2]{k}  \eu[-\im k\cdot x] \tilde{f}(k) \\
%         % f(x) &= \int \del{k}  \eu[-\im k\cdot x] \tilde{f}(k) \\
%         f(x) &=  \integ[4][(2\ppi)^4]{k} \eu[-\im k\cdot x]\tilde{f}(k),  \\
%         % f(x) &= \integ{\frac{\diff[4]k}{(2\ppi)^4}} \eu[-\im k\cdot x] \tilde{f}(k) \\
%         % f(x) &= \int \! \frac{d^4\! k}{(2\ppi)^4} \, \eu[-\im k\cdot x] \tilde{f}(k) \\
%         \tilde{f}(k) &= \integ[4]{x} \eu[\im k \cdot x] f(x) .
%     \end{split}
% \end{equation}
% Here, $k\cdot x = k\_{\sigma} x\^{\sigma} = g\_{\rho\sigma} k\^{\rho} x\^{\sigma}$. 
We use the following convention for the continuous Fourier transform of $f(\vec{x})$, $\tilde{f}(\vec{k})$, and its inverse, where $\vec{x}$ and $\vec{k}$ are three-vectors:
\begin{equation}\label{eq:notation:Fourier_transform_and_inverse}
    \begin{split}
        \tilde{f}(\vec{k}) &={\mathop{\mathcal{F}}_{\vec{k}}} {\cclosed{f(\vec{x})}}  =\integ[3]{x}\eu[\im \vec{k}\cdot \vec{x}]f(\vec{x}), \\
        f(\vec{x}) &=  {\mathop{\mathcal{F}}_{\vec{x}}}^{-1}  {\cclosed{\tilde{f}(\vec{k})}}  =\integ[3][(2\ppi)^3]{k} \eu[-\im \vec{k}\cdot \vec{x}]\tilde{f}(\vec{k}).
    \end{split}
\end{equation}
% Here, $k\cdot x = k\_{\sigma} x\^{\sigma} = g\_{\rho\sigma} k\^{\rho} x\^{\sigma}$. 
Note that this gives $\partial\_i \leftrightarrow +\im k\_i$.





\subsubsection{Special functions} %
\comment{Will fix this.} %
We will denote various cylinder functions by \blahblah see~\cref{app:cylinder}

\begin{tabular*}{\linewidth}{r c l}
    regular Bessel functions: & $\Cylindrical[]$; & $\Cylindrical[\nu]^{(i)}(x)$  \\
    spherical Bessel functions: & $\Cylindrical[][z]$; &  $\Cylindrical[n][z]^{(i)}(x) = \sqrt{\ppi/(2x)} \,\Cylindrical[n+1/2]^{(i)}(x) $ \\
    Riccati--Bessel functions: & $\Cylindrical[][R]$; &  $\Cylindrical[n][R]^{(i)}(x) =-(-1)^{i}\cdot x\,\Cylindrical[n][z]^{(i)}(x) $ 
\end{tabular*}
Here, $\nu\in \Complex$ and $n\in\Integer$ denotes order, while $i=1,2$ refers to the \emph{$i$th kind}. \speak{Too messy!}

% \begin{align}
%     \zn^{(i)}(x) &= \sqrt{\frac{\ppi}{2x}} \Zv[n+1/2]^{(i)}(x) \\
%     \Rn^{(i)}(x) &= (-1)^{i-1} x\zn^{(i)}(x)
% \end{align}
% \begin{equation}
%     \begin{split}
%         \lambda \quad \widebar{\lambda} \hbar \quad \rlap{X}-- \\
%         \text{X\hspace{-0.8em}\raisebox{0.75pt}{\,--}} \\
%         \lambdabar \equiv \lambda /(2\ppi) ; (\lambda = 1/f) \iff k = 2\ppi/\lambda = 1/\lambdabar
%     \end{split}
% \end{equation}

% \section*{Frequently used abbreviations}
\section*{Acronyms}
\subimport{./}{abbrv.tex}
% \comment{Glossary?}k

\section*{Nomenclature}
\subimport{./}{symbols.tex}
