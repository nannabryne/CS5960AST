% ||||||||||||||||||||||||||||||||||||||||
% |||||| 7.X Symmetron domain walls ||||||
% ||||||||||||||||||||||||||||||||||||||||

% ------------------------------------------------
% labels: \label[type]{[type]:871:symm_dws:[name]}
% ------------------------------------------------


% ----------------
\newcommand{\eqregimenum}[1]{{\footnotesize{\textsf{\textbf{({#1})}}}}}
\newcommand{\brchi}{\breve{\chi}}
% ----------------

From~\cref{sec:CFTgrav:symmetron} we have the (a)symmetron effective potential and \dots


Using the symmetron model to represent a domain wall in an FRLW universe, $\sq \phi=V\ped{eff,\phi}$ becomes
\begin{equation}
    -a^{-2} \bclosed{ \ddot{\phi} + 2\mathcal{H} \dot{\phi} - \vec{\nabla}^2\phi } = \lambda \phi^3 +\mu^2 \pclosed{ \upsilon-1}\phi.
\end{equation}
From here, we will use $\chi=\phi/\phi_\infty$ and $\chi_\pm=\sqrt{1-\upsilon}$. We recall that $\upsilon=\rho\ped{m}/\rho_{\mathrm{m}\ast} = (a_\ast/a)^3$. 

Prior to SSB, the scalar field is trivial, and so we move on to consider $\chi$ from this critical point where the quartic term turns negative and the $\mathsf{Z_2}$ symmetry is spontaneously broken. 

\subsection{Quasistatic limit}
    We can solve
    \begin{equation}
        \vec{\nabla}^2 \chi \simeq  + \pclosed{ \mu^3/\sqrt{\lambda} } \cdot a^2\bclosed{ \chi^2 - (1-\upsilon) }\chi
    \end{equation}
    to obtain the solution in the limit where spatial gradient plays a much larger role that time derivatives. We let $\chi=\chi(a, z)$ and use the well-established~\citep[see e.g.][]{llinaresDomainWallsCoupled2014} expression for a domain wall when $a\propto \tau^\alpha$
    \begin{equation}
        \chi(a, z) = \sqrt{1-\upsilon} \tanh{\pclosed{ \frac{a(z-z\ped{dw})}{2L\ped{C}} \sqrt{1-\upsilon}}}.
    \end{equation}
    $L\ped{C}={(\sqrt{2}\mu)}^{-1}$ is known as the Compton wavelength of the symmetron, a measurement that will become significant later when \blahblah.
    We write $a=a_\ast (\tau/\tau_\ast)^\alpha$, which gives $\upsilon=(a_\ast/a)^3=(\tau_\ast/\tau)^{3\alpha}$.
    \begin{figure}[h]\label[fig]{fig:871:symm_dws:quasi_chi}
        \centering
        \includegraphics[width=\linewidth]{dummy_normal.png}
        %%%%%%%%%%
        \caption{Quasistatic evolution of the domain wall represented by $\chi$.}
        %%%%%%%%%%%%%
    \end{figure}


    \phpar[sårbarheter etc., thickness]
    % We take the thickness of the wall a
    % The thickness of the wall, $\delta\ped{dw}$, can be taken as the 
    % \begin{equation}
    %     \tanh{\pclosed{ \frac{z-z\ped{dw}}{\sqrt{2}\delta\ped{dw}} }}
    % \end{equation}
    % i.e.
    % \begin{equation}
    %     \delta\ped{dw} = \frac{2L\ped{C}}{a\sqrt{1-\upsilon}}
    % \end{equation}

    \paragraph{Basic properties.} %
    % We take the wall thickness parameter $\delta\ped{dw}$ to be the $z$-coordinate at which the argument in the $\tanh$-function is $1/\sqrt{2}$, i.e.
    % % \begin{equation}
    % %     \sqrt{2}\delta\ped{dw} = \frac{2L\ped{C}}{a\sqrt{1-\upsilon}}.
    % % \end{equation}
    % \begin{equation}
    %     a\delta\ped{dw} = \delta_\infty \pclosed{1-\upsilon}^{-1/2}.
    % \end{equation}
    % The surface tension is in this case given by
    % \begin{equation}
    %     \sigma\ped{dw} = \sigma_\infty \pclosed{1-\upsilon}^{3/2}.
    % \end{equation}
    %
    \citet{vilenkinCosmicStringsOther1994} define the wall thickness as the $(z-z\ped{dw})$-coordinate at which the argument in the $\tanh$-function is $1/\sqrt{2}$, which is $\delta_\infty  \equiv \mu^{-1}= \sqrt{2}L\ped{C}$. The surface tension of such a conventional wall \nc{is estimated $\sigma_\infty \equiv \frac{2\sqrt{2}}{3} \mu^3/\lambda$}[maybe background section?]. Extrapolated to expanding spacetime, we get
    \begin{equation}
        a\delta\ped{dw} = \delta_\infty \pclosed{1-\upsilon}^{-1/2} \quad \text{and} \quad \sigma\ped{dw} = \sigma_\infty \pclosed{1-\upsilon}^{3/2}
    \end{equation}
    as expressions for the comoving thickness $\delta\ped{dw}$ and the surface energy density $\sigma\ped{dw}$. 
    % The surface tension is in this case given by
    % \begin{equation}
    %     \sigma\ped{dw} = \sigma_\infty \pclosed{1-\upsilon}^{3/2}.
    % \end{equation}
    % Conventional walls, walls in Minkowski space, have
    The system of $N$ kinks and $M$ antikinks is represented by the field $\phi = \phi_\infty \chi$ with
    \begin{equation}
        \chi = \prod_{i}^{N} \chi\ped{k}(z-k_i) \prod_{j}^{M} \widebar{\chi}\ped{k}(z-\widebar{k}_j),
    \end{equation}
    where
    \begin{equation}
        \chi\ped{k}(z-k_i) = \sqrt{1-\upsilon} \tanh{\pclosed{ \frac{a(z-k_i)}{\sqrt{2}\delta_{\mathrm{k},i}} }}
    \end{equation}
    and $\widebar{\chi}\ped{k} = - \chi\ped{k}$. 




\subsection{Asymptotic limit}
    We let $\brchi$ denote the field value far away from the wall, well inside the domain. Here,
    \begin{equation}\label[eq]{eq:871:symm_dws:eom_asym_chi_s}
        \ddot{\brchi} +  2\mathcal{H} \dot{\brchi} = - \pclosed{ \mu^3/\sqrt{\lambda} } \cdot a^2\bclosed{ \brchi^2 + \upsilon - 1 }\brchi
    \end{equation}
    controls the evolution of the field strength. The trivial solution becomes unstable after the phase transition, and the field may fall into any of the two vacua. We take a look at one of the minima. The positive minimum, which was zero at PT, goes as $\chi_+ = \sqrt{1-\upsilon}$. Now, the rate at which $\chi_+$ moves from its initial value, blows up at the phase transition, but decays rapidly when approaching the limit value. We solve this issue by parting the equation in two regimes; \eqregimenum{I} a quick, non-adiabatic regime in which $V\ped{eff}$ changes faster than what $\chi$ can possibly follow and \eqregimenum{II} an adiabatic regime where $V\ped{eff}$ changes much slower than $\chi$. What happens is that the field rolls towards the minimum and begins to oscillate around it whilst following its slow drift. \checkthis{Plagiarism? (Julian's notes)} The oscillation amplitude is decided by the initial conditions of the field.

    \phpar[Why do we not want oscillations?]

    We can rewrite~\cref{eq:871:symm_dws:eom_asym_chi_s} in terms of the time coordinate $\chi_+ =\sqrt{1-\upsilon}$,
    \begin{equation}\label[eq]{eq:871:symm_dws:eom_asym_chi_chiplus}
        \dv[2]{\brchi}{\chi_+} - \frac{1}{\chi_+ \pclosed{1-\chi_+^2}} \dv{\brchi}{\chi_+} + m^2 \frac{\chi_+^2 \pclosed{\brchi^2-\chi_+^2}}{\pclosed{1-\chi_+^2}^3}\brchi = 0,
    \end{equation}
    where
    \begin{equation}\label[eq]{eq:871:symm_dws:eom_asym_chi_chiplus_m_val}
        m= \frac{2\mu}{3\mathcal{H}_\ast \pclosed{1+\mathfrak{z}_\ast}} = \frac{\sqrt{2}a_\ast^{3/2}}{3\xi_\ast}.
    \end{equation}
    The idea is to use this solution as boundary conditions for $\chi$:
    % \begin{equation}
    %     \chi(\tau, z\to \pm \infty) =\pm \brchi(\tau) \quad \land \quad \dot{\chi}(\tau, z\to \pm \infty) =\pm \dot{\brchi}(\tau)
    % \end{equation}
    \begin{equation}
        \chi|_{z\to \pm\infty} =\pm \brchi \quad \land \quad \dot{\chi}|_{z\to \pm\infty} =\pm \dot{\brchi}
    \end{equation}
    % We solve in two regimes, each solution expanded around \eqregimenum{I} $\chi_+=0$ and \eqregimenum{II} $\chi_+=1$. We get 
    % \begin{subequations}
    %     \begin{equation}
    %         \brchi \simeq\begin{cases}
    %             \brchi^{\text{\eqregimenum{I}}}  & \chi_+ \leq \chi_+^{\mathrm{match}} \\
    %             \brchi^{\text{\eqregimenum{II}}}  & \chi_+ \geq \chi_+^{\mathrm{match}} \\
    %         \end{cases}
    %     \end{equation}
    %     where
    %     \begin{align}
    %         %\brchi &\stackrel{\chi_+\sim 0}{\simeq} 
    %         \brchi^{\text{\eqregimenum{I}}}  &= \chi_\ast + \frac{\mathcal{C}}{2}\chi_+^2 + \frac{\mathcal{C}-\chi_\ast^3 m^2}{8} \chi_+^4 \\ %& \equiv \brchi^b
    %         %\brchi &\stackrel{\chi_+\sim 1}{\simeq} 
    %         \brchi^{\text{\eqregimenum{II}}} &= \chi_+ + \frac{8(3-m^2)}{m^4} \pclosed{\chi_+ -1}^3 + \frac{1440 - 636m^2 + 41m^4}{2m^6} \pclosed{\chi_+ -1}^4 %\\%& \equiv \brchi^a\\%+ \BigO{ \pclosed{\chi_+ -1}^5} \\
    %     \end{align}
    % \end{subequations}
    We solve in two regimes, each solution expanded around \eqregimenum{I} $\chi_+=0$ and \eqregimenum{II} $\chi_+=1$:
    \begin{subequations}
        \begin{align}
            %\brchi &\stackrel{\chi_+\sim 0}{\simeq} 
            \brchi^{\text{\eqregimenum{I}}}  &= \chi_\ast + \frac{\mathcal{C}}{2}\chi_+^2 + \frac{\mathcal{C}-\chi_\ast^3 m^2}{8} \chi_+^4 \\ %& \equiv \brchi^b
            %\brchi &\stackrel{\chi_+\sim 1}{\simeq} 
            \brchi^{\text{\eqregimenum{II}}} &= \chi_+ + \frac{8(3-m^2)}{m^4} \pclosed{\chi_+ -1}^3 + \frac{1440 - 636m^2 + 41m^4}{2m^6} \pclosed{\chi_+ -1}^4 %\\%& \equiv \brchi^a\\%+ \BigO{ \pclosed{\chi_+ -1}^5} \\
        \end{align}
    \end{subequations}



    \subsection{Examples}
        We employ the 

        We solve~\ref{eq:871:symm_dws:eom_asym_chi_s} numerically for various sets of initial conditions to demonstrate \blahblah
        \begin{figure}[h]\label[fig]{fig:871:symm_dws:asym_chi}
            \centering
            \includegraphics[width=\linewidth]{dummy_normal.png}
            %%%%%%%%%%
            \caption{Evolution of the minima $\brchi$ .}
            %%%%%%%%%%%%%
        \end{figure}






    
    




\subsection{Compton wavelength}



Much like GR, this thesis is highly non-linear. It is near impossible to preserve a causal structure, to write it ``chronologically,'' if one also aims to divide it into subjects for readability. 