% |||||||||||||||||||||||||||||||||||
% |||||| 7.X Dynamic modelling ||||||
% |||||||||||||||||||||||||||||||||||

% ------------------------------------------------
% labels: \label[type]{[type]:871:dynamics:[name]}
% ------------------------------------------------


When translating a theory consisting of nice, continous functions with theoretical limits to a discrete, numerical system, there are \blahblah. When said theory also includes a phase transition, it does not become any easier. Phase transitions are manifestly computational headaches, and we are limited by \blahblah. The discontinuity introdused by the phase transition is even more complicated to replicate in simulations. 

\phpar[The code (gevolution etc.) (maybe own section?)]

\phpar[BCs]


We will have a look some of the most pressing matters when aiming to test the theory from~\cref{chap:pertwalls} in simulations. 



\subsection{Spatial and temporal resolution}



\subsection{Memory}




