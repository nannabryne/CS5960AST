% |||||||||||||||||||||||||||||||
% |||||| X.X gwasevolution ||||||
% |||||||||||||||||||||||||||||||

% --------------------------------------------
% labels: \label[type]{[type]:871:gwas:[name]}
% --------------------------------------------


% \newcommand{\lcoord}[1]{\mathtt{#1}} % code lattice coordinate
\newcommand{\lcoord}[1]{\mathtt{#1}} % code lattice coordinate


% \subsection{\tmptitle{Equations, solvers}}



The code \texttt{gwasevolution}\footnote{Written by \citeauthor{christiansenAsevolutionRelativisticNbody2023}.} is an extended version of the massively parallelised $N$-body relativistic code \texttt{gevolution}~\citep{adamekGevolutionCosmologicalNbody2016}. 


The general setup is a three-dimensional box of equal side lengths $L$ on a lattice of $N^3$ points, giving a spatial resolution of $\Delta x=L/N$. %We refer to the points in the configuration space as 

The mapping from code configuration space to comoving coordinates is:
\begin{equation}
    \begin{split}
        \vec{x} & = \lcoord{x}_{\lcoord{i}, \lcoord{j}, \lcoord{k}} \cdot \Delta x \\
        (x, y, z)&= (\lcoord{i}, \lcoord{j}, \lcoord{k})\cdot \Delta x
    \end{split}
\end{equation}
Meanwhile, the corresponding mapping in Fourier space is given in terms of the fundamental frequency $k\ped{f}=2\ppi/L$:
\begin{equation}
    \begin{split}
        \vec{k} & = \lcoord{k}_{\lcoord{u}, \lcoord{v}, \lcoord{w}} \cdot k\ped{f}\\
        (k_x, k_y, k_z)&=   (\lcoord{u}, \lcoord{v}, \lcoord{w})\cdot k\ped{f}
    \end{split}
\end{equation}
\comment{comment about hermitian symmetry?}



\subsection{Initial configuration}
    For the purpose of exploring the validity of our equations, the simulation setup needs be controlled \comment{no stochasticity}. This is why we cannot initialise a cosmological scenario where a phase transition is \cringe{in the cards (on the horizon)}; we would never achieve a domain wall lying perfectly at $z=z_0+\epsilon_\ast \sin{py}$. 

    Ignore the perturbation for now. The 3D simulation box of side lengths $L$ has in total $N^3$ lattice points.\footnote{Two dimensions would suffice for this problem, but \texttt{gevolution} only takes cubic boxes.} %Separating the box in two domains calls for two domain walls.
    To preserve the periodic boundary conditions, we need at least two walls. We place one topological defect at $z=L/2$ and its counterpart (the antikink) at $z=0$. The simplest way to achieve this is by initialising the scalar field $\chi$ ($\texttt{achi}$ in the code) with
    \begin{equation}\label[eq]{eq:871:gwas:achi_IC_type0}
        \chi_+ \tanh{ \pclosed{ \frac{a(z-z\ped{dw})}{2L\ped{C}} \chi_+ } }.
    \end{equation}


    % We choose to have one topologicalm
    % To create a stable, thin, planar wall configuration, we 


    % We start with a stable, thin, planar wall configuration (no perturbation).



\subsection{Discrete Fourier space}
    \blahblah

    $h\_{ij}(\eta, \vec{k}) = \Lambda\_{ijkl}(\vec{k}) h\_{kl}(\eta,\vec{k})$ where $\vec{k}$ is \emph{not} defined $\vec{k} = 2\ppi (\lcoord{u},\lcoord{v},\lcoord{w})/ L $

\subsection{Options/parameters}
    % Now to the spe
    We now turn to the details of the simulations we are to perform. For starters, we provide options and compiler flags to make certain that fields $\chi$ and $h\_{ij}$ are computed. 

    
    \paragraph{Numerics.} %
    There are choices to be made concerning temporal and spatial resolution, and what integration method to use. Of solvers, there are a handful to choose from, for instance Leap-Frog or fourth order Runge-Kutta. The spatial resolution $\Delta x= L/N$ is set by the users choice of box side length $L$ in $\text{Mpc}/h_0$ and the number of grid points $N$ in each direction. The temporal resolution\emph{s} vary for some field updates. They are controlled by the \emph{Courant factors}
    \begin{equation}
        C\ped{f} = v\ped{g} \Delta \tau / \Delta x 
    \end{equation}
    where \blahblah
    % For the scalar field, we have $v^{\chi}=$

    \comment{$L$ and $\redshift\ped{init}$ difficult to place in category.}

    
    \paragraph{Physics.} %
    More exciting are the parameters that directly relates to our theoretical setup and cosmological scenarios. We can set the simulation box size, $L$, 
        \subparagraph{Asymmetron parameters.} %
        The code takes the phenomenological parameters $\mathrm{As}_\ast = \{a_\ast, \xi_\ast, \beta_+, \beta_- \}$ and maps them to Lagrangian parameters $\mathrm{As}_\mathcal{L} = \{ M, \mu, \lambda, \kappa \}$ as described in~\nc{some section or appendix}.


        \subparagraph{Perturbation parameters.} % 
        We added some options to the initialisation-part of the code so to easily create the idealised scenario we want. It has options for this kind of perturbation in the $z$-direction:
        \begin{equation}
            \epsilon = \epsilon_p(\tau) \mathop{\mathrm{tri}}{\cclosed{ p_x x + p_y y }}; \quad \mathop{\mathrm{tri}} \in \{ \sin, \cos \},
        \end{equation}
        for which the user can provide initial amplitude $\epsilon_\ast$ and perturbation scale in terms of scaled, integer wavenumbers $m_{\lcoord{i},\lcoord{j}} = p_{x,y} /k\ped{f}$. 
        
    We need to be careful \blahblah







\comment{Surface tension ``tension'': Sensitivity to this.}

