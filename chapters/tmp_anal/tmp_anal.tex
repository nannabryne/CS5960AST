% !TEX root = ../../thesis.tex

% -----------------------------------
% labels: \label{[type]:[this chap key]:[name]}
% -----------------------------------

\tmptitle{Possibly different Ch. name: ``GWs from perturbed flat DW'' or something }
\tmptitle{``Dynamics of Domain Walls''??}

\begin{bullets}
    \item BenteBent (thin) $\to$ KatjaKaj (thick)
\end{bullets}

We consider a relatively general four-dimensional spacetime $(\mathscr{M}, g\_{\mu\nu})$ with continuous metric and associated coordinate system $\{x\^\mu \}$. \blahblah \citep{ishibashiEquationMotionDomain1999} 


\section{Dynamics of Domain Walls in the Thin-Wall Limit}
{

Let the $(2+1)$-dimensional submanifold $\varSigma$ embedded in $\mathscr{M}$ represent the thin domain wall we henceforth shall refer to as ``BenteBent'' (or BB). This hypersurface divides the manifold into two separate regions ($\mathscr{M}_{\pm}$), allowing us to write $\mathscr{M} = \mathscr{M}_+ \cup  \varSigma  \cup \mathscr{M}_-$. Allow indices $a,b,c,\dots$ run over $0,1,2$, and assign the coordinate system $\{y\^a\}$ to $\varSigma$. Now, the world-volume metric
\begin{equation}
    \gamma\_{ab} = g\_{\mu\nu} \pdv{x\^\mu}{y\^a}  \pdv{x\^\nu}{y\^b}
\end{equation}
is the induced metric on $\varSigma$. \comment{Hmmm, maybe something wrong here.} \citep{carrollSpacetimeGeometryIntroduction2019}



We write the covariant action of the domain wall BenteBent as
\begin{equation}
    S\ped{BB}  = -\sigma \integ[3]{y\sqrt{-\gamma}}[\varSigma],
\end{equation}
where 





}