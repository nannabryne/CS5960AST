% !TEX root = ../../thesis.tex

% -----------------------------------
% labels: \label{[type]:[this chap key]:[name]}
% -----------------------------------



\newpage


\begin{draft}%
    Consider a planar domain wall in the $xy$-plane in a flat FRW universe, represented by a scalar field $\phi(\eta, \vec{x})$ and a potential $V(\phi)$. The background metric is
    \begin{equation}
        d\bar{s}^2 = \bar{g}\indices{_{\mu\nu}} \diff \bar{x}\indices{^\mu} \bar{x}\indices{^\nu} = -\diff t^2 + a^2(t) \Krondelta{_{ij}} \diff x^i \diff x^j = a^2 (\eta) \cclosed{-\diff\eta^2  + \diff x^2 +\diff  y^2 + \diff z^2}.
    \end{equation}
    The solution to $\bar{\sq} \phi=\dv*{V}{\phi}$ is denoted $\bar{\phi}(\eta, z)$. We let indices $a,b,c=1,2$ and $i,j,k,l,\ldots =1,2,3$. Now we add a linear perturbation $\zeta(\eta, x^a)$ to the wall such that
    \begin{equation}
        \phi(\eta, \vec{x}) = \bar{\phi}(\eta, z; \zeta(\eta, x^a)) = \bar{\phi}(\eta, z; 0) + \zeta(\eta, x^a) \pdv{\bar{\phi}}{z}\bigg|_{\zeta=0} + \BigO{\zeta^2}.
    \end{equation}
    Furthermore, Fourier transforming \provethis{show this!} the spatial components gives
    \begin{equation}
        \phi(\eta,\vec{k}) = \int\! \diff[3]x \, \eu[\im k_i x^i] \phi(\eta,\vec{x})= \bclosed{ (2\ppi)^2 \Diracdelta[2]{k\loind{a}}- \im k\loind{3} \zeta(\eta, k\loind{a}) } \bar{\phi}(\eta, k\loind{3}; 0) + \order{\zeta^2 }.
    \end{equation}

    % \phpar[expanation of/reference to how to obtain the following:]
    The TT-part of the energy-momentum tensor is \provethis{refer to some section}
    \begin{equation}
        \tensor*{T}{^{\mathrm{TT}}_{ij}}(\eta, \vec{k}) = \tensor{\Lambda}{_{ij,kl}}(\hat{\vec{k}}) \int \frac{\diff[3] p}{(2\ppi)^3} p_k p_l \phi(\eta, \vec{p}) \phi(\eta, \vec{k}-\vec{p}).
    \end{equation}
    We define a quantity $\tensor{t}{_{kl}}$ by
    \begin{equation}
        \tensor*{T}{^{\mathrm{TT}}_{ij}}(\eta, \vec{k}) =  \tensor{\Lambda}{_{ij,kl}}(\hat{\vec{k}}) \pclosed{  \frac{1}{2\ppi} \cdot\tensor{t}{_{kl}}(\eta, \vec{k})  + \BigO{\zeta^2} },
    \end{equation}
    and the additional function 
    \begin{equation}
        \mathfrak{I}_n(\eta, q_0) = \int_{\mathbb{R}} \!\diff q \, q^n \bar{\phi}(\eta,q;0) \bar{\phi}(\eta, q_0-q;0).
    \end{equation}
    After some manipulation \provethis{show this!}, we get the following:
    \begin{subequations}
        \begin{align}
            \tensor{t}{_{ab}} (\eta, \vec{k}) &= k\loind{a} k\loind{b} \bclosed{-\im \zeta(\eta, k\loind{c})} \mathfrak{I}_1(\eta, k\loind{3}) \\
            \tensor{t}{_{a3}} (\eta, \vec{k}) &= k\loind{a} \bclosed{-\im \zeta(\eta, k\loind{c})} \mathfrak{I}_2(\eta, k\loind{3}) \\
            \tensor{t}{_{33}} (\eta, \vec{k}) &= k\loind{3} \bclosed{-\im \zeta(\eta, k\loind{c})} \mathfrak{I}_2(\eta, k\loind{3}) + (2\ppi)^2\Diracdelta[2]{k\loind{a}} \mathfrak{I}_2(\eta,k\loind{3})
        \end{align}
    \end{subequations}
    \comment{There are some \emph{small} constraint on the perturbation from this. Need to be commented!}

    Gravitational waves sourced by this field is -- to first order in $\zeta$ -- given by
    \begin{equation}
    \begin{split}
        a\tensor{h}{_{ij}}(\eta, \vec{k}) &= \frac{16\ppi G\ped{N}}{k} \int_{\eta\ped{i}}^{\eta} \! \diff \eta'\, \sin{\pclosed{k\bclosed{\eta-\eta'}}} a(\eta')  \tensor*{T}{^{\mathrm{TT}}_{ij}}(\eta', \vec{k}) \\
        &= \frac{8 G\ped{N}}{k} \tensor{\Lambda}{_{ij,kl}}(\hat{\vec{k}}) \int_{\eta\ped{i}}^{\eta} \! \diff \eta'\, \sin{\pclosed{k\bclosed{\eta-\eta'}}} a(\eta') t\loind{kl}(\eta', \vec{k}) + \BigO{\zeta^2}.
    \end{split}
    \end{equation}
    Remaining is the $\Lambda\loind{ij,kl}t\loind{kl}$-elements, which in total are \checkthis{6} terms per $ij$, due to symmetry in $t\loind{kl}$: %\comment{Gauge freedom implies 4, does it not? Maybe only in $h_{ij}$.} 
    \begin{equation}
    \begin{split}
        \Lambda\loind{ij,kl} (\hat{\vec{k}}) t\loind{kl} (\eta,\vec{k}) &=2\cclosed{%
            \Lambda\loind{ij,12} t\loind{12} + \Lambda\loind{ij,13} t\loind{13} + \Lambda\loind{ij,23} t\loind{23}} (\eta, k\hat{\vec{k}}) \\
            &\phantom{=}+ \cclosed{
                \Lambda\loind{ij,11} t\loind{11} + \Lambda\loind{ij,22} t\loind{22} + \Lambda\loind{ij,33} t\loind{33}
            }(\eta, k\hat{\vec{k}})
    \end{split}
    \end{equation}

    

\end{draft}



\newpage

\section{General Formalism}


% \begin{equation}
%     a\tensor{h}{_{ij}}(\eta, \vec{k}) = \frac{16\ppi G\ped{N}}{k} \int \dots \tensor*{T}{^{\mathrm{TT}}_{ij}}
% \end{equation}




