% !TEX root = ../../thesis.tex

% -----------------------------------
% labels: \label{[type]:[this chap key]:[name]}
% -----------------------------------

% ############LOCAL MACROS############
\newcommand*{\mfa}{\mathfrak{a}} % fraktur a; index 0, 1, 2
\newcommand*{\mfb}{\mathfrak{b}} % fraktur b; index 0, 1, 2
\newcommand*{\mfc}{\mathfrak{c}} % fraktur c; index 0, 1, 2
\newcommand*{\Lam}[2]{\ProjectionLambda{#1}{#2}}
% ####################################

\newpage


\begin{draft}%
    Consider a planar domain wall in the $xy$-plane in a flat FRW universe, represented by a scalar field $\phi(\eta, \vec{x})$ and a potential $V(\phi)$. The action of this theory is
    \begin{equation}
        S = \integ[4]{x\sqrt{-g}} \cclosed{ 16\ppi G\ped{N}\mathcal{R} - \frac{1}{2}\phi\^{;\mu}\phi\_{;\mu} + V(\phi)}.
    \end{equation}
    The background metric is
    \begin{equation}
        d\widebar{s}^2 = \widebar{g}\indices{_{\mu\nu}} \diff \widebar{x}\indices{^\mu} \diff\widebar{x}\indices{^\nu} = -\diff t^2 + a^2(t) \Krondelta{_{ij}} \diff x^i \diff x^j = a^2 (\eta) \cclosed{-\diff\eta^2  + \diff x^2 +\diff  y^2 + \diff z^2}.
    \end{equation}
    The solution to $\widebar{\sq} \phi=\dv*{V}{\phi}$ is denoted $\widebar{\phi}(\eta, z)$. We let indices $a,b,c=1,2$ and $i,j,k,l,\ldots =1,2,3$. Now we add a linear perturbation $\zeta(\eta, x^a)$ to the wall such that
    \begin{equation}
        \phi(\eta, \vec{x}) = \widebar{\phi}(\eta, z; \zeta(\eta, x^a)) = \widebar{\phi}(\eta, z; 0) + \zeta(\eta, x^a) \pdv{\widebar{\phi}}{z}\bigg|_{\zeta=0} + \BigO{\zeta^2}.
    \end{equation}
    \comment{Remember eqs for $\zeta$!} 
    Furthermore, Fourier transforming \provethis{show this!} the spatial components gives
    \begin{equation}
        \phi(\eta,\vec{k}) = \integ[3]{x} \eu[\im k\_{i} x\^{i}] \phi(\eta,\vec{x})= \bclosed{ (2\ppi)^2 \Diracdelta[2]{k\_{a}}- \im k\_{3} \zeta(\eta, k\_{a}) } \widebar{\phi}(\eta, k\_{3}; 0) + \order{\zeta^2 }.
    \end{equation}

    % \phpar[expanation of/reference to how to obtain the following:]
    The TT-part of the energy-momentum tensor is \provethis{refer to some section} \comment{NB: $g$ cannot have cross terms!!}
    \begin{equation}
        \tensor*{T}{^{\mathrm{TT}}_{ij}}(\eta, \vec{k}) = \Lam{ij}{kl}(\hat{\vec{k}})  \integ[3][(2\ppi)^3]{p}  p\_{k} p\_{l} \phi(\eta, \vec{p}) \phi(\eta, \vec{k}-\vec{p}).
    \end{equation}
    We define a quantity $\tensor{t}{_{kl}}$ by
    \begin{equation}
        \tensor*{T}{^{\mathrm{TT}}_{ij}}(\eta, \vec{k}) =  \Lam{ij}{kl}(\hat{\vec{k}}) \pclosed{  \frac{1}{2\ppi} \cdot\tensor{t}{_{kl}}(\eta, \vec{k})  + \BigO{\zeta^2} },
    \end{equation}
    and the additional function 
    \begin{equation}
        \mathfrak{I}_n(\eta, q_0) = \integ{q}[\mathbb{R}] q^n \widebar{\phi}(\eta,q;0) \widebar{\phi}(\eta, q_0-q;0).
    \end{equation}
    After some manipulation \provethis{show this!}, we get the following:
    \begin{subequations}
        \begin{align}
            \tensor{t}{_{ab}} (\eta, \vec{k}) &= k\_{a} k\_{b} \bclosed{-\im \zeta(\eta, k\_{c})} \mathfrak{I}_1(\eta, k\_{3}) \\
            \tensor{t}{_{a3}} (\eta, \vec{k}) &= k\_{a} \bclosed{-\im \zeta(\eta, k\_{c})} \mathfrak{I}_2(\eta, k\_{3}) \\
            \tensor{t}{_{33}} (\eta, \vec{k}) &= k\_{3} \bclosed{-\im \zeta(\eta, k\_{c})} \mathfrak{I}_2(\eta, k\_{3}) + (2\ppi)^2\Diracdelta[2]{k\_{a}} \mathfrak{I}_2(\eta,k\_{3})
        \end{align}
    \end{subequations}
    \comment{There are some \emph{small} constraint on the perturbation from this. Need to be commented!}

    Gravitational waves sourced by this field is -- to first order in $\zeta$ -- given by
    \begin{equation}
    \begin{split}
        a\tensor{h}{_{ij}}(\eta, \vec{k}) &= \frac{16\ppi G\ped{N}}{k} \integ{\eta'}[\eta\ped{i}][\eta] \sin{\pclosed{k\bclosed{\eta-\eta'}}} a(\eta')  \tensor*{T}{^{\mathrm{TT}}_{ij}}(\eta', \vec{k}) \\
        &= \frac{8 G\ped{N}}{k} \Lam{ij}{kl}(\hat{\vec{k}}) \integ{\eta'}[\eta\ped{i}][\eta] \sin{\pclosed{k\bclosed{\eta-\eta'}}} a(\eta') t\_{kl}(\eta', \vec{k}) + \BigO{\zeta^2}.
    \end{split}
    \end{equation}
    Remaining are the $\Lam{ij}{kl}t\_{kl}$-elements, which in total are \checkthis{6} terms per $ij$, due to symmetry in $t\_{kl}$: %\comment{Gauge freedom implies 4, does it not? Maybe only in $h_{ij}$.} 
    \begin{equation}
    \begin{split}
        \Lam{ij}{kl} (\hat{\vec{k}}) t\_{kl} (\eta,\vec{k}) &= \cclosed{%
        \pclosed{%
            \Lam{ij}{12} + \Lam{ij}{21}} t\_{12} %
        + \pclosed{%
            \Lam{ij}{13} + \Lam{ij}{31} } t\_{13}  %
        + \pclosed{%
                \Lam{ij}{23} + \Lam{ij}{32} } t\_{23} }(\eta, k\hat{\vec{k}}) \\
            &\phantom{=}+ \cclosed{
                \Lam{ij}{11} t\_{11} + \Lam{ij}{22} t\_{22} + \Lam{ij}{33} t\_{33}
            }(\eta, k\hat{\vec{k}})
    \end{split}
    \end{equation}
    All of these are on the form
    \begin{equation}
        -\im \zeta(\eta, k\_{a}) \times \cclosed{k^2 k\^{2} k\ap{2} \mathfrak{I}_1(\eta, k\_{3}) A\_{ij}(\hat{\vec{k}}) + k \mathfrak{I}_2(\eta, k\_{3}) B\_{ij}(\hat{\vec{k}})},
    \end{equation}
    leaving
    \begin{equation}
        ah\_{ij} (\eta, \vec{k}) = 8G\ped{N} \bclosed{k A\_{ij}(\hat{\vec{k}}) \mathcal{I}_1(\eta, \vec{k}; \eta\ped{i}) + B\_{ij}(\hat{\vec{k}}) \mathcal{I}_2(\eta, \vec{k}; \eta\ped{i}) }
    \end{equation}
    where
    \begin{equation}
        \mathcal{I}_n(\eta, \vec{k}; \eta\ped{i}) = -\im \integ{\eta'}[\eta\ped{i}][\eta] a(\eta')  \sin{\pclosed{k\pclosed{\eta-\eta'}}} \times \zeta(\eta', k\_{a}) \mathfrak{I}_n(\eta', k\_{3}).
    \end{equation}
    Furthermore, we can show \provethis{proof!} that $A\_{ij}(\vec{n}) = -n\_{3} B\_{ij}(\vec{n})\equiv +2n\_{3} C\_{ij}(\vec{n}) $ for $\abs{\vec{n}}^2=n\_{1}\!^2 +n\_{2}\!^2 +n\_{3}\!^2 = 1$, allowing for the slightly simpler expression
    \begin{equation}
        ah\_{ij} (\eta, \vec{k}) = 4G\ped{N}  C\_{ij}(\hat{\vec{k}}) \bclosed{ k\_{3} \mathcal{I}_1(\eta, \vec{k}; \eta\ped{i}) - \mathcal{I}_2(\eta, \vec{k}; \eta\ped{i}) },
    \end{equation}
    where\checkthis{:
    % \begin{equation}
    %     \begin{split}
    %         A\_{11} = \frac{1}{2} n\_{3}\!^2 \bclosed{ n\_{1}\!^2 n\_{3}\!^2 - n\_{2}\!^2 } \quad    &A\_{12} = \frac{1}{2} n\_{1}n\_{2} n\_{3}\!^2 \bclosed{ n\_{3}\!^2 +1} \\
    %         A\_{22} = \frac{1}{2} n\_{3}\!^2 \bclosed{ n\_{2}\!^2 n\_{3}\!^2 - n\_{1}\!^2 }  \quad &A\_{13} =\frac{1}{2} n\_{1} n\_{3}\!^3 \bclosed{ n\_{3}\!^2 -1} \\
    %         A\_{33} = \frac{1}{2} n\_{3}\!^2 \bclosed{ \pclosed{1 - n\_{3}\!^2}^2 } \quad  &A\_{23} =\frac{1}{2} n\_{2} n\_{3}\!^3 \bclosed{ n\_{3}\!^2 -1}
    %     \end{split}
    % \end{equation}
    % \begin{equation}
    % \begin{split}
    %     A\_{ab}(\vec{n}) &= \frac{1}{2}n\_{3}\!^2 \bclosed{ n\_{a} n\_{b}\pclosed{n\_{3}\!^2 +1} + \Krondelta{\_{ab}}\pclosed{n\_{3}\!^2 -1} } \\
    %     A\_{a3}(\vec{n}) &= \frac{1}{2}n\_{1} n\_{3}\!^3\bclosed{ n_3^2 -1 } \\
    %     A\_{33}(\vec{n}) &= \frac{1}{2} n\_{3}\!^2 \bclosed{ \pclosed{1-n_3^2}^2 }
    % \end{split}
    % \end{equation}
    \begin{equation}
    \begin{split}
        C\_{ab}(\vec{n}) &= n\_{3} \bclosed{ n\_{a} n\_{b}\pclosed{n\_{3}\!^2 +1} - \Krondelta{_{ab}}\pclosed{1-n\_{3}\!^2 } } \\
        C\_{a3}(\vec{n}) &= -n\_{a} n\_{3}\!^2\pclosed{ 1-n\_{3}\!^2} \\
        C\_{33}(\vec{n}) &=  n\_{3}\!^2 \pclosed{1-n\_{3}\!^2}^2 
    \end{split}
    \end{equation}
    }


    Redshift $\mathfrak{z}_\ast=2 \therefore a(\eta\ped{i}) = (1+\zz_\ast)^{-1}=1/3$



    $ds^2 =a^2(\eta) \pclosed{\Krondelta{_{\mu\nu}}  + h\_{\mu\nu} } \diff x\^\mu \diff x\^\nu, x\^0 = \eta$

    $u\_{\mfa}x\^{\mfa}, \mfa=0,1,2$

    $u\_{\imath}x\^{\imath}, \imath=0,1,2 \flat \flat $



    \important{Important references:}~\citep[p.~145]{vachaspatiKinksDomainWalls2006},~\citep[p.~291]{vilenkinCosmicStringsDomain1985},~\citep[p.~375]{vilenkinCosmicStringsOther1994}

    

\end{draft}



\newpage

\section{General Formalism}


% \begin{equation}
%     a\tensor{h}{_{ij}}(\eta, \vec{k}) = \frac{16\ppi G\ped{N}}{k} \int \dots \tensor*{T}{^{\mathrm{TT}}_{ij}}
% \end{equation}




