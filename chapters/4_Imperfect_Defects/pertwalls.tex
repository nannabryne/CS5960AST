%%%%%%%%%%%%%%%%%%%%%%%%%%%%%%%%%%%%%%%
%%%%%% Ch. 4: Imperfect Defects  %%%%%%
%%%%%%%%%%%%%%%%%%%%%%%%%%%%%%%%%%%%%%%


% ---------------------------------------
% labels: \label{[type]:pertwalls:[name]}
% ---------------------------------------


% ¨¨¨¨¨¨¨¨¨¨¨¨¨¨¨¨¨¨¨¨¨¨¨¨¨¨¨¨¨¨¨¨¨¨¨¨¨
% LOCAL MACROS:
\newcommand*\hypsurf{\ALIAShypsurf}             % hypersurface
\newcommand*\sppt{\ALIASsppt}                   % spatial part of pert.
\newcommand*\pert{\ALIASpert}                   % perturbed quantities
\newcommand*\hypacc{\widehat}                   % accent on hypersurface quantities
\newcommand*\Ft{\ALIASFt}                       % Fourier transform
\newcommand*\dummy{\ALIASdummy}                 % Dummy variable
\newcommand*\Tw{\ALIASTw}
\newcommand*\Twf{\ALIASTwf}
% ¨¨¨¨¨¨¨¨¨¨¨¨¨¨¨¨¨¨¨¨¨¨¨¨¨¨¨¨¨¨¨¨¨¨¨¨¨




% ////////////////// intro //////////////////



Any cosmologically relevant domain wall will be thin compared to the horizon~\citep{pressDynamicalEvolutionDomain1989}. Are they sufficiently thin so that the thin-wall approximation holds? If so, we should be able to study the dynamics of such walls by viewing them as $(2+1)$-dimensional timelike hypersurfaces in a spacetime of $3+1$ dimensions. 
By Nambu--Goto theory and an action principle, one can derive an equation governing the dynamics of this worldsheet~\citep{vilenkinCosmicStringsOther1994}. In fact, this is the same for any hypersurface of $d$ dimensions, thus relevant for several types of defects. In particular, this concerns the codimension-one defects, as higher-codimension defects cannot generally be thought of as hypersurfaces, but submanifolds in any case~\citep{vachaspatiKinksDomainWalls2006}. 




To substantiate the applicability of this theory, we begin in a more general picture than what we eventually will require. We consider a $d$-dimensional timelike ($\varsigma = +1$ in~\cref{eq:GR:diffgeo:varsigma_hypsurf}) submanifold $\hypsurf$ embedded in a smooth pseudo-Riemannian $n$-dimensional manifold $\Manifold$. %
We let $\hypsurf$ have codimension one ($d=n-1$) and split the ambient space into two separate hypervolumes. Now, $\hypsurf$ is a timelike hypersurface of $\Manifold$ that can be interpreted as the $d$-dimensional analogue of an infinitely thin domain wall. 


% Instead of 

We will use this to explore the dynamics of domain walls in the thin-wall limit, where the emphasis is not on the symmetry-breaking scalar field $\phi$, but of the position and evolution of the domain wall itself. There exist analytical solutions to the equation of motion for linear perturbations to the \emph{planar} wall position in expanding spacetime. We derive this equation for conventional, static walls and propose a version of this applicable to the defect during phase transition (when the surface energy density is changing). 

Now, static, symmetric walls do not produce gravitational radiation by themselves. We will use the dynamics of a wiggly domain wall as source to tensorial metric perturbations, a.k.a. gravitational waves.


% We can find an equation of motion for leading order distortions to the wall normal coordinate, eventually feeding asymmetry to the Nambu--Goto stress--energy tensor, %
% % (derived from the action in~\cref{eq:cosmo:defects:Z2_action}), 
% necessarily causing spacetime distortions that may or may not live on to paint its autograph in a gravitational wave observation on Earth. How this signature looks, is for us to figure out. 






The chapter is outlined as follows. In~\cref{sec:pertwalls:eom_wall} we review the general formula for dynamics of topological defects. In~\cref{sec:pertwalls:thinwall} we look at domain walls in a general spacetime, as well as perturbations to planar domain walls in an FLRW universe. We implement the symmetron description of these walls in~\cref{sec:pertwalls:mywalls} and solve the linear perturbation in a matter-dominated universe. In~\cref{sec:pertwalls:gws} we review the gravitational waves sourced by the Nambu--Goto stress--energy tensor.




% /////////////////////////////////////////// 






% ////////////////// sections //////////////////


% General formula in NG theory
% \section{\tmptitle{Kink dynamics / General formula / Formal treatment}}\label{sec:pertwalls:eom_wall}
\section{Formal treatment of defect dynamics}\label{sec:pertwalls:eom_wall}
    {\subimport{./}{eom_wall.tex}}

% Planar DWs in FLRW universe
% \section{Dynamics of planar domain walls in expanding universe}\label{sec:pertwalls:thinwall}
\section{Domain-wall dynamics}\label{sec:pertwalls:thinwall}
    {\subimport{./}{thinwall.tex}}

% Symmetron DWs (planar, FLRW)
\section{Symmetron walls}\label{sec:pertwalls:mywalls}
    {\subimport{./}{mywalls.tex}}

% Sourcing of GWs
\section{From domain wall wiggles to spacetime ripples}\label{sec:pertwalls:gws}
    {\subimport{./}{gws.tex}}




% //////////////////////////////////////////////
