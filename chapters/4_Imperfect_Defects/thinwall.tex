% |||||||||||||||||||||||||||||||||||||||||||||||
% |||||| 4.2 Dynamics of thin domain walls ||||||
% |||||||||||||||||||||||||||||||||||||||||||||||

% ------------------------------------------------
% labels: \label{[type]:pertwalls:thinwall:[name]}
% ------------------------------------------------






So far, we have addressed the very general Nambu--Goto picture. %
% From here, we specify defects to be $(2+1)$-dimensional i in $(3+1)$
From here, we turn our attention to $(2+1)$-dimensional topological defects living in $(3+1)$-dimensional spacetime, i.e.~domain walls. Let $v=v_+-v_- =0$. %
We describe the wall position in $\Manifold$ with $x\^\mu = X\^\mu(\xi\^a)$ where $\mu=0,1,2,3$ and $a=0,1,2$., and impose coordinates that satisfy
\begin{equation}
    {ds}^2 =  {\diff Y}^2  + \gamma\_{ab} {\diff \xi\^a}{\diff \xi\^b},
\end{equation}
where $Y$ is the normal distance $\hypsurf$~\citep{vilenkinCosmicStringsOther1994}. With~\cref{eq:pertwalls:thinwall:induced_metric} and the identity \( {\delta \gamma} = \gamma \gamma\^{ab} {\delta\gamma_{ab}}  \), we obtain~\citep{vilenkinCosmicStringsOther1994}
\begin{equation}\label{eq:pertwalls:thinwall:eom_domain_wall_general}
    \hypacc{{\sq}} X\^\mu
    % + n\_\mu\ChristoffelSym{\mu}{\kappa\tau} \gamma\^{ab} e\indices*{^\kappa_a}e\indices*{^\tau_b} %
    + \ChristoffelSym{\mu}{\kappa\tau} \gamma\^{ab} X\indices{^\kappa_{,a}}X\indices{^\tau_{,b}} %
     = 0,
\end{equation}
which is a repetition of~\cref{eq:pertwalls:eom_wall:eom_defect} with degenerate vacua. %
According to~\citet{vilenkinCosmicStringsOther1994}, to the best of the authors' knowledge,
% \speak{(This was~\citeyear{vilenkinCosmicStringsOther1994}, so maybe outdated?)}, 
no known gauge choice can reduce~\cref{eq:pertwalls:thinwall:eom_domain_wall_general} to an exactly solvable linear equation. %, to the best of the authors' knowledge. %
If $X\^\mu$ is a known solution, we can study a linear perturbation to this instead, with the perturbed wall trajectory in~\cref{eq:pertwalls:eom_wall:linear_pert_wall_coord}.



% \hlineSep




\subsection{Planar walls in expanding spacetime}

% From here, we turn our attention to $(2+1)$-dimensional topological defects living in $(3+1)$-dimensional conformally flat spacetime.

% We consider the unperturbed wall position to be
Assume we have the unperturbed wall parallel to the $xy$-plane of an expanding spacetime
\begin{equation}
    {ds}^2 = a^2 {ds}^2\rvert\nped{M} = a^2 \pclosed{- {\diff \tau}^2 + {\diff \vec{x}}^2  }, 
\end{equation}
where $\vec{x}=(x,y,z)$ are the comoving Cartesian coordinates. %
We fix the gauge $ \xi^a = (\tau, x, y)$ so that $z=z\ped{w}=z_0+\epsilon(\tau, x, y) $ uniquely describes the wall motion. 
% the function $\zeta(\tau, x, y)$ uniquely describes the wall motion. 
The Nambu--Goto action reads
\begin{equation}\label{eq:pertwalls:thinwall:Nambu_Goto_action_dw_FLRW}
    S\nped{NG} = -\sigma \integ[3]{\xi} \sqrt{-\gamma} = - \sigma \integ{\tau}a^3(\tau)\integ{x}\integ{y}  h(\tau,x,y)
\end{equation}
where 
\begin{equation}
    h \triangleq a^{-3}\sqrt{-\gamma}%\rvert\ped{unperturbed} 
    = \sqrt{1 + \epsilon\^{,a}\epsilon\_{,a} }.
    % = \sqrt{1 + \partial\^a X_\bot \partial\_a X_\bot  }.
\end{equation}
The equation of motion for the wall becomes~\citep{vilenkinCosmicStringsOther1994}
\begin{equation}\label{eq:pertwalls:thinwall:eom_domain_wall_FLRW}
    % \pclosed{ \partial_\tau 3 \mathcal{H} } (f\partial_\tau z)
    % \epsilon_{,\tau\tau} + 3\mathcal{H} \epsilon_{,\tau} - \epsilon_{,xx} - \epsilon_{,yy} = 0.
    % -\partial\^a ( \partial\_a X_\bot /f) + 3 \mathcal{H} (\partial_\tau X_\bot /f) = 0.
    -\partial\^a ( \epsilon\_{,a}/h) + 3 \mathcal{H} \epsilon_{,\tau} /h = 0,
\end{equation}
where $\mathcal{H}= a_{,\tau}/a$ is the conformal Hubble factor. %
We let $\epsilon$ be a linear perturbation such that $\epsilon^2 \to 0$ %$\pert{\zeta} = z_0 + \epsilon(\tau, x, y)$. 
and consider power-law expansion, $a\propto \tau^\alpha$. 
% We Taylor expand $f$ to find $f = 1 + \mathscr{O}(\epsilon^2)$ and 
Now,
\begin{equation}\label{eq:pertwalls:thinwall:epsilon_eom_simple}
    \epsilon_{,\tau\tau} + 3\mathcal{H} \epsilon_{,\tau} - \epsilon_{,xx} - \epsilon_{,yy} = 0
\end{equation}
is the equation for the first-order displacement field. %
This is a separable partial differential equation (PDE) and we assume solutions of the form $\epsilon(\tau, x, y) = \varepsilon(\tau)\sppt(x,y)$. Now, we separate~\cref{eq:pertwalls:thinwall:epsilon_eom_simple} into
\begin{align}\label{eq:pertwalls:thinwall:varepsilon_and_E_eoms}
    \ddot{\varepsilon} + \mathcal{D}(\tau) \dot{\varepsilon} + p^2 \varepsilon = 0 \quad\text{and}\quad  \sppt_{,xx} + \sppt_{,yy} + p^2 \sppt = 0,
\end{align}
where $\mathcal{D}(\tau) = \mathcal{H}(\tau)$. %
The spatial part has solutions that are linear combinations of $\sin{(p_x x+p_y y)}$ and $\cos{(p_x x+p_y y)}$, with $p_x^2+p_y^2 = p^2$. %
The solutions to this equation are of the form $\varepsilon = \tau^\nu \Cylindrical[\nu](p\tau)$ with $ \nu=(1-3\alpha)/2$ (see~\cref{app:cylinder},~\cref{eq:cylinder:Bessel_eq_abc}).


% \hlineSep






% \begin{equation}
%     \xi^a = (\tau, x, y)
% \end{equation}










% Imposing Cartesian coordinates $x\^\mu=(\tau,x,y,z)$ and




% $g\_{\mu\nu} = a^2 (\tau) \eta\_{\mu\nu}$, we know that we are dealing with domain walls in a flat FLRW universe.






% We place a thin domain wall at $z=z_0$, which gives the embedding $X\^\mu(\xi\^a)=(\xi^\tau, \xi^x, \xi^y, z_0)$ of $\hypsurf$ in $\Manifold$, with $\xi\^a= x\^a$. The comoving normal vector is taken to be
% %
% % The unit normal vector is %$n\^\mu = a^{-1}\Krondelta*{^\mu_z}$.
% $n\^\mu = \Krondelta*{^\mu_z}$, such that $a^2 \eta\_{\mu\nu}n\^\mu n\^\nu=a^2$ and $n\_\mu = \eta_{\mu\nu}n\^\nu$. %The induced metric
% \hl{Here is a misunderstanding!}


% Now the groundwork is laid for the scenario that which this project is all about. That is, we consider a conformally flat spacetime with %metric $g\_{\mu\nu}= a^2 \eta\_{\mu\nu}$
% line element
% \begin{equation}
%     {ds}^2 = g\_{\mu\nu} \diff x\^\mu \diff x\^\nu = a^2 \eta\_{\mu\nu} \diff x\^\mu \diff x\^\nu = a^2(\tau) \cclosed{ -{\diff \tau}^2 + {\diff x}^2 + {\diff y}^2 + {\diff z}^2  }.
% \end{equation}
% \hlineSep
% A solution to the eom for $z(\tau)$ is
% % \hlineSep
% We place a thin domain wall at $z$-coordinate $z_0$, represented by a hypersurface $\hypsurf$, whose induced metric is $\gamma\_{ab}= g\_{\mu\nu} x\indices{^\mu_{,a}} x\indices{^\nu_{,b}}$.
% We assert the solution $x\^\mu \supset x_\bot(\tau)n\^\mu =z_0 \Krondelta{^{\mu z}}$ to~\cref{eq:pertwalls:eom_wall:eom_extrinic_curvature} with $v=0$.\footnote{
%     In~\citet{garrigaPerturbationsDomainWalls1991} it is shown that $z=z_0$ is the only viable solution in the analogous case in Minkowski space.
% } %



% The perturbed equation is obtained by a similar action principle as before,
% \begin{equation}\label{eq:pertwalls:thinwall:pert_variation_x_normal_coord}
%     X\^\mu  \to \pert{X}^\mu = X\^\mu + \epsilon n\^\mu.
% \end{equation}
% We simplify our equations tremendously now that we keep $n\^\mu$ unaffected by the variation, and solve the one-dimensional problem
% \begin{equation}
%     \hypacc{\sq} \pert{X}^{z} + \ChristoffelSym{z}{\kappa \tau}\gamma\^{ab}  \pert{X}\indices{^\kappa_{,a}} \pert{X}\indices{^\tau_{,b}} + \frac{\rho}{\sigma} = 0.
% \end{equation}
% %with
% % \begin{equation}
% %     x_\bot \to \pert{x}_\bot = x_\bot + \epsilon.
% % \end{equation}
% % only we keep $n\^\mu$ unaffected by the variation.
% The resulting equation of motion for the scalar perturbation to leading order reads
% \begin{equation}
%     a^2\hypacc{\sq} \epsilon  - 2\mathcal{H} \dot{\epsilon} = 0
% \end{equation}
% or
% \begin{equation}\label{eq:pertwalls:thinwall:epsilon_eom_simple}
%     \epsilon_{,\tau\tau} + 3\mathcal{H} \epsilon_{,\tau} - \epsilon_{,xx} - \epsilon_{,yy} = 0.
% \end{equation}
% \comment{Decide which form. Also include calculation details.}

% This is a separable partial differential equation (PDE) and we assume solutions of the form $\epsilon(\tau, x, y) = \varepsilon(\tau)\sppt(x,y)$. Now, we separate~\cref{eq:pertwalls:thinwall:epsilon_eom_simple} into
% % \begin{subequations}
%     \begin{align}\label{eq:pertwalls:thinwall:varepsilon_and_E_eoms}
%         \ddot{\varepsilon} + f(\tau) \dot{\varepsilon} + p^2 \varepsilon = 0 \quad\text{and}\quad  \sppt_{,xx} + \sppt_{,yy} + p^2 \sppt = 0,
%     \end{align}
% % \end{subequations}
% where $f(\tau)=3\mathcal{H}$. %
% % %
% \hlineSep
% %
% Now, $z = z_0 + \epsilon(y\^a)$ and
% \begin{equation}
%     \mathrm{D}^2 \epsilon + \ChristoffelSym{3}{\kappa\tau} q\^{ab} \pclosed{ \Krondelta{^\kappa_a}\Krondelta{^\tau_b} + 2\Krondelta{^\kappa_a} \Krondelta{^\tau_3} \epsilon\_{,b}  +  \Krondelta{^\kappa_3} \Krondelta{^\tau_3} \epsilon\_{,a}   \epsilon\_{,b}    } + \frac{\Delta v}{\sigma} n\^\mu= 0
% \end{equation}
% \blahblah \comment{Set $\Delta v=0$ etc.}
% Eventually, we arrive at
% \begin{equation}
%     \ddot{\epsilon} + 3\mathcal{H}\dot{\epsilon} -  \pclosed{\partial_x^2 +\partial_y^2}\epsilon= 0.  %\bclosed{ \pdv[2]{}{x}  + \pdv[2]{}{y}  } \epsilon = 0.
% \end{equation}
% With the ansatz $\epsilon=\varepsilon_p(\tau) \sppt(x,y)$ we can solve this in terms of eigenvalues $-p^2$,
% \begin{equation}\label{eq:pertwalls:eom_epsp_s_MD_simple}
%     \ddot{\epsilon} + 3\mathcal{H}\dot{\epsilon}+ p^2 \epsilon = 0,% \pclosed{\partial_x^2 +\partial_y^2} \\sppt(x,y) = -p^2 \\sppt(x,y)
% \end{equation}
% where $(\partial_x^2 +\partial_y^2) \sppt(x,y) = -p^2 \sppt(x,y) $.
% The spatial part has solutions that are linear combinations of $\sin{(p_x x+p_y y)}$ and $\cos{(p_x x+p_y y)}$, with $p_x^2+p_y^2 = p^2$. %
% % For a universe with $a\propto \tau^\alpha$, 
% The solutions to this equation are of the form $\varepsilon = \tau^\nu \Cylindrical[\nu](p\tau)$ with $ \nu=(1-3\alpha)/2$.








% * * * * * * * * * * * * * * * * * * * * * * * * * * * * * * * * * * * * * * * * * * * * * 
% Energy and mom.
% * * * * * * * * * * * * * * * * * * * * * * * * * * * * * * * * * * * * * * * * * * * * * 

\subsubsection{Energy and momentum} %

% \paragraph{Non-vanishing thickness.} % 
To some extent, we can account for a possibly non-vanishing wall half-width $l\sim\delta$ by choosing a Gaussian function instead of a Dirac delta distribution. Simply substituting
\begin{equation}
    \Diracdelta(r) \to   \varPhi_l (r) \equiv\frac{1}{\sqrt{2\ppi}l} \exp{ -\frac{r^2}{2l^2} }
\end{equation}
in~\cref{eq:pertwalls:eom_wall:SE_tensor_NG} does the trick, and restores $\Diracdelta(r)$ in the limit where $l\to 0$. This is understood in context with~\cref{sec:cosmo:defects:dws}, where we argue that $l=\delta/\sqrt{2}$ gives a suitable Gaussian profile. %, and we define 
% \begin{equation}
%     \varPhi_\delta (r) \
% \end{equation}
Note that this is not the same as going beyond the thin-wall limit, but rather an approximation that includes a thickness that is small enough to not alter the dynamics. %
Domain walls of cosmological relevance will have thickness much smaller than the horizon~\citep{garrigaPerturbationsDomainWalls1991}. However, some models will have instances in time when this is not true, and the thickness in fact varies~\citep{hinterbichlerScreeningLongRangeForces2010}.

% \subsection{Stress--energy tensor}\label{sec:pertwalls:thinwall:SE_tensor}
We perform the variation in~\cref{eq:pertwalls:eom_wall:SE_tensor_NG} for the scenario in question in~\cref{app:walls:SE_tensor}, to find the SE tensor associated with the wall motion. %The detailed calculation is found in~\cref{app:walls:SE_tensor}\rcomment{Maybe not...}. 
The non-vanishing spatial components are
\begin{equation}\label{eq:pertwalls:thinwall:SE_tensor}
    \begin{split}
        % T\_{ab}(\tau, \vec{x})
        \Tw\_{AB}&=  -a \sigma\ped{w} \varPhi_l(z-z\ped{w}) \cdot \Krondelta{_{AB}} ,\\
        % T\_{(i3)}(\tau, \vec{x})
        \Tw\_{(A3)}&= -a \sigma\ped{w}\varPhi_l(z-z\ped{w} ) \cdot\epsilon\_{,A}, %\partial\_{i} \epsilon
    \end{split}
\end{equation}
where indices $A,B=1,2$ and $z\ped{w}=z_0 + \epsilon$. The tensor $\Tw\^{\mu\nu}$ refers to the stress--energy tensor associated with a particular domain wall by~\cref{eq:pertwalls:eom_wall:SE_tensor_NG}.

% We will consider $\Tw\^{\mu\nu} = T\^{\mu\nu}\rvert\nped{NG}$ to be the stress--energy 





% * * * * * * * * * * * * * * * * * * * * * * * * * * * * * * * * * * * * * * * * * * * * * 
% Time-dep.
% * * * * * * * * * * * * * * * * * * * * * * * * * * * * * * * * * * * * * * * * * * * * * 


\subsection{Time-dependent surface tension}\label{sec:pertwalls:thinwall:time_dep_surface_tension}
    Until now, the theory is model-independent in the sense that we have not assumed any particular type of domain wall (or defect, in general). This dependence is encoded in the \emph{constant} surface tension $\sigma$ and difference in vacuum energies $v=v_+-v_-$. %
    % It is plain to see that the $\sigma$-dependence disappears when $v=0$.
    It is plain to see from~\cref{eq:pertwalls:eom_wall:eom_extrinic_curvature} that the absence of energy bias removes the surface-tension dependence. %
    But what if the surface tension is not constant?

    % What connects the eom for $\epsilon$ to a model of discrete symmetry break, is mainly the surface

    % Until now, the equations have been model-independent, given constant surface tension ($\dot{\sigma}= 0$) and no energy bias ($v_+ - v_- = 0$).
    If we allow the surface tension to vary, $\sigma=\sigma(\tau)$, %$\stackrel{\tau\to \infty}{\to} \sigma_\infty$
    we need to put this inside of the integral in the Nambu--Goto action in~\cref{eq:pertwalls:thinwall:Nambu_Goto_action_dw_FLRW}. The calculation is presented in~\cref{app:walls:dynamics}. We immediately see that this is equivalent to letting $a^3\to a^3 \sigma$, which amounts to
    % with the substitution
    % \begin{equation}
    %     3\mathcal{H} =  \frac{3}{a} \dv{a}{\tau} \to \frac{3}{a\sigma^{1/3}} \dv{a\sigma^{1/3}}{\tau} =  3\frac{\dot{a}}{a} +  \frac{\dot{\sigma}}{\sigma} = 3\dv{\ln{a}}{\tau} + \dv{\ln{\sigma}}{\tau} %3\dot{\ln{a}} + \dot{\ln{\sigma}}
    %     % 3\mathcal{H} =  3 \dot{a}/a \to  3 \pclosed{a\sigma^{1/3}}^{-1}\dv{a\sigma^{1/3}}{\tau} =  3 \dot{a}/a  + \dot{\sigma}/\sigma
    % \end{equation}
    \begin{equation}
        \mathcal{D}(\tau) = \frac{3}{a\sigma^{1/3}} \dv{a\sigma^{1/3}}{\tau} =  3\frac{\dot{a}}{a} +  \frac{\dot{\sigma}}{\sigma} = 3\dv{\ln{a}}{\tau} + \dv{\ln{\sigma}}{\tau}
    \end{equation}
    % %This is equivalent to substituting $a\to a \sigma^{1/3}$, so that we only need to change $\mathcal{H} \to \mathcal{H}_\Upsilon$ \dots
    % If we define $\mathcal{H}_\Upsilon = \Upsilon^{-1} \dot{\Upsilon} $, we see that with $\Upsilon=a \sigma^{1/3}$, we get
    % \begin{equation}
    %     % \ddot{\epsilon} + \pclosed{ 3\frac{\dot{a}}{a} + \frac{\dot{\sigma}}{\sigma} } \dot{\epsilon} + k^2 \epsilon = 0.
    %     \ddot{\epsilon} + \pclosed{ 3\mathcal{H}_a + \mathcal{H}_\sigma  } \dot{\epsilon} + k^2 \epsilon = 0.
    % \end{equation}
    % If we write
    % \begin{equation}
    %     \ddot{\epsilon} + 3\hat{\mathcal{H}}\dot{\epsilon} - \vec{\nabla}^2 \epsilon = 0,
    % \end{equation}
    % and let
    in~\cref{eq:pertwalls:thinwall:varepsilon_and_E_eoms}. We get
    \begin{equation}
        \ddot{\varepsilon} + \pclosed{3\dot{a}/a + \dot{\sigma}/\sigma}\dot{\varepsilon} +p^2 \varepsilon = 0.
    \end{equation}
    This extra term will introduce the model-dependence since the surface tension is given by (see~\cref{app:walls:surface_tension})%$We show in~\cref{app:walls:surface_tension} that the surface tension in the thin-wall approximation is
    \begin{equation}\label{eq:pertwalls:thinwall:surface_tension}
        \sigma\simeq \integ{\phi}[\phi_-][\phi_+] \sqrt{ 2V\ped{eff}(\phi)-2V\ped{eff}(\phi_\pm) },
    \end{equation}
    % ---the dependence on $V\ped{eff}(\phi)$
    % seeing as the surface tension in the thin-wall limit is given by
    % \begin{equation}\label{eq:pertwalls:thinwall:surface_tension}
    %     \sigma = \integ{\phi}[\phi_-][\phi_+] \sqrt{ 2V\ped{eff}(\phi)-2V\ped{eff}(\phi_\pm) },
    % \end{equation}
    where $V\ped{eff}(\phi)$ is the effective potential of the scalar field theory.






% \phpar[constant surface tension]






% We follow~\citet{garrigaPerturbationsDomainWalls1991} and~\citet{ishibashiEquationMotionDomain1999}. The world sheet $\hypsurf$ divides \Manifold~into two submanifolds $\Manifold_{\pm}$ such that $\mathscr{M} = \mathscr{M}_+ \cup  \hypsurf \cup \mathscr{M}_-$. That is to say, a domain wall holds a world sheet separating two vacua. We take \Manifold~ to be smooth and $(N+1)$-dimensional, and let $\hypsurf$ be a smooth also and $((N-1)+1)$-dimensional. Consequently, $\hypsurf$ is a timelike hypersurface in \Manifold.


% %
% % \citep{ishibashiEquationMotionDomain1999,garrigaPerturbationsDomainWalls1991}


% \begin{bullets}
%     \item Vary DW action
%     \item Goal: E.O.M. for physically relevant component (epsilon basically)
%     \item Expression for energy--momentum tensor
%     \item Extension to non-thin walls
%     \item Extension to Asymmetron or introduction of energy bias
%     \item What does thin mean? Why is the tension indep. of width?
% \end{bullets}


% % The generalisation to $(N+1)$ dimensions is straight-forward.
% We invoke a smooth coordinate system $\{x\^\mu\}$ ($\mu=0,1,\dots,N$) of the spacetime $(\Manifold, g\_{\mu\nu})$ in a neighbourhood of $\hypsurf$. The embedding of $\hypsurf$ in $\Manifold$ is $x\^\mu = x\^\mu(y\^a)$, where the coordinate system $\{y\^a\}$ ($a=0,1,\dots,N-1$) parametrises $\hypsurf$.
% The induced metric on $\hypsurf$ is
% \begin{equation}\label{eq:pertwalls:thinwall:induced_metrid}
%     q\_{ab} = g\_{\mu\nu} e^\mu_a e^\nu_b; \quad e^\mu_a \equiv \pdv{x\^\mu}{y\^a}% g\_{\mu\nu}\pdv{x\^\mu}{y\^a}\pdv{x\^\nu}{y\^b}
% \end{equation} \provethis{argue!}

% We let $\sigma$ represent the surface energy density of the wall---a quantity we will discuss i much more detail later---and $v_\pm$ the vacuum energy densities of $\mathscr{M}_\pm$. The complete action of the coupled system is
% \begin{equation}
%     S = \underbrace{- \sigma \integ[N]{y\sqrt{-q}}[\hypsurf] }_{S\ped{NG}} %
%     - v_+ \integ[N+1]{x\sqrt{-g}}[\mathscr{M}_+] %
%     - v_- \integ[N+1]{x\sqrt{-g}}[\mathscr{M}_-] %
%     \underbrace{+ \frac{M\nped{Pl}^2}{2}\integ[N+1]{x\sqrt{-g}}[\mathscr{M}] \mathcal{R}}_{S\ped{EH}}.
% \end{equation}
% \comment{Comment about Nambu-Goto action. Maybe add matter and $\phi$ actions?}

% % The action for a thin domain wall is famously~\citep[e.g.][]{vachaspatiKinksDomainWalls2006} the Nambu-Goto action $S\ped{NG}$,
% % \begin{equation}
% %     S\ped{dw} = -\sigma \integ[N]{y\sqrt{-h}}[\hypsurf],
% % \end{equation}
% % where $\sigma$ is the wall's energy per unit area, henceforth called ``surface tension''. The action for the coupled system


% Under small changes in $x\^\mu$ on $\hypsurf$, $x^\mu \to x\^\mu + \delta x\^\mu$, we obtain the equation
% \begin{equation}
%     \mathrm{D}\^a e\indices*{^\mu_a} + \ChristoffelSym{\mu}{\kappa\tau} q\^{ab} e\indices*{^\kappa_a}  e\indices*{^\tau_b} +\frac{v_+-v_-}{\sigma} n\^\mu = 0,
% \end{equation}
% or equivalently,
% \begin{equation}
%     % \sq_\hypsurf %
%     \mathrm{D}\_a \mathrm{D}\^a %
%     x\^\mu+ \ChristoffelSym{\mu}{\kappa\tau} q\^{ab} \pdv{x^\kappa}{y\^a} \pdv{x^\tau}{y\^b}  +\frac{v_+-v_-}{\sigma} n\^\mu = 0,
% \end{equation}
% where $\mathrm{D}\_a$ is the covariant derivative with respect to $q\_{ab}$.

% \checkthis{The part of $\delta x\^\mu$ that is tangential to $\hypsurf$ are diffeomorphisms on $\hypsurf$ ($y\^a \to y\^a + \delta y\^a$).} The only physically meaningful component is the transverse one; %Let us write $x\^\mu = $
% \begin{equation}
%    n\_\mu\mathrm{D}\^a \mathrm{D}\_a x\^\mu + n\_\mu \ChristoffelSym{\mu}{\kappa\tau} q\^{ab} e\indices*{^\kappa_a}  e\indices*{^\tau_b} +\frac{\Delta v}{\sigma} = 0.
% \end{equation}


% Without loss of generality we let \dots $n\^\mu = n\^N$ \blahblah







% \subsection{Planar domain walls in FLRW spacetime}





% %%%%%%%%%%%%%%%%%%%%%%%%%%%%%%%%%%%%%%%%%%%%%%%%%%%%%%%%%%%%%%%%%%%%%%%%%%%%
% %%%%%%%%%%%%%%%%%%%%%%%%%%%%%%%%%%%%%%%%%%%%%%%%%%%%%%%%%%%%%%%%%%%%%%%%%%%%
% % \begin{draft}
% % \subsection{Expanding universe. (my scenario)}




% % \end{draft}
% %%%%%%%%%%%%%%%%%%%%%%%%%%%%%%%%%%%%%%%%%%%%%%%%%%%%%%%%%%%%%%%%%%%%%%%%%%%%
% %%%%%%%%%%%%%%%%%%%%%%%%%%%%%%%%%%%%%%%%%%%%%%%%%%%%%%%%%%%%%%%%%%%%%%%%%%%%



% % \section{Stress--energy tensor of domain walls}
