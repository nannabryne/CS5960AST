% |||||||||||||||||||||||||||||||||||
% |||||| 4.1 General framework ||||||
% |||||||||||||||||||||||||||||||||||

% ------------------------------------------------
% labels: \label{[type]:pertwalls:eom_wall:[name]}
% ------------------------------------------------



% ¨¨¨¨¨¨¨¨¨¨¨¨¨¨¨¨¨¨¨¨¨¨¨¨¨¨¨¨¨¨¨¨¨¨¨
% LOCAL MACROS:
% \newcommand*\Ft{\ALIASFt}
% ¨¨¨¨¨¨¨¨¨¨¨¨¨¨¨¨¨¨¨¨¨¨¨¨¨¨¨¨¨¨¨¨¨¨¨



% \begin{bullets}
%     \item Vary DW action
%     \item Goal: E.O.M. for physically relevant component (epsilon basically)
%     \item Expression for energy--momentum tensor
%     \item Extension to non-thin walls 
%     \item Extension to Asymmetron or introduction of energy bias
%     \item What does thin mean? Why is the tension indep. of width?
% \end{bullets}



% \comment{\citet{garrigaPerturbationsDomainWalls1991} and~\citet{ishibashiEquationMotionDomain1999}}
% \comment{Technical section}

In this fairly technical section we follow~\citet{garrigaPerturbationsDomainWalls1991} and~\citet{ishibashiEquationMotionDomain1999}. %
In mathematical terms, we consider a smooth spacetime
\begin{equation}
    \Manifold = (\Real^{n-1,1}, g\_{\mu\nu}) \supset \hypsurf=(\Real^{d-1,1}, \gamma\_{ab})
\end{equation}
where $d = n-1$ and indices $a,b,c =0,1,\dots d-1$, while Greek indices as usual runs from 0 to $n-1$. Let the world sheet $\hypsurf$ divide \Manifold~into two hypervolumes $\Manifold_{\pm}$ such that $\mathscr{M} = \mathscr{M}_+ \cup  \hypsurf \cup \mathscr{M}_-$. 
We invoke a smooth coordinate system $\{x\^\mu\}$ of the spacetime in a neighbourhood of $\hypsurf$. The embedding of $\hypsurf$ in $\Manifold$ is $x\^\mu = X\^\mu(\xi\^a)$, where the coordinate system $\{\xi\^a\}$ parametrises $\hypsurf$. 
The induced metric on $\hypsurf$ is
% \begin{equation}\label{eq:pertwalls:eom_wall:induced_metric}
%     % \gamma\_{ab} = g\_{\mu\nu} e^\mu_a e^\nu_b; \quad e^\mu_a \equiv \pdv{x\^\mu}{\xi\^a}
%     \gamma\_{ab} = g\_{\mu\nu} x\indices{^\mu_{,a}}x\indices{^\nu_{,b}}
% \end{equation} 
\begin{equation}\label{eq:pertwalls:thinwall:induced_metric}
    \gamma\_{ab} = g\_{\mu\nu} e\indices*{^\mu_{a}} e\indices*{^\nu_{b}}; \quad e\indices*{^\mu_{a}}\equiv X\indices{^\mu_{,a}}= \pdv*{X\^\mu}{\xi\^a}.% g\_{\mu\nu}\pdv{x\^\mu}{\xi\^a}\pdv{x\^\nu}{\xi\^b} 
\end{equation}
% \provethis{argue!}
%
Here, $e\indices*{^\mu_a}$ are the tangent vectors and $n\^\mu$ the unit-normal vector pointing from $\hypsurf$ to $\Manifold_+$, obeying
\begin{equation}
    g\_{\mu\nu} n\^\mu n\^\nu = 1  \quad \text{and}\quad  g\_{\mu\nu} e\indices*{^\mu_a} n\^\nu = 0.
\end{equation}
The action for the system is~\citep{ishibashiEquationMotionDomain1999}
% \begin{equation}
%     S = \underbrace{- \sigma \integ[d]{y\sqrt{-\gamma}}[\hypsurf] }_{S\nped{NG}} %
%     \underbrace{- v_+ \integ[n]{x\sqrt{-g}}[\mathscr{M}_+] %
%     - v_- \integ[n]{x\sqrt{-g}}[\mathscr{M}_-]}_{S\ped{vac}}. %
%     %\underbrace{+ \frac{M\nped{Pl}^2}{2}\integ[n]{x\sqrt{-g}}[\mathscr{M}] \mathcal{R}}_{S\ped{EH}}.
% \end{equation}
\begin{equation}\label{eq:pertwalls:eom_wall:total_action}
    S = - \sigma \integ[d]{\xi\sqrt{-\gamma}}[\hypsurf] %
    - v_+ \integ[n]{x\sqrt{-g}}[\mathscr{M}_+] %
    - v_- \integ[n]{x\sqrt{-g}}[\mathscr{M}_-]. %
    %\underbrace{+ \frac{M\nped{Pl}^2}{2}\integ[n]{x\sqrt{-g}}[\mathscr{M}] \mathcal{R}}_{S\ped{EH}}.
\end{equation}
The first term is called the Nambu--Goto action $S\nped{NG}$, with $\sigma$ representing the constant positive energy density of the defect in its rest frame. The rest is the vacuum action $S\ped{vac}$ given through the constant potential energy densities $v_\pm$ of $\Manifold_\pm$. %
% We consider

Let us consider the variation of $S$ under small changes in the world sheet, $X\^\mu \to X\^\mu + \delta X\^\mu$. Since only transverse motion is physically relevant, we can write the variation in terms of the small, but otherwise arbitrary, function $\psi(\xi\^a)$;
\begin{equation}\label{eq:pertwalls:eom_wall:variation_x_normal_coord}
    X\^\mu \to X\^\mu + \psi n\^\mu.
    % \delta x\^\mu = \psi n\^\mu.
\end{equation}
% Under variation of $S$ with respect to tiny changes in $x^\mu(\xi\^a)$ on $\hypsurf$, 
% \begin{equation}
%     x\^\mu \to x\^\mu + \delta x\^\mu,
% \end{equation}
The equation of motion for the normal coordinate is then~\citep{ishibashiEquationMotionDomain1999,garrigaPerturbationsDomainWalls1991}
% \begin{equation}
%     \sq_\hypsurf x\^\mu  %
%     + \ChristoffelSym{\mu}{\kappa\tau} \gamma\^{ab} e\indices*{^\kappa_a}e\indices*{^\tau_b} %
%     +\frac{v}{\sigma} n\^\mu = 0,
% \end{equation}
\begin{equation}\label{eq:pertwalls:eom_wall:eom_defect}
    % n\_\mu\sq_\hypsurf x\^\mu  %
    n\_\mu \hypacc{{\sq}} X\^\mu 
    % + n\_\mu\ChristoffelSym{\mu}{\kappa\tau} \gamma\^{ab} e\indices*{^\kappa_a}e\indices*{^\tau_b} %
    + n\_\mu\ChristoffelSym{\mu}{\kappa\tau} \gamma\^{ab} X\indices{^\kappa_{,a}}X\indices{^\tau_{,b}} %
    +\frac{v}{\sigma} = 0,
\end{equation}
where $v=v_+-v_-$ is the difference in vacuum energies, %$\hypacc{\sq}$
% \footnote{
%     $\hypacc{\sq} = \hypacc{\nabla}\^a  \hypacc{\nabla}\_a  = \frac{1}{\sqrt{-\gamma}} \partial\_a (\sqrt{-\gamma} \gamma\^{ab}\partial\_b)$, $\gamma= \hypacc{g}$.
% } %
\begin{equation}
    \hypacc{\sq} = \hypacc{\nabla}\^a  \hypacc{\nabla}\_a  = \frac{1}{\sqrt{-\gamma}} \partial\_a (\sqrt{-\gamma} \gamma\^{ab}\partial\_b)
\end{equation}
is the d'Alembertian associated with $\hypsurf$, %
and $\ChristoffelSym{\mu}{\kappa\tau}$ are the spacetime Christoffel symbols. %The only physical observable is the normal component, 
%
This can be written in the simple form 
\begin{equation}\label{eq:pertwalls:eom_wall:eom_extrinic_curvature}
    \gamma\^{ab} \hypacc{K}\_{ab} = -\frac{v}{\sigma} ,
\end{equation}
where the extrinsic curvature tensor (\cref{eq:GR:diffgeo:extrinsic_curvature}) is $\hypacc{K}_{ab} = - e\indices*{^\mu _a}e\indices*{^\nu _b} \nabla\_\nu n\_\mu$.
% \begin{equation}
%     \hypacc{K}_{ab} = - e\indices*{^\mu _a}e\indices*{^\nu _b} \nabla\_\nu n\_\mu.
% \end{equation}



% This is where it gets a bit complicated. \blahblah To proceed, we let $x\^\mu = x_\bot n\^\mu$ be a solution of~\cref{eq:pertwalls:eom_wall:eom_extrinic_curvature}. When 
% we assume the solution $x_\bot (\tau)=z_0$



% A simple solution to~\cref{eq:pertwalls:eom_wall:eom_extrinic_curvature} is a 


% \subsubsection{Exact solutions}
    % \iftime{Mention Minkowski, $v=0$, strings, gauge invariance.}

    \paragraph{Minkowski.} %
    In~\citet{garrigaPerturbationsDomainWalls1991} it is shown that a planar domain wall oriented perpendicular to the $z$ axis in Minkowski space follows the trajectory $z=z(t)$ whose equation of motion is 
    \begin{equation}
        \frac{z_{,tt}}{{(1-z_{,t}^2)}^{3/2}} = \frac{v}{\sigma}.
    \end{equation}
    With $v=0$ the solution is $z(t)=0$ in a suitable Lorentz frame. %\iftime{Comment about constant velocity?}
    With non-degenerate vacua, the solution is the hyperbola~\citep{garrigaPerturbationsDomainWalls1991}.
    

    % \paragraph{Strings in expanding spacetime.} %
    % Fixing conformal gauge \dots
    % \iftime{Fill in.}





\subsection{Linearised perturbations}\label{sec:pertwalls:eom_wall:linearised_pert}
    Let $X\^\mu$ solve the equation of motion for the defect. We denote the perturbed solution $\pert{X}\^\mu$. Only transverse motion is physically observable, and so
    \begin{equation}\label{eq:pertwalls:eom_wall:linear_pert_wall_coord}
        \pert{X}\^\mu = X\^\mu + \delta X\^\mu = X\^\mu + \epsilon(\xi\^a) n\^\mu,%
        \footnote{Formally, this is the same as the variation above, but now we are dealing with a physical perturbation as opposed to a mathematical variation.}
    \end{equation}
    where the wall displacement variable $\epsilon$ is a linear perturbation ($\epsilon^2 \simeq 0$), and $n\^\mu$ is the unit-normal of the \emph{unperturbed} worldsheet~\citep{vilenkinCosmicStringsOther1994}. %
    To find the equation of motion for $\epsilon$ we apply~\cref{eq:pertwalls:eom_wall:linear_pert_wall_coord} to~\cref{eq:pertwalls:eom_wall:eom_extrinic_curvature} and solve order by order. If the ambient space is Minkowski, this amounts to~\citep{garrigaPerturbationsDomainWalls1991}
    \begin{equation}
        % -\hypacc{\sq} \epsilon + \pclosed{\hypacc{R}-v^2/\sigma^2}\epsilon  = 0.
        \hypacc{\sq} \epsilon + \hypacc{K}\^{ab}\hypacc{K}\_{ab} \epsilon = 0; \quad g\_{\mu\nu}= \eta\_{\mu\nu}.
    \end{equation}

    


    % \phpar[to linear pert.]
    % The traject
    % The simplest solution to~\cref{eq:pertwalls:eom_wall:eom_extrinic_curvature} is $z=z_0$, 
    % i.e. a flat domain wall in the $xy$-plane.



% \hlineSep





% \paragraph{Linearised perturbations.} %
% Let 


% To make this defect more defect, we could either solve the equation for a time-dependent normal coordinate \emph{or} add a small perturbation to the trivial solution and solve the perturbed equation. The latter is arguably much more trivial.



% We eventually want to consider a time-dependent surface tension, and therefore look at the simplified variations.

% \hlineSep



\subsection{Energy and momentum}
    
    From the Nambu--Goto action $S\nped{NG}$, we can construct an effective energy--momentum tensor associated with the defect
    % We find the domain wall (hypersurface in $(3+1)$ dimensions) 
    %Lagrangian density as $S\nped{NG} =\integ[4]{x} \mathcal{L}\nped{NG}$, allowing us to compute the associated SE tensor
    % \begin{equation}\label{eq:pertwalls:eom_wall:SE_tensor_NG}
    %     T\^{\mu\nu}\big|_{\text{\tiny{NG}}} = \frac{2}{\sqrt{-g}} \Fdv{\mathcal{L}_{\text{\tiny{NG}}}}{g\_{\mu\nu}}.
    % \end{equation}
    \begin{equation}\label{eq:pertwalls:eom_wall:SE_tensor_NG_general}
        T\^{\mu\nu}\big|\nped{NG} = \frac{2}{\sqrt{-g}} \Fdv{S\nped{NG}}{g\_{\mu\nu}}.
    \end{equation}
    % Assume the 
    Consider $X\^\mu= \xi\^a e\indices*{^\mu_a} + X_\bot n\^\mu$. 
    By rewriting the action to the spacetime integral
    % With the rewriting 
    \begin{equation}
        S\nped{NG} = - \sigma \integ[n]{x\sqrt{-\gamma}}[\Manifold] \Diracdelta( r); \quad r\equiv n\_\mu x\^\mu  - X_\bot,
    \end{equation}
    where the Dirac delta function essentially eliminates any spacetime event that is \emph{not} the defect. The variation $g\_{\mu\nu} \to g\_{\mu\nu}+ \delta g\_{\mu\nu}$ gives~\cite{vachaspatiKinksDomainWalls2006,vilenkinCosmicStringsOther1994}
    \begin{equation}
        T\^{\mu\nu}\big|\nped{NG} = \frac{\sigma}{\sqrt{-g}} \integ[d]{\xi} \sqrt{-\gamma} \gamma\^{ab} X\indices{^\mu_{,a}} X\indices{^\nu_{,b}} \Diracdelta(r ),
    \end{equation}
    % alternatively
    or equivalently
    \begin{equation}\label{eq:pertwalls:eom_wall:SE_tensor_NG}
        % {T\^{\mu\nu}}\rvert\nped{NG} 
        T\^{\mu\nu}\big|\nped{NG}  = \frac{\sigma \Diracdelta(r )}{\sqrt{-g}\sqrt{-\gamma}} \Fdv{\gamma}{g\_{\mu\nu}}.
    \end{equation}

    % \speak{Maybe mention \( {\diff \gamma} = \gamma \gamma\^{ab} {\diff \gamma_{ab}}  \)?}

    % We will consider $\Tw\^{\mu\nu} = T\^{\mu\nu}\rvert\nped{NG}$ to be the stress--energy 

    % \comment{Maybe define $T_{\mathrm{w}}^{\mu\nu}$ or $W\^{\mu\nu}$?}

    
    % \begin{equation}
    %     \Pi\_{ij} = T\_{ij}- p g\_{ij} = T\_{ij} + \frac{2}{3} \rho g\_{ij}
    % \end{equation}


    % \speak{%
    % I strongly believe $T\ap{TT}_{ij} = \Pi\ap{TT}_{ij}$. Maybe use ``wall displacement variable''?
    % %
    % }


    % \begin{equation}
    %     S\nped{NG} = -\sigma \integ[d]{\xi} \sqrt{-\gamma} \gamma\^{ab} X\indices{^\mu_{,a}} X\indices{^\nu_{,b}} \Diracdelta[n](x\^\rho - X\^\rho)
    % \end{equation}


    
    % We identify the Lagrangian by rewriting the Nambu-Goto action,
    % \begin{equation}
    %     S\nped{NG} =- \sigma \integ[3]{y \sqrt{-\gamma}} = -  \sigma \integ[4]{x \sqrt{-\gamma}} \Diracdelta[4](x\^\mu-X\^\mu),
    % \end{equation}
    % and perform the variation in~\cref{eq:pertwalls:eom_wall:SE_tensor_NG} to find an explicit expression for $T\^{\mu\nu}$.
    % The Lagrangian for the domain wall in $(3+1)$ dimensions is obtained via
    % \begin{equation}
    %     S\nped{NG} =\integ[4]{x} \mathcal{L}\nped{NG}= - \sigma \integ[3]{y \sqrt{-\gamma}} = - \integ[4]{x \sqrt{-g}}  \sigma \Diracdelta[4](x\^\mu-X\^\mu).
    % \end{equation}
    % The SE tensor becomes
    % %The SE tensor for the domain wall in $(3+1)$ is %obtained through the Nambu-Goto action $S\nped{NG} = \integ[4]{x} \mathcal{L}\nped{NG}$
    % \begin{equation}
    %     T\^{\mu\nu}\big|_{\text{\tiny{NG}}} = \frac{2}{\sqrt{-g}} \Fdv{\mathcal{L}_{\text{\tiny{NG}}}}{g\_{\mu\nu}} = ,
    % \end{equation}
    % where
    % \begin{equation}
    %     S\nped{NG} =\integ[4]{x} \mathcal{L}\nped{NG}= - \sigma \integ[3]{y \sqrt{-\gamma}} = - \integ[4]{x \sqrt{-g}}  \sigma \Diracdelta[4](x\^\mu-X\^\mu)
    % \end{equation}
    % The SE tensor for the domain wall is obtained through the Nambu-Goto action $S\nped{NG} = \integ[4]{x} \mathcal{L}\nped{NG}$
    % % % \begin{equation}
    %     T\^{\mu\nu}\big|_{\text{\tiny{NG}}} = \frac{2}{\sqrt{-g}} \Fdv{\mathcal{L}_{\text{\tiny{NG}}}}{g\^{\mu\nu}} 
    % \end{equation}



    % \subsubsection{\tmptitle{Thick domain walls}}
        
    %     To some extent, we can account for a possibly non-vanishing wall half-width $l\sim\delta$ by choosing a Gaussian function instead of a Dirac delta distribution. Simply substituting
    %     \begin{equation}
    %         \Diracdelta(r) \to   \varPhi_l (r) \equiv\frac{1}{\sqrt{2\ppi}l} \exp{ -\frac{r^2}{2l^2} }
    %     \end{equation}
    %     in~\cref{eq:pertwalls:eom_wall:SE_tensor_NG} does the trick, and restores $\Diracdelta(r)$ in the limit where $l\to 0$. This is understood in context with~\cref{sec:cosmo:defects:dws}, where we argue that $l=\delta/\sqrt{2}$ gives a suitable Gaussian profile. %, and we define 
    %     % \begin{equation}
    %     %     \varPhi_\delta (r) \
    %     % \end{equation}
    %     Note that this is not the same as going beyond the thin-wall limit, but rather an approximation that includes a thickness that is small enough to not alter the dynamics.

    %     Domain walls of cosmological relevance will have thickness much smaller than the horizon~\citep{garrigaPerturbationsDomainWalls1991}. However, some models will have instances in time when this is not true, and the thickness in fact varies. 
        % \phpar[meaning of thickness]


        % \hlineSep
        % In the case of $x\^a = \xi\^a$, this looks like
        % \begin{equation}
        %     \Diracdelta(x\^{\nu_\ast}  - X\^{\nu_\ast}) \to  \varPhi_l( x\^{\nu_\ast}  - X\^{\nu_\ast}) = \frac{1}{\sqrt{2\ppi}l} \exp{- \frac{ \pclosed{ x\^{\nu_\ast}  - X\^{\nu_\ast} }^2 }{2l^2} },
        % \end{equation}
        % where $\Diracdelta(x\^{\nu_\ast}  - X\^{\nu_\ast}) $ is retrieved by taking the limit %$\lim_{l\to 0}\varPhi$
        % $l\to 0$.

        % The domain wall Lagrangian is written
        % \begin{equation}
        %     \mathcal{L}\nped{NG} = -  \sigma\sqrt{-g}\, \varPhi_l(x\^{\nu_\ast}  - X\^{\nu_\ast}),
        % \end{equation}
        % so that we would have the SE tensor
        % \begin{equation}
        %     \begin{split}
        %         T\^{\mu\nu} \big|\nped{NG} &= - \frac{2\sigma \,\varPhi_l(x\^{\nu_\ast}  - X\^{\nu_\ast})}{\sqrt{-g}} \Fdv{\sqrt{-\gamma}}{g\_{\mu\nu}} \\
        %         &= \frac{\sigma \,\varPhi_l(x\^{\nu_\ast}  - X\^{\nu_\ast})}{\sqrt{-g}\sqrt{-\gamma}} \Fdv{\gamma}{g\_{\mu\nu}}.
        %     \end{split}
        % \end{equation}


       


