
%%%%%%%%%%%%%%%%%%%%%%%%%%%%%%%%%%%%%%%%%%%%%%%%%%%%%%%%%%
%%%%%%% Ch. 2: Classical Field Theory and Gravity  %%%%%%%
%%%%%%%%%%%%%%%%%%%%%%%%%%%%%%%%%%%%%%%%%%%%%%%%%%%%%%%%%%


% -------------------------------------
% labels: \label{[type]:CFTgrav:[name]}
% -------------------------------------



% ////////////////// intro //////////////////


Alongside quantum mechanics, Einstein's theory of gravity---general relativity (GR)---is widely accepted as the most accurate description of our surroundings. GR can be formulated from a geometrical point of view, or it can be viewed as a classical field theory. In the former approach we meet geometrical tools such as the geodesic equation, whereas the latter allows the application of field-theoretical methods. This chapter lays emphasis on the field interpretation of GR. 

\phpar[Two perspectives insightful; better overall understanding of aspects of concepts in GR]


% ///////////////////////////////////////////



\section{General relativity}\label{sec:CFTgrav:GR}
    {\subimport{./}{GR.tex}}



% \section{\tmptitle{Energy-momentum density}}




\section{\tmptitle{Conformal field theory}}
    {\subimport{./}{conformal_trafos.tex}}

% \section{\tmptitle{Scalar--Tensor Theories (maybe subsec. of prev.)}}






\section{Perturbation theory}\label{sec:CFTgrav:pert}


\section{Classical solitons}\label{sec:CFTgrav:solitons}
    {
        \begin{bullets}
            \item topological defects \begin{itemize}
                \item System of kinks: $\phi(x) = \phi_\infty^{1-(N+M)} \prod_{i=1}^{N} \phi\ped{k}(x-k_i) \prod_{j=1}^{M} \bar{\phi}\ped{k}(x-\bar{k}_j)$ \citep{vachaspatiKinksDomainWalls2006}
            \end{itemize}
            \item basic properties?
        \end{bullets}
    }










\clearpage
\newpage
% \section{Chameleon models}\label{sec:CFTgrav:chameleon}
%     {\subimport{./}{chameleon.tex}}

\section{Symmetron model}\label{sec:CFTgrav:symmetron}
    {\subimport{./}{symmetron.tex}}




