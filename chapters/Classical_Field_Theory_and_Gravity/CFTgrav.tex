
%%%%%%%%%%%%%%%%%%%%%%%%%%%%%%%%%%%%%%%%%%%%%%%%%%%%%%%%%%
%%%%%%% Ch. 2: Classical Field Theory and Gravity  %%%%%%%
%%%%%%%%%%%%%%%%%%%%%%%%%%%%%%%%%%%%%%%%%%%%%%%%%%%%%%%%%%


% -------------------------------------
% labels: \label{[type]:CFTgrav:[name]}
% -------------------------------------



% ////////////////// intro //////////////////


Alongside quantum mechanics, Einstein's theory of gravity---general relativity (GR)---is widely accepted as the most accurate description of our surroundings. GR can be formulated from a geometrical point of view, or it can be viewed as a classical field theory. In the former approach we meet geometrical tools such as the geodesic equation, whereas the latter allows the application of field-theoretical methods. This chapter lays emphasis on the field interpretation of GR. 

\phpar[Two perspectives insightful; better overall understanding of aspects of concepts in GR]


% ///////////////////////////////////////////



\section{General Relativity}\label[sec]{sec:CFTgrav:GR}
    {\subimport{./}{GR.tex}}



\section{\tmptitle{Conformal Field Theory}}
    {\subimport{./}{conformal_trafos.tex}}

% \section{\tmptitle{Scalar--Tensor Theories (maybe subsec. of prev.)}}


\section{Perturbation Theory}\label[sec]{sec:CFTgrav:pert}


\section{Classical Solitons}\label[sec]{sec:CFTgrav:solitons}







