% ||||||||||||||||||||||||||||||||||||
% |||||| 2.1 General Relativity ||||||
% ||||||||||||||||||||||||||||||||||||

% ----------------------------------------
% labels: \label{[type]:CFTgrav:GR:[name]}
% ----------------------------------------


The Einstein--Hilbert action in vacuum is \comment{check Planck mass def.}
\begin{equation}
    S\ped{EH} = \shalf \Planck{M}^2 \int \! \diff[4]x \,\sqrt{-\abs{g}}\,  \mathcal{R}, 
\end{equation}
where $\mathcal{R} = g\up{\mu\nu}\mathcal{R}\lo{\mu\nu}$. By varying $S\ped{EH}$ with respect to $g\lo{\mu\nu}$ one obtains the equation of motion
\begin{equation}
    \mathcal{G}\lo{\mu\nu} \equiv \mathcal{R}\lo{\mu\nu} + \frac{1}{2} g\lo{\mu\nu} \mathcal{R} = 0.
\end{equation}
Thus, we interpret GR as a \emph{classical} field theory where the tensor field $g\lo{\mu\nu}$ is the gravitational field, \cringe{with the particle realisation named ``graviton''}.



\begin{bullets}
    \item scalar field (ST theories) 
\end{bullets}

