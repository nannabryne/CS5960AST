% |||||||||||||||||||||||||||||||||||||||||||||||
% |||||| 6.1 Dynamics of Thin Domain Walls ||||||
% |||||||||||||||||||||||||||||||||||||||||||||||

% ------------------------------------------------
% labels: \label{[type]:pertwalls:thinwall:[name]}
% ------------------------------------------------



% ------------------------------------------------
\newcommand*\hypsurf{\ensuremath{\varSigma}}    % hypersurface
% ------------------------------------------------




\phpar[constant surface tension]



We follow~\citet{garrigaPerturbationsDomainWalls1991} and~\citet{ishibashiEquationMotionDomain1999}. The world sheet $\hypsurf$ divides \Manifold~into two submanifolds $\Manifold_{\pm}$ such that $\mathscr{M} = \mathscr{M}_+ \cup  \hypsurf \cup \mathscr{M}_-$. That is to say, a domain wall holds a world sheet separating two vacua. We take \Manifold~ to be smooth and $(N+1)$-dimensional, and let $\hypsurf$ be a smooth also and $((N-1)+1)$-dimensional. Consequently, $\hypsurf$ is a timelike hypersurface in \Manifold. 


%
% \citep{ishibashiEquationMotionDomain1999,garrigaPerturbationsDomainWalls1991}


\begin{bullets}
    \item Vary DW action
    \item Goal: E.O.M. for physically relevant component (epsilon basically)
    \item Expression for energy--momentum tensor
    \item Extension to non-thin walls 
    \item Extension to Asymmetron or introduction of energy bias
    \item What does thin mean? Why is the tension indep. of width?
\end{bullets}


% The generalisation to $(N+1)$ dimensions is straight-forward.
We invoke a smooth coordinate system $\{x\^\mu\}$ ($\mu=0,1,\dots,N$) of the spacetime $(\Manifold, g\_{\mu\nu})$ in a neighbourhood of $\hypsurf$. The embedding of $\hypsurf$ in $\Manifold$ is $x\^\mu = x\^\mu(y\^a)$, where the coordinate system $\{y\^a\}$ ($a=0,1,\dots,N-1$) parametrises $\hypsurf$.
The induced metric on $\hypsurf$ is
\begin{equation}\label{eq:pertwalls:thinwall:induced_metrid}
    h\_{ab} = g\_{\mu\nu} e^\mu_a e^\nu_b; \quad e^\mu_a \equiv \pdv{x\^\mu}{y\^a}% g\_{\mu\nu}\pdv{x\^\mu}{y\^a}\pdv{x\^\nu}{y\^b} 
\end{equation} \provethis{argue!}


The action for a thin domain wall is famously~\citep[e.g.][]{vachaspatiKinksDomainWalls2006} the Nambu-Goto action
\begin{equation}
    S\ped{dw} = -\sigma \integ[N]{y\sqrt{-h}}[\hypsurf],
\end{equation}
where $\sigma$ is the wall's energy per unit area, henceforth called ``surface tension''.











%%%%%%%%%%%%%%%%%%%%%%%%%%%%%%%%%%%%%%%%%%%%%%%%%%%%%%%%%%%%%%%%%%%%%%%%%%%%
%%%%%%%%%%%%%%%%%%%%%%%%%%%%%%%%%%%%%%%%%%%%%%%%%%%%%%%%%%%%%%%%%%%%%%%%%%%%
\begin{draft}
\subsection{Expanding universe. (my scenario)}




\end{draft}
%%%%%%%%%%%%%%%%%%%%%%%%%%%%%%%%%%%%%%%%%%%%%%%%%%%%%%%%%%%%%%%%%%%%%%%%%%%%
%%%%%%%%%%%%%%%%%%%%%%%%%%%%%%%%%%%%%%%%%%%%%%%%%%%%%%%%%%%%%%%%%%%%%%%%%%%%
