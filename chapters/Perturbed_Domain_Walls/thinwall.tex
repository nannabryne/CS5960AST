% |||||||||||||||||||||||||||||||||||||||||||||||
% |||||| 6.1 Dynamics of Thin Domain Walls ||||||
% |||||||||||||||||||||||||||||||||||||||||||||||

% ------------------------------------------------
% labels: \label{[type]:pertwalls:thinwall:[name]}
% ------------------------------------------------



% ------------------------------------------------
% \newcommand*\hypsurf{\ensuremath{\varSigma}}    % hypersurface
% ------------------------------------------------






% \phpar[constant surface tension]



% We follow~\citet{garrigaPerturbationsDomainWalls1991} and~\citet{ishibashiEquationMotionDomain1999}. The world sheet $\hypsurf$ divides \Manifold~into two submanifolds $\Manifold_{\pm}$ such that $\mathscr{M} = \mathscr{M}_+ \cup  \hypsurf \cup \mathscr{M}_-$. That is to say, a domain wall holds a world sheet separating two vacua. We take \Manifold~ to be smooth and $(N+1)$-dimensional, and let $\hypsurf$ be a smooth also and $((N-1)+1)$-dimensional. Consequently, $\hypsurf$ is a timelike hypersurface in \Manifold. 


% %
% % \citep{ishibashiEquationMotionDomain1999,garrigaPerturbationsDomainWalls1991}


% \begin{bullets}
%     \item Vary DW action
%     \item Goal: E.O.M. for physically relevant component (epsilon basically)
%     \item Expression for energy--momentum tensor
%     \item Extension to non-thin walls 
%     \item Extension to Asymmetron or introduction of energy bias
%     \item What does thin mean? Why is the tension indep. of width?
% \end{bullets}


% % The generalisation to $(N+1)$ dimensions is straight-forward.
% We invoke a smooth coordinate system $\{x\^\mu\}$ ($\mu=0,1,\dots,N$) of the spacetime $(\Manifold, g\_{\mu\nu})$ in a neighbourhood of $\hypsurf$. The embedding of $\hypsurf$ in $\Manifold$ is $x\^\mu = x\^\mu(y\^a)$, where the coordinate system $\{y\^a\}$ ($a=0,1,\dots,N-1$) parametrises $\hypsurf$.
% The induced metric on $\hypsurf$ is
% \begin{equation}\label{eq:pertwalls:thinwall:induced_metrid}
%     q\_{ab} = g\_{\mu\nu} e^\mu_a e^\nu_b; \quad e^\mu_a \equiv \pdv{x\^\mu}{y\^a}% g\_{\mu\nu}\pdv{x\^\mu}{y\^a}\pdv{x\^\nu}{y\^b} 
% \end{equation} \provethis{argue!}

% We let $\sigma$ represent the surface energy density of the wall---a quantity we will discuss i much more detail later---and $v_\pm$ the vacuum energy densities of $\mathscr{M}_\pm$. The complete action of the coupled system is
% \begin{equation}
%     S = \underbrace{- \sigma \integ[N]{y\sqrt{-q}}[\hypsurf] }_{S\ped{NG}} %
%     - v_+ \integ[N+1]{x\sqrt{-g}}[\mathscr{M}_+] %
%     - v_- \integ[N+1]{x\sqrt{-g}}[\mathscr{M}_-] %
%     \underbrace{+ \frac{M\nped{Pl}^2}{2}\integ[N+1]{x\sqrt{-g}}[\mathscr{M}] \mathcal{R}}_{S\ped{EH}}.
% \end{equation}
% \comment{Comment about Nambu-Goto action. Maybe add matter and $\phi$ actions?}

% % The action for a thin domain wall is famously~\citep[e.g.][]{vachaspatiKinksDomainWalls2006} the Nambu-Goto action $S\ped{NG}$, 
% % \begin{equation}
% %     S\ped{dw} = -\sigma \integ[N]{y\sqrt{-h}}[\hypsurf],
% % \end{equation}
% % where $\sigma$ is the wall's energy per unit area, henceforth called ``surface tension''. The action for the coupled system


% Under small changes in $x\^\mu$ on $\hypsurf$, $x^\mu \to x\^\mu + \delta x\^\mu$, we obtain the equation
% \begin{equation}
%     \mathrm{D}\^a e\indices*{^\mu_a} + \ChristophelSym{\mu}{\kappa\tau} q\^{ab} e\indices*{^\kappa_a}  e\indices*{^\tau_b} +\frac{v_+-v_-}{\sigma} n\^\mu = 0,
% \end{equation}
% or equivalently,
% \begin{equation}
%     % \sq_\hypsurf %
%     \mathrm{D}\_a \mathrm{D}\^a %
%     x\^\mu+ \ChristophelSym{\mu}{\kappa\tau} q\^{ab} \pdv{x^\kappa}{y\^a} \pdv{x^\tau}{y\^b}  +\frac{v_+-v_-}{\sigma} n\^\mu = 0,
% \end{equation}
% where $\mathrm{D}\_a$ is the covariant derivative with respect to $q\_{ab}$.

% \checkthis{The part of $\delta x\^\mu$ that is tangential to $\hypsurf$ are diffeomorphisms on $\hypsurf$ ($y\^a \to y\^a + \delta y\^a$).} The only physically meaningful component is the transverse one; %Let us write $x\^\mu = $
% \begin{equation}
%    n\_\mu\mathrm{D}\^a \mathrm{D}\_a x\^\mu + n\_\mu \ChristophelSym{\mu}{\kappa\tau} q\^{ab} e\indices*{^\kappa_a}  e\indices*{^\tau_b} +\frac{\Delta v}{\sigma} = 0.
% \end{equation}


% Without loss of generality we let \dots $n\^\mu = n\^N$ \blahblah


% \subsection{Planar domain walls in FLRW spacetime}





% %%%%%%%%%%%%%%%%%%%%%%%%%%%%%%%%%%%%%%%%%%%%%%%%%%%%%%%%%%%%%%%%%%%%%%%%%%%%
% %%%%%%%%%%%%%%%%%%%%%%%%%%%%%%%%%%%%%%%%%%%%%%%%%%%%%%%%%%%%%%%%%%%%%%%%%%%%
% % \begin{draft}
% % \subsection{Expanding universe. (my scenario)}




% % \end{draft}
% %%%%%%%%%%%%%%%%%%%%%%%%%%%%%%%%%%%%%%%%%%%%%%%%%%%%%%%%%%%%%%%%%%%%%%%%%%%%
% %%%%%%%%%%%%%%%%%%%%%%%%%%%%%%%%%%%%%%%%%%%%%%%%%%%%%%%%%%%%%%%%%%%%%%%%%%%%



% % \section{Stress--energy tensor of domain walls}





% Viewed as a hypersurface in $(3+1)$ dimensions, we align a domain wall with the first three dimensions of the

Now the ground work is laid for the scenario that which this project i all about. That is, we consider a conformally flat spacetime with %metric $g\_{\mu\nu}= a^2 \eta\_{\mu\nu}$ 
line element
\begin{equation}
    {ds}^2 = g\_{\mu\nu} \diff x\^\mu \diff x\^\nu = a^2 \eta\_{\mu\nu} \diff x\^\mu \diff x\^\nu = a^2(\tau) \cclosed{ -{\diff \tau}^2 + {\diff x}^2 + {\diff y}^2 + {\diff z}^2  }.
\end{equation}
We place a thin domain wall at $z$-coordinate $z_0$, represented by a hypersurface $\hypsurf$, whose induced metric is $q\_{ab}= g\_{\mu\nu} x\indices{^\mu_{,a}} x\indices{^\nu_{,b}}$. Now, $z = z_0 + \epsilon(y\^a)$ and 
\begin{equation}
    \mathrm{D}^2 \epsilon + \ChristophelSym{3}{\kappa\tau} q\^{ab} \pclosed{ \Krondelta{^\kappa_a}\Krondelta{^\tau_b} + 2\Krondelta{^\kappa_a} \Krondelta{^\tau_3} \epsilon\_{,b}  +  \Krondelta{^\kappa_3} \Krondelta{^\tau_3} \epsilon\_{,a}   \epsilon\_{,b}    } + \frac{\Delta v}{\sigma} n\^\mu= 0
\end{equation}
\blahblah \comment{Set $\Delta v=0$ etc.}

Eventually, we arrive at
\begin{equation}
    \ddot{\epsilon} + 3\mathcal{H}\dot{\epsilon} -  \pclosed{\partial_x^2 +\partial_y^2}\epsilon= 0  %\bclosed{ \pdv[2]{}{x}  + \pdv[2]{}{y}  } \epsilon = 0.
\end{equation}
With the ansatz $\epsilon=\epsilon_p(\tau) \sppt(x,y)$ we can solve this in terms of eigenvalues $-p^2$,
\begin{equation}\label[eq]{eq:pertwalls:untitled1:eom_eps_s_MD_simple}
    \ddot{\epsilon} + 3\mathcal{H}\dot{\epsilon}+ p^2 \epsilon = 0,% \pclosed{\partial_x^2 +\partial_y^2} \\sppt(x,y) = -p^2 \\sppt(x,y) 
\end{equation}
where $(\partial_x^2 +\partial_y^2) \sppt(x,y) = -p^2 \sppt(x,y) $. 
For a universe with $a\propto \tau^\alpha$, the solution to this equation is of the form $\epsilon_p(\tau) = x^\gamma \Sylindrical[\gamma](p\tau)$, $\gamma=(1-3\alpha)/2$.


\subsection{Stress-energy tensor}
    We perform the variation in \nc{Eq. XX}[of q]. The detailed calculation can be found in \nc{appendix X}[app. with derivations]. The non-vanishing components are
    \begin{equation}
        \begin{split}
            T\_{ab}(\tau, \vec{x})  &=  -a \sigma \, \varPhi(z-z\ped{dw}) \, \eta\_{ab} \\
            T\_{(i3)}(\tau, \vec{x})  &= -a \sigma \,\varPhi(z-z\ped{dw} ) \,\epsilon\_{,i} %\partial\_{i} \epsilon
        \end{split}
    \end{equation}
    where $a,b=0,1,2$.



\subsection{Non-constant surface tension}
    Until now, these equations are model-independent, given constant surface tension ($\dot{\sigma}= 0$) and no energy bias ($v_+ - v_- = 0$). 
    If we allow the surface tension to vary, $\sigma=\sigma(\tau)$, %$\stackrel{\tau\to \infty}{\to} \sigma_\infty$
    we need to put this inside of the integral in the Nambu-Goto action. We immidiately see that this is equivalent to letting $a^3\to a^3 \sigma$, and with the substitution 
    \begin{equation}
        3\mathcal{H} =  \frac{3}{a} \dv{a}{\tau} \to \frac{3}{a\sigma^{1/3}} \dv{a\sigma^{1/3}}{\tau} =  3\frac{\dot{a}}{a} +  \frac{\dot{\sigma}}{\sigma} = 3\dot{\ln{a}} + \dot{\ln{\sigma}}
        % 3\mathcal{H} =  3 \dot{a}/a \to  3 \pclosed{a\sigma^{1/3}}^{-1}\dv{a\sigma^{1/3}}{\tau} =  3 \dot{a}/a  + \dot{\sigma}/\sigma
    \end{equation}
    % %This is equivalent to substituting $a\to a \sigma^{1/3}$, so that we only need to change $\mathcal{H} \to \mathcal{H}_\Upsilon$ \dots 
    % If we define $\mathcal{H}_\Upsilon = \Upsilon^{-1} \dot{\Upsilon} $, we see that with $\Upsilon=a \sigma^{1/3}$, we get
    % \begin{equation}
    %     % \ddot{\epsilon} + \pclosed{ 3\frac{\dot{a}}{a} + \frac{\dot{\sigma}}{\sigma} } \dot{\epsilon} + k^2 \epsilon = 0.
    %     \ddot{\epsilon} + \pclosed{ 3\mathcal{H}_a + \mathcal{H}_\sigma  } \dot{\epsilon} + k^2 \epsilon = 0.
    % \end{equation}
    % If we write
    % \begin{equation}
    %     \ddot{\epsilon} + 3\hat{\mathcal{H}}\dot{\epsilon} - \vec{\nabla}^2 \epsilon = 0,
    % \end{equation}
    % and let 
    in \nc{Eq. XXX}, we get
    \begin{equation}
        \ddot{\epsilon} + \pclosed{3\dot{a}/a + \dot{\sigma}/\sigma}\dot{\epsilon} +p^2 \epsilon = 0.
    \end{equation}
    This extra term will introduce the model-dependence, seeing as the surface tension in the thin-wall limit is given by
    \begin{equation}
        \sigma = \integ{\phi}[\phi_-][\phi_+] \sqrt{ 2V\ped{eff}(\phi)-2V\ped{eff}(\phi_\pm) },
    \end{equation}
    where $V\ped{eff}$ is the effective potential of the theory. 

