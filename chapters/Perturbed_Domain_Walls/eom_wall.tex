% |||||||||||||||||||||||||||||||||||
% |||||| 5.1 General framework ||||||
% |||||||||||||||||||||||||||||||||||

% ------------------------------------------------
% labels: \label{[type]:pertwalls:eom_wall:[name]}
% ------------------------------------------------





We follow~\citet{garrigaPerturbationsDomainWalls1991} and~\citet{ishibashiEquationMotionDomain1999}. The world sheet $\hypsurf$ divides \Manifold~into two submanifolds $\Manifold_{\pm}$ such that $\mathscr{M} = \mathscr{M}_+ \cup  \hypsurf \cup \mathscr{M}_-$. That is to say, a domain wall holds a world sheet separating two vacua. We take \Manifold~ to be smooth and $(N+1)$-dimensional, and let $\hypsurf$ be a smooth also and $((N-1)+1)$-dimensional. Consequently, $\hypsurf$ is a timelike hypersurface in \Manifold. 


%
% \citep{ishibashiEquationMotionDomain1999,garrigaPerturbationsDomainWalls1991}


\begin{bullets}
    \item Vary DW action
    \item Goal: E.O.M. for physically relevant component (epsilon basically)
    \item Expression for energy--momentum tensor
    \item Extension to non-thin walls 
    \item Extension to Asymmetron or introduction of energy bias
    \item What does thin mean? Why is the tension indep. of width?
\end{bullets}


% The generalisation to $(N+1)$ dimensions is straight-forward.
We invoke a smooth coordinate system $\{x\^\mu\}$ ($\mu=0,1,\dots,N$) of the spacetime $(\Manifold, g\_{\mu\nu})$ in a neighbourhood of $\hypsurf$. The embedding of $\hypsurf$ in $\Manifold$ is $x\^\mu = x\^\mu(y\^a)$, where the coordinate system $\{y\^a\}$ ($a=0,1,\dots,N-1$) parametrises $\hypsurf$.
The induced metric on $\hypsurf$ is
\begin{equation}\label{eq:pertwalls:thinwall:induced_metrid}
    q\_{ab} = g\_{\mu\nu} e^\mu_a e^\nu_b; \quad e^\mu_a \equiv \pdv{x\^\mu}{y\^a}% g\_{\mu\nu}\pdv{x\^\mu}{y\^a}\pdv{x\^\nu}{y\^b} 
\end{equation} \provethis{argue!}

We let $\sigma$ represent the surface energy density of the wall---a quantity we will discuss i much more detail later---and $v_\pm$ the vacuum energy densities of $\mathscr{M}_\pm$. The complete action of the coupled system is
\begin{equation}
    S = \underbrace{- \sigma \integ[N]{y\sqrt{-q}}[\hypsurf] }_{S\ped{NG}} %
    - v_+ \integ[N+1]{x\sqrt{-g}}[\mathscr{M}_+] %
    - v_- \integ[N+1]{x\sqrt{-g}}[\mathscr{M}_-] %
    \underbrace{+ \frac{M\nped{Pl}^2}{2}\integ[N+1]{x\sqrt{-g}}[\mathscr{M}] \mathcal{R}}_{S\ped{EH}}.
\end{equation}
\comment{Comment about Nambu-Goto action. Maybe add matter and $\phi$ actions?}

% The action for a thin domain wall is famously~\citep[e.g.][]{vachaspatiKinksDomainWalls2006} the Nambu-Goto action $S\ped{NG}$, 
% \begin{equation}
%     S\ped{dw} = -\sigma \integ[N]{y\sqrt{-h}}[\hypsurf],
% \end{equation}
% where $\sigma$ is the wall's energy per unit area, henceforth called ``surface tension''. The action for the coupled system


Under small changes in $x\^\mu$ on $\hypsurf$, $x^\mu \to x\^\mu + \delta x\^\mu$, we obtain the equation
\begin{equation}
    \mathrm{D}\^a e\indices*{^\mu_a} + \ChristophelSym{\mu}{\kappa\tau} q\^{ab} e\indices*{^\kappa_a}  e\indices*{^\tau_b} +\frac{v_+-v_-}{\sigma} n\^\mu = 0,
\end{equation}
or equivalently,
\begin{equation}
    % \sq_\hypsurf %
    \mathrm{D}\_a \mathrm{D}\^a %
    x\^\mu+ \ChristophelSym{\mu}{\kappa\tau} q\^{ab} \pdv{x^\kappa}{y\^a} \pdv{x^\tau}{y\^b}  +\frac{v_+-v_-}{\sigma} n\^\mu = 0,
\end{equation}
where $\mathrm{D}\_a$ is the covariant derivative with respect to $q\_{ab}$.

\checkthis{The part of $\delta x\^\mu$ that is tangential to $\hypsurf$ are diffeomorphisms on $\hypsurf$ ($y\^a \to y\^a + \delta y\^a$).} The only physically meaningful component is the transverse one; %Let us write $x\^\mu = $
\begin{equation}
   n\_\mu\mathrm{D}\^a \mathrm{D}\_a x\^\mu + n\_\mu \ChristophelSym{\mu}{\kappa\tau} q\^{ab} e\indices*{^\kappa_a}  e\indices*{^\tau_b} +\frac{\Delta v}{\sigma} = 0.
\end{equation}


Without loss of generality, we may align $\hypsurf$ with e.g.~the first $N-1$ dimensions of $\Manifold$, i.e.~$e\indices*{^{\mu}_{a}}=\Krondelta*{^\mu_a} + \Krondelta*{^\mu_{\nu_\ast}}\epsilon\_{,a}$ and $n\^\mu = \Krondelta*{^\mu_{\nu_\ast}}$, with $\nu_\ast = N$. We let $x\^{\mu} = \Krondelta*{^\mu_a} y\^a + \Krondelta*{^\mu_{\nu_\ast}} (\epsilon(y\^a) + \zeta ) $ be the embedding function, where $\zeta$ is the $\nu_\ast$-coordinate of $\hypsurf$ in $\Manifold$. Now,
\begin{equation}
    % \mathrm{D}^2 x\^{\nu_\ast} + \ChristophelSym{{\nu_\ast}}{\kappa\tau} q\^{ab} e\indices*{^\kappa_a}  e\indices*{^\tau_b} +\frac{\Delta v}{\sigma} = 0.
    \mathrm{D}^2 \epsilon + \ChristophelSym{{\nu_\ast}}{\kappa\tau} q\^{ab} \Krondelta{^\kappa_a} \Krondelta{^\tau_b} +\frac{\Delta v}{\sigma} = 0.
 \end{equation}
 \important{FIX ME!!! }

\comment{Will remove several of these equations.}
% Without loss of generality we let \dots $n\^\mu = n\^N$ \blahblah

Wall position $X\^\mu = X_0^\mu + \epsilon N\^{\mu}=\Krondelta*{^\mu_a}y\^a + \Krondelta*{^\mu_{\nu_\ast}} \pclosed{\zeta  +...}$


\important{NB! This is ``general'' for topological defects, should call $\sigma$ somthing different.}

\subsection{Energy and momentum}
    We find the domain wall (hypersurface in $(3+1)$ dimensions) Lagrangian as $S\nped{NG} =\integ[4]{x} \mathcal{L}\nped{NG}$, allowing us to compute the associated SE tensor
    \begin{equation}
        T\^{\mu\nu}\big|_{\text{\tiny{NG}}} = \frac{2}{\sqrt{-g}} \Fdv{\mathcal{L}_{\text{\tiny{NG}}}}{g\_{\mu\nu}}.
    \end{equation}
    We identify the Lagrangian by rewriting the Nambu-Goto action,
    \begin{equation}
        S\nped{NG} =- \sigma \integ[3]{y \sqrt{-q}} = -  \sigma \integ[4]{x \sqrt{-q}} \Diracdelta[4](x\^\mu-X\^\mu),
    \end{equation}
    and perform the variation in \nc{Eq.XX} to find an explicit expression for $T\^{\mu\nu}$.
    % The Lagrangian for the domain wall in $(3+1)$ dimensions is obtained via
    % \begin{equation}
    %     S\nped{NG} =\integ[4]{x} \mathcal{L}\nped{NG}= - \sigma \integ[3]{y \sqrt{-q}} = - \integ[4]{x \sqrt{-g}}  \sigma \Diracdelta[4](x\^\mu-X\^\mu).
    % \end{equation}
    % The SE tensor becomes
    % %The SE tensor for the domain wall in $(3+1)$ is %obtained through the Nambu-Goto action $S\nped{NG} = \integ[4]{x} \mathcal{L}\nped{NG}$
    % \begin{equation}
    %     T\^{\mu\nu}\big|_{\text{\tiny{NG}}} = \frac{2}{\sqrt{-g}} \Fdv{\mathcal{L}_{\text{\tiny{NG}}}}{g\_{\mu\nu}} = ,
    % \end{equation}
    % where
    % \begin{equation}
    %     S\nped{NG} =\integ[4]{x} \mathcal{L}\nped{NG}= - \sigma \integ[3]{y \sqrt{-q}} = - \integ[4]{x \sqrt{-g}}  \sigma \Diracdelta[4](x\^\mu-X\^\mu)
    % \end{equation}
    % The SE tensor for the domain wall is obtained through the Nambu-Goto action $S\nped{NG} = \integ[4]{x} \mathcal{L}\nped{NG}$
    % % % \begin{equation}
    %     T\^{\mu\nu}\big|_{\text{\tiny{NG}}} = \frac{2}{\sqrt{-g}} \Fdv{\mathcal{L}_{\text{\tiny{NG}}}}{g\^{\mu\nu}} 
    % \end{equation}



    \subsubsection{Thickness?}
        To some extent, we can account for a possibly non-vanishing wall width $l$ by choosing a Gaussian funciton instead of a Dirac-Delta distrubution. In the case of $x\^a = y\^a$, this looks like
        \begin{equation}
            \Diracdelta(x\^{\nu_\ast}  - X\^{\nu_\ast}) \to  \varPhi_l( x\^{\nu_\ast}  - X\^{\nu_\ast}) = \frac{1}{\sqrt{2\ppi}l} \exp{- \frac{ \pclosed{ x\^{\nu_\ast}  - X\^{\nu_\ast} }^2 }{2l^2} },
        \end{equation}
        where $\Diracdelta(x\^{\nu_\ast}  - X\^{\nu_\ast}) $ is retrieved by taking the limit %$\lim_{l\to 0}\varPhi$
        $l\to 0$.

        The domain wall Lagrangian is written
        \begin{equation}
            \mathcal{L}\nped{NG} = -  \sigma\sqrt{-g}\, \varPhi_l(x\^{\nu_\ast}  - X\^{\nu_\ast}),
        \end{equation}
        so that we would have the SE tensor
        \begin{equation}
            \begin{split}
                T\^{\mu\nu} \big|\nped{NG} &= - \frac{2\sigma \,\varPhi_l(x\^{\nu_\ast}  - X\^{\nu_\ast})}{\sqrt{-g}} \Fdv{\sqrt{-q}}{g\_{\mu\nu}} \\
                &= \frac{\sigma \,\varPhi_l(x\^{\nu_\ast}  - X\^{\nu_\ast})}{\sqrt{-g}\sqrt{-q}} \Fdv{q}{g\_{\mu\nu}}.
            \end{split}
        \end{equation}
        


