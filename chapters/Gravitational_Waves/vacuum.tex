% |||||||||||||||||||||||||||||||||||||||||||||||
% |||||| 3.2 Gravitational Waves in Vacuum ||||||
% |||||||||||||||||||||||||||||||||||||||||||||||

% ----------------------------------------
% labels: \label{[type]:GWs:vacuum:[name]}
% ----------------------------------------



%%%%%%%%%%%%%%%%%%%%%%%%%%%%%%%%%%%%%%%%%%%%%%%%%%%%%%%%%%%%%%%%
% \newcommand*\ah{\bar{h}}
%%%%%%%%%%%%%%%%%%%%%%%%%%%%%%%%%%%%%%%%%%%%%%%%%%%%%%%%%%%%%%%%




In an expanding, flat universe, we choose the TT gauge such that $h\indices{^i_i}=0$ and $h\indices{_{ij,j}}=0$. The unsourced equation reads
\begin{equation}
    \sq h_P(x) = 0 \Rightarrow \ddot{h}_P (x) + 2 \mathcal{H}\dot{h}_P(x) - \vec{\nabla}^2 h_P(x) = 0,
\end{equation}
and thus describes a damped harmonic oscillator. 
We specify $a\propto \tau^\alpha$ and $\alpha\in \Integer$ and turn to Fourier space,
\begin{equation}
    \ddot{\ah}_P(\tau, \vec{k}) + 2\frac{\alpha}{\tau} \dot{h}_P(\tau, \vec{k})  + k^2 h_P(\tau, \vec{k})  =0.
\end{equation}
In defining $\ah_P\equiv ah_P$ and $u\equiv k\tau$ we can recognise the convenient form
\begin{equation}
    \dv[2]{\ah_P}{u} + \bclosed{1 - \frac{\alpha(\alpha - 1)}{u^2}} \ah_P  = 0
\end{equation}
whose general solution is a linear combination of the first and second order Riccati-Bessel functions, $\ah_P(\tau, \vec{k}) = A_P(k) \RiccatiBessel[n][1](k\tau) + B_P(k)\RiccatiBessel[n][2](k\tau)$ where $n=\alpha-1$.
% We follow~\cref{sec:CFTgrav:conformal_trafos:anal_sols} to find the general solution for this equation in Fourier space. We specify $\alpha \in \Integer$ and get
% \begin{equation}
%     h_P(\tau, \vec{k}) = \frac{A_k \sphBessel[\abs{\alpha}](k\tau) + B_k \sphBessel[-\abs{\alpha}](k\tau)}{\tau^{\alpha-1}}; \quad k= \abs{\vec{k}}.
% \end{equation}
% \comment{OR} we solve for the comoving GWs, 
% $\ah_P$, defining $u=k\tau$
% \begin{equation}
%     \pdv[2]{\ah_P}{u} + \bclosed{1 - \frac{\alpha(\alpha - 1)}{u^2}} \ah_P  = 0
% \end{equation}
% If $n=\alpha-1$ is an integer, the solution is
% \begin{equation}
%     \ah_P(\tau, \vec{k}) = A \RiccatiBessel[n][1](k\tau) + B\RiccatiBessel[n][2](k\tau),
% \end{equation}

\par
\dots
where
\begin{equation}
    \begin{split}
        \RiccatiBessel[n][1](u) &= u \sphBessel[n][1](u) = \sqrt{\frac{\ppi u}{2}} \Bessel[n+ \frac{1}{2}][1](u)  \\
        \RiccatiBessel[n][2](u) &=- u \sphBessel[n][2](u) = -\sqrt{\frac{\ppi u}{2}} \Bessel[n+ \frac{1}{2}][2](u)  
    \end{split}
\end{equation}
are the Riccati-Bessel functions.