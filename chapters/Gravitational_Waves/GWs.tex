
%%%%%%%%%%%%%%%%%%%%%%%%%%%%%%%%%%%%%%%%%%
%%%%%%% Ch. 3: Gravitational Waves %%%%%%%
%%%%%%%%%%%%%%%%%%%%%%%%%%%%%%%%%%%%%%%%%%


% ---------------------------------
% labels: \label{[type]:GWs:[name]}
% ---------------------------------


% ############LOCAL MACROS############
\newcommand*{\Chr}[2]{\ChristophelSym{#1}{#2}}  % Christoffel symbols        
\newcommand*{\barChr}[2]{%
    \ChristophelSym[\widebar{\Gamma}]{#1}{#2}}  % Christoffel symbols (background)
\newcommand*{\barg}{\widebar{g}}                % background metric
\newcommandx*{\hTT}[1]{h\indices*{^{\mathrm{TT}}_{ij}}}
\newcommandx*{\ah}{\mathfrak{h}}
\newcommand*{\piG}{\ppi G\ped{N}}
\newcommand*{\Lam}[2]{\ProjectionLambda{#1}{#2}}% Lambda tensor
% ####################################





% ////////////////// intro //////////////////


The term ``gravitational waves'' refers to the \nc{tensor perturbations to the background metric}. These ``waves'' are spacetime \normalsize{distortions} whose name comes from the fact that \checkthis{they obey the wave equation}.


% ///////////////////////////////////////////



\section{Linearised Gravity}\label[sec]{sec:GWs:lingrav}
    {\subimport{./}{lingrav.tex}}


\section{Polarisation and Decomposition of Gravitational Waves}\label[sec]{sec:GWs:decomp}
    {\subimport{./}{decomp.tex}}


\section{Gravitational Waves in Vacuum}\label[sec]{sec:GWs:vacuum}
    {\subimport{./}{vacuum.tex}}




\section{Generation of Gravitational Waves}\label[sec]{sec:GWs:generation}
    {\subimport{./}{generation.tex}}