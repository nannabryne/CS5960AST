


% \begin{bullets}
%     \item Surface-tension tension: There is a strong dependence on $\sigma$.
%     \item what I would do if time
%     \item actual observables
%     \item concluding remarks
% \end{bullets}



% In the following, we address the results 


% The equation of motion for the wall perturbation $\epsilon$ i

% We consider the 
The Nambu--Goto formulation with time-varying surface tension gave rise to an analytically solvable equation of motion for the planar-wall perturbation in the flat FLRW universe dominated by non-relativistic matter. Simulations showed qualitatively good correspondence with the full theory. Simulations~\texttt{1}--\texttt{5} and \texttt{7} showed maximum errors corresponding to $(16\pm 5)\%$ of the initial amplitude, and 
% $ \Delta \varepsilon  \lesssim [0.11,0.21]\epsast$. 
% $ \Delta \varepsilon  \lesssim (0.16 \pm 0.05)\epsast$. 
There exists loose restrictions for the maximum amplitude of the perturbation and of the scalar field fluctuations. 

Validations of the gravitational-wave calculations were inconclusive. Our simulation results showed patterns in the gravitational signature that were directly connected to the nature of the wall displacement field, but the exact relations were unclear. 


There is a substantial amount of work left if we ever want to compare this to actual gravitational-wave observations.

The results overall are intriguing, and show promise of future analytical estimations of gravitational waves from topological defects. The time-varying surface tension is effectively an additional damping in the equation of motion for the wall displacement field, and during a symmetron phase-transition, there exists explicit solutions to this equation.
% allows for explicit solutions to the new equation of motion 


\speak{Connect to intro}

% \paragraph{Gravitational waves.} %
% The results regarding gravitational radiation from perturbed, planar domain walls in a matter-dominated universe are inconclusive. More thorough analysis is required for \blahblah.
% To compare 

% They point in the right direction, but \blahblah


% \speak{TO-DO: Maybe hard-code reference list! + Fix 2 places with \url{something.no} (month year)}


% \section{Applications}
%     Similar lines-of-thought should hold for phase transitions in a variety of systems, amongst others the very early universe and particle-physics scenarios. \iftime{Write more.}




\section{Summary}
    We have solved analytically the equation of motion for a pertubed planar domain wall in a matter-dominated universe during a symmetron phase-transition, using the Nambu--Goto theory. This solution was put to the test in full-theory simulations 

    % Open questions
    What it means for the gravitational waves to have any other perturbation $\epsilon= \varepsilon(\tau) \sin{(py + \varphi)}$ is unclear. A starting point for such an analysis was provided in~\cref{app:walls:SE_tensor_alt:general}.


\section{Future work}
    Additional validations and analyses are necessary. We refer to suggestions in~\cref{sec:whatif:discussion:cont_verification}. 
    
    Most urgent is possibly the need for an intuitive statistical measure of the gravitational waves. 
    
    
    \iftime{Write more.}

\speak{\section*{TO DO (end)}}
\speak{\paragraph{Referance list:} Maybe hard-code reference list! + Fix 2 places with \url{something.no} (month year)}
\speak{\paragraph{Verb tenses:} Consider adjusting in some chapters/sections.}


