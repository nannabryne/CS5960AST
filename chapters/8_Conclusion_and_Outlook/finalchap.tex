%%%%%%%%%%%%%%%%%%%%%%%%%%%%%%%%%%%%%%%%%%%%
%%%%%% Ch. 8: Conclusion and Outlook  %%%%%%
%%%%%%%%%%%%%%%%%%%%%%%%%%%%%%%%%%%%%%%%%%%%

% ---------------------------------------
% labels: \label{[type]:finalchap:[name]}
% ---------------------------------------

% ¨¨¨¨¨¨¨¨¨¨¨¨¨¨¨¨¨¨¨¨¨¨¨¨¨¨¨¨¨¨¨¨¨¨¨¨¨¨¨¨¨¨
\newcommand{\wallsep}{\ALIASwallsep}
\newcommand\hpA{\ALIAShpA}
\newcommand\hpB{\ALIAShpB}
\newcommand\hpC{\ALIAShpC}
\newcommand\hpAB{\ALIAShpAB}
\newcommand\hpCR{\ALIAShpCR}
\newcommand\hpCI{\ALIAShpCI}
\newcommand\epsA{\ALIASepsA}
\newcommand\epsB{\ALIASepsB}
\newcommand\epsC{\ALIASepsC}
% ¨¨¨¨¨¨¨¨¨¨¨¨¨¨¨¨¨¨¨¨¨¨¨¨¨¨¨¨¨¨¨¨¨¨¨¨¨¨¨¨¨¨



% \begin{bullets}
%     \item Surface-tension tension: There is a strong dependence on $\sigma$.
%     \item what I would do if time
%     \item actual observables
%     \item concluding remarks
% \end{bullets}



% \section{Concluding remarks}
    The Nambu--Goto formulation with time-varying surface tension gives rise to an analytically solvable equation of motion for the planar-wall perturbation in the flat FLRW universe dominated by non-relativistic matter. Simulations with two walls per box length $L_\#$ show qualitatively good correspondence with the full theory. Simulations~\simnum{1}--\simnum{5} and \simnum{7} show average and maximum errors corresponding to $(8\pm 3)\%$ and $(16\pm 5)\%$ of the initial perturbation strength $\epsast$, respectively. In restricting $\epsast <\wallsep/5$ (i.e.~excluding simulation~\simnum{2}), where $\wallsep=L_\#/2$ is the distance between every wall centre, we get $(7\pm 1)\%$ and $(13\pm 2)\%$ instead. 

    The model-dependence in the thin-wall limit comes from the time-dependent part of the surface tension, directly related to the theory of phase-transition. In the conventional picture (\cref{eq:PT:symm_dws:chi_w_quasistatic_FLRW}), symmetron \emph{parameters} are factored out and do not affect the thin-wall motion. On the other hand, strength of fifth-force oscillations depends on $a_\ast$ and $\xi_\ast$ (see $m_\ast$ in~\cref{eq:stablesym:m_ast_def}) and can in theory affect the damping term in the equation of motion for $\varepsilon$ (see~\cref{app:walls:surface_tension:new_limits}).



    Large fluctuations in the symmetron field can be avoided by tweaking the initial conditions to our favour, as elaborated in~\cref{app:stablesym}. This is successfully tested with a homogeneous simulation (\cref{fig:results:achi:achi_no_wall}), but time limitations will have us stop there. We acknowledge that this improvement would not necessarily affect our results particularly, but it gives a flexibility to control future similar toy-model experiments. 


    We present only non-numerical comparative analyses on the subject of spin-2 perturbations. 
    As such, validations of the gravitational-wave calculations are inconclusive. Toy-model simulation results show patterns in the gravitational signature that are directly connected to the nature of the wall displacement field, but the exact relations are unclear. %\rephrase{Novel quantitative analyses is required.}
    The error propagating from the difference in wall positions ($\Delta \varepsilon = \abs*{\epsA-\epsC}$) is significant for the gravitational-wave modes, and we therefore use the simulated wall position as input to the formula~\cref{eq:PT:gwas:mywaves_complete_formula} for the comparison.


    The results overall are intriguing, and show promise of future 
    % \hyperlink{sentence1}%
    {analytical estimations of gravitational waves from topological defects}, 
    % which was one of the aims of this thesis. 
    which is the general idea~\citep{saikawaReviewGravitationalWaves2017}. %
    The time-varying surface tension is effectively an additional damping in the equation of motion for the wall displacement field, and during a symmetron phase-transition, there exist explicit solutions to this equation for idealised universes. Completing the framework with novel comparative analyses---both qualitative and quantitative---can potentially elucidate the dependence on symmetron parameters systematically. This would open a window of dynamical ranges and parameter spaces that simulations alone cannot encompass.



    A substantial amount of work remains if we ever want to compare this to actual gravitational-wave observations. We have not investigated the particular implications for future experiments in this thesis, but other literature~\citep{ferreiraGravitationalWavesDomain2023,clementsDetectingDarkDomain2023,paulGravitationalWaveSignatures2021,nakayamaGravitationalWavesDomain2017,hiramatsuEstimationGravitationalWave2014,hiramatsuGravitationalWavesCollapsing2010} support the hypothesis that domain wall signatures can be probed, potentially also discarded.%\comment{or non-found}. 

    % allows for explicit solutions to the new equation of motion 






    % \speak{Connect to intro}

    % \paragraph{Gravitational waves.} %
    % The results regarding gravitational radiation from perturbed, planar domain walls in a matter-dominated universe are inconclusive. More thorough analysis is required for \blahblah.
    % To compare 

    % They point in the right direction, but \blahblah


    % \speak{TO-DO: Maybe hard-code reference list! + Fix 2 places with \url{something.no} (month year)}


    % \section{Applications}
    %     Similar lines-of-thought should hold for phase transitions in a variety of systems, amongst others the very early universe and particle-physics scenarios. \iftime{Write more.}




\section{Summary of work}
    In this thesis we have solved analytically the equation of motion for a perturbed planar domain wall in a matter-dominated universe during a symmetron phase-transition, using the Nambu--Goto theory. This solution was put to the test in full field-theory toy-model simulations, along with an estimation of the gravitational-wave modes generated from this motion. We investigated thoroughly the ``$\epsA$ vs. $\epsC$''-anomaly in the limited number of simulation experiments presented. We have laid a solid foundation for complementary simulations with \asgrd{} using tweaked symmetron initial conditions and continued gravitational-wave analysis.





    % and  \blahblah
    
    % arguments for further analyses of ($\hpB$ vs. $\hpC$).
 
    % \speak{Prototype, toy model}
    % Verificatu

    % Open questions

    \subsection{Open questions}
    What it means for the gravitational waves to have any other perturbation $\epsilon= \varepsilon(\tau) \sin{(py + \varphi)}$ is unclear. A starting point for such an analysis was provided in~\cref{app:walls:SE_tensor_alt:general}. 
    We have \blahblah


    




\section{Implications and applications}
    Additional validations and analyses are necessary. We refer to suggestions in~\cref{sec:whatif:cont}. 
    Most urgent is possibly the need for an intuitive statistical measure of the gravitational waves. 



    Similar lines-of-thought should be applicable to flat spacetimes where phase transition is triggered at a critical temperature instead of a critical density. 
    
    
    \iftime{Write more.}



% \speak{\section*{TO DO (end)}}
% \speak{\paragraph{Referance list:} Maybe hard-code reference list! + Fix 2 places with \url{something.no} (month year)}
% \speak{\paragraph{Verb tenses:} Consider adjusting in some chapters/sections.}


