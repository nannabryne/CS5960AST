% ||||||||||||||||||||||||||||||||
% |||||| 7.X Open questions ||||||
% ||||||||||||||||||||||||||||||||

% ----------------------------------------------
% labels: \label{[type]:whatif:questions:[name]}
% ----------------------------------------------







What it means for the gravitational waves to have any other perturbation $\epsilon= \varepsilon(\tau) \sin{(py + \varphi)}$ remains unclear. A starting point for such an analysis was provided in~\cref{app:walls:SE_tensor_alt:general}. Varying symmetron parameters, or scalar-field theory as such, was the original wish, but other analyses was prioritised. 

We have seen how the time-dependence of the surface tension affects the dynamics of the wall. After some time, when the derivative of the surface tension goes to zero, the evolution is equivalent to scalar field in \blahblah, oblivious to the initial evolution. 
Gravitational waves are more complicated and generally ``remembers'' 

How this motion imprints on the gravitational waves and how this differs from what happens when we let the surface te is not studied in the 






% We have not studied the implications for biased domain walls, but one can 