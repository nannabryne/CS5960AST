% |||||||||||||||||||||||||||||||||||
% |||||| 5.4 Simulation setups ||||||
% |||||||||||||||||||||||||||||||||||

% --------------------------------------------
% labels: \label[type]{[type]:PT:sims:[name]}
% --------------------------------------------





% The simulations that were performed in this project and are referenced in the coming discussions are listed in~\cref{tab:PT:sims:sim_setups} for reference. 

\phpar[simulation details]


% \begin{description}
%     \item[type-0]\label[it]{it:results:something:type0} Use~\cref{eq:PT:gwas:achi_IC_type0} for $\chi$ and its conformal time derivative for $\dot{\chi}$ at some time shortly after $\redshift_\ast$
%     \item[type-$\star$]\label[it]{it:results:something:typestar} Use tweaked initial conditions for field,  
% \end{description}

% \comment{Reasoning behind choices of parameters:}




% \begin{description}
%     \item[Box size:] On one hand, we want the comoving box to be large so that the domain wall thickness $\delta\ped{dw}$ is much smaller than the box size $L_\#$ and the separation between the main wall and its counterpart (the ``anti-wall'') are fare away from each other. On the other hand, to resolve the Compton wavelength $L\nped{C}$, we require $\Delta_\# \leq \lambda\nped{C}$ and in turn $N_\# \geq L_\#/L\nped{C}$.
%     % \item[Compton wavelength] is \emph{not} resolved. On one  
%     \item[Initial time:] The normal quasistatic $\tanh$-profile (\cref{eq:PT:symm_dws:chi_w_quasistatic_FLRW}) approximates the initial configuration to $\breve{\chi}_\ast\to0$ and $\breve{\chi}'_\ast \to \infty$, which does not actually work out. With the analysis in~\cref{sec:PT:symm_dws:asymptotic} we can impose suitable initial conditions so that the field rolls into the minima in the most effective way. However, this affects the expression for the domain wall surface tension and \blahblah 
% \end{description} 

Below, we describe the reasoning behind the choice of parameters. As a starting point, we consider the fiducial set of symmetron parameters $a_\ast =0.33$, $\xi_\ast = 3.33\times 10^{-4}$ and $\beta_\ast = 1$, following~\citet{christiansenCosmologicalSimulationsPhase2024}. This gives Compton wavelength $L\nped{C}\simeq 1~\Mpch$ and asymptotic wall thickness $\delta_\infty \simeq \sqrt{2} ~\Mpch$.
%We refer to this as $\vec{\theta}_\ast\ap{fid}=(0.33, 3.33\times 10^{-1}, 1)$.


\subsubsection{Simulation box}
    % In the toy model, we are considering ``infinitely'' broad domain walls with vanishing thickness. We need therefore a simulation box of size that is comparable to the universe, 
    % Size of universe ~ 5000 Mpc/h0

    We use a comoving simulation box size that is comparable to the size of the universe---order $\mathscr{O}(\mathrm{Gpc}/h_0)$---so that $\delta\ped{w}\ll L_\#$ and the separation between the walls is $\sim L_{\#}/2$. If we want to resolve the Compton wavelength, this requires $N_\# \geq L_\#/L\nped{C} \gtrsim L~[\Mpch]$. However, since we are modelling the formation, the walls will \blahblah


    % On one hand, we want the comoving box to be large so that the domain wall thickness $\delta\ped{dw}$ is much smaller than the box size $L_\#$ and the separation between the main wall and its counterpart (the ``anti-wall'') are far away from each other. On the other hand, to resolve the Compton wavelength $L\nped{C}$, we require $\Delta_\# \leq L\nped{C}$ and in turn $N_\# \geq L_\#/L\nped{C}$.


\subsubsection{Initial configuration}

    The normal quasistatic $\tanh$-profile (\cref{eq:PT:symm_dws:chi_w_quasistatic_FLRW}) approximates the initial configuration to $\breve{\chi}_\ast\to0$ and $\breve{\chi}'_\ast \to \infty$, which does not actually work out. With the analysis in~\cref{sec:PT:symm_dws:asymptotic} we can impose suitable initial conditions so that the field rolls into the minima in the most effective way. We need to keep in mind that this affects the surface tension and wall width.
    %  However, this affects the expression for the domain wall surface tension and \blahblah 


\subsubsection{Nature of wall perturbation}
    We choose to only perturb one wall so that the time of collision between propagating gravitational waves is at $ \tau\ped{init} + L_\#/2$.\footnote{
        Perturbing both walls would give collision time $\tau\ped{init} + L_{\#}/4$.
    } %
    We would like $\varepsilon_\ast \ll L_\#/2$, but simply by looking at~\cref{fig:pertwalls:untitled1:demo_analytical_epsilon} we can assume that this would require a very high spatial resolution. As a result, we choose to exaggerate the initial perturbation amplitude and risk higher-order effects. 

    


    % First of all, we restrict $\varepsilon_\ast < L_{\#}/4$.





\subsection{Catalogue}

    Every simulation used a 4th order Runge-Kutta solver \blahblah 


    \begin{table}[h]
        \import{tables/Methodology/}{sim_setups.tex}
        \caption{Details about each simulation addressed in~\cref{part:findings}. Each simulation is labelled \simnum{0} -- \simnum{7}. See~\cref{sec:PT:gwas} for description of parameters. \comment{Should maybe give fiducial set of parameters (sim. \simnum{1}) and only give deviations from this.}}
        \label[tab]{tab:PT:sims:sim_setups}
    \end{table}


    \comment{Some extra simulations for convergence testing!}




% \subsection{Caveats}
%     We address some of the most obvious problems and paradoxes when simulating these domain walls.
%     %
%     \begin{description}
%         \item[Box size:] On one hand, we want the comoving box to be large so that the domain wall thickness $\delta\ped{dw}$ is much smaller than the box size $L_\#$ and the separation between the main wall and its counterpart (the ``anti-wall'') are fare away from each other. On the other hand, to resolve the Compton wavelength $L\nped{C}$, we require $\Delta_\# \leq \lambda\nped{C}$ and in turn $N_\# \geq L_\#/L\nped{C}$.
%         % \item[Compton wavelength] is \emph{not} resolved. On one  
%         \item[Initial time:] The normal quasistatic $\tanh$-profile (\cref{eq:PT:symm_dws:chi_w_quasistatic_FLRW}) approximates the initial configuration to $\breve{\chi}_\ast\to0$ and $\breve{\chi}'_\ast \to \infty$, which does not actually work out. With the analysis in~\cref{sec:PT:symm_dws:asymptotic} we can impose suitable initial conditions so that the field rolls into the minima in the most effective way. However, this affects the expression for the domain wall surface tension and \blahblah 
%     \end{description} 



% $\chi\rvert_{a}$