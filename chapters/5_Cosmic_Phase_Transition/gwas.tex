% |||||||||||||||||||||||||||||||
% |||||| 5.3 gwasevolution ||||||
% |||||||||||||||||||||||||||||||

% -------------------------------------
% labels: \label{[type]:PT:gwas:[name]}
% -------------------------------------



% ¨¨¨¨¨¨¨¨¨¨¨¨¨¨¨¨¨¨¨¨¨¨¨¨¨¨¨¨¨¨¨
% LOCAL MACROS:
\newcommand{\epsA}{\ALIASepsA}
\newcommand{\epsB}{\ALIASepsB}
\newcommand{\epsC}{\ALIASepsC}
\newcommand{\hpAB}{\ALIAShpAB}
\newcommand{\hpA}{\ALIAShpA}
\newcommand{\hpB}{\ALIAShpB}
\newcommand{\hpC}{\ALIAShpC}
% ¨¨¨¨¨¨¨¨¨¨¨¨¨¨¨¨¨¨¨¨¨¨¨¨¨¨¨¨¨¨¨





To accurately compare the soliton's evolution with the Nambu--Goto prediction, it is essential to begin with an initial configuration closely approximating the exact solution to the equations of the full field theory. Without this, any deviations from the Nambu--Goto dynamics might simply result from inaccuracies in the initial setup.

That said, we do not actually have these exact solutions in curved spacetime~\lcomment{Revise calculation, I forgot it\dots}. However, with sufficient tweaking, we should be able to design our desired scenario in a~\gevolution{} simulation box.
% \speak{Section about the configuration, particular case}
This~\lcnamecref{sec:PT:gwas} elaborates on the detailed simulation configuration that is the toy model.





\subsection{Initial configuration}\label{sec:PT:gwas:initial_config}
% In order to accurately compare the evolution of the soliton with the Nambu--Goto prediction, we need to start with an initial condition that is close enough to an exact solution of the full field theory equations. Otherwise, the possible deviations from the Nambu--Goto dynamics could just be due to the lack of precision in the initial configuration.
% To accurately compare the soliton's evolution with the Nambu--Goto prediction, it is essential to begin with an initial condition closely approximating the exact solution of the full field theory equations. Without this precision, any deviations from the Nambu--Goto dynamics might simply result from inaccuracies in the initial setup.

    The purpose of the simulations is to test the applicability of the thin-wall approximation for the wall evolution (\cref{eq:pertwalls:mywalls:eps_s_complete_MD}) and the corresponding gravitational waves (\cref{sec:pertwalls:gws}). Therefore, the simulation setup imitates a toy model, not a realistic scenario. To create this idealised cosmological scenario we initialise the simulation box as described below.

    % For the purpose of exploring the validity of our equations, the simulation setup needs be controlled \comment{no stochasticity}. This is why we cannot initialise a cosmological scenario where a phase transition is \cringe{in the cards (on the horizon)}; we would never achieve a domain wall lying perfectly at $z=z_0+\epsast \sin{py}$. 

    We assume wall profiles of the form
    \begin{subequations}
        \begin{align}
            \chi\ped{w}(a\ped{i}, z- z\ped{w}) &= {\breve{\chi}\tanh{\pclosed{\frac{a\breve{\chi}}{2L\nped{C}} (z-z\ped{w}) }} }\Bigg|_{a=a\ped{i}}, \\
            q\ped{w}(a\ped{i}, z- z\ped{w}) &= a^2 \dv{\chi\ped{w}}{\tau}\bigg|_{a=a\ped{i}},
        \end{align}
    \end{subequations}
    where $a$, $\breve{\chi}$ and $z\ped{w}$ generally depends on time, and $z\ped{w}$ also on spatial coordinates $x$ and $y$. Note that with $\breve{\chi}=\chi_+$ we get the familiar quasi-static formula in~\cref{eq:PT:symm_dws:chi_w_quasistatic_FLRW}. We refer to~\cref{app:stablesym} for details.

    % \paragraph{($\mathtt{achi}$)} %
    In an FLRW universe a (semi-)stable symmetron domain wall is approximated~\cref{eq:PT:symm_dws:chi_w_quasistatic_FLRW} (or~\cref{eq:stablesym:chi_w_quasistatic_FLRW_updated}). To preserve periodic boundary conditions, there needs to be an even number of walls present, where half are walls and the other half are anti-walls.\footnote{Made-up term used as the domain-wall analogue of the antikink in~\cref{sec:cosmo:defects:ex_Z2_kink}.} %
    % ($N=M$ in~\cref{eq:cosmo:defects:many_kinks}). 
    We choose two, and place the wall of interest at $z_0 = L_\#/2$, and its counterpart %---the anti-wall---
    at $\widebar{z}_0=0$, both aligned with the $xy$-plane. %The periodicity of the system \blahblah
    

    % Ignore the perturbation for now. The 3D simulation box of side lengths $L$ has in total $N^3$ lattice points.\footnote{Two dimensions would suffice for this problem, but \texttt{gevolution} only takes cubic boxes.} %Separating the box in two domains calls for two domain walls.
    % To preserve the periodic boundary conditions, we need at least two walls. We place one topological defect at $z=L/2$ and its counterpart (the antikink) at $z=0$. The simplest way to achieve this is by initialising the scalar field $\chi$ ($\texttt{achi}$ in the code) with eq XX.
    % % \begin{equation}\label[eq]{eq:PT:gwas:achi_IC_type0}
    %     \chi_+ \tanh{ \pclosed{ \frac{a(z-z\ped{dw})}{2L\nped{C}} \chi_+ } }.
    % \end{equation}
    Now say we add a displacement $\epsilon= \epsilon(\tau, x, y)$ to the middle wall. %
    % The system of defects is then
    % \begin{equation}
    %     \chi(a,z) = \prod_{i\in \Integer} \chi\ped{w}(a, z- [z_0 + \epsilon+ iL_\#] )\prod_{j\in\Integer} \bclosed{-\chi\ped{w}(a, z-jL_\#) },
    % \end{equation}
    % assuming sufficient spatial separation.
    Assuming sufficient spatial separation, the system of defects is given by (see~\cref{eq:cosmo:defects:many_kinks})
    \begin{equation}
        \chi(a,\vec{x}) = \breve{\chi}^{1-2N}\prod_{n= -\infty}^\infty \chi\ped{w}(a, z- z\ped{w}^{n} )\prod_{m=-\infty}^{\infty} {\widebar{\chi}\ped{w}(a, z-\widebar{z}\ped{w}^m) },
    \end{equation}
    where
    \begin{equation}\label{eq:PT:gwas:wall_antiwall_positions}
        z\ped{w}^n = z_0 + \epsilon + nL_\# %
        \quad\text{and}\quad %
        \widebar{z}\ped{w}^m = \widebar{z}_0+ mL_\#.
    \end{equation}
    \Cref{fig:PT:gwas:demo_achi_periodicity} demonstrates how $\chi(a\ped{i}, z)$ varies with comoving coordinate $z$, in the absence of wall perturbation.
    \begin{figure}[t]
        \centering
        \includegraphics[width=\linewidth]{Methodology/demo_achi_periodicity.png}
        \caption{Demonstration of the periodicity of $\chi(a\ped{i},z)$ on the lattice. The shaded region represents the box coordinates.
        % We only simulate $z\in[0,L_{\#})$, but the system is theoretically 
        }%
        \label{fig:PT:gwas:demo_achi_periodicity}
    \end{figure}
    
    
    

    % \phpar[about tweaked BCs]


    % \paragraph{($\mathtt{aq}$)} %
    Setting the initial conditions on the field $q = a^2 \dot{\chi}$ is a matter of algebra. The expression is cleanest when $\dot{z}\ped{w}=\dot{\epsilon}=0$, i.e. for $a\ped{i}=a_\ast$. See~\cref{app:stablesym:initialisation} for detailed explanation. %a bit more subtle. In short, 


    % \comment{Write about $\mathtt{aq}$ field.} 


    



    % We choose to have one topologicalm
    % To create a stable, thin, planar wall configuration, we 


    % We start with a stable, thin, planar wall configuration (no perturbation).



    \paragraph{Initial time.} %
    % So far, we have not \blahblah
    We need to be careful when choosing the exact redshift to initiate the simulation. Clearly, the conventional expression~\cref{eq:PT:symm_dws:chi_w_quasistatic_FLRW} does not work if we set initial redshift $\redshift\ped{i}=\redshift_\ast$. Initiating only a few time steps later allows us to use this after all, but we need to make sure the wall and anti-wall do not collide. Whether this is the case depends on several parameters. Some sets of initial conditions will induce large fifth-force oscillations, and we should try to avoid this.

    Using the tweaked initial conditions on the asymptotic fields as boundary conditions on the quasi-static field in~\cref{eq:stablesym:chi_w_quasistatic_FLRW_updated} opens for the possibility of initialising as close to phase transition as we want. %
    We observe that $\breve{\chi}\ap{ideal}\to \chi_+ $ after some time, and so another strategy to reduce oscillations is to initialise \emph{after} this non-adiabatic phase. The drawbacks are that we might lose important information about the source of the gravitational waves and that $\dot{z}\ped{w}\not{approx}0$.

\subsection{Energy and momentum}\label{sec:PT:gwas:SE_tensor}
    With this exact setup, we can write the total domain-wall SE tensor on the form %
    % $T\_{\mu\nu}(\tau, \vec{x})= \sum_{n}{T^{\mathrm{w}(n)}_{\mu\nu}(\tau, \vec{x})} + \sum_m {\widebar{T}^{\mathrm{w}(m)}_{\mu\nu}(\tau, \vec{x})} $, 
    $T\_{\mu\nu}= \sum_{n}{\Tw^n_{\mu\nu}} + \sum_m {\widebar{\Tw}^{m}_{\mu\nu}} $, 
    % where $T\ap{w}$ and 
    i.e.~as a sum over the SE tensors associated with each wall and anti-wall. Explicitly,
    \begin{equation} 
        \begin{split}
            T\_{ab}(\tau, \vec{x}) &= -a \sigma \eta\_{ab} \sum_{n} \bclosed{\varPhi_{l}(z-z\ped{w}^n)+ \varPhi_{l}(z-\widebar{z}\ped{w}^n) },  \\
            T\_{iz}(\tau, \vec{x}) &= -a \sigma \sum_{n} \bclosed{\varPhi_{l}(z-z\ped{w}^n) \partial\_i z\ped{w}^n + \cancel{ \varPhi_{l}(z-\widebar{z}\ped{w}^n) \partial\_i \widebar{z}\ped{w}^n} },
        \end{split}
    \end{equation}
    from~\cref{sec:pertwalls:gws:Fourier_SE_tensor}, with $z\ped{w}^n$ and $\widebar{z}\ped{w}^n$ are given in~\cref{eq:PT:gwas:wall_antiwall_positions}. Here, $\sigma$ and $\delta=\sqrt{2}l$ refers to the surface tension and thickness of any wall in the system.

    \subsubsection{Continuous and discrete Fourier space}
    The Fourier analogue reads 
    \begin{equation}
        \begin{split}
            \Ft{T}\_{ab}(\tau, \vec{k}) &= -a\sigma \eta\_{ab} \sum_{n}\bclosed{ \mathscr{D}^n(\tau, k_z) \integ[2]{x} \eu[\im k_z \epsilon] \eu[\im (k_x x +k_y y)]  + (2\ppi)^2\widebar{\mathscr{D}}^n (\tau, k_z) \Diracdelta(k_x) 
            \Diracdelta(k_y)} ,\\
            \Ft{T}\_{i z}(\tau, \vec{k}) &= -a\sigma \sum_{n} \mathscr{D}^n(\tau, k_z) \integ[2]{x} \epsilon\_{,i} \eu[\im k_z \epsilon] \eu[\im (k_x x +k_y y)],
        \end{split}
    \end{equation}
    where $\mathscr{D}^n= \eu[\im k_z L_\#/2]\cdot \widebar{\mathscr{D}}^n$ and $\widebar{\mathscr{D}}^n = \exp{ \im k_z nL_\#- k_z^2 l^2/2 }$ from~\cref{sec:pertwalls:gws:Fourier_SE_tensor}. This essentially gives a periodicity in $k_z$ in the SE tensor that goes as $n\cdot k_\#$. %
    % The Fourier analogue reads
    % \begin{equation}
    %     \begin{split}
    %         \Ft{T}\_{ab}(\tau, \vec{k}) &= -a\sigma \eta\_{ab} \sum_{n}\bclosed{ \mathscr{D}\ped{w}^n(\tau, k_z) \integ[2]{x} \eu[\im k_z \epsilon] \eu[\im (k_x x +k_y y)]  + (2\ppi)^2\widebar{\mathscr{D}}\ped{w}^n (\tau, k_z) \Diracdelta(k_x) 
    %         \Diracdelta(k_y)} ,\\
    %         \Ft{T}\_{i z}(\tau, \vec{k}) &= -a\sigma \sum_{n} \mathscr{D}\ped{w}^n(\tau, k_z) \integ[2]{x} \epsilon\_{,i} \eu[\im k_z \epsilon] \eu[\im (k_x x +k_y y)],
    %     \end{split}
    % \end{equation}
    % where $\mathscr{D}\ped{w}^n= \eu[\im k_z L_\#/2]\cdot \widebar{\mathscr{D}}\ped{w}^n   $ and $\widebar{\mathscr{D}}\ped{w}^n = \exp{ \im k_z nL_\#- (k_z \delta\ped{w}(\tau))^2/2 }$. This essentially gives a periodicity in $k_z$ in the SE tensor that goes as $nk_\#$. %
    % We see that $\mathscr{D}\ped{w}^n(\tau, k_z=\lcoord{w}k_{\#}) = (-1)^{\lcoord{w}} \exp{-2(\ppi \lcoord{w} \delta\ped{w}(\tau)/L_{\#})^2}\in \Real$. 
    We define
    \begin{equation}
        \mathtt{D}(\tau, \lcoord{w}) \equiv \sum_{n} \Diracdelta(\lcoord{w}-n) \mathscr{D}^n(\tau, \lcoord{w}k_\#) = \bbclosed{\lcoord{w}\in\Integer}  (-1)^{\lcoord{w}} \exp{- \lcoord{w}^2 [k_\# \delta(\tau)]^2/4 } \in \Real,
    \end{equation} 
    which gives 
    \begin{equation}
        \begin{split}
            \Ft{T}\_{ab} = - a\sigma \eta\_{ab} \mathtt{D} \cdot \mathop{\mathcal{F}}_{k_x,k_y}{\cclosed{\eu[\im k_z \epsilon]}}, \quad \Ft{T}\_{iz} = - a\sigma \mathtt{D} \cdot  \mathop{\mathcal{F}}_{k_x,k_y}{\cclosed{\epsilon\_{, i} \eu[\im k_z \epsilon]}} ,
        \end{split}
    \end{equation}
    as expressions for the walls' displacement field's contribution to the total energy and momentum. %
    We used the shorthand notation $\mathop{\mathcal{F}}_{k_i}$ for Fourier transforms~\lcomment{Ref. to some sec.!}.
    % as the SE tensor representing the contribution from tk
    % \begin{equation}\label{eq:PT:gwas:pi_plus_Ft_series}
    %     \Ft{\pi}\_{ij}(\tau, \vec{k})
    %     % = a^2  \ProjectionLambda{ij}{kl}(\vec{k})\Ft{T}\_{kl}(\tau, \vec{k}) 
    %     = \mathtt{D}(\tau, k_z/k_\#) \times \Ft{\pi}\ap{w0}_{ij}(\tau, \vec{k})
    %     % \bclosed{\Ft{\pi}\ap{w}_{ij}(\tau, \vec{k})}_{\delta\ped{w}\to 0}
    % \end{equation}
    % and see that $\mathtt{D}\in \Real$.
    % for the TT-projected SE tensor associated with series of walls. We used $\Ft{\pi}\ap{w0}$ to represent an infinitely  %Note that 



    % Thus, if we consider $k_y \neq 0 $, 
    
    
    % \checkthis{This periodicity will eventually give a periodicity in $k_z$ in the SE tensor that goes as $nk_\#$.} 
    % \rephrase{We see that $\mathscr{D}\ped{w}^n(\tau, k_z=\lcoord{w}k_{\#}) = (-1)^{\lcoord{w}} \exp{-2(\ppi \lcoord{w} \delta\ped{w}(\tau)/L_{\#})^2}\in \Real$.}

    We now get the same result as in~\cref{sec:pertwalls:gws:Fourier_SE_tensor} in letting $\epsilon = \varepsilon(\tau)\sin{py}$ and considering $k_y\neq 0$. Substituting $\mathscr{D}\ped{w}\to \mathtt{D}$ in~\cref{eq:pertwalls:gws:pi_w_plus_Ft}, we get the TT-projected SE tensor for the whole system, and put into~\cref{eq:pertwalls:gws:H_P_12} gives the gravitational waves sourced by the displacement in the wall normal coordinate. We elaborate on this for $\alpha=2$ in the coming subsection.



% \subsection{Gravitational waves}
\subsection{Ripples in matter-dominated spacetime}\label{sec:PT:gwas:gws}
    % We collect 
    Consider matter-domination, $\alpha=2$, and set $\epsilon=\varepsilon(\tau)\sin{py}$. %
    % \begin{equation}
    %     \Ft{\pi}_+(\tau, \vec{k})\rvert\nped{NG} =2{\ppi}^2    %\quad k_y \neq  0
    % \end{equation}
    % We consider the matter-dominated unive
    \begin{subequations}\label{eq:PT:gwas:mywaves_complete_formula}%
    The resulting transverse, traceless tensor perturbations are given by $\Ft{h}\_{ij}= \Ft{h}_+e^{+}_{ij} + \Ft{h}_\times e^{\times}_{ij}$. From~\cref{sec:pertwalls:gws:Fourier_SE_tensor} we have $\Ft{h}_\times=0$, so the gravitational waves are monochromatic plus-polarised waves given by
    \begin{equation}
        a(\tau)\Ft{h}_+(\tau, \vec{k}) = H_+^1(\tau, \vec{k}) + H_+^2(\tau, \vec{k}),
    \end{equation}
    % where $H^{1,2}_+$ are 
    where
    \begin{equation}
        \begin{split}
            H_+^1(\tau, \vec{k}) &= + \RiccatiBessel[1](k\tau)  \integ{\dummy{\tau}}[\tau_\ast][\tau] \RiccatiNeumann[1](k\dummy{\tau})f_+(\dummy{\tau}, \vec{k}) , \\
            H_+^2(\tau, \vec{k}) &= - \RiccatiNeumann[1](k\tau)  \integ{\dummy{\tau}}[\tau_\ast][\tau] \RiccatiBessel[1](k\dummy{\tau})f_+(\dummy{\tau}, \vec{k}),
        \end{split}
    \end{equation}
    and
    % given by~\cref{eq:pertwalls:gws:H_P_12} and 
    \begin{equation}
        f_+(\tau, \vec{k}) = \frac{16\ppi G\nped{N} a^3(\tau) \Ft{\pi}_+(\tau, \vec{k})}{k^2}.
    \end{equation}
    \end{subequations}
    %
    \begin{subequations}\label{eq:PT:gwas:mywaves_pi_plus_series}%
    Now consider the TT-projected SE tensor associated with an infinitely thin wall located the origin. We use~\cref{eq:pertwalls:gws:pi_w_plus_Ft} to write
    \begin{equation}
        a(\tau)\Ft{\pi}\ap{w0}_+(\tau, \ell, \vartheta) = 2\ppi^2\sigma\ped{w}(\tau) \cos^2{\vartheta} \cdot (-1)^\ell\Bessel[\ell]\big[\ell \tan{\vartheta} \cdot p \varepsilon(\tau)\big],
    \end{equation}
    where $\ell = k_y/p \in \Integer$ and $\vartheta = \arctan{k_z/k_y}\in [0,2\ppi)$. %
    We make further use of this parametrisation 
    % $\vec{k}=(k_x,k_y,k_z) \to (\ell, \zeta)$ 
    and write the periodic system's total TT-projected SE tensor as
    \begin{equation}
        \Ft{\pi}_+(\tau, \vec{k}) = \Diracdelta(k_x)\Diracdelta(k_y-\ell p) \bbclosed{\ell \in \Integer }  \times \Ft{\pi}_+(\tau, \ell, \vartheta = \arctan{k_z/k_y}),
    \end{equation}
    where
    \begin{equation}
        \Ft{\pi}_+(\tau, \vec{\ell}) = \mathtt{D}(\tau, \ell \tan{\vartheta} p /k_\#) \times\Ft{\pi}_+\ap{w0}(\tau, \vec{\ell}),
    \end{equation}
    with $\vec{\ell} \equiv (\ell, \vartheta)$. % \in \Integer \cross \Real$. %
    \end{subequations}
    % Furthermore, we use the parametrisation $\vec{k}=(k_x,k_y,k_z) \to (\ell, \zeta)$ and write
    % % \begin{equation}
    % %     (\ell, \zeta) = 
    % %     \begin{cases}
    % %         (k_y/p, k_z/k_y)
    % %     \end{cases}
    % % \end{equation}
    % % we write
    % % \begin{equation}
    % %     a(\tau)\Ft{\pi}_+(\tau, \vec{k}) = 
    % % \end{equation}
    % \begin{equation}
    %     \Ft{\pi}_+(\tau, \vec{k}) = \Diracdelta(k_x)\Diracdelta(k_y-\ell p) \bbclosed{\ell \in \Integer }  \times \Ft{\pi}_+(\tau, \ell, \zeta=k_z/k_y),
    % \end{equation}
    % where
    % \begin{equation}
    %      \Ft{\pi}_+(\tau, \ell, \zeta) = \mathtt{D}(\tau, \ell\zeta p /k_\#)\Ft{\pi}_+\ap{w0}(\tau, \ell, \zeta)
    % \end{equation}
    % and $\Ft{\pi}_+\ap{w0}$ represents the TT-projected SE tensor associated with an infinitely thin wall located the origin. In particular,
    % \begin{equation}
    %     a(\tau)\Ft{\pi}\ap{w0}_+(\tau, \ell, \zeta) =  \frac{2\ppi^2}{1+\zeta^2 }\sigma\ped{w}(\tau)  \cdot (-1)^\ell\Bessel[\ell]\big(\ell \zeta p \varepsilon(\tau)\big).
    % \end{equation}
    % % \begin{equation}
    % %     a(\tau)\Ft{\pi}\ap{w}_+(\tau, \ell, \zeta) =  \frac{2\ppi^2}{1+\zeta^2 }\sigma\ped{w}(\tau) \mathscr{D}\ped{w}(\tau, p\ell \zeta) \cdot (-1)^\ell\Bessel[\ell]\big(\ell \zeta p \varepsilon(\tau)\big).
    % % \end{equation}
    % % and
    % % \begin{equation}
    % %     \mathscr{D}\ped{w}(\tau, k_z) = \exp{ \im k_z z_0 - (k_z \delta\ped{w})^2/2 }.
    % % \end{equation}
    % % We showed above that the periodic series is taken into account by
    % % \begin{equation}
    % %     \Ft{\pi}_+(\tau, \vec{k}) = \Diracdelta(k_z/k_\#-\lcoord{w})\bbclosed{ \lcoord{w}\in \Integer}\times  \Ft{\pi}\ap{w}_+(\tau, \vec{k}) 
    % % \end{equation} 
    % % where
    % % \begin{equation}
    % %     \mathscr{D}\ped{w}(\tau, k_z=\lcoord{w}k_\#) = \exp{ \im k_z z_0 - (k_z \delta\ped{w})^2/2 } \stackrel{\lcoord{w} \in \Integer}{\longrightarrow}(-1)^{\lcoord{w}} \exp{- (\lcoord{w} k_\# \delta\ped{w})^2}.
    % % \end{equation}
    % \comment{Comment about $\zeta$!} %
    Now,~\crefrange{eq:PT:gwas:mywaves_complete_formula}{eq:PT:gwas:mywaves_pi_plus_series} gives $\Ft{h}_+$ for a given input function $\varepsilon(\tau)$. %$\mathtt{\Delta} \mathtt{e}$
    
   









\subsection{Code output and interpretation}\label{sec:PT:gwas:data_comparison}
    % The code automatically outputs scalar quantities like extremal values of $\chi$ and $q$, as well as their box-averaged values. Besides this, we request two-dimensional snapshots of the field in question at every-something time step. 
    % \phpar[about background output and HDF5 output]

    We use the outputted two-dimensional snapshots of $\chi$ at lattice coordinate $\lcoord{i}=4$ (chosen randomly) to track the simulated wall position, considered as the coordinates $(y,z)$ of the minimum value of $\abs{\chi}$ (ignoring the anti-wall). We define $\epsC$ as the aforementioned quantity at a $y$-coordinate for which $\sin{py}=1$, subtracting the initial position $z_0=L_\#/2$. The confidence region of this is then $\pm \Delta_\#/2$ in the $z$-direction. \iftime{Comment about temporal resolution.}

    The code outputs two-dimensional snapshots of tensor perturbations $\Ft{h}\_{ij}$ at the same rate, sliced at $\lcoord{u}=0$. We use only $\Ft{h}_{xx}=\hpC$, but sporadically check that $2\hpC^2 =\Ft{h}\^{ij}\Ft{h}\_{ij} $. 


    The formula for the gravitational waves in the thin-wall limit,~\cref{eq:PT:gwas:mywaves_complete_formula}, allows for any function $\varepsilon(\tau)$ as input. Therefore, we will use both the solution to the equation of motion~\cref{eq:pertwalls:mywalls:eom_eps_s_MD}, call it $\epsA$, and the simulated result $\epsB$. We will have to keep in mind that the latter is poorly resolved. 

    The semi-analytical solution $\hpAB$ is obtained by means of the simple trapezoidal scheme, specifically the \textit{Numpy} library's \texttt{cumptrapz} method \rcomment{More details?}. We let $\hpA$ and $\hpB$ signify the solution to~\crefrange{eq:PT:gwas:mywaves_complete_formula}{eq:PT:gwas:mywaves_pi_plus_series} with $\epsA$ and $\epsB$ as input, respectively.


    % \comment{Explain how we obtain $\varepsilon$ and how we compute $\Ft{h}_{+}$}















    % \hlineSep

    % We have the SE tensor $T\_{\mu\nu}(\tau, \vec{x})= \sum_{n}{T^{\mathrm{w}(n)}_{\mu\nu}(\tau, \vec{x})} + \sum_m {\widebar{T}^{\mathrm{w}(m)}_{\mu\nu}(\tau, \vec{x})} $, which we can write as~(see~\cref{sec:pertwalls:gws:Fourier_SE_tensor})
    % \begin{equation} 
    %     \begin{split}
    %         T\_{ab}(\tau, \vec{x}) &= -a \sigma \eta\_{ab} \sum_{n} \bclosed{\varPhi_{\delta\ped{w}/2}(z-z\ped{w}^n)+ \varPhi_{\delta\ped{w}/2}(z-\widebar{z}\ped{w}^n) },  \\
    %         T\_{iz}(\tau, \vec{x}) &= -a \sigma \sum_{n} \bclosed{\varPhi_{\delta\ped{w}/2}(z-z\ped{w}^n) \partial\_i z\ped{w}^n + \cancel{ \varPhi_{\delta\ped{w}/2}(z-\widebar{z}\ped{w}^n) \partial\_i \widebar{z}\ped{w}^n} },
    %     \end{split}
    % \end{equation}
    % where $z\ped{w}^n$ and $\widebar{z}\ped{w}^n$ are given in~\cref{eq:PT:gwas:wall_antiwall_positions}. %
    % Going to Fourier space, we have
    % \begin{equation}
    %     \begin{split}
    %         \Ft{T}\_{ab}(\tau, \vec{k}) &= -a\sigma \eta\_{ab} \sum_{n}\bclosed{ \mathscr{D}\ped{w}^n(\tau, k_z) \integ[2]{x} \eu[\im k_z \epsilon] \eu[\im (k_x x +k_y y)]  + \widebar{\mathscr{D}}\ped{w}^n (\tau, k_z) \Diracdelta(k_x) 
    %         \Diracdelta(k_y)} ,\\
    %         \Ft{T}\_{i z}(\tau, \vec{k}) &= -a\sigma \sum_{n} \mathscr{D}\ped{w}^n(\tau, k_z) \integ[2]{x} \epsilon\_{,i} \eu[\im k_z \epsilon] \eu[\im (k_x x +k_y y)],
    %     \end{split}
    % \end{equation}
    % where $\mathscr{D}\ped{w}^n= \eu[\im k_z L_\#/2]\cdot \widebar{\mathscr{D}}\ped{w}^n   $ and $\widebar{\mathscr{D}}\ped{w}^n = \exp{ \im k_z nL_\#- (k_z \delta\ped{w}(\tau))^2/8 }$.

    % % \boxed{\text{For $k_z=\mathtt{w}2\ppi/L_{\#}$ this means an additional factor $(-1)^{\mathtt{w}}$}}

    % \speak{%
    %     \paragraph{\dots} %
    %     For $k_z=\mathtt{w}2\ppi/L_{\#}$ this means an additional factor $(-1)^{\mathtt{w}}$.%
    % }

    % Periodicity $\leadsto$
    % \[ T\_{\mu\nu}(\tau, \vec{k}) = k_{\#}\sum_n \Diracdelta( k_z - n k_\# ) T\ap{ww}_{\mu\nu}(\tau, \vec{k}) \]


    % \begin{equation}
    %     \begin{split}
    %         T\_{ab}(\tau, \vec{k}) &= -a\sigma \eta\_{ab} \sum_{n}\bclosed{ \mathscr{D}\ped{w}^n(\tau, k_z) \integ[2]{x} \eu[-\im k_z \epsilon] \eu[\im (k_x x +k_y y)]  + \widebar{\mathscr{D}}\ped{w}^n (\tau, k_z) \Diracdelta(k_x) 
    %         \Diracdelta(k_y)} ,\\
    %         T\_{i z}(\tau, \vec{k}) &= -a\sigma \sum_{n} \mathscr{D}\ped{w}^n(\tau, k_z) \integ[2]{x} \epsilon\_{,i} \eu[-\im k_z \epsilon] \eu[\im (k_x x +k_y y)],
    %     \end{split}
    % \end{equation}


    % \subsubsection{Discrete Fourier space}
    % The discrete Fourier transform (DFT) $\tilde{f}(\lcoordk)$ of $f(\lcoordx)$ is given by
    % \begin{equation}\label{eq:PT:gwas:DFT_definition}
    %     f(\lcoordx) = \frac{1}{N_\#^3} \sum_{\lcoordk } \eu[-2\ppi \im \lcoordk \lcoordx/N_\# ]\tilde{f}(\lcoordk) , \quad \tilde{f}(\lcoordk) = \sum_{\lcoordx} \eu[+2\ppi \im \lcoordk \lcoordx/N_\#  ]f(\lcoordx).
    % \end{equation}
    % We see that $\mathscr{D}\ped{w}^n(\tau, k_z=\lcoord{w}k_{\#}) = (-1)^{\lcoord{w}} \exp{-(2\ppi \lcoord{w} \delta\ped{w}(\tau)/L_{\#})^2/8}\in \Real$.
    % For the stress--energy tensor this means
    % \begin{equation}
    %     \tilde{T}\_{ab}(\tau, \lcoordk) = 
    % \end{equation}




    % \begin{draft}
        
    

        % \begin{equation}
        %     \mathscr{D}\ped{w}^n = \exp{ -\im k_z (z_0 + nL) - (k_z \delta\ped{w})/8 } %\quad \text{and} \quad 
        %     % \widebar{\mathscr{D}}\ped{w}^n = \exp{ -\im k_z (\widebar{z}_0 + nL) + (k_z \delta\ped{w})/8 } 
        %     % \stackrel{()}{\mathscr{D}}
        % \end{equation}
        % and $\widebar{\mathscr{D}}\ped{w}^n$ the same with $\widebar{z}_0$.
        % \begin{equation}
        %     \widebar{\mathscr{D}}\ped{w}^n = \exp{ -\im k_z nL_\#- (k_z \delta\ped{w})/8 }.  %\quad \text{and} \quad 
        % \end{equation}
        % and $\widebar{\mathscr{D}}\ped{w}^n$ the same with $\widebar{z}_0$.



        % \speak{I think I messed it up; should I go through the Dirac comb directly?}
    
    




% \subsection*{Discrete Fourier space}
    % The discrete Fourier transform (DFT) $\tilde{f}(\lcoord{k}_{\lcoord{u}, \lcoord{v},\lcoord{w}})$ of $f(\lcoord{x}_{\lcoord{i}, \lcoord{j}, \lcoord{k}})$ is given by
    % \begin{equation}
    %     f(\lcoord{x}) = \frac{1}{N_\#^3} \sum_{\lcoord{k}} \eu[-2\ppi \im \lcoord{k}\lcoord{x}/N_\# ]\tilde{f}(\lcoord{k}) , \quad \tilde{f}(\lcoord{k}) = \sum_{\lcoord{x}} \eu[+2\ppi \im\lcoord{k}\lcoord{x}/N_\# ]f(\lcoord{x}).
    % \end{equation}
    % \comment{OR:} 
    % The discrete Fourier transform (DFT) $\tilde{f}(\lcoordk)$ of $f(\lcoordx)$ is given by
    % \begin{equation}
    %     f(\lcoordx) = \frac{1}{N_\#^3} \sum_{\lcoordk } \eu[-2\ppi \im \lcoordk \lcoordx/N_\# ]\tilde{f}(\lcoordk) , \quad \tilde{f}(\lcoordk) = \sum_{\lcoordx} \eu[+2\ppi \im \lcoordk \lcoordx/N_\#  ]f(\lcoordx).
    % \end{equation}




    % \blahblah

    % $h\_{ij}(\eta, \vec{k}) = \Lambda\_{ijkl}(\vec{k}) h\_{kl}(\eta,\vec{k})$ where $\vec{k}$ is \emph{not} defined $\vec{k} = 2\ppi (\lcoord{u},\lcoord{v},\lcoord{w})/ L $


    % \blahblah


    
    % Let $T^0_{ij}(\tau, \vec{x})$ be the SE tensor for a wall or an anti-wall at $z=0$. For our setup, this amounts to
    % \begin{equation}
    %     T\_{ij}(\tau, \vec{x}) = \sum_{n=0}^{\infty} \bclosed{ \Diracdelta(z-z\ped{w}^{n}) + \Diracdelta(z-\widebar{z}\ped{w}^{n})  } T^0_{ij}(\tau, \vec{x})
    % \end{equation}


    % \hlineSep

    % \deleteme{
    % The wall--anti-wall system
    % \begin{equation}
    %     T\ap{ww}_{ij}(\tau, \vec{x}) = T\ap{w}_{ij}(\tau, \vec{x}) + \widebar{T}\ap{w}_{ij}(\tau, \vec{x})
    % \end{equation}
    % \begin{equation}
    %     T\_{ij}(\tau, \vec{x}) = \sum_{n}^{\infty}\Diracdelta(\vec{x}\cdot\hat{\vec{z}} -nL)  T\ap{ww}_{ij}(\tau, \vec{x})
    % \end{equation}
    % \begin{equation}
    %     T\_{ij}(\tau, \vec{x}) = \sum_{n}^{\infty} T^{\mathrm{w}}_{ij} (\tau, \vec{x}-\hat{\vec{z}}z\ped{w}^n ) + \sum_{m}^{\infty} \widebar{T}^{\mathrm{w}}_{ij} (\tau, \vec{x}-\hat{\vec{z}}\widebar{z}\ped{w}^m ) 
    % \end{equation}
    % \begin{equation}
    %     T\_{ij}(\tau, \vec{x}) = %
    %     \sum_{n}^{\infty} \Diracdelta(\vec{x}\cdot\hat{\vec{z}} -z\ped{w}^n) T^{\mathrm{w}}_{ij} (\tau, \vec{x}) 
    %     + \sum_{m}^{\infty} \Diracdelta(\vec{x}\cdot\hat{\vec{z}} -\widebar{z}\ped{w}^m) \widebar{T}^{\mathrm{w}}_{ij} (\tau, \vec{x
    %     }) 
    % \end{equation}}
    % \begin{equation}
    %     T\_{ij}(\tau, \vec{x}) = \sum_{n}^{\infty} \Diracdelta(z\ped{w}-z\ped{w}^n) T^{\mathrm{w}}_{ij} (\tau, \vec{x}) 
    %     % T^{\mathrm{w}(n)}_{ij}(\tau,\vec{x}-z\ped{w}^{n}\hat{\vec{z}}) +
    % \end{equation}
% \end{draft}




% +++++++++++++++++++++++++++++++++




    


% +++++++++++++++++++++++++++++++++







% \begin{draft}

% \subsection{\tmptitle{Options/parameters}}
%     % Now to the spe
%     \rephrase{We now turn to the details of the simulations we are to perform. For starters, we provide options and compiler flags to make certain that fields $\chi$ and $h\_{ij}$ are computed.} 

    
%     \paragraph{Numerics.} %
%     There are choices to be made concerning temporal and spatial resolution, and what integration method to use. Of solvers, there are a handful to choose from, for instance Leap-Frog or fourth order Runge--Kutta. The spatial resolution $\Delta_\#= L_\#/N_\#$ is set by the users choice of box side length $L_\# $ in $\text{Mpc}/h_0$ and the number of grid points $N_\#$ in each direction. The temporal resolution\emph{s} vary for some field updates. They are controlled by the \emph{Courant factors}
%     \begin{equation}
%         C\ped{f} = v\ped{g} \Delta \tau / \Delta_\#
%     \end{equation}
%     where \blahblah \comment{I realise that I do not understand this entirely ...}
%     % For the scalar field, we have $v^{\chi}=$


%     \paragraph{Physics.} % 
%     We need to choose a background for evolving the relativistic equations. For our case, this is a flat \textLambda{}CDM universe with $\Omega\ped{m0}=1$. %The (a)symmetron parameters are given 
%     The code takes the phenomenological parameters $\mathrm{As}_\ast = \{a_\ast, \xi_\ast, \beta_+, \beta_- \}$ and maps them to Lagrangian parameters $\mathrm{As}_\mathcal{L} = \{ M, \mu, \lambda, \kappa \}$ as described in~\nc{some section or appendix}. We initialise the fields $\mathtt{achi}\cor \chi$ and $\mathtt{q} \cor q=a^2 \dot{\chi}$ by providing the code with suitable \blahblah at $\redshift\ped{i}$.


%     % The code allows us to choo,se initial time in redshift




%     \comment{$L$ and $\redshift\ped{i}$ difficult to place in category.}

    
%     \paragraph{Physics.} %
%     More exciting are the parameters that directly relates to our theoretical setup and cosmological scenarios. %We can set the simulation box size, $L$, 
%         \subparagraph{Asymmetron parameters.} %
%         The code takes the phenomenological parameters $\mathrm{As}_\ast = \{a_\ast, \xi_\ast, \beta_+, \beta_- \}$ and maps them to Lagrangian parameters $\mathrm{As}_\mathcal{L} = \{ M, \mu, \lambda, \kappa \}$ as described in~\nc{some section or appendix}.


%         \subparagraph{Perturbation parameters.} % 
%         We added some options to the initialisation-part of the code so to easily create the idealised scenario we want. It has options for this kind of perturbation in the $z$-direction:
%         \begin{equation}
%             \epsilon = \varepsilon(\tau) \mathop{\mathrm{tri}}{\cclosed{ p_x x + p_y y }}; \quad \mathop{\mathrm{tri}} \in \{ \sin, \cos \},
%         \end{equation}
%         for which the user can provide initial amplitude $\epsast$ and perturbation scale in terms of scaled, integer wavenumbers $m_{\lcoord{i},\lcoord{j}} = p_{x,y} /k_\#$. 

%         We introduce a ``curvature'' parameter $\Upsilon^{\AC}=  100m_{\lcoord{j}}\epsilon /L_\#$ which will be useful for later discussion. 


        
%     We need to be careful \blahblah

% \end{draft} 





% % \comment{Surface tension ``tension'': Sensitivity to this.}

