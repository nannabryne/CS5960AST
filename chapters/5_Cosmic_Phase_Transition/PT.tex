%%%%%%%%%%%%%%%%%%%%%%%%%%%%%%%%%%%%%%%%%%%%
%%%%%% Ch. 5: Cosmic Phase Transition %%%%%%
%%%%%%%%%%%%%%%%%%%%%%%%%%%%%%%%%%%%%%%%%%%%


% --------------------------------
% labels: \label{[type]:PT:[name]}
% --------------------------------


% ¨¨¨¨¨¨¨¨¨¨¨¨¨¨¨¨¨¨¨¨¨¨¨¨¨¨¨¨¨¨¨¨¨¨¨¨¨¨¨¨¨¨
% LOCAL MACROS:
\newcommand{\lcoord}{\ALIASlcoord} % code lattice coordinate
\newcommand{\lcoordx}{\ALIASlcoordx} % code lattice coordinate
\newcommand{\lcoordk}{\ALIASlcoordk} % code lattice coordinate
\newcommand*\Ft{\ALIASFt}
\newcommand\dummy{\ALIASdummy}
\newcommand*\Tw{\ALIASTw}
\newcommand*\Twf{\ALIASTwf}
% ¨¨¨¨¨¨¨¨¨¨¨¨¨¨¨¨¨¨¨¨¨¨¨¨¨¨¨¨¨¨¨¨¨¨¨¨¨¨¨¨¨¨






% ////////////////// intro //////////////////

% \begin{bullets}
%     \item Testing the framework
%     \item Different theory
%     \item Comment on annihilation?
%     \item First, go through the field description (modified gravity, Symmetron)
% \end{bullets}





Discrete symmetry breakage leads to formation of domain walls~\citep{vilenkinCosmicStringsOther1994}. %
We may apply thin-wall dynamics when the thickness parameter $\delta$ is negligible compared to other dimensions of the problem, primarily the horizon size~\citep{vilenkinCosmicStringsOther1994}. Otherwise, we must employ the full field theory. % where the scalar field equation reads $\sq \phi = V_{,\phi}$
During a phase transition in which the dynamics is governed by a reflection-invariant Lagrangian in $3+1$ dimensions, the system quickly divides into alternating positive and negative vacuum domains. %
Separating these are domain walls, which during transition is thinning from initially (as we will see) infinite thickness.\footnote{Rather a naive extrapolation of the thickness interpretation, in~\cref{eq:PT:symm_dws:surface_tension_and_thickness_FLRW}.}  

% The thin-wall approximation only holds if the thickness parameter $\delta$ is negligeble 
% The thin-wall approximation is not suited to describe the dynamics at this stage, and is \cringe{certainly not able to tell the whole story.} 

% A Lagrangian that is reflection-invariant in a scalar field theory, and then suddenly is not any more, describes a system that undergoes a phase transition. In the case of $3+1$ dimensions, the system is quickly divided into alternating positive and negative vacuum domains. Separating these are stable domain walls---the newborn soliton solution for the scalar field. During the early stages of formation, a domain wall in theory reduces its thickness from initially infinite to $\sim \delta_\infty = \mathscr{O}(\mu^{-1})$---a source of various problems both simulation-wise and for the theory. 

% These walls may annihilate in the presence of an asymmetry \comment{Or if is an approximation?}, a feature that would weaken the theoretical arguments \comment{}

% \pensive{If these walls could talk\dots}



% \pensive{Possible title of thesis: \textbf{``If Domain Walls Could Talk''} or \textbf{``Gravitational Response if Domain Walls Could Talk''} or \textbf{``If Domain Walls Could Talk, What Would Gravitational Waves Say?''}}


% \pensive{Content: The Watcher on The Wall, }


% A variety of challenges present themselves when simulating cosmological scenarios. \blahblah


% \comment{Say something about flat expanding spacetime!}



In this chapter we attack the field-theoretical approach to describing the motion of domain walls, during and after phase transition. We eventually take leave of one spatial dimension, and so the framework can also describe strings~\citep{blanco-pilladoDynamicsDomainWall2023}. %
Previous work~\citep{blanco-pilladoDynamicsDomainWall2023} has been done on Minkowski background in the static picture. The following analysis assumes a conformally flat, homogeneous and isotropic background, particularly one with scale factor $a\propto \tau^\alpha$. We consider the entire phase transition, i.e.~the actual formation of defects, which alters the dynamics in the Nambu--Goto picture. %
We build up towards simulations of toy scenarios for which we set $\alpha=2$.


This chapter first, in~\cref{sec:PT:symm_dws}, presents the scalar-field formulation of the defect and defect formation. We address the code that was used in~\cref{sec:PT:code} and how we design the simulative experiments in~\cref{sec:PT:gwas}. In~\cref{sec:PT:sims} we present details about the particular simulations we perform.



% \comment{Important refs: \citep{blanco-pilladoDynamicsDomainWall2023}}

% ///////////////////////////////////////////



% ****************** SECTIONS ******************

\section{\(\Zn\) symmetry-break}\label{sec:PT:symm_dws}
    {\subimport{./}{symm_dws.tex}}


\section{Dynamic modelling}\label{sec:PT:code} %
    {\subimport{./}{code.tex}}

% \section{\tmptitle{\texttt{gwasevolution} (On the lattice?)}}\label{sec:PT:gwas}
%     {\subimport{./}{gwas.tex}}

\section{\tmptitle{Toy model design}}\label{sec:PT:gwas}
% \section{\tmptitle{Designing the prototype}}\label{sec:PT:gwas}
    {\subimport{./}{gwas.tex}}


\section{Simulation setups}\label{sec:PT:sims}
    {\subimport{./}{sims.tex}}

% **********************************************

% \begin{draft}
%     %
%     A variety of challenges present themselves when simulating cosmological scenarios. 


%     %
%     \section*{Symmetron}
%         From~\cref{sec:CFTgrav:symmetron} we have the (a)symmetron effective potential and \dots

        
%         Using the symmetron model to represent a domain wall in an FRLW universe, $\sq \phi=V\ped{eff,\phi}$ becomes
%         \begin{equation}
%             -a^{-2} \bclosed{ \ddot{\phi} + 2\mathcal{H} \dot{\phi} - \vec{\nabla}^2\phi } = \lambda \phi^3 - \mu^2 \pclosed{ 1 - \upsilon}\phi.
%         \end{equation}
%         From here, we will use $\chi=\phi/\phi_\infty$ and $\chi_\pm=\sqrt{1-\upsilon}$. We recall that $\upsilon=\rho\ped{m}/\rho_{\mathrm{m}\ast} = (a_\ast/a)^3$. 

%         Prior to SSB, the scalar field is trivial, and so we move on to consider $\chi$ from this critical point.

%         \subsection*{Quasistatic limit}
%             We can solve
%             \begin{equation}
%                 \vec{\nabla}^2 \chi =  \lambda \phi_\infty^3 \cdot a^2\bclosed{ \chi^2 - (1-\upsilon) }\chi
%             \end{equation}
%             to obtain the solution in the limit where spatial gradient plays a much larger role that time derivatives. We let $\chi=\chi(a, z)$ and use the well-established~\citep[see e.g.][]{llinaresDomainWallsCoupled2014} expression for a domain wall when $a\propto \eta^\alpha$
%             \begin{equation}
%                 \chi(a, z) = \sqrt{1-\upsilon} \tanh{\pclosed{ \frac{az}{2L\nped{C}} \sqrt{1-\upsilon}}}.
%             \end{equation}
%             We write $a=a_\ast (\eta/\eta_\ast)^\alpha$, which gives $\upsilon=(a_\ast/a)^3=(\eta_\ast/\eta)^{3\alpha}$.


%         \subsection*{Asymptotic limit}
%             We let $\breve{\chi}$ denote the field value far away from the wall, well inside the domain.


%         % \nc{need ref to some sec. about (a)Symmetron}



%         % The field-like description of an Asymmetron domain wall \dots
%         % \begin{equation}
%         %     S\ped{As} = \integ[4]{x\sqrt{-g}} \cclosed{ \frac{1}{2}M\nped{Pl}^2 \mathcal{R} - \frac{1}{2}\nabla\_\mu \phi \nabla \phi  - V(\phi)  } + S\ped{m}
%         % \end{equation}

%         % \blahblah

%         % We are left with the eom
%         % \begin{equation}
%         %     \sq \phi = \dv{V\ped{eff}}{\phi}
%         % \end{equation}
%         % with
%         % \begin{equation}
%         %     V\ped{eff}(\phi) = \frac{\lambda}{4}\phi^4 - \frac{\kappa}{3}\phi^3 -\frac{\mu^2}{2} \pclosed{1-\upsilon  } \phi^2 ,
%         % \end{equation}
%         % where $\upsilon= \rho\ped{m}/(\mu^2M^2)$. Now
%         % \begin{equation}
%         %     {V\ped{eff}}\_{,\phi}= \lambda\phi^3 + \kappa \phi^2 - \mu^2 \pclosed{1- \upsilon }\phi,
%         % \end{equation}
%         % which equated to zero gives the vacuum expectation values
%         % \begin{equation}
%         %     \phi_0 = 0 \quad \lor \quad \phi_\pm = \phi_\infty \pclosed{\bar{\kappa} \pm  \sqrt{\bar{\kappa}^2 +  1- \upsilon }}\phi%\frac{\kappa \pm \sqrt{\kappa^2 + 4\lambda\mu^2 \pclosed{1- \rho\ped{m}/(\mu^2M^2)} } }{2\lambda}
%         % \end{equation}
%         % where we defined $\bar{\kappa} = \kappa / (2\mu \lambda) $ and $\phi_\infty = \mu/\sqrt{\lambda}$.
        



%         % % Using $g\_{\mu\nu} = a^2 \eta\_{\mu\nu}$, we get
%         % % \begin{equation}
%         % %     \ddot{\phi} + 2 \mathcal{H} \dot{\phi} - \vec{\nabla}^2 \phi = -a^2 \pclosed{  \lambda\phi^3 + \kappa \phi^2 - \mu^2 \pclosed{1- \frac{\rho\ped{m}}{\mu^2M^2} } }
%         % % \end{equation}
%         \subsection*{Asymptotic solution}
%             We look at the eom for $\phi$ where spatial gradients are neglected. 
        
        









% \end{draft}