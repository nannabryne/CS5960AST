% |||||||||||||||||||||||||||||||||||
% |||||| 5.X Dynamic modelling ||||||
% |||||||||||||||||||||||||||||||||||

% ------------------------------------------------
% labels: \label[type]{[type]:PT:dynamics:[name]}
% ------------------------------------------------


% When translating a theory consisting of nice, continuous functions with theoretical limits to a discrete, numerical system, there are \blahblah. When said theory also includes a phase transition, it does not become any easier. Phase transitions are manifestly computational headaches, and we are limited by \blahblah. The discontinuity introduced by the phase transition is even more complicated to replicate in simulations. 


% From the field-theoretical perspective 


The list of assumptions and terms starting with ``quasi'' has become uncomfortably long. Most of these have been tested before or are otherwise motivated by similar established phenomena in Minkowski space. %the ``Why not?''-aspect. 
We want to test how the thin-wall NG dynamics associates with the field-theoretically formulated dynamics of a cosmic phase transition, explicitly in a generic spacetime. Cosmological contexts seldom care about spacetimes other than FLRW. The effect of curvature is insignificant in the Friedmann equations, thus irrelevant in \emph{our} universe, and we can safely assume (conformal) flatness. \comment{Bold statement, no?}




% Hypothetically non-flat universes obey the same  

% They also welcome the flatness assumption

% , and as the flatness assumption is more than welcome,




% \phpar[The code (gevolution etc.) (maybe own section?)]

% \phpar[BCs]


% We will have a look some of the most pressing matters when aiming to test the theory from~\cref{chap:pertwalls} in simulations. 



% \subsection{Spatial and temporal resolution}



% \subsection{Memory}




