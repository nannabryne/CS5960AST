%%%%%%%%%%%%%%%%%%%%%%%%%%%%%%%%%%%%%%%%
%%%%%% Ch. 3: Kinks in Cosmology  %%%%%%
%%%%%%%%%%%%%%%%%%%%%%%%%%%%%%%%%%%%%%%%


% -----------------------------------
% labels: \label{[type]:cosmo:[name]}
% -----------------------------------




% ////////////////// intro //////////////////







% \phpar[about modified gravity etc.--reason for modification--auxiliary scalar fields and expansion]


% Scalar field theories 



% Candidate theories for physics beyond the \LCDM~model include modified gravity theories and quintessence. %

 
% A possible consequence of theories that predict phase transitions is the creation of topological defects such as cosmic strings and domain walls. 





% Symmetry breaking occurs during phase transitions more often than not, and the distinction between these closely related concepts is insignificant \grammar[for our purposes]{in this context}. Some theories that exhibit such a feature predicts the existence of~\emph{topological defects}.  %(which we will discuss shortly). %\blahblah
% \paragraph{Topological defects.} %



When a system spontaneously changes from one physical state to another due to external conditions such as temperature or pressure reach certain critical points, it is said to undergo a \newconcept{phase transition}. Water for example changes from liquid to gas at the critical temperature $100^\circ\unit{C}$, given atmospheric pressure. Spontaneous symmetry breaking (SSB), an important concept in especially particle physics and cosmology, occurs when the physical laws possess a certain symmetry, but the vacuum state does not~\citep{kolbEarlyUniverse1990}.

%
%
% The existence of topological defects is predicted by several theories in various branches of physics, such as particle physics, cosmology and material sciences.
%
Topological defects, or topological solitons, are ubiquitous and manifest in a variety of forms across different scales and systems. Their hypothesised existence in exotic environments such as the early universe is further motivated by proofs of their existence in everyday life, such as whorls, loops and arches in our fingerprints~\citep{fumeronIntroductionTopologicalDefects2023}. % Topological defects appear 
% We find them in the irregular structure of foams in our household cleaning products. 
% Whorls, loops and arches in fingerprints are examples of topological defects in the patterns formed by our own skin~\citep{fumeronIntroductionTopologicalDefects2023}, and 
% On larger scales there are for instance fault lines between tectonic plates. 
It was in fact a very mundane event in 1834 that induced the first formal description of a~\newconcept{soliton}; John Russell Scott (1808--1882) keenly observed a hump of water moving from the prow of a boat rapidly down a canal whilst preserving its shape~\citep{vachaspatiKinksDomainWalls2006}. Such solitary waves that can interact with others and restore its shape were later named ``solitons''.






% ///////////////////////////////////////////





% \begin{bullets}
%     \item SECTIONS: %
%         \begin{description}
%             \item[Quintessence:] Auxiliary fields
%             \item[Domain walls:]
%         \end{description}
% \end{bullets}










% ****************** SECTIONS ******************


% Topological defects
\section{Cosmological defects}\label{sec:cosmo:defects}
{\subimport{./}{defects.tex}}


% Quintessence
\section{\tmptitle{Quintessence/Scalar field theories}}\label{sec:cosmo:quintessence}
{\subimport{./}{quintessence.tex}}


\section{Classification of topological defects}
    \iftime{Write short about classification, homotopy group, etc., \citep{vilenkinCosmicStringsOther1994,fumeronIntroductionTopologicalDefects2023}}



% **********************************************





% \clearpage
% \newpage

% \begin{draft}
% {
% \section{Symmetron model}




% The general picture presents quintessence as a group of scalar--tensor theories---a subgroup of modifications to gravity in which a scalar field is added to the total action. By going through all possible covariants with maximum second order time derivatives in four dimensional spacetime, one arrives at the most general formulation of these type of theories, the \emph{Hordenski theory}. Said theory is summed up by the total Lagrangian density $\mathcal{L}\ped{tot} = \mathcal{L}\ped{m} + M\nped{Pl}^2 \sum_{i=2}^{5} \mathcal{L}_i$, where $\mathcal{L}_i$ are built up by derivatives of the scalar field, the Einstein tensor and arbitrary functionals of the scalar field and its kinetic term.\footnote{%
% With the usual notation convention and arbitrary functionals $F_i$, the Lagrangian densities $\mathcal{L}_i$ are given by the following:
% \begin{equation*}
%     \begin{split}
%         \mathcal{L}_2 &= F_2(\phi, X) \\
%         \mathcal{L}_3 &= F_3 (\phi, X)\sq \phi \\
%         \mathcal{L}_4 &= F_4 (\phi, X)\mathcal{R} + F_{4,X}(\phi, X) \pclosed{ {(\sq \phi)}^2 - \phi\_{;\mu\nu} \phi\^{;\mu\nu}} \\
%         \mathcal{L}_5 &= F_5 \mathcal{G}\_{\mu\nu}\phi\^{;\mu\nu} - \sfrac{1}{6} F_{5,X}(\phi, X)  \pclosed{ {(\sq \phi)}^3 + 
%         2\phi\indices{^\nu_{;\mu}}\phi\indices{^\mu_{;\nu}} - 3  \phi\_{;\mu\nu} \phi\^{;\mu\nu} \sq \phi} 
%     \end{split}
% \end{equation*}%
% } %
% General relativity is retrieved with $\mathcal{L}_4 = \mathcal{R}/2$ and $\mathcal{L}_{2,3,5}=0$. Adding $\mathcal{L}_2 = X(\phi)-V(\phi)$ gives quintessence. Performing relevant variations eventually gives an equation of motion that is indistinguishable from standard GR with $\mathcal{L}\ped{m} \to \mathcal{L}\ped{m} + \mathcal{L}_2$ in the stress--energy tensor on the rhs of the Einstein equation. Thus, quintessence models are not modified gravity theories, but adds to the total matter in the universe. 


% \paragraph{Conformal duality / Jordan \& Einstein frames.} %
% (Some text about these.) Jordan frame (Jf.) and Einstein frame (Ef.) 


% \paragraph{Screening.} %
% \blahblah


% \begin{bullets}
%     \item Restructure of this chapter
%     \item Remember: origin of symmetron, screening, firth force, asymmetron, phase transition
% \end{bullets}


% The origin of the symmetron model is \blahblah


% The $\mathsf{Z}_2$ symmetry $\phi \to -\phi$ is broken by the symmetron potential $V(\phi)=\lambda\phi^4 / 4 - \mu^2\phi^2/2$. \citep{hinterbichlerScreeningLongRangeForces2010}


% The \textit{a}symmetron model is a generalisation of the symmetron in which a cubic term is added to the potential. A nonzero cubic term corresponds to a system where one potential well is deeper than the other. We write the asymmetron \checkthis{vacuum} potential as
% %
% % \begin{equation}\label[eq]{eq:CFTgrav:symmetron:asymmetron_potential}
% %     V(\phi) = \frac{\lambda}{4} \phi^4 + \frac{\kappa}{3}\phi^3 - \frac{\mu^2}{2} \phi^2
% % \end{equation}
% % \begin{empheq}[box={\mybluebox[2pt][2pt]}]{equation}%\label[eq]{eq:CFTgrav:symmetron:asymmetron_potential}
% %     V(\phi) = \frac{\lambda}{4} \phi^4 + \frac{\kappa}{3}\phi^3 - \frac{\mu^2}{2} \phi^2
% % \end{empheq}
% \boxedeq{eq:CFTgrav:symmetron:asymmetron_potential}{%
% V(\phi) = \frac{\lambda}{4} \phi^4 - \frac{\kappa}{3}\phi^3 - \frac{\mu^2}{2} \phi^2 + V_0%
% }
% for which the symmetron is retrieved when $\kappa=0$. In~\cref{fig:CFTgrav:symmetron:test} we demonstrate the impact of varying each coupling term separately.
% % \begin{figure}[h]\label{fig:CFTgrav:symmetron:test}
% %     \import{figs/}{demo002.tex}
% % % \end{figure}
% % \begin{figure}[h]\label{fig:CFTgrav:symmetron:test}
% %     \centering
% %     \includegraphics[width=\linewidth]{asymmetron_demo.pdf}
% %     %%%%%%%%%%
% %     \caption{Asymmetron potential for different parameter choices.}
% %     %%%%%%%%%%%%%
% % \end{figure}


% % General framework

% \subsection{The symmetron action}
% % 

% % \paragraph{Jordan \& Einstein frames.} %
% % (Some text about these.) Jordan frame (Jf.) and Einstein frame (Ef.) 


% % The action
% We begin with the general chameleon action in the Einstein frame
% \begin{equation}
%     S =S\nped{SE} + S_\phi  + S\ped{m}= \integ[4]{x\sqrt{-g}} \cclosed{ \frac{M\nped{Pl}}{2} \mathcal{R} + X(\phi) - V(\phi)} + S\ped{m}[\tilde{g}\_{\mu\nu} , \psi]
% \end{equation} 
% where $X = -\frac{1}{2} \phi\^{;\mu} \phi\_{;\mu}$ %$X= -\frac{1}{2} \nabla\^\mu \phi \nabla\_\mu \phi $
% is viewed as the kinetic energy of the chameleon, and $S\ped{m}$ is the Jordan frame matter action. In truth, the Jf. fields $\tilde{g}\_{\mu\nu}$ and $\psi$ are% \emph{sums} over \blahblah


% The matter fields couple to the Jordan frame metric $\tilde{g}\_{\mu\nu} = A^2(\phi)g\_{\mu\nu}$. 

% In Jf., the SE tensor $\tilde{T}\_{\mu\nu} = -\pclosed{2/\sqrt{-\tilde{g}}} \Fdv*{\mathcal{L}\ped{m}}{\tilde{g}\^{\mu\nu}}$ is covariantly conserved; $\tilde{\nabla}\_{\mu}\tilde{T}\indices{^{\mu}_{\nu}}=0$. By varying the Ef. action, we find the symmetron obeys
% \begin{equation}
%     \sq \phi -V_{,\phi} + A^3(\phi) A_{,\phi} \cdot\tilde{g}\^{\mu\nu} \tilde{T}\_{\mu\nu}=0 .
% \end{equation}



% We let $A(\phi)=1+\Delta A= 1 + \phi^2/(2M^2)$ be the (a)symmetron conformal factor. The Ef. energy density $\rho=A^3\tilde{\rho}$

% \paragraph{Matter density in Einstein frame.} %
% % Assuming there is no interaction between the matter fields $\phi_i$, 
% The Jf. SE tensor is covariantly conserved; $\tilde{\nabla}\_{\mu}\tilde{T}\indices{^{\mu}_{\nu}}=0$. Several systems, such as galaxies, allow us to assume  ....
% We then find the trace $\tilde{g}\^{\mu\nu} \tilde{T}\_{\mu\nu}= -\sum_i \tilde{\rho}_i(1-w_i) =-\tilde{\rho}$, and the eom for $\phi$ becomes
% \begin{equation}
%     \sq \phi = V_{,\phi} + \rho A_{,\phi}.
% \end{equation}
% It is customary to define the effective potential s.t. $\sq \phi = V\ped{eff,\phi}$, i.e.
% \begin{equation}
%     V\ped{eff}(\phi) = V(\phi) + \rho A(\phi) =  \frac{\lambda}{4} \phi^4 - \frac{\kappa}{3}\phi^3 + \frac{\mu^2}{2}\pclosed{\frac{\rho}{\mu^2M^2}-1} \phi^2 + V_0.
% \end{equation}
% \blahblah
% This potential becomes unstable when $\rho\leq \mu^2M^2\equiv \rho_\ast$, and the field rolls into either of the two vacua. 


% \paragraph{Symmetron potential.} %
%     The (a)symmetron effective potential captures the complete phase transition. Let $\upsilon\equiv\rho\ped{m}/(\mu^2M^2)$. By setting $V\ped{eff,\phi}=0$, we find the vacuum expectation values
%     \begin{equation}
%         \phi_0 = 0 \quad \lor \quad \phi_\pm = \phi_\infty \pclosed{\bar{\kappa} \pm  \sqrt{\bar{\kappa}^2 +  1- \upsilon }},%\frac{\kappa \pm \sqrt{\kappa^2 + 4\lambda\mu^2 \pclosed{1- \rho\ped{m}/(\mu^2M^2)} } }{2\lambda}
%     \end{equation}
%     where we defined $\bar{\kappa} = \kappa / (2\mu \sqrt{\lambda}) $ and $\phi_\infty = \mu/\sqrt{\lambda}$. Note that for the symmetron ($\kappa=0$), since the field is real, VEV is zero before SSB. We determine the stability of these vacua by evaluating $V\ped{eff,\phi\phi}$ at $\phi=\phi_0,\phi_\pm$ and see that $\phi_0$ remains stable until $\rho_\ast$. 


%     % \begin{enumerate}[label=(\roman*)]
%     %     \item before SSB ($\rho>\rho_\ast$) VEV is 0
%     %     \item at SSB ($\rho=\rho_\ast$) VEV is 0 (unstable)
%     %     \item after SSB ($\rho<\rho_\ast$) VEV are $\phi_\pm$
%     % \end{enumerate}

% %     the symmetry break at $\rho_\ast$ such that 


% % The field-like description of an Asymmetron domain wall 
% % \begin{equation}
% %     S\ped{As} = \integ[4]{x\sqrt{-g}} \cclosed{ \frac{1}{2}M\nped{Pl}^2 \mathcal{R} - \frac{1}{2}\nabla\_\mu \phi \nabla \phi  - V(\phi)  } + S\ped{m}
% % \end{equation}

% % \blahblah

% % We are left with the eomx
% % \begin{equation}
% %     \sq \phi = \dv{V\ped{eff}}{\phi}
% % \end{equation}
% % with
% % \begin{equation}
% %     V\ped{eff}(\phi) = \frac{\lambda}{4}\phi^4 - \frac{\kappa}{3}\phi^3 -\frac{\mu^2}{2} \pclosed{1-\upsilon  } \phi^2 ,
% % \end{equation}
% % where $\upsilon= \rho\ped{m}/(\mu^2M^2)$. Now
% % \begin{equation}
% %     {V\ped{eff}}\_{,\phi}= \lambda\phi^3 + \kappa \phi^2 - \mu^2 \pclosed{1- \upsilon }\phi,
% % \end{equation}
% % which equated to zero gives the vacuum expectation values
% % \begin{equation}
% %     \phi_0 = 0 \quad \lor \quad \phi_\pm = \phi_\infty \pclosed{\bar{\kappa} \pm  \sqrt{\bar{\kappa}^2 +  1- \upsilon }}\phi%\frac{\kappa \pm \sqrt{\kappa^2 + 4\lambda\mu^2 \pclosed{1- \rho\ped{m}/(\mu^2M^2)} } }{2\lambda}
% % \end{equation}
% % where we defined $\bar{\kappa} = \kappa / (2\mu \lambda) $ and $\phi_\infty = \mu/\sqrt{\lambda}$.




% % \subsection{Fifth-force}
% %     \blahblah

%     }
% \end{draft}