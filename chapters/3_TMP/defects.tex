% |||||||||||||||||||||||||||||||||||||||||||||
% |||||| 3.1 Topological defects ||||||
% |||||||||||||||||||||||||||||||||||||||||||||


% -------------------------------------------
% labels: \label{[type]:cosmo:defects:[name]}
% -------------------------------------------




\begin{bullets}
    \item Cover: motivation, classical kink solution (+ translational symmetry +antikinks+ series of kinks)
    \item about topological solitons / defects
    \item symmetry breaking
    \item Properties: NG action, width, mass/tension
\end{bullets}

\phpar[about topological defects]

\rephrase{Kink solutions placed in spacetime with more than one spatial dimension, they become extended, planar structures (or membranes), that which we call ``domain walls''. \cite{vachaspatiKinksDomainWalls2006}} Domain walls possess richer dynamics than kinks, and \blahblah 

\pensive{Is there a difference between topological solitons and topological defects?}



The only type of topological defect that is directly relevant to this project is the domain wall, specifically the $\mathsf{Z_2}$ type. These are two-dimensional topological defects that occurs where a discrete symmetry is broken.\footnote{%
Likewise, cosmic strings and monopoles are products of axial/cylindrical and spherical symmetry breaking, respectively.%
} %
% A domain wall is in many ways a \emph{kink} in $3+1$ dimensions, and a kink 



To give an idea of the basic properties of topological defects, we present an example. More thorough derivations can be found in~\citet{vachaspatiKinksDomainWalls2006}. \comment{Later on, we will derive it for DWs in FRW spacetime.}

\subsection{Example: (stationary) \( \Zn \) kinks}\label{sec:cosmo:defects:ex_Z2_kink}
    The king of kinks, the so-called ``\Zn~kink,'' can be described through a scalar field $\phi$ with the action $S = S\nped{EH} + S\ped{\Zn}$,
    \begin{equation}\label{eq:cosmo:defects:Z2_action}
        % S = \integ[2]{x} \cclosed{ \frac{1}{2} \partial\^\mu \phi \partial\_\mu \phi - V(\phi) },
        S\ped{\Zn} = \integ[n+1]{x}\sqrt{-g} \cclosed{ -\frac{1}{2} g\^{\mu\nu} \partial\_\mu \phi \partial\_\nu \phi - V(\phi) },
    \end{equation}
    where \( V(\phi) \) is the two-fold degenerate potential $V(\phi)=\lambda (\phi^2-\eta^2)^2 $. % $V(\phi)=\lambda \phi^4 /4 - \mu^2 \phi^2 /2$. %
    The eom $\sq \phi = V_{,\phi}$ can be derived from variation of $S$ with respect to $\phi$. For simplicity, we consider Minkowski spacetime with $1+1$ dimensions where $\eta\_{\mu\nu} = \text{diag}=(-1,1)$ is the metric. The eom reads
    \begin{equation}
        -\partial_t^2 \phi  + \partial_x^2 \phi = \lambda (\phi^2 -\eta^2)\phi.
    \end{equation}
    Setting time derivatives to zero, and imposing boundary conditions $\phi(x\to \pm \infty)= \pm \eta$, we obtain a class of static solutions
    \begin{equation}\label{eq:cosmo:defects:phi_k_Z2}
        \phi\ped{k}(x;x_0)=  \eta \tanh{\pclosed{\sqrt{\frac{\lambda}{2}}\eta (x-x_0) }},
    \end{equation}
    where $x_0$ is the position of the kink. \comment{Translational invariance $\phi\ped{k}(x;x_0)=\phi\ped{k}(x-x_0)$}
    

    % \paragraph{Topological charge.} %
    % There is a conserved current \( j\^\mu \) gives rise to 


    % \subsubsection{System of kinks}
    \paragraph{Multi-kink field.} %
    Without commenting further, we state that this kink has \emph{topological charge} $Q=1$ (\citet[see][Ch.~1]{vachaspatiKinksDomainWalls2006} for discussion). This comes from the boundary conditions, and thus similar arguments constructs solutions with $Q=-1$ by swapping the boundaries; $\phi(x\to \pm \infty) = \mp \eta$. This is the \emph{antikink} solution $\widebar{\phi}\ped{k}(x)=- \phi\ped{k}(x)$. A feature of the \Zn~kinks is that one cannot have a system with topological charge $\abs{Q}>1$. For sufficiently separated kinks and antikinks located at $x_i$ and $\widebar{x}_j$, respectively, we write \citep{vachaspatiKinksDomainWalls2006}:
    \begin{equation}\label{eq:cosmo:defects:many_kinks}
        \phi(x) = \frac{\eta}{\eta^{N+M}} \prod_{i}^{N} \phi\ped{k}(x-x_i) \prod_{j}^{M} \widebar{\phi}\ped{k}(x-\widebar{x}_j),
    \end{equation}
    where $\abs{N-M}\leq 1$ and $x_i < \widebar{x}_j < x_{i+1}$. This describes the allowed system of $N$ kinks and $M$ antikinks aligned in an alternating structure.


    \subsubsection{Basic properties}

        The energy of the kink is obtained integrating over the energy density, i.e.% $E = \integ{x}T\indices{^0_0}$
        \begin{equation}\label{eq:cosmo:defects:energy_of_Z2_kink}
            E = \integ{x} T\indices{^0_0} = \frac{2\sqrt{2 }}{3} \lambda \eta^3.
        \end{equation}
        \comment{Check sign.}
        We define the width of the kink to be the argument where the tanh function equals $\tanh{1/\sqrt{2}}$,
        \begin{equation}
            w = \frac{1}{\eta\sqrt{\lambda}}.
        \end{equation}
        Most of the energy is confined within $x \in x_0 +[-w/2, w/2]$. See~\cref{fig:cosmo:defects:Z2_kink_demo} for illustrative explanation.
        \begin{figure}[H]\label{fig:cosmo:defects:Z2_kink_demo}
            \centering
            \includegraphics[width=\linewidth]{Background/Z2_kink_demo.png}
            \caption{Demonstration of the $\Zn$ kink \blahblah.}
        \end{figure}

    

        % From 
        % \citet{vachaspatiKinksDomainWalls2006} defines the 





\subsection{\tmptitle{Domain walls}}\label{sec:cosmo:defects:dws}
    The kink solution in~\cref{sec:cosmo:defects:ex_Z2_kink} put in two more spatial dimensions (setting $\eta\_{\mu\nu}=\text{diag}(-1,1,1,1)$) is a planar $\Zn$ domain wall. The energy in~\cref{eq:cosmo:defects:energy_of_Z2_kink} is now a surface energy density better known as the \emph{surface tension} of the wall, denoted $\sigma$. 


    
    For later convenience, we will define $\sigma_\infty$ and $\delta_\infty$ as the surface tension and wall width as their solutions in the stationary $\Zn$ scenario, respectively. In terms of the mass scale $\mu=\eta\sqrt{\lambda}$, this amounts to
    % For the stationary $\Zn$ wall, we have 
    \begin{equation}\label{eq:cosmo:defects:sigma_delta_inf}
        \sigma_\infty \equiv \frac{2\sqrt{2 }}{3} \frac{\mu^3}{\lambda} \quad\text{and}\quad \delta_\infty \equiv \frac{1}{\mu}.\footnote{In~\cref{part:method}, we set $\phi_\infty = \eta$.}
    \end{equation}




    % \subsubsection{Kink dynamics in the thin-wall limit}
        \phpar[Nambu-Goto action]






% \subsection{Stess--energy tensor}






\subsection{Defect formation}\label{sec:cosmo:defects:formation}
    \comment{Maybe own section?}

    \begin{bullets}
        \item Effective potential
        \item First-order phase transitions
        \item Biased phase transitions?
    \end{bullets}



    Defect formation, symmetry breaking and phase transitions are tightly related phenomena. \comment{Write in general terms.}
    To study phase transitions it is helpful to use the \emph{effective potential} that takes into account the interaction between some field and a background. An effective potential can ensure that a scalar field does not really show itself until some critical point at which the vacuum state goes from being trivial to \blahblah



    \subsubsection{Energy bias}
        Existence of discrete vacua implicates existence of domain walls. The degeneracy of these vacua ensures the stability of such walls. If we imagine a slight break in this degeneracy, that is, if one vacuum is favoured over another, biased domain walls form~\citep{vachaspatiKinksDomainWalls2006}. \blahblah







    



    
    




    