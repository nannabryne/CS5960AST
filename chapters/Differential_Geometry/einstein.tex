% |||||||||||||||||||||||||||||||||||||
% |||||| 2.X Einstein's equation ||||||
% |||||||||||||||||||||||||||||||||||||

% ----------------------------------------------
% labels: \label{[type]:diffgeo:einstein:[name]}
% ----------------------------------------------




% \blahblah intro


How does the gravitational field affect how matter behaves, and in what way is matter controlling the gravitational field? Newtonian gravity proposes very good answers to these questions: The acceleration of an object in a gravitational potential $\Phi$ is
\begin{equation}
    \vec{a} = -\vec{\nabla} \Phi,
\end{equation}
and said field is governed by the matter density $\rho$ through the Poisson equation
\begin{equation}
    \vec{\nabla}^2\Phi =  4 \ppi G\nped{N} \,\rho.
\end{equation}
% In physics, the answer you get is hugely dependent on \emph{how the question was asked}. 
In physics, the answer to a question is highly dependent on \emph{how the question was asked}. 
A common misconception is that Newtonian gravity was disproven by Einstein. Newton was simply telling a different story; a story about dynamics in non-relativistic systems.\footnote{Which, to be fair, are most common on Earth.} %Einstein's theory of general relativity proposes both different questions and answers to gravitational physics.
Einstein confronted gravitational physics with different but analogous questions, and subsequently more complex answers than Newton. General relativity explains how curvature of spacetime influences matter, manifesting as gravity, and in what way energy and momentum affects spacetime to create curvature. In mathematical terms, these are the geodesic equation 
\begin{equation}
    \dv[2]{x\^\mu}{\lambda} = - \Gamma\indices*{^\mu_{\rho\sigma}} \dv{x\^\rho}{\lambda} \dv{x\^\sigma}{\lambda} 
\end{equation}
and Einstein's equation
\begin{equation}
    \mathcal{G}\_{\mu\nu} = 8 \ppi G\nped{N} \, T\_{\mu\nu}.
\end{equation}
These can be obtained by 

\comment{Not too long chapter, but want to mention the ``naive'' tankegang from which these can be obtained (minimal coupling etc.).}

\citep[Ch.~4]{carrollSpacetimeGeometryIntroduction2019}


% Using equivalent argumentation as in the Newtonian formulation, we arrive at generalised versions of the aforementioned equations


% How does the gravitational field affect how matter behaves, and in what way is matter controlling the gravitational field? In the Newtonian formulation of gravitational physics, this is answered through
% \begin{enumerate}
%     \item the acceleration of an object in a gravitational field; $\vec{a}=-\vec{\nabla} \Phi$,
%     \item and the Poisson's differential equation for $\Phi$ in terms of matter density $\rho$; $\vec{\nabla}^2\Phi =  4 \ppi G\nped{N} \rho $.
% \end{enumerate}



% Newtonian gravity is 

% Gravitational physics in the Newtonian formulation  

% In the Newtonian formulation

% The Newtonian formulation of gravity is via a gravitational potential $\Phi$ obeying $\vec{a}=-\vec{\nabla} \Phi$ and $\vec{\nabla}^2\Phi =  4 \ppi G\nped{N} \rho $


% In physics, the answer to any question depends on how the question is asked. 