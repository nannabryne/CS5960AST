
%%%%%%%%%%%%%%%%%%%%%%%%%%%%%%%%%%%%%%%%%%%%%
%%%%%%% Ch. 2: Differential Geometry  %%%%%%%
%%%%%%%%%%%%%%%%%%%%%%%%%%%%%%%%%%%%%%%%%%%%%


% -------------------------------------
% labels: \label{[type]:diffgeo:[name]}
% -------------------------------------



% ////////////////// intro //////////////////



\phpar

To develop a classical field theory, we require a handful of mathematical \grammar[concepts]{structures} from differential geometry.

A classical field theory consists of the following mathematical structures:
\begin{description}
    \item[Spacetime] \dots
\end{description}

A spacetime is a \emph{smooth manifold} with or without additional mathematical structures.



The aim of this somewhat technical chapter is to provide the necessary mathematical tools to get a sufficient grasp of general relativity as a classical field theory.


% ///////////////////////////////////////////


\section{\tmptitle{Manifolds, tensors, etc.}}
    {\subimport{./}{manifolds.tex}}



\section{Riemannian geometry}\label{sec:diffgeo:riemann}
    {\subimport{./}{riemann.tex}}


\section{Conformal geometry}\label{sec:diffgeo:conformal}
    {\subimport{./}{conformal.tex}}





\section{Einstein's equation}\label{sec:diffgeo:einstein}
    {\subimport{./}{einstein.tex}}
