% !TEX root = ../../thesis.tex

% -----------------------------------
% labels: \label{[type]:[this chap key]:[name]}
% -----------------------------------

\tmptitle{Possibly different Ch. name: ``GWs from perturbed flat DW'' or something }
\tmptitle{``Dynamics of Domain Walls''??}

\begin{bullets}
    \item BenteBent (thin) $\to$ KatjaKaj (thick)
    \item \textit{A story of two brothers (?)}
\end{bullets}

We consider a relatively general four-dimensional spacetime $(\mathscr{M}, g\_{\mu\nu})$ with continuous metric and associated coordinate system $\{x\^\mu \}$. \blahblah \citep{ishibashiEquationMotionDomain1999} 


\section{Dynamics of Domain Walls in the Thin-Wall Limit}
{

Let the $(2+1)$-dimensional submanifold $\varSigma$ embedded in $\mathscr{M}$ represent the thin domain wall we henceforth shall refer to as ``BenteBent'' (or BB). This hypersurface divides the manifold into two separate regions ($\mathscr{M}_{\pm}$), allowing us to write $\mathscr{M} = \mathscr{M}_+ \cup  \varSigma  \cup \mathscr{M}_-$. Allow indices $a,b,c,\dots$ to run over $0,1,2$, and assign the coordinate system $\{y\^a\}$ to $\varSigma$. The wall location is $X\^\mu=X\^\mu (y\^a)$, such that any spacetime point is expressed
\begin{equation}
    x^\mu = X\^\mu (y\^a) + \xi N\^\mu (y\^a); \quad  \xi \in \Real, \,  N\^\mu \perp X\^\mu.
\end{equation}
Now, the world-volume metric
\begin{equation}
    \gamma\_{ab} = g\_{\mu\nu}\pdv{X\^\mu}{y\^a}  \pdv{X\^\nu}{y\^b}
\end{equation}
is the induced metric on $\varSigma$. \comment{Hmmm, maybe something wrong here.} \citep{carrollSpacetimeGeometryIntroduction2019} 



We write the covariant action of the domain wall BenteBent as
\begin{equation}
    S\ped{BB}  = -\sigma \integ[3]{y\sqrt{-\gamma}}[\varSigma] = -\sigma \integ[4]{x}[\mathscr{M}]  \Diracdelta[4]{x\^\mu-X\^\mu} \sqrt{-\gamma}; \quad \gamma = \det{\gamma\_{ab}},
\end{equation}
where $\sigma$ is the surface tension of the wall. 





}

\section{\tmptitle{IDK}}

\begin{multline}
    ah\_{ij}(\eta, \vec{k}) = \text{(const.)} \cdot  \Diracdelta{k\_2} \Diracdelta{\ell\in \Integer} \cdot K\_{ij}\cdot \eu[-\pclosed{w_0 k\_3}^2/2] \cdot I; \quad \ell = -k\_1 /u\_1 ; \\
    \text{(const.)}= 32\ppi^3 G\nped{N} \sigma_0 {\pclosed{a\ped{i}/\eta\ped{i}^\alpha}}^2,\quad
    K\_{ij} = k^{-6} {k\_1}^2 \bclosed{ \Krondelta{_{ij}} \pclosed{k^2-2k\_i k\_j} + k\_i k\_j},\\
    I = \integ{\tau'}[\tau\ped{i}][\tau] \mathcal{G}_{\nu}(\tau, \tau') {\tau'}^{2\nu+1} \Bessel[\ell]\pclosed{k\_3 \bar{\epsilon}(\tau'; u)};\quad \tau=k\eta, \, \nu = \alpha+1/2
\end{multline}