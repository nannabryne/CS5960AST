


\paragraph{Constants and units.} %
We use \checkthis{`natural units'} where $\hbar = c = 1$, where $\hbar$ is the reduced Planck constant and $c$ is the speed of light in vacuum.\comment{Planck units? Set $k\ped{B}=G\ped{N}=1$?} The Newtonian constant of gravitation $G\ped{N}$ is referenced explicitly, and we use Planck units such as the Planck mass $\Planck{M} = \pclosed{\hbar c /G\ped{N}}^{\shalf} = G\ped{N}^{-\shalf} \sim 10^{-8} \unit{kg}$. 
%

\paragraph{Tensors.} %
The metric signature $(-,+,+,+)$ is considered, i.e.~$\det~\!\![g\indices{_{\mu\nu}}] \equiv \abs{g} < 0 $. The Minkowski metric is denoted $\eta\indices{_{\mu\nu}}$, whereas a general metric is denoted $g\indices{_{\mu\nu}}$. A four-vector $p\indices{^{\mu}} = $

$[\eta\_{\mu\nu}] = \text{diag}(-1, 1, 1, 1)$

$f\_{,\mu}\equiv \partial\_\mu f = \pdv{f}{x\^\mu}$

\subparagraph[Gamma]{Christophel symbols.} %
The Christophel symbols or ``connections'' are written
\begin{equation}
    \ChristophelSym{\rho}{\mu\nu} = \frac{1}{2} g\^{\rho\sigma} \pclosed{ g\_{\mu\sigma,\nu} + g\_{\mu\sigma,\nu} - g\_{\mu\nu,\sigma}  }
\end{equation}

\subparagraph[Lambda]{``Lambda tensor.''} %
We express the \textit{Lambda tensor}---sometimes called the ``projection operator''---that projects onto the TT gauge \nc{refer to sec.} as
\begin{equation}
% \begin{split}
    \ProjectionLambda{ij}{kl}(\vec{n}) = P\_{ik}(\vec{n})P\_{jl}(\vec{n}) - \frac{1}{2}P\_{ij}(\vec{n})P\_{kl}(\vec{n});
    \quad P\_{ij}(\vec{n}) = \Krondelta{_{ij}}-n\_i n\_j %\land \vec{n}^2 = n\_1\!^2 + n\_2\!^2+ n\_3\!^2 = 1
% \end{split}
\end{equation}
% $\forall \vec{n}^2 = n\_1\!^2 + n\_2\!^2+ n\_3\!^2 = 1$
$\forall\, \vec{n}$ of unit length; $\vec{n}^2 = n\_1\!^2 + n\_2\!^2+ n\_3\!^2 = 1$. %Note that the notation is somewhat different from the convention 
% Other literature may use a comma (`,') instead of a 
We use a dot (`$.$') instead of the more conventional comma (`$,$') to distinguish from the Minkowskian partial derivative.
% $[\tens{\eta}] = \text{diag}(-1, 1, 1, 1)$

\paragraph{Fourier transforms.} %
We use the following convention for the Fourier transform of $f(x)$, $\tilde{f}(k)$, and its inverse, where $x$ and $k$ are Lorentz four-vectors:
\begin{equation}
    \begin{split}
        % f(x) &= \int \del[2 3]{k}  \eu[-\im k\cdot x] \tilde{f}(k) \\
        % f(x) &= \int \del[2]{k}  \eu[-\im k\cdot x] \tilde{f}(k) \\
        % f(x) &= \int \del{k}  \eu[-\im k\cdot x] \tilde{f}(k) \\
        f(x) &=  \integ[4][(2\ppi)^4]{k} \eu[-\im k\cdot x]\tilde{f}(k)  \\
        % f(x) &= \integ{\frac{\diff[4]k}{(2\ppi)^4}} \eu[-\im k\cdot x] \tilde{f}(k) \\
        % f(x) &= \int \! \frac{d^4\! k}{(2\ppi)^4} \, \eu[-\im k\cdot x] \tilde{f}(k) \\
        \tilde{f}(k) &= \integ[4]{x} \eu[\im k \cdot x] f(x) 
    \end{split}
\end{equation}
Here, $k\cdot x = k\_{\sigma} x\^{\sigma} = g\_{\rho\sigma} k\^{\rho} x\^{\sigma}$.


% \begin{equation}
%     \begin{split}
%         \lambda \quad \widebar{\lambda} \hbar \quad \rlap{X}-- \\
%         \text{X\hspace{-0.8em}\raisebox{0.75pt}{\,--}} \\
%         \lambdabar \equiv \lambda /(2\ppi) ; (\lambda = 1/f) \iff k = 2\ppi/\lambda = 1/\lambdabar
%     \end{split}
% \end{equation}

% \section*{Frequently used abbreviations}
\section*{Acronyms}
\subimport{./}{abbrv.tex}


\section*{Nomenclature}
\subimport{./}{symbols.tex}