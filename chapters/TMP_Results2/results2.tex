%%%%%%%%%%%%%%%%%%%%%%%%%%%%%%%%%%%%%%%%%%%%%%
%%%%%%% Ch. X: Results (2) %%%%%%%
%%%%%%%%%%%%%%%%%%%%%%%%%%%%%%%%%%%%%%%%%%%%%%




% ////////////////// intro //////////////////




\comment{Maybe this chapter should be post-simulation work? I.e. implications? }

% In this chapter, we discuss and present the tensorial results, that is $h\_{ij}$ and $T\_{ij}$.



% The Equation of equations,
% \begin{equation}
%     ah_+ = H_1 + H_2
% \end{equation}
% \begin{align}
%     H_1(\tau, \vec{k}) &= +\psi_2(k\tau) \integ{\theta}[\eta_\ast][\tau]  \psi_1(k\theta) S(\theta, \vec{k}) \\
%     H_2(\tau, \vec{k}) &= -\psi_1(k\tau) \integ{\theta}[\eta_\ast][\tau]  \psi_2(k\theta) S(\theta, \vec{k})
% \end{align}
% \begin{align}
%     \psi_1(x) &= \sqrt{x}\Bessel[\nu](x) \\
%     \psi_2(x) &= \sqrt{x}\Bessel[\nu][2](x) \\
% \end{align}
% (Above also true for free waves)


% Now, source dependence: Let $\epsilon = \epsilon_q(\tau) \sin{qy}$ 
% \begin{equation}
%     S(\tau, \vec{k}) = 16\ppi^4 G\nped{N} \Diracdelta(k\_x) \sum_{n\in\Integer} \Diracdelta(k_y + nq) (nq/k)^2 \cdot a^4(\tau) \sigma(\tau) \cdot W(\tau,k\_z) \Bessel[-n] \big[\epsilon_q(\tau) k\_z \big]
% \end{equation}





% ///////////////////////////////////////////






% \paragraph{About the output from the code ...}
%     We quickly see from calculations that $h\_{ij}(\tau, \vec{k})\in \Real$. The code will have it differently, however, and consistently produces non-negligible imaginary components. Likely, this has to do with the different Fourier conventions used by hand and by code. We have not been able to resolve this completely (i.e.~find a suitable mapping), and so we present only the magnitude of the strain.


% \paragraph{Computing the semi-analytical expression ...}
%     To find $H_{1,2}(\tau, \vec{k})$ we need to use a numerical solver, and for this we chose \textit{Numpy}'s  \texttt{cumtrapz}; a method for integrating cumulatively with the trapezoidal rule. 
    
%     %\texttt{odeint} from the \texttt{integrate}-module from the \texttt{Python} library \texttt{Scipy}. 




% ///////////////////////////////////////////




% \section{Mode by mode}
%     We extract the relevant output from \texttt{gwasevolution} to compare with the analytical calculations. Nothing is assumed about the temporal part of the wall normal coordinate, so we may insert any function as $\epsilon_q(\tau)$ into \nc{Eq. XXX}[main expr.]. This is a huge advantage since the results from~\cref{chap:results} are not perfect. 
    
%     An even bigger advantage would be to have the code output the wall position as a near-continuous function of time, but we only have the profile extracted from $\mathtt{achi}$ animation outputs, giving it a function with \blahblah


%     \paragraph{A few take-aways.} %
%     There are some results that need be mentioned, but not necessarily presented plots. 
%     \subparagraph{Periodicity in $y$-mode.} %
%     The outputted \texttt{hijFT} shows significantly smaller strains for $K_Y\neq n m_Y$ than for $K_Y = n m_Y$, something we interpret as a corroboration to the Dirac--delta factor in \nc{Eq. XXX}[main expr.]. It being non-zero may be a result of numerical error, but it is likely also related to the issue with \nc{the wavenumber ambiguity}. 
%     \subparagraph{$K_Y=0$ is non-zero.} %
%     For whatever reason, the code insists there are significant tensor perturbation propagating in the $Z$-direction. This is not what we expected from calculations, where $\vec{k}=(0,0,k\_{z})$ corresponds to zero strain.




% \section{Tensor perturbations}\label[sec]{sec:results2:h11}
%     {\subimport{./}{h11.tex}}

    

% \section{Sensitivity to changes in \dots}



So the framework is not rock solid, but it is definitely \emph{something} true about the equations. We reserve this chapter for the ifs, buts and maybes. To keep the discussion at bay, we focus only on the $+$-polarised wave of the tensor perturbation.


\begin{bullets}
    \item insert sensible parameters
\end{bullets}



\section{Limits of the framework}
    Let us review the equations this thin-wall approximation is built on. 
    We want everything up to the Fourier SE tensor to be analytically solvable, at least to some level that resembles the actual situation, like when using $\sigma \propto (1-\upsilon)^{3/2}$ in this project. \comment{The behaviour is recognisable at this stage.} 
    Okay, so far, so good. We found that for a two-dimensional topological defect in a conformally flat spacetime, we have the SE tensor
    \begin{equation}
        T\_{\mu\nu} = 
    \end{equation}
    (z-plane) 
    % The result is that any function $\varrho (x\^a)$ that satisfies $-\im \partial\_{a} \leftrightarrow p\_a$ for which we have a analytical formula for the Fourier transforms of 
    % \begin{equation}
    %     \eu[\im c \varrho(x\^a)]\quad \text{and} \quad  \partial\_a \eu[\im c \varrho(x\^a)]
    % \end{equation}



\section{Superpositions}
    \begin{bullets}
        \item Adding propagating waves on torus
        \item What would happen if there were two such perturbations? or several pert. walls?
    \end{bullets}
    