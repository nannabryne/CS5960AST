% |||||||||||||||||||||||||||||||
% |||||| 1.2 Preliminaries ||||||
% |||||||||||||||||||||||||||||||




% -----------------------------------------
% labels: \label{[type]:into:prelim:[name]}
% -----------------------------------------



% ¨¨¨¨¨¨¨¨¨¨¨¨¨¨¨¨¨¨¨¨¨¨¨¨¨¨¨¨¨¨¨
% LOCAL MACROS:
\newcommand\pert{\ALIASpert}
% ¨¨¨¨¨¨¨¨¨¨¨¨¨¨¨¨¨¨¨¨¨¨¨¨¨¨¨¨¨¨¨



% ////////////////////// intro //////////////////////

We assume that the reader is familiar with basic concepts from mathematical methods in physics such as the variational principle, tensors, Green's functions and cylindrical functions. Classical field theory and linear perturbation theory will not be covered from ground level. Other than this, the thesis will be relatively self-contained, but understanding of basic modern cosmology will be an advantage. 
For reference, we cover a few of these concepts in the subsections below.

% \comment{Comment on Lorentz invariance and gauge invariance??}


% ///////////////////////////////////////////////////







% classical field theory, linear perturbation theory and mathematical meth


% It is assumed that the reader is familiar with variational calculus and linear perturbation theory.

% \rephrase{In the following, we briefly (re)capture some concepts that are important starting points for the rest of the thesis.}




% \subsection{GR}

% \begin{bullets}
%     \item variational calculus/ varying action
%     \item action
%     \item pert. theory?
%     \item line element
%     \item gauge invariance
%     \item FRW cosmology
%     \item \underline{classical field theory}
% \end{bullets}


% \begin{enumerate}
%     \item Special relativity 
%     \item Classical field theory \begin{itemize}
%         \item Newtonian gravity
%     \end{itemize}
%     \item Mathematical tools
% \end{enumerate}




\subsection{Special relativity}
    The \newconcept{four-vector} $x\^\mu=(t, \vec{x})$ in Minkowski space $\Real[3,1]$ encompasses time $t=x\^0$ and position $\vec{x}=x\^i= (x\^1, x\^2, x\^3)$. The {metric tensor} (defined in~\cref{sec:GR:diffgeo}) in SR is $g\_{\mu\nu}= \eta\_{\mu\nu} \equiv \text{diag} (-1,+1,+1,+1)$. The four-vector $x\^\mu$ is ``naturally raised,'' whereas the derivative
    \begin{equation}
        \partial\_\mu \equiv \pdv{}{x\^\mu} =\pclosed{ \pdv{}{t}, \vec{\nabla} }
    \end{equation}
    is ``naturally lowered.'' The contraction between two four-vectors is independent of choice of reference system, commonly known as Lorentz-invariant.
    % \comment{Lorentz invariance.}





\subsection{Classical field theory}
    % Consider Lorentzian \blahblah. We formulate a theory in terms of the Lorentz invariant action
    % \begin{equation}
    %     S = \int \! \dx[4]\, \mathcal{L},
    % \end{equation}
    % with $\mathcal{L}$ being the \emph{Lagrangian}\footnote{Technically, it is the Lagrangian \emph{density}, an insignificant \grammar[point]{distinction} for these purposes.} of the theory. This Lorentz invariant quantity is composed of scalar terms only.
    % \comment{Fix}
    We formulate a theory of scalar fields $\Phi(x)$ in four-dimensional Minkowski spacetime in terms of the Lorentz-invariant action
    \begin{equation}
        % S = \integ[4]{x} \mathcal{L}(\{\phi_i\}, \{ \partial\_{\mu} \phi_i\}),
        S = \integ[4]{x} \mathcal{L}(\Phi, \partial\_\mu \Phi),
    \end{equation}
    with $\mathcal{L}$ being the \emph{Lagrangian density} of the theory, a function of the set of fields $\Phi$ and its first derivatives. We will refer to $\mathcal{L}$ simply as the Lagrangian, as is customary when working with fields. 

    % This translates to a general (i.e.~curved) spacetime with the substitution of partial to covariant derivative and the construction of a Lorentz-invariant Lagrangian,
    This translates to a general (i.e.~curved) $n$-dimensional spacetime through the construction of a Lorentz-invariant Lagrangian $\hat{\mathcal{L}}$,
    % For a general (i.e.~curved) spacetime, \blahblah $\partial\_{\mu} \to \nabla\_{\mu} $ \blahblah to construct a Lorentz invariant Lagrangian,
    \begin{equation}
        % S = \integ[4]{x}\underbrace{ \mathcal{L}(\{\phi_i\}, \{ \nabla\_{\mu} \phi_i\})}_{\text{not scalar}} =  \int \! \underbrace{\dx[4] \, \sqrt{-\abs{g}}}_{\text{scalar}} \,\underbrace{\hat{\mathcal{L}}(\{\phi_i\}, \{ \nabla\_{\mu} \phi_i\})}_{\text{scalar}}, 
        % S = \integ[4]{x}\underbrace{ \mathcal{L}(\Phi,  \partial\_{\mu} \Phi)}_{\text{not scalar}} =  \int \! \underbrace{\dx[4] \, \sqrt{-g}}_{\text{scalar}} \,\underbrace{\hat{\mathcal{L}}(\Phi, \nabla\_{\mu} \Phi)}_{\text{scalar}},
        S = \integ[n]{x}\mathcal{L}(\Phi,  \partial\_{\mu} \Phi) =  \integ[n]{x\sqrt{-g}}\hat{\mathcal{L}}(\Phi,  \nabla\_{\mu} \Phi) 
    \end{equation}
    where $g\equiv \det~\!(g\_{\mu\nu})$ is the determinant of the metric, and 
    \begin{equation}
        d^n \! x \sqrt{-g} =  {\diff x}^0 \wedge {\diff x}^1 \wedge \cdots\wedge  {\diff x}^{n-1}\sqrt{-g}
    \end{equation} 
    is the invariant volume element~\citep{carrollSpacetimeGeometryIntroduction2019}. %
    What was applied here, was the \newconcept{minimal-coupling principle}, 
    and we may refer to $\mathcal{L}$ as the minimally-coupled Lagrangian (density). Loosely speaking, this principle involves replacing the Minkowski metric with the curved-spacetime metric ($\eta\_{\mu\nu}\to g\_{\mu\nu}$), and partial derivatives to covariant derivatives ($\partial\_{\mu} \to \nabla\_{\mu} $).
    
    % where $\hat{\mathcal{L}}$ is a scalar.
    % \comment{Maybe specify that this is only for scalar fields? Or include other fields?}

    % \comment{Action princible}


    \paragraph{Action principle.} %
    We use variational calculus to obtain the fields' equations of motion. For a scalar field $\phi$, we consider the change $\phi \to \phi + \delta \phi$ and apply this to the action, such that $S\to S+ \delta S$. %Identifying terms of leading order ($\mathscr{O}(\delta \phi)$), 
    After isolating $S$ and gathering terms up to leading order, we set $\delta S = 0$ and obtain an equation for $\phi$ in terms of first and second derivatives. 

    % Often it is necessary to use \checkthis{Stokes' theorem} to set certain terms to zero. The argument \blahblah \comment{Fix or remove!}
    % We gather zero-order terms and isolate $S$
    % \begin{enumerate}
    %     \item Gather zero-order terms and isolate $S$.
    %     \item Identify leading-order terms to find $\delta S= \mathscr{O}(\delta \phi)$.
    %     \item Set all higher-order terms to zero, $\mathscr{O}((\delta \phi))=0$
    % \end{enumerate}

% \subsection{Classical Field Theory}
    % The action
    % \begin{equation}
    %     S\ped{ST} = S\ped{EH} + S_{\phi} + S\ped{m}=  \int \!\diff[4] x \, \sqrt{-\abs{g}} \cclosed{ \frac{\Planck{M}^2}{2} R - \frac{1}{2}\nabla\indices{_{\rho}} \phi \nabla\indices{^{\rho}} \phi - V(\phi)} + S\ped{m}
    % \end{equation}

    % GR as we know it is reconstructed when varying $S\ped{E}$ with respect to the metric $g\_{\mu\nu}(x)$



    % \subsubsection{Newtonian gravity}
    %     \blahblah



\subsection{Mathematical tools}
% \subsubsection{Perturbation theory}
%     % \blahblah
%     We write that to order $o$ in some small parameter $\epsilon$, we have
%     \begin{equation}
%         \pert{q} \equiv \sum_{i=0}^o \delta^{(i)} q =  q^{(0)} + \epsilon q^{(1)}  + \epsilon^2 q^{(2)} + \dots +  \epsilon^o q^{(o)}, 
%     \end{equation}
%     where 

%     Consider $Q(q)$. We write
%     \begin{equation}
%         \pert{Q} \equiv \sum_{i=0}^o {}^{(i)} Q,
%     \end{equation}
%     where ${}^{(i)}Q$ is the $i$th order perturbation of $Q$, not necessarily $\epsilon^i Q$

%     \comment{Fix or remove.}
    




    \subsubsection{Method of Green's functions}
    A linear ordinary differential equation (ODE) $\mathop{\mathrm{L}}_x f(x)=g(x)$ assumes a linear differential operator $\mathop{\mathrm{L}}$, an unknown function $f$, and a right-hand side $g$ that forms the inhomogeneous part of the ODE. The \newconcept{Green's function} $G$ for the ODE (or $\mathop{\mathrm{L}}$) manifests as any solution to $\mathop{\mathrm{L}}_x G(x, y) = \Diracdelta(x-y)$. Now
    \begin{equation}
        f(x) = \integ{y} G(x, y) g(y)
    \end{equation}    
    solves the original ODE. 

    % \paragraph{Homogeneous initial conditions.} %
    % If $f(x_0)=\dv*{f}{x}\rvert_{x=x_0}=0$, the solution reads
    % \begin{equation}
    %     f(x) = \integ{y}[x_0][x] G(x,y)g(y).
    % \end{equation}



    % \paragraph{Pulse signal.} %

    
    % \deleteme{If $\mathop{\mathrm{L}}$ is translation invariant (invariant under $x\mapsto x+a $)---which is equivalent to $\mathop{\mathrm{L}}$ having constant coefficients---we can write $G(x,y)=G(x-y)$ and %\provethis{show?}
    % \begin{equation}
    %     f(x) = (G \ast g )(x) =\integ{y}G(x-y)g(y)
    % \end{equation}
    % solves $\mathop{\mathrm{L}_x}f(x)=g(x)$. }
    
    % \deleteme{%
    % Let $f_i^{\mathrm{(0)}}$, $i=1,2,3,\dots$ be solutions to the homogeneous ODE, i.e.~$\mathop{\mathrm{L}_x} f_i^{\mathrm{(0)}}=0$. Then, by the superposition principle, $f(x) + \sum_{i}c_i f_i^{\mathrm{(0)}}$ is also a solution of the original, inhomogeneous equation.}

    % \deleteme{
    % \paragraph{Pulse signal.} Consider the very common scenario where the source is a temporary pulse;
    % \begin{equation}
    %     g(x) = \begin{cases}
    %         g(x), & x_0 \leq x \leq x_1, \\
    %         0, & x\geq x_1.
    %     \end{cases}
    % \end{equation}}


    % \comment{Fix or remove.}

    % This is a test.


    \subsubsection{Topology}
    The \newconcept{$n$-torus} $\mathbb{T}^n$ is the $n$-fold product of the circle $\mathbb{S}^1$, %$\mathbb{T}^n=\mathbb{S}^1 \cross \mathbb{S}^1 \cross \cdots \mathbb{S}^1 \, (n\geq 2)$ 
    ~\citep{lancasterQuantumFieldTheory2014} 
    \begin{equation}
        \mathbb{T}^n = \underbrace{\mathbb{S}^1 \cross \cdots \cross \mathbb{S}^1}_{n} \equiv (\mathbb{S}^1)^n.
    \end{equation}
    ``Gluing'' pairs of opposite-facing sides of a cube---which is to impose periodic boundary conditions---gives a 3-torus. %
    3-tori have no boundaries and are equivalent to infinite flat spaces with a periodically repeated cubic templates~\citep{carrollSpacetimeGeometryIntroduction2019}.
    
    
    % infinite Euclidean space obtained by periodically repeating the exact perturbation pattern of a single cubic template
    
    % and are obtained by ``gluing'' pairs of opposite-facing sides of a cube. The periodic boundaries are interpreted \blahblah
    



    \subsubsection{The cyclic group}
    The cyclic group, denoted $\Zn[n]$, describes a symmetry of a plane figure that remains invariant after a rotation of $2\ppi/n$ radians (or $360^\circ/n$ degrees), where $n\in \Integer$ ($n$ is integer)~\citep{lancasterQuantumFieldTheory2014}. We consider discrete $\Zn[2]$-symmetries in this thesis, also called reflectional symmetry, which amounts to invariance under change of sign: $\phi \to\eu[\im 2\ppi/2] \phi  = -\phi$.








\subsection{Cosmic scales}

    % We shall in this thesis where cosmic time is 

    % \begin{bullets}
    %     \item expansion rate, cosmic time, conformal time
    %     \item why is flat assumption OK?
    %     \item redshift $\redshift=a_0/a-1$
    % \end{bullets}
    % The universe expands with the rate $a(t\phys)$ at physical or rather \emph{cosmic} time $t\phys$. \comment{Explain cosmic time and $t\phys=0$.} This work sincerely favours the use of \emph{conformal} time, also known as the comoving horizon. As $t$ is a neat variable name, we shall \emph{not}  reserve it for the cosmic time, as is the most common use, but rather let it refer to the conformal time coordinate. 

    % The standard model of cosmology, the \hypertarget{abbrv:lcdm}{\textLambda{}CDM}

    % In standard model 

    % There are several common choices of time reference frames in cosmological context.
    % The standard model of cosmology offers several measurements of time that are convenient for various purposes. 
    \subsubsection{Chronology}
    The standard model of cosmology provides multiple time measurements that are useful for different applications. 
    Cosmic time $t$ is often impractical to use in physical cosmology, but it gives a certain intuition that other measures of time may lack. In this thesis, we give cosmic time estimates (in units of (giga)years) from the perspective of a flat \textLambda{}CDM model consisting of $32\%$ non-relativistic matter and $68\%$ vacuum energy today, with Hubble constant $67\unit{km/s/Mpc}$: 
    \begin{equation}\label{eq:intro:prelim:cosmic_time_parameter_default}
        \mathrm{h}=0.67, \quad \Omega\ped{m0} = 0.32, \quad \Omega_{\Lambda 0} = 1 -  \Omega\ped{m0} = 0.68,
    \end{equation}
    which is an appropriate approximate concordance model of cosmology. 

    % The expansion of the universe manifests in a 


    The Doppler shift in electromagnetic waves arising from the recession of distant objects is %the \newconcept{cosmic redshift}:
    \begin{equation*}
        \text{redshift} = 
        \frac{\text{observed wavelength} - \text{emitted wavelength} }{\text{emitted wavelength} }.
    \end{equation*}
    % \begin{equation*}
    %     \text{redshift}= 
    %     \frac{\text{observed wavelength} - \text{emitted wavelength} }{\text{emitted wavelength} } =
    %     % \frac{\text{observed wavelength}}{\text{emitted wavelength} } 
    %     \frac{\text{distance now}}{\text{distance then}}-1.
    % \end{equation*}
    This shift is determined by the change in proper distance between two comoving points, 
    \begin{equation*}
        \frac{\text{proper distance now}}{\text{proper distance then}} = \frac{\text{scale factor now}\cross \text{comoving distance}}{\text{scale factor then}\cross \text{comoving distance}}.
    \end{equation*}
    This \newconcept{cosmic scale factor} is denoted $a= a(t)$. Normalised such that ``now'' means ``today'', we get 
    \begin{equation}
        1 + \redshift = a_0/a(t),
    \end{equation}
    where $a_0 = a(t_0)\equiv 1$ is the scale factor today, and $\redshift$ the \newconcept{cosmic redshift}. 
    % to $a_0 \equiv a(t_0)$
    % Now, the \newconcept{cosmic redshift} is defined 
    % \begin{equation*}
    %     % \text{cosmic redshift} = 
    %     \frac{\text{scale factor now}}{\text{scale factor then}} - 1
    % \end{equation*}
    % % \newconcept{cosmic redshift}. 
    % The \newconcept{cosmic redshift} describes the universe' expansion, and 
    We denote it in this thesis with $\redshift$ to distinguish it from the Cartesian spatial $z$-coordinate, as they both are frequently used variables. %The cosmic time today, $t_0 \approx 13.8$ gigayears (per~\cref{eq:intro:prelim:cosmic_time_parameter_default}), is the coordinate time since $a(t\equiv 0)=0$, defined as the time of Big Bang. 
    To give some examples, redshifts $\redshift = 0$ and $\redshift=2$ correspond the universe at $t=t_0 \approx 13.8$ (today) and $t\approx 3.2$ gigayears old, respectively, defined as the coordinate time that has passed since $a(t\equiv 0)=0$.
    % It is calculated as 




    % The Doppler shift arising from the recession of distant objects is the \newconcept{cosmic redshift}%
    % \footnote{
    %     We will use $\redshift$ to distinguish the cosmic redshift from the Cartesian spatial coordinate $z$ in $\vec{x}=(x,y,z)$, as they will both be used frequently.
    % } %
    % $\redshift$, %
    % defined through
    % \begin{equation}
    %     1 + \redshift = a(t_0)/a(t),
    % % \end{equation}
    % where $t$ is cosmic time with value $t_0\approx 13.8$ gigayears today, and $a$ is the dimensionless \newconcept{cosmic scale factor}. The concept of cosmic time is defined as the coordinate time passed since $a(t\equiv 0)=0$, 
    % % The cosmic-time term is defined via $a(t\equiv 0)= 0$, 
    % which is to say that we define the time of Big Bang as this point in time. 
    In this thesis we make use of \newconcept{conformal time} $\tau$ for which ${\diff t} = a {\diff \tau}$ holds %
    %  giving
    % {\newcommand{\dummy}{}
    % \begin{equation}
    %     \tau = \integ{\dummy{t}}\frac{1}{a(\dummy{t})} =  \integ{\dummy{a}} \frac{1}{\dummy{a}^2 H(\dummy{a})}
    % \end{equation}
    % }%
    in units with $c=1$. We will use $\dot{}\equiv\dv*{}{\tau}$ as a shorthand notation for derivatives with respect to conformal time. %Another important
    Other common measures of time include the physical Hubble factor $H=1/a \dv*{a}{t}$, and conformal Hubble factor $\mathcal{H}= aH = \dot{a}/a$.

    % To give some numbers, a simplified concordance model of cosmology describes a flat universe consisting of $32\%$ non-relativistic matter and $68\%$ vacuum energy today, corresponds to a $t_0\approx 13.8$ gigayears old universe. At redshift $\redshift = 2$, the scale factor is $a = 1/3$ and the universe is $t\approx 3.26$ gigayears old.
    % \comment{Rewrite.}


    \subsubsection{Comoving and physical scales}
    We use comoving measures of disctances and relate these to comoving wavenumbers $k$, related to comoving wavelength $\lambda$. We adapt the convention in~\citet{maggioreGravitationalWavesVol2018} and use \newconcept{reduced wavelengths}
    \begin{align}
        \lambdabar \equiv   1/k  = \lambda/(2\ppi).
    \end{align}
    % We refer to sub- and super-horizon modes as the modes for which $k\tau \ll 1 $
    Comoving scales remain constant in time. We refer to modes with wavelengths larger than the horizon, $\lambdabar\gg \mathcal{H}^{-1}(a)$, as \newconcept{super-horizon} modes, where $\mathcal{H}=1/a \dv*{a}{\tau}$ is the conformal Hubble factor (\cref{eq:GR:lcdm:first_Friedmann_conformal}) or the ``comoving horizon.'' %
    Modes well inside the horizon, $\lambdabar\ll \mathcal{H}^{-1}(a)$ are \newconcept{sub-horizon}. %
    Now, we obtain physical measurements via $k\ped{ph}(a) = k /a$ and $\lambdabar\ped{ph}(a) = a \lambdabar$.

    




\subsection{Cosmological simulations}
    % \phpar[Nyquist frequency, fundamental frequency, DFTs]

    Simulations of cosmological scenarios are a topic of high-performance computing (HPC) and usually performed on three-dimensional lattices with periodic boundaries (i.e.  3-tori, $\mathbb{T}^3$). Say we have such a cubic lattice of side lengths $L$, with $N$ points in each direction, giving a spatial resolution of $\Delta=L/N$. %
    The \newconcept{fundamental frequency} is the smallest frequency mode, which in terms of wavenumber is $k\ped{IR}=2\ppi/L$. %\footnote{We use $\text{wavenumber} = \text{wavelength}^{-1} =  2 \ppi \times \text{frequency}$} 
    The \newconcept{Nyquist frequency}, given by the angular frequency $k\ped{UV}=k\ped{IR} N/2 = \ppi N/L$, is the highest resolvable frequency.

    % The choice of $L$ and $N$ involves a trade-off between the desire to capture large-scale structures happening on cosmological scales ($\gtrsim \mathrm{Mpc}/h$) and the need for 

    \subsubsection{High-performance computing}

    There are two main parallelisation techniques used in HPC; the distributed-memory standard Message Passing Interface (\newconcept{MPI}), and the shared-memory Open Multi-Processing (\newconcept{OpenMP}) application. %
    The MPI standard functions on parallel computing architectures, such as computer clusters. In lattice simulations, this amounts to a clever division of tasks on different parts of the lattice. \comment{OpenMP?} %Large simulations 
    % With OpenMP, tasks are threaded on the 

    % It is utterly unpractical to perform high-resolution relativistic $N$-body simulations without techniques from high-performance computing (HPC). %
    % We dedicate this subsection to the \blahblah
    % The computational expense of a simulation is in principle decided by the most time-consuming task of floating point operations and data access. Today, we are not particularly limited by computer arithmetic, but rather memory bandwidth. Furthermore, \blahblah (series vs. parallel)
    % % most time-consuming task  
    % % Computational expense in this contkext is in principle governed by a
    % There are two main parallelisation techniques used in HPC. The Message Passing Interface (MPI) standard functions on parallel computing architectures, such as computer clusters. In lattice simulations, it is sensible to \blahblah}

    

    

    
    

