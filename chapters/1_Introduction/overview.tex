% ||||||||||||||||||||||||||
% |||||| 1.1 Overview ||||||
% ||||||||||||||||||||||||||

% -------------------------------------------------
% labels: \label{[type]:into:overview:[name]}
% -------------------------------------------------




% \begin{bullets}
%     \item Intro to GR, cosmo, DE, GWs
% \end{bullets}
% General relativity (GR) is widely accepted as the best theory for gravity yet. Astronomical observations, GPS


% > GR
General relativity (GR) is a fundamental theory of gravitation proposed by Albert Einstein (1979--1955) in 1915. It describes gravity not as a force, but as a curvature of spacetime caused by mass and energy, fundamentally altering our understanding of both space and time. The theory is supported by numerous experiments and observations, such as the bending of light around massive objects (gravitational lensing) and the precise orbit of Mercury. Additionally, GR is essential for the accurate functioning of the Global Positioning System (GPS), as it allows for the necessary corrections to account for the differences in the passage of time between satellites in orbit and receivers on Earth.
%
% > SR
Special relativity (SR)---proposed by the same physicist in 1905---is understood as a special case of GR where the background spacetime is Minkowski. 
%
% > Newtonian gravity ? 


% > Cosmology
In cosmology, GR forms the backbone of our understanding of the universe on large scales. The theory predicts the expanding universe, which has been confirmed by the redshift observations of distant galaxies, leading to the development of the Big Bang theory. GR also accounts for the formation of large-scale structures like galaxies and galaxy clusters, and provides the framework for understanding cosmic phenomena such as black holes and gravitational waves. The solutions to Einstein's field equations under various conditions have led to a rich set of predictions and observations, including the cosmic microwave background radiation (CMB), which is a relic from the early universe and a critical piece of evidence supporting the Big Bang model.



% > DE, CDM
% The 
% Dark energy is postulated to 
The standard model of cosmology is the Lambda cold dark matter model (\textLambda{}CDM) that postulates the existence of a cosmological constant (\textLambda{}), widely used synonymously with dark energy, and cold dark matter (CDM), besides normal matter. Dark energy is used to explain the contemporary accelerated expansion of the universe, whereas dark matter accounts for the observed gravitational effects that ordinary matter alone cannot elucidate. %


Besides the explanation of dark matter and dark energy, there are other shortcomings to these theories. %
Of most cosmological relevance are the smallness (``vacuum catastrophe'') and coincidence problems associated with the cosmological constant, as well as the Hubble tension (\cref{sec:GR:lcdm}). We will address some of these in this thesis. 


 
Amongst the more popular strategies to overcome the problems of modern cosmology, is to add extra degrees of freedom. Speculative scalar fields that possess certain symmetries tend to predict phase transitions, which in turn can produce topological defects. \speak{Two--three sentences about domain walls.}




% horizon, homogeneity, flatness and monopole problems, as well as the Hubble tension (\cref{sec:GR:lcdm:problems}). We will address some of these in this thesis. 
%
% \comment{Mention BH problem? Or?}

% \comment{Comment about inflation? Solve horizon, homogeneity, flatness and monopole problems }






\subsection{Key literature}
    Let us mention a few foundational works that have been instrumental in shaping the theoretical and methodological approach in this project. % 
    \textit{Spacetime and Geometry} by~\citeauthor{carrollSpacetimeGeometryIntroduction2019}~(\citeyear{carrollSpacetimeGeometryIntroduction2019}) is the baseline for the majority of the differential-geometry analyses and GR discussions. A large part of the gravitational-wave discussion is inspired by \textit{Gravitational Waves} volumes 1 and 2 by~\citeauthor{maggioreGravitationalWavesVol2007}~(\citeyear{maggioreGravitationalWavesVol2007,maggioreGravitationalWavesVol2018}). \textit{Kinks and Domain Walls} by~\citeauthor{vachaspatiKinksDomainWalls2006}~(\citeyear{vachaspatiKinksDomainWalls2006}) provide the basis for our understanding of topological defects.

    % Articles of particular relevance are~\citet{christiansenAsevolutionRelativisticNbody2023,llinaresDomainWallsCoupled2014} \blahblah

    % This project 



    % We use~\citet{carrollSpacetimeGeometryIntroduction2019} for GR \blahblah, \citet{maggioreGravitationalWavesVol2007,maggioreGravitationalWavesVol2018} for 

    % \citet{vachaspatiKinksDomainWalls2006}