
% !TEX root = ../../thesis.tex

% -----------------------------------
% labels: \label{[type]:intro:[name]}
% -----------------------------------

\begin{bullets}
    \item GOALS: \begin{itemize}
        \item Gather framework about GWs from DWs
        \item Remove the need for very expensive $N$-body simulations with (semi-)analytical predictions
        \item Extract as much information as possible from the NANOgrav spectra thingy
    \end{itemize}
    \item WHY RELEVANT: \begin{itemize}
        \item NANOgrav data wihoo
        \item Simulations in this regard are hugely expensive, and will not allow us to constrain the parameters of a model
    \end{itemize}
\end{bullets}

% We use the convention that $(-,+,+,+)$ is the metric signature.

% \Group{SO}{3} is a group 

% \textsf{SO(3)} is a group

% $\Group{SO}{3}$ is a group

% The order \BigO{1} is large

% The order $\BigO{1}$ is large



% The Planck unit \Planck{M}


% We have a GW with $\rho\ped{GW}$ or $\rho\ped{gw}$  \dots  $\rho\subGW$

% \begin{equation}
% \begin{split}
%     f(x) &= \int \! \frac{d^4\! k}{(2\ppi)^4} \, \eu[-\im k\cdot x] \tilde{f}(k) \\
%     \tilde{f}(k) &= \int \! d^4\! x \, \eu[\im k \cdot x] f(x) 
% \end{split}
% \end{equation}


% \begin{equation}
%     \tilde{h}^{\prime \prime}_\circledast + 2\mathcal{H}\tilde{h}^{\prime}_\circledast + k^2 \tilde{h}_\circledast = 16\ppi G\nped{N} a^2 \tilde{\sigma}_\circledast; \quad \circledast = +, \times
% \end{equation}





% \begin{equation}
%    \pclosed{\tilde{h}\ap{TT}}\indices{_{ij}}  (\eta, \vec{k}) = \sum_{\circledast = +, \times} \tensor*{e}{^{\circledast}_{ij}}(\hat{\vec{k}}) \tensor{\tilde{h}}{_\circledast}(\eta, \vec{k})  %\tensor{{\tilde{h}}}{_{ij}} 
% \end{equation}


% \begin{equation}
%     \tensor{\tilde{h}^{\text{\tiny{TT}}}{}}{_{ij}}  (\eta, \vec{k}) = \sum_{\circledast = +, \times } \tensor*{e}{^{\circledast}_{ij}}(\hat{\vec{k}}) \tensor{\tilde{h}}{_\circledast}(\eta, \vec{k})  %\tensor{{\tilde{h}}}{_{ij}} 
% \end{equation}




% \underscore

% \texttt{pert\underscore{}i}

% \circumflex{i}


% The thesis is structured as follows. \cref{part:bckg} 

The thesis is divided into three parts. %
\cref{part:bckg} presents the relevant background theory, including concepts from differential geometry to topological defects. %
In~\cref{part:method} (pre-sims / a priori) we describe how the framework was designed and the way it is to be tested in simulations. %
Finally, in~\cref{part:findings} (post-sims), we present the results from simulations and calculations, with comparisons and discussions. %
Prior to all of this, we describe some basic concepts in the coming section.


\begin{draft}
    \paragraph{Final product.} %
    We end up with a framework that describes the motion of a small perturbation to the normal coordinate of a planar domain wall in a matter-dominated universe, with possibility for generalisation. The model assumes the symmetron potential, but this is easily changed to another scalar field model with discrete symmetry by changing the surface tension. The equations are tested by comparison to toy scenarios in cosmological simulations. 
    The final part---the resulting tensor perturbations to the metric---has room for improvement, or at least needs to be better tested more thoroughly against simulations. There is no doubt that the signature from such a wall position distortion is detectable \emph{in simulations}, but there are other features of the resulting tensor modes that remain mysterious. 
\end{draft}




\section{Preliminaries}\label{sec:intro:prelim}%
    {\subimport{./}{prelim.tex}}


