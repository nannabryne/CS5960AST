% ||||||||||||||||||||||||||||||||
% |||||| 7.X Open questions ||||||
% ||||||||||||||||||||||||||||||||

% ----------------------------------------------
% labels: \label{[type]:whatif:questions:[name]}
% ----------------------------------------------







What it means for the gravitational waves to have any other perturbation $\epsilon= \varepsilon(\tau) \sin{(py + \varphi)}$ remains unclear. A starting point for such an analysis was provided in~\cref{app:walls:SE_tensor_alt:general}. 

Varying symmetron parameters, or scalar-field theory as such, was the original wish, but other analyses was prioritised. Exactly how the gravitational waves can constrain the symmetron model is not discussed in this thesis.

We have seen how the time-dependence of the surface tension affects the dynamics of the wall. %
After some time, when the derivative of the surface tension goes to zero, the equation of motion is~\cref{eq:pertwalls:thinwall:varepsilon_and_E_eoms} only with initial conditions from the first part of the evolution. The gravitational-wave dynamics are a little more intricate, and we have not studied the impact of letting $\sigma$ vary with time. If it should turn out that this initial evolution of $\varepsilon$ does not really affect the gravitational-wave signature, then maybe using time-dependent $\sigma$ is an unnecessary complication. We could then adjust the initial conditions in~\cref{eq:pertwalls:gws:H_P_12} and use the simpler solution $\varepsilon(s)= {s}^{-5/2} \Cylindrical[-5/2](\omega s)$, where the constants are determined after evaluating $\epsA$ at some time point $s\gtrsim s\ped{sow}$ (see~\cref{sec:pertwalls:mywalls:sol_MD}).


In summary, we have seen that there are still many paths to continue our work, in particular understanding better \comment{\dots}, which might help to establish at last \comment{\dots}






% is equivalent to scalar field in 


% \blahblah, oblivious to the initial evolution. 
% Gravitational waves are more complicated and generally ``remembers'' 

% How this motion imprints on the gravitational waves and how this differs from what happens when we let the surface te is not studied in the 






% We have not studied the implications for biased domain walls, but one can 