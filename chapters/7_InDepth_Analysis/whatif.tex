%%%%%%%%%%%%%%%%%%%%%%%%%%%%%%%%%%%%%%%%%%%%%%
%%%%%% Ch. 7: Ifs, Buts and Maybes %%%%%%
%%%%%%%%%%%%%%%%%%%%%%%%%%%%%%%%%%%%%%%%%%%%%%


% -----------------------------------------------
% labels: \label{[type]:whatif:[name]}
% -----------------------------------------------




% ¨¨¨¨¨¨¨¨¨¨¨¨¨¨¨¨¨¨¨¨¨¨¨¨¨¨¨¨¨¨¨¨¨¨¨¨
% LOCAL MACROS:
\newcommand\Ft{\ALIASFt}
\newcommand\ah{\ALIASah}
\newcommand\lcoord{\ALIASlcoord}
\newcommand\lcoordx{\ALIASlcoordx}
\newcommand\lcoordk{\ALIASlcoordk}
\newcommand\hpA{\ALIAShpA}
\newcommand\hpB{\ALIAShpB}
\newcommand\hpC{\ALIAShpC}
\newcommand\hpAB{\ALIAShpAB}
\newcommand\hpCR{\ALIAShpCR}
\newcommand\hpCI{\ALIAShpCI}
\newcommand\epsA{\ALIASepsA}
\newcommand\epsB{\ALIASepsB}
\newcommand\epsC{\ALIASepsC}
\newcommand\wallsep{\ALIASwallsep}
% ¨¨¨¨¨¨¨¨¨¨¨¨¨¨¨¨¨¨¨¨¨¨¨¨¨¨¨¨¨¨¨¨¨¨¨¨




% ////////////////// intro //////////////////





In this chapter, we take a few steps back and look at our work from a more critical point of view. 
We provide a general discussion of the toy-model analysis---both methodically and result-wise---in~\cref{sec:whatif:discussion,sec:whatif:cont} with regrets and proposed methods of verification. In~\cref{sec:whatif:framework} we address the framework as is, agnostic to any potential (in)validations, and discuss its implications and potential. We summarise with some open questions in~\cref{sec:whatif:questions}. 
% \comment{FIXME}
% We refer to the 



% It is assumed in this chapter that the \rephrase{theoretical framework checks out.} What would that mean? And how is it applicable to realistic cosmological scenarios? We dedicate this chapter to suggestions for improvements and implications. In the last section we discuss what simulative experiments we would have performed had we more time.




% \begin{bullets}
%     \item insert sensible parameters
%     \item propose different spatial parts 
%     \item implications
% \end{bullets}



% \pensive{Maybe for large scales, scalar field fluctuations contribute a lot? They should peak at the frequency corresponding to the mass scale $m=\sqrt{2}\mu \sqrt{1 - \upsilon }$?}





% ///////////////////////////////////////////











% ****************** SECTIONS ******************

\section{Project reflection}\label{sec:whatif:discussion}
    {\subimport{./}{discussion.tex}}


\section{Continued assessment}\label{sec:whatif:cont}
    {\subimport{./}{cont.tex}}


\section{Limitations and possibilities}\label{sec:whatif:framework}
    {\subimport{./}{framework.tex}}


\section{Open questions}\label{sec:whatif:questions}
    {\subimport{./}{questions.tex}}

% **********************************************




    
