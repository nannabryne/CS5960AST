% |||||||||||||||||||||||||||
% |||||| 6.4 Hindsight ||||||
% |||||||||||||||||||||||||||

% -----------------------------------------------------
% labels: \label{[type]:results:hindsight:[name]}
% -----------------------------------------------------


Some technicalities regarding our choices is necessary to address at this point. 
It seemed natural to begin with a set of parameters that was deemed sensible before, i.e. $a_\ast = 0.33$, $\xi_\ast = 3.33\times 10^{-4}$ and $\beta_\ast = 1$ from~\citet{christiansenAsevolutionRelativisticNbody2023}. %
The initial idea was to start simulations at initial redshift $\redshift\ped{i}=2.00$ with SSB happening only a few time steps before this, at $\redshift_\ast = 1/a_\ast - 1 = 2 + 1/33 \simeq 2.0303$. This way, the ``true'' wall width $\delta\ped{w} = (a\mu\chi_+)^{-1}$ is not infinite at the start, and we may use the quasi-static approximation for $\chi$ and $q$ with $\breve{\chi}= \chi_+$. We used the analytical expression for $\varepsilon$ and $\dot{\varepsilon}$ to find suitable values at initial time. To get an idea of the impact of larger oscillations around minima, we attempted initialising at redshift $2.02$, where the walls were dangerously close to colliding, but where separated just enough. 



It was not until later that we decided to tweak the initial conditions to our advantage, and performed the analysis in~\cref{sec:PT:symm_dws:asymptotic}. Now we could initialise as close to PT as we wanted. The catch was now that maybe the surface tension was too different from what we used in~\cref{sec:pertwalls:mywalls}, so we spent some time adjusting the expression for $\sigma\ped{w}$ and studying the effect on $\varepsilon$. The result was that only when the fifth-force oscillations are exaggeratively large was there significant changes in $\varepsilon$. 


If we were to do it all over again, it might make sense to set %
% $a_\ast = 0.4$ and $\redshift\ped{i}=\redshift_\ast = 1.5$, %
for example 
$a_\ast = 2/5$ and $\redshift\ped{i}=\redshift_\ast = 3/2$, %
and use the initial conditions $\breve{\chi}=\chi_\ast$ and $\breve{q}=q_\ast$ from~\cref{app:untitled2}, with $\varepsilon=\epsast$ and $\dot{\varepsilon}=0$. %
The advantage that $\dot{z}\ped{w}=0$ is not only that $q$ is less complicated, but we avoid having to include it in $\chi$ as well. To account for a moving wall, we should technically perform a Lorentz transformation on the argument to the hyperbolic tangent~\citep{vachaspatiKinksDomainWalls2006,blanco-pilladoDynamicsDomainWall2023}. In the kink scenario from~\cref{sec:cosmo:defects:ex_Z2_kink},~\cref{eq:cosmo:defects:phi_k_Z2} becomes~\citep{vachaspatiKinksDomainWalls2006}
\begin{equation}
    \phi\ped{k}(t, x ) = \eta \tanh{\pclosed{\sqrt{\frac{\lambda}{2}}\eta  \frac{x - vt}{\sqrt{1-v^2}} }},
\end{equation}
where $v$ is the speed of the moving defect.

% % That would \blahbla
% \begin{equation}
%     \chi\ped{w} =\breve{\chi} \tanh{ \pclosed{ \frac{a\breve{\chi}}{2L\nped{C}}Z  } }
% \end{equation}
% with
% \begin{equation}
%     Z = \frac{z-z\ped{w} - \dot{z}\ped{w}\tau  }{\sqrt{1-\dot{z}\ped{w}^2}}, (NOT\,CORRECT)
% \end{equation}
% i.e.~perform a Lorentz transformation on the argument to the hyperbolic tangent~\citep{vachaspatiKinksDomainWalls2006,blanco-pilladoDynamicsDomainWall2023}. 





% At PT and the short time period thereafter in which the potential changes rapidly, we

\speak{Unsure where this section is best placed.}

\iftime{Describe plateau in beginning, and the reasoning for discarding those results. Explain how this plateau showed up again in the thin-wall limit when changing the surface tension.}

