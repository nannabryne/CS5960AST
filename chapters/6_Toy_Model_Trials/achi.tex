% |||||||||||||||||||||||||||||||||
% |||||| 6.1 Symmetron field ||||||
% |||||||||||||||||||||||||||||||||

% ------------------------------------------------
% labels: \label{[type]:results:achi:[name]}
% ------------------------------------------------

%%%%%%%%%%%%%%%%%%%%%%%%%%%%%%%%%%%%%%%%%%%%%%%%%%
% \newcommand{\lbl}[1]{\textsf{\textbf{#1}}}
% \newcommand{\brphi}{\breve{\phi}}
\newcommand{\brchi}{\breve{\chi}}
%%%%%%%%%%%%%%%%%%%%%%%%%%%%%%%%%%%%%%%%%%%%%%%%%%



\begin{figure}[h]
    \centering
    \includegraphics[width=\linewidth]{Findings/achi_no_wall.pdf}
    \caption{The evolution of the symmetron field in the asymptotic limit. The symmetron parameters are the fiducial ones. The simulation results are from the simulation with no walls. We see $\chi_+$ in solid grey, $\brchi\rvert\ped{opt.}$ in solid blue, $\avg{\brchi}$ in solid red and $\brchi\ap{ideal}$ in dashed yellow.}
    \label{fig:results:achi:achi_no_wall}
\end{figure}
In the simplest case where there are no walls present, and the scalar field takes the same value everywhere, the evolution is described solely by~\cref{eq:PT:symm_dws:eom_asym_chi_s}. In this case, we expect there to be good correspondence between theory and simulation. We find that the optimal path for the symmetron to minimise oscillations occurs if initial conditions are given by~\cref{eq:stablesym:optimal_chi_breve}. This result is demonstrated in~\cref{fig:results:achi:achi_no_wall}. 
We distinguish between the idealised path---call it $\brchi\ap{ideal}$---given by~\cref{eq:stablesym:optimal_chi_breve}, and the ``optimal'' path that solves~\cref{eq:PT:symm_dws:eom_asym_chi_s} with initial conditions $\brchi\rvert_{a\ped{i}} = \brchi\ap{ideal}\rvert_{a\ped{i}}$ (and likewise with derivatives), that differ slightly for different choices of $a\ped{i}$. 



With that established, we move on to wall--anti-wall simulations as described in~\cref{sec:PT:gwas} and~\cref{tab:PT:sims:sim_setups}. %
In general, the symmetron field $\chi$ will at SSB roll into one of the two minima, depending on the sign of the field right before it happens. The strength of the oscillations around the true minima depends on both the initial field value and its time derivative. In our specific simulations, this is deterministic. %
We show in~\cref{fig:results:achi:bckg_qnts} a handful of background quantities from various simulations. %
\begin{figure}[h]
    \centering
    \includegraphics[width=\linewidth]{Findings/bckg_qnts.pdf}
    \caption{Background quantities. \comment{Not too happy about this plot... FIX LABELS!}}
    \label{fig:results:achi:bckg_qnts}
\end{figure}
The simulations with earlier initialisation (simulations~\simnum{4} \&~\simnum{7}) give enhanced fifth-force oscillations. This is consistent with numerical solutions to the asymptotic equations with similar initial conditions. We take a closer look at the symmetron field in~\cref{fig:results:achi:indepth_achi_aq}. %$\mathord{\textnormal{\ttfamily\textphi}}$\textdelta$\mathtt{\&}$ 
% We will have a closer look at the \blahblah.
% The maximum ``speed'' of the field seems to be connected to the parameters $\epsast$ and $p$. Naively we assess that the maximal $\epsast$ \blahblah


% ------------------------------
% ----------- FIGURE -----------
\begin{figure}[ht]
    \centering
    %
    \begin{subfigure}[b]{\linewidth}
        \centering
        \includegraphics[width=\linewidth]{Findings/indepth_achi.pdf}
        \caption{Yellow solid and red dash-dotted graphs show results from numerically solving the asymptotic~\cref{eq:PT:symm_dws:eom_asym_chi_s} with optimal (\cref{eq:stablesym:optimal_chi_breve}) and simulation (\cref{tab:PT:sims:sim_setups}) initial conditions, respectively. The solid and dashed lines show respective simulation output, in form of maximum and averaged values. All are functions of conformal time $s=\tau/\tau_\ast$.}
        \label{fig:results:achi:indepth_achi}
    \end{subfigure}
    %
    \hfill
    \begin{subfigure}[b]{\linewidth}
        \centering
        \includegraphics[width=\linewidth]{Findings/indepth_aq.pdf}
        \caption{The graphs show the outputted maximum (solid) and averaged (dashed) values of $q$ in units $a^2 \epsast \xi_\ast^{-1}$, all as functions of $t_\omega = p(\tau-\tau_\ast)$.}
        \label{fig:results:achi:indepth_aq}
    \end{subfigure}
    % %
    \caption{The asymptotic symmetron field $\breve{\chi}$ and its time derivative $\dot{\breve{\chi}}$.}
    \label{fig:results:achi:indepth_achi_aq}
\end{figure}
% ------------------------------

\paragraph{Symmetron field.} %
% In it is demonstrated how the   
\Cref{fig:results:achi:indepth_achi} shows two things. First, the asymptotic evolution of the symmetron field from simulations correspond to numerical solutions to~\cref{eq:PT:symm_dws:eom_asym_chi_s} with equal initial conditions. We come to this conclusion by combining the maximum value of $\chi$ and the smoothness of the averaged $\chi^2$. Second, the best possible path %
% \footnote{
%     Not to be confused with the idealised path with zero oscillations. It is the numerical solution to~\cref{eq:PT:symm_dws:eom_asym_chi_s} with initial conditions from said idealised path.
% } 
is very close to the actual evolution in simulations initialised at redshift $2.00$, which makes re-runs less useful. On the other hand, we see that simulation~\simnum{7} is expected to benefit quite a lot from such a tweaking of initial conditions. This also applies for simulation~\simnum{4}.


\paragraph{Velocity field.} %
It turned out to be very convenient to plot certain quantities over the peculiar time parameter $t = \omega(s-1)=p(\tau-\tau_\ast)$. One of the quantities that inhabits such a nature is the asymptotic velocity field $\breve{q}$, or in actuality the maximum of $q$. %
The similarities in~\cref{fig:results:achi:indepth_aq} are not difficult to spot. We plot the field as $q~[a^2 \epsast/\xi_\ast]$ %
% in units of \hl{XXX $1/(\epsast \mu)$} 
to emphasise how the speed scales as the initial perturbation amplitude (and symmetron length scale). We can look at~\cref{fig:results:achi:bckg_qnts} to see that this is a feature of the perturbation, as simulation~\simnum{0} lacks said feature. The next section will show how this periodicity is strongly connected to the wall position. The average absolute value tells us more about the fields' stability, where we see a link between these lines' and the asymptotic scalar field's strength of oscillations around the positive minimum.

    % \begin{figure}[h!]
    %     \centering
    %     \includegraphics[width=\linewidth]{Findings/indepth_aq.pdf}
    %     \caption{The ...}
    %     \label{fig:results:achi:indepth_aq}
    % \end{figure}




% \subsection{Some title}
%     The solution to the
%     % \begin{figure}[h]\label{fig:results:achi:achi_sim1}
%     %     \centering
%     %     \includegraphics[width=\linewidth]{Findings/achi_sim1.pdf}
%     %     %%%%%%%%%%
%     %     \caption{$\brchi$ as function of conformal time $s=\tau/\tau_\ast$.}
%     %     %%%%%%%%%%%%%
%     % \end{figure}



%     These results, and simply the fact that they are \blahblah
    

%     The dimensionless time variable $t_p = (\tau-\tau\ped{i}) \cdot p$ or $t_p = (\tau-\tau_\ast) \cdot p= (s-1)u$

    

\subsection{Review} % of \( \chi \)}
    In the absence of topological defects, we see near perfect correspondence between predicted and simulated scalar field $\chi=\breve{\chi}$ (\cref{fig:results:achi:achi_no_wall}). Presence of walls alters the maximum field value (\cref{fig:results:achi:indepth_achi}), due to the ``bump'' in the profile (also to be spotted in~\cref{fig:results:epsilon:wall_profile_2D}). This feature is much more prominent when the walls are perturbed (cf.~two left-most panels in~\cref{fig:results:achi:indepth_achi}). %
    We could have extracted the field value from the snapshots around lattice $z$-coordinate $\lcoord{k}\sim 3 N_\#/4$, but this is unnecessarily complicated and time-consuming, and the temporal resolution is suboptimal. 

    In any case, we see from the \emph{smooth} average-squared field-values in~\cref{fig:results:achi:indepth_achi}, that the overall oscillations are very close to what we expect
    We believe the aforementioned bumps to be a result of the quasi-static approximation (\cref{sec:PT:symm_dws:quasi-static}), which assumes time derivatives to be close to zero.

    The path the field can take that inhabits the smallest possible fifth-force oscillations is close to what happens in simulations~\simnum{0}--\simnum{3} and~\simnum{5}. The same cannot be said for simulations~\simnum{4} and~\simnum{7}.
    