%%%%%%%%%%%%%%%%%%%%%%%%%%%%%%%%%%%%%
%%%%%% Ch. X: Toy Model Trials %%%%%%
%%%%%%%%%%%%%%%%%%%%%%%%%%%%%%%%%%%%%


% -------------------------------------------
% labels: \label[type]{[type]:results:[name]}
% -------------------------------------------

%%%%%%%%%%%%%%%%%%%%%%%%%%%%%%%%%%%%%%%%%%%%%
\newcommand{\lbl}[1]{\textsf{\textbf{#1}}}
\newcommand{\completelbl}[4]{%
\textbf{#1)}%
\textbf{#2:}%
\lbl{#3.#4}%
}
% \newcommand{\completelbl}[4]{%
% \textbf{#1:}%
% % \textbf{#2:}%
% \lbl{#3.#4}%
% $\to${#2}
% }
\newcommand{\lcoord}{\ALIASlcoord}
%%%%%%%%%%%%%%%%%%%%%%%%%%%%%%%%%%%%%%%%%%%%%



\pensive{Title: \textbf{Simulative Insights}; \textbf{Toy Model Trials}; \textbf{Synthetic Experiments}; }

% ////////////////// intro //////////////////




This chapter provides simulative insights to our theoretical framework. We both present and discuss results from~\cref{tab:PT:sims:sim_setups}, and compare this to the theory. The analysis is split into three parts: In~\cref{sec:results:achi} we study the background evolution through box-averaged scalar quantities. Less primitive analysis proceeds in~\cref{sec:results:epsilon} where the domain wall dynamics of the two theories is compared. Finally, in~\cref{sec:results:h11} we look at what sort of gravitational waves we expect from this phenomenon. 






% ///////////////////////////////////////////










% ****************** SECTIONS ******************

% Background field
\section{Background evolution / Symmetron field}\label{sec:results:achi}
    {\subimport{./}{achi.tex}}


% Wall evolution
\section{Domain wall dynamics}\label{sec:results:epsilon}
    {\subimport{./}{epsilon.tex}}

% GWs
\section{\tmptitle{Gravitational waves}}\label{sec:results:h11}
    {\subimport{./}{h11.tex}}


% **********************************************

