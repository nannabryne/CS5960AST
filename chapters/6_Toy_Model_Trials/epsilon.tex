% ||||||||||||||||||||||||||||||||||||||
% |||||| 6.2 Domain-wall dynamics ||||||
% ||||||||||||||||||||||||||||||||||||||

% ---------------------------------------------------
% labels: \label{[type]:results:epsilon:[name]}
% ---------------------------------------------------




In the following, we ignore the anti-wall at the box's boundaries and focus on the middle wall.
Simulation-wise, the wall's position is tracked by the minimum value of $\abs{\chi}$, i.e.~the $z$-coordinate at which the field is closest to zero. %We ignore the anti-wall that is left unperturbed.
We keep in mind that we do not expect a perfect match between the simulated and analytical wall perturbation, partly because the analytical equations are only valid for perturbations of leading order, but also because the thin-wall limit is expected to fail at the early stages of the phase transition. A more tangible way to look at it is to consider the unperturbed unit normal vector that we put along the $z$-axis; just by looking at the simulations (\cref{fig:results:epsilon:wall_profile_2D}), we can see that this is clearly not the case, at least close to phase transition.


We reduce the problem from three dimensions to two, and then again to one dimension by considering a suitable slice in the $y$-direction %
% (where $\sin{(2 \ppi m_{\lcoord{j}} \lcoord{j} )} =1$) 
(where $\sin{py} =1$) %
at which we take the $z$-coordinate of the minimal absolute value of the scalar field $\chi$. An example of the two-dimensional perspective is shown in~\cref{fig:results:epsilon:wall_profile_2D}. This picture is more or less the same for all simulations, at least when comparing by-eye. 

% ------------------------------
% ----------- FIGURE -----------
\begin{figure}[ht]
    \centering
    %
    \begin{subfigure}[b]{\linewidth}
        \centering
        \includegraphics[width=\linewidth]{Findings/wall_profile_2D.pdf}
        \caption{The domain wall evolution in the $yz$-plane of the simulation box. We indicate the aforementioned slice with a grey vertical line. The dash-dotted green graph is~$\epsA[\epsilon](\tau(a), y)$.}
        \label{fig:results:epsilon:wall_profile_2D}
    \end{subfigure}
    %
    \hfill
    \begin{subfigure}[b]{\linewidth}
        \centering
        \includegraphics[width=\linewidth]{Findings/achi_eps_1D_analysis.pdf}
        \caption{\figpanel{Left panel}~The scalar field value along $y$-coordinate $y=96\Delta_\#$ at different redshifts following the colour bar in the right panel. \figpanel{Right panel}~The wall coordinate as function of time.}
        \label{fig:results:epsilon:achi_eps_1D_analysis}
    \end{subfigure}
    % %
    \caption{Demonstration of the interpretation of the results from simulation~\simnum{1}.}
    \label{fig:results:epsilon:from_achi_to_epsilon}
\end{figure}
% ------------------------------


We observe a ``bump'' in the quasi-static $\tanh$-solution in each simulation, manifesting in plots like the left panel of~\cref{fig:results:epsilon:achi_eps_1D_analysis} around $z\sim z_0,\widebar{z}_0$, and in the maximum value of $\chi$ in~\cref{fig:results:achi:indepth_achi}. This effect is seemingly more prominent in simulations with $\epsast >0$, i.e.~everyone except~\simnum{0}. 

The simulated wall position graph is not perfectly overlapping with the analytical one, as the right panel of~\cref{fig:results:epsilon:achi_eps_1D_analysis} shows. Simulated walls show a tendency to evolve slower, at least initially, realised in a phase difference between $\epsC$ from simulation and $\epsA$ from the thin-wall approximation.%$\varepsilon(\tau)$ and . 


In~\cref{fig:results:epsilon:eps_diff_sims_combi} we show the difference between the Nambu--Goto prediction---which for $e=\varepsilon/\epsast$ vs. $t_\omega=\omega (s-1)$ is the same any setup---and the field theory. Simulations with similar perturbation setup (cf.~simulations~\simnum{1},~\simnum{2},~\simnum{4},~\simnum{7}) but different box parameters give similar wall evolution. 
% Note that there is a bump in the simulations with larger scalar field oscillations (simulations~\simnum{4},~\simnum{7})~\rcomment{Invisible...}. %
Larger initial amplitude (simulation~\simnum{2}) increases the difference, as does larger scale parameter (simulation~\simnum{5}). %
The bottom panel of~\cref{fig:results:epsilon:eps_diff_sims_combi} is remarkably alike~\cref{fig:results:achi:indepth_aq}.
% ----------------------------------------
% ---------------- FIGURE ----------------
\begin{figure}[h]
    \centering
    \includegraphics[width=\linewidth]{Findings/eps_diff_sims_combi.pdf}
    \caption{Functions of the scaled time parameter $t_\omega = \omega (s-1) = p(\tau-\tau_\ast)$.~\figpanel{Top panel}~The wall extremal position normalised to the initial amplitude, $e=\varepsilon/\epsast$.~\figpanel{Bottom panel}~The absolute difference between the wall position from calculations and simulations, $\Delta e = \abs*{\epsA-\epsB}/\epsast$.}
    \label{fig:results:epsilon:eps_diff_sims_combi}
    %%%%%%%%%%%%%
\end{figure}
% ----------------------------------------


We see that initial amplitude actually does matter in simulations, cf.~simulation \simnum{1} vs.~\simnum{2}. The thin-wall approximation disagrees with this, but it can be argued that this relates to the validity of the linear perturbation size. %does not say this, however, in fact it says the opposite.%; \checkthis{the equation of motion is scale invariant, and thus unchanged by translations.} 
% It is therefore hard to argue that this motion is possible to reproduce by adjusting terms in the equation of motion. 
% This points to higher-order effects or resolution issues as candidate explanations for the discrepancy, rather than it being a problem with the damping term in the equation of motion. Otherwise, this can just as easily be a manifestation of the approximation that the Nambu--Goto description actually is~\rcomment{Still bad sentence...}. 
This suggests that higher-order effects or resolution issues are more likely explanations for the discrepancy, rather than a problem with the damping term in the equation of motion. Alternatively, this discrepancy could simply be an indication of the inherent Nambu--Goto. %approximation in the Nambu--Goto descripition.


% Otherwise, it is a manifestation of the \grammar[]{disagreement} between the Nambu--Goto and the full field theories. \comment{bad sentence}

We observe that the initial perturbation amplitudes in simulations~\simnum{2} and~\simnum{4} are similar in comoving units, with $\epsast =0.12L_\#\approx 123~\Mpch$ and $ \epsast =0.08L_\# = 112~ \Mpch$, respectively. The large-box simulation gives results close to those with $\epsast =0.08L_\#$, whereas simulation~\simnum{2} shows different wall dynamics. The point is that as far as the linearised-perturbation analysis goes, the order is given relative to the comoving horizon $\tau=\tau_\ast$. This gives substance to the hypothesis that the discrepancy might be due to interactions with the anti-wall. On the other hand, when factoring in the perturbation scale parameter, we can think of the perturbation in simulation~\simnum{4} as equal to those of~\simnum{1},~\simnum{3}, etc. A better upper length scale may then be $\sim \wallsep$ as opposed to $\tau_\ast$.

    
    
% \rephrase{If not due to numerical error, this is necessarily either a consequence of the field-like description or possibly another damping term in the equation of motion for $\varepsilon$. In the latter scenario, one could guess that the expression for the surface tension is not flawless (something else would insinuate that the expansion term is wrong, which is not the case.) With better spatial resolution, there was no improvement for this part. Initialising simulations even closer to symmetry break enhanced oscillations and increased the phase difference. Increasing the box size---and scaling all parameters thereafter---did not have any effect in this matter.} \comment{Old analysis, needs to be updated.}


    

   


    % \begin{figure}[h]\label{fig:results:epsilon:eps_diff_sims}
    %     \centering
    %     \includegraphics[width=\linewidth]{Findings/eps_diff_sims.pdf}
    %     %%%%%%%%%%
    %     \caption{The absolute difference between the wall position from calculations and simulations as functions of time.}
    %     %%%%%%%%%%%%%
    % \end{figure}

    % This might indicate that the deviation is due to higher-order effects \comment{and the linearised analysis fails.}

    % We acknowledge that there might be effects such as inter-kink and possibly intra-kink ($y$-direction) forces at play. \boxed{\textsf{ii}}



    % \comment{Possible explanation: Field evolution not independent in $y$-direction, at least when $\epsast$ is comparable to $1/p$}
    % \begin{bullets}
    %     \item Inter-kink forces, and possibly intra-kink forces.
    % \end{bullets}


    % \pensive{The Wiener process is scale-invariant\dots}

   
    % \deleteme{\subsection*{Adjusting the equation of motion}}
        % We take a closer look at the wall evolution in one particular simulation. We discussed the effect of changing the damping term in the equation of motion for \dots
        % (Study effect of changing damping term.)
        % \textbf{[FIGURE]}}


        % \begin{figure}[h]
        %     \centering
        %     \includegraphics[width=\linewidth]{dummy_normal1.8.png}
        %     %%%%%%%%%%
        %     \caption{The wall position as depicted by \dots.}
        %     \label{fig:results:epsilon:epsilon_sim1}
        % \end{figure}

    % \iftime{Should I include old results that are very bad? Where $\epsast\sim L/4$.}

    \subsection{Review}% of \(\epsilon\)}
        %
        The full field-theoretical description of the wall evolution agrees with the thin-wall approximation to some level. We find that the maximum error is $ \Delta \varepsilon \approx (0.16 \pm 0.05)\epsast$ between simulations \simnum{1}--\simnum{5} and \simnum{7}. Similarly, we find average values $\Delta \varepsilon \approx (0.08 \pm 0.03)\epsast$. The general tendency is that the oscillatory behaviour in the Nambu--Goto description is quicker and somewhat weaker than what simulations would have it. Both show characteristics of a damped harmonic oscillator with wavelength $\gtrsim 2\ppi / t_\omega$. Upgrading the spatial and temporal resolution with about $15\%$---as done with simulation~\simnum{1} and~\simnum{3}---does not show significant changes in the evolution of the wall displacement parameter.
        
        Minimising scalar-field oscillations does not seem to affect the wall evolution particularly. This can be seen by comparing simulations \simnum{1}, \simnum{3}, \simnum{4} and \simnum{7} in~\cref{fig:results:epsilon:eps_diff_sims_combi} which all have the same relative initial amplitude, but different levels of fifth-force oscillations (\cref{fig:results:achi:indepth_achi}). However, changing the ``curvature'' of the wall, seems to change the overall behaviour of the displacement parameter $\varepsilon$. In particular, increasing the parameter $\Upsilon^{\AC}_\ast$ from 16 to 18 (simulation~\simnum{3}) or 24 (simulation~\simnum{5}). We use this as a naive quantification of the ``badness'' of the equation of motion for $\varepsilon$; larger amplitude and wavenumber correspond roughly to increased risk of higher-order effects. The unperturbed wall normal coordinate is aligned with the $z$-axis, and $\Upsilon^{\AC}_\ast$ is a rough measure of the perturbed wall's normal coordinate, relative to this. 
        % the larger amplitude, the more likely we are to observe higher-order effects, and the larger wavenumber, the farther we are from the unperturbed wall normal coordinate along the $z$-axis.
        % $n\^\mu \sim \deltaup\indices*{^\mu _z}$,
        %  roughly speaking~\lcomment{I dont like this sentence.}. %

        The wall--anti-wall system may inhabit inter-kink forces if the walls are not sufficiently far apart~\citep{vachaspatiKinksDomainWalls2006}, and this could contribute as an external force in the equation of motion for the wall perturbation. Since such forces would be attractive and position-dependent~\citep{vachaspatiKinksDomainWalls2006}, this could explain why $\epsB$ is ``slower'' and ``deeper'' than $\epsA$. %
        There might also be ``intra-kink'' forces at play, that is interactions between different points on the same wall. The parameter $\Upsilon^{\AC}_\ast$ can provide an indication as to the risk of this happening, at least in the $y$-direction. The walls are not infinitely thin and one can imagine that internal forces may contribute in the $z$-direction.


        % \iftime{It is difficult to see in~\cref{fig:results:epsilon:eps_diff_sims_combi}, but we notice that higher levels of scalar field fluctuations (simulations~\simnum{4} and~\simnum{7}) correspond to simulations where the wall evolution is \textit{discontinuous} in the very beginning. This is apparently not affecting the motion later. The two simulations in question are also characterised by initialisation closer to SSB, a feature that can technically be source of \emph{both} of these phenomena.}

        We have \emph{not} thoroughly tested whether $\epsB[\epsilon](\tau, x,y)=\epsB(\tau) \sin{py}$ holds over time in simulations. From~\cref{fig:results:epsilon:wall_profile_2D}, and similar results, we can see that the sine profile seems to be quite well-preserved throughout the evolution. 


    % \subsubsection{Why not?}
    %     In this section, we want to provide answers to the most likely questions the reader might have. Near on any question starting with ``why did you not ...'' may be answered ``because of temporal and computational limitations.'' 
    %     \paragraph{Perturbation amplitude.} %
    %     Why was it not increased to better resolve the motion in space? Recall that there needs to be \emph{two} walls present, and the kink profile should really not affect the antikink profile. In the quasi-static limit, the wall's thickness goes as $\sim {(a\chi_+)}^{-1}$, i.e. from infinitely large at symmetry break. 





