% ||||||||||||||||||||||||||||||||||||||
% |||||| 6.2 Domain wall dynamics ||||||
% ||||||||||||||||||||||||||||||||||||||

% ---------------------------------------------------
% labels: \label[type]{[type]:results:epsilon:[name]}
% ---------------------------------------------------




Ignore the anti-wall the box's boundaries. 
Simulation-wise, the wall's position is tracked by the minimum value of $\abs{\chi}$, i.e.~the $z$-coordinate at which the field is closest to zero. %We ignore the anti-wall that is left unperturbed.
We keep in mind that we do not expect a perfect match between the simulated and analytical wall perturbation, mostly because the analytical equations are only valid for perturbations of leading order. A more tangible way to look at is to consider the unit normal vector $n\^\mu$ that we put along the $z$-axis; just by looking at the simulations (\cref{fig:results:epsilon:wall_profile_2D}), we can see that this is clearly not the case, at least close to phase transition.


We reduce the problem from three dimensions to two with cylindrical symmetry, and then again to one dimension by considering a suitable slice in the $y$-direction and taking the coordinate of the minimal absolute value of the scalar field $\chi$. An example of the two-dimensional perspective is shown in~\cref{fig:results:epsilon:wall_profile_2D}. This picture is more or less the same for all simulations, at least when comparing by-eye.     

% ------------------------------
% ----------- FIGURE -----------
\begin{figure}[ht]
    \centering
    %
    \begin{subfigure}[b]{\linewidth}
        \centering
        \includegraphics[width=\linewidth]{Findings/wall_profile_2D.pdf}
        \caption{The domain wall evolution in two dimensions. We indicate the aforementioned slice with a green vertical line.}
        \label{fig:results:epsilon:wall_profile_2D}
    \end{subfigure}
    %
    \hfill
    \begin{subfigure}[b]{\linewidth}
        \centering
        \includegraphics[width=\linewidth]{Findings/achi_eps_1D_analysis.pdf}
        \caption{\figpanel{Left panel}~The scalar field value along $y$-coordinate XXX at different redshifts. \figpanel{Right panel}~The wall coordinate as function of time. Note that the colour bar share the same axis as the \blahblah.}
        \label{fig:results:epsilon:achi_eps_1D_analysis}
    \end{subfigure}
    % %
    \caption{Demonstration of results from simulation 1.}
    \label{fig:results:epsilon:from_achi_to_epsilon}
\end{figure}
% ------------------------------


\rephrase{We see that the quasistatic $\tanh$-solution varies in applicability as it occurs ``bumps'' around each wall after some time.} 

% --: constant surface tension \dots
% 0: surface tension $\sim \pclosed{ 1 - \upsilon }^{3/2}$ \dots
% 1: surface tension different? \dots
% X: ?

% Labels: \textsf{\textbf{Tn.i}} = Type T, equation/method n, initial conditions i


% Initial conditions: \textsf{\textbf{a}} $\epsilon_p(\tau_\ast)= \varepsilon_\ast$, $\dot{\epsilon} (\tau_\ast)=0$.


% \begin{figure}[h]\label{fig:results:epsilon:achi_eps_1D_analysis}
%     \centering
%     \includegraphics[width=\linewidth]{Findings/achi_eps_1D_analysis.pdf}
%     %%%%%%%%%%
%     \caption{\textit{Left panel:}~The scalar field value along $y$-coordinate XXX at different redshifts. \textit{Right panel:}~The wall coordinate as function of time. Note that the colour bar share the same axis as the \blahblah}
%     %%%%%%%%%%%%%
% \end{figure}

% \subsection{Technical note}
    % % \noindent\rule{\textwidth}{2pt}

    % This section's focus is on the solutions to
    % \begin{equation}
    %     \ddot{\epsilon}_p + \pclosed{ 3\dot{a}/a + \dot{\sigma}/\sigma } \,\dot{\epsilon}_p + p^2 \epsilon_p= 0.
    % \end{equation}
    % That is, we compare solutions obtained with different methods (\lbl{A}/\lbl{N}/\lbl{S}), variants (\lbl{--}/\lbl{0}/\lbl{1}/\lbl{X}) and initial conditions (\lbl{a}/\lbl{b}/\lbl{c})
    % %
    % Types:
    % \begin{description}
    %     \item[A] Completely analytical solution to eom.
    %     \item[N] Numerical solution (\texttt{odeint}) to eom. 
    %     \item[S] Simulated result.
    % \end{description}
    % % With type %\ref{itm:whatif:epsilon:A}  her
    % Subtype:
    % \begin{description}
    %     \item[--] Eom for $\epsilon_p(\tau)$ with $\sigma= \sigma_\infty$.
    %     \item[0] Eom for $\epsilon_p(\tau)$ with $\sigma= \sigma_\infty \pclosed{ 1 - \upsilon }^{3/2}$.
    %     \item[1] Eom for $\epsilon_p(\tau)$ with $\sigma= \sigma_\infty/ 2 \cdot \pclosed{ 3(1-\upsilon)- \breve{\chi}^2} \breve{\chi}$.
    %     \item[X] Where $\abs{\chi}$ takes its minimum value.
    % \end{description}
    % Initial conditions ($\tau_0$, $\epsilon_p(\tau_0)$, $\dot{\epsilon}_p(\tau_0)$) :
    % \begin{description}
    %     % \item[a] $\epsilon_p(\tau_\ast)= \varepsilon_\ast$, $\dot{\epsilon} (\tau_\ast)=0$
    %     % \item[b] $\epsilon_p(\tau\ped{init}) = \epsilon_p(\tau\ped{init})$ from \textsf{\textbf{A0.a}}
    %     \item[a] ($\tau_\ast$, $\varepsilon_\ast$, $0$) 
    %     \item[b] ($\tau\ped{init}$, $\epsilon_p(\tau\ped{init})$ from \lbl{A0.a}, $\dot{\epsilon}_p(\tau\ped{init})$ from \lbl{A0.a})
    % \end{description}




    % \noindent\rule{\textwidth}{2pt}


    % With $$k
    % \begin{description}
    %     \item[A0] Completely analytical solution to the naive eom for $\epsilon_p(\tau)$, i.e. Eq.~XXX from~\cref{sec:pertwalls:untitled1}.
    %     \item[N0] Numerical solution (\texttt{odeint}) to naive eom for $\epsilon_p(\tau)$. 
    %     \item[SX] The $z-L/2$-coordinate where $\abs{\chi}$ from simulation takes its minimum value. %We resolve $\Delta \tau / $  $\Delta x$ 
    % \end{description}



% \subsection{Idk}
    % It does not seem as if the 
    % \begin{bullets}
    %     \item The initial amplitude affects the phase of the simulated wall evolution, according to simulations.
    %     \item In any case, the wall pos. evolution is quite consistent for different levels of oscillations
    %     \item $\varepsilon_\ast > 1/p$: We will see great impact of changing $\varepsilon_\ast$
    %     \item Figure showing difference in position graphs
    %     \item Appendix with background quantities?
    % \end{bullets}   
    % %
    %
    %
    %
    The simulated wall position graph is not perfectly overlapping with the analytical one. Simulated walls show a tendency to evolve slower, at least initially, manifesting in a phase difference between $\varepsilon(\tau)$ from simulation and thin-wall approximation. 
    \rephrase{If not due to numerical error, this is necessarily either a consequence of the field-like description or possibly another damping term in the eom for $\varepsilon$. In the latter scenario, one could guess that the expression for the surface tension is not flawless (something else would insinuate that the expansion term is wrong, which is not the case.) With better spatial resolution, there was no improvement for this part. Initialising simulations even closer to symmetry break enhanced oscillations and increased the phase difference. Increasing the box size---and scaling all parameters thereafter---did not have any effect in this matter.} \comment{Old analysis, needs to be updated.}
    

    We saw that initial amplitude actually did matter in simulations, cf.~simulation \simnum{1} vs.~\simnum{2}. The thin-wall approximation does not say this, however, in fact it says the opposite; \checkthis{the EOM is scale invariant, and thus unchanged by translations.} It is therefore hard to argue that this motion is possible to reproduce by adjusting terms in the EoM. 

   


    % \begin{figure}[h]\label{fig:results:epsilon:eps_diff_sims}
    %     \centering
    %     \includegraphics[width=\linewidth]{Findings/eps_diff_sims.pdf}
    %     %%%%%%%%%%
    %     \caption{The absolute difference between the wall position from calculations and simulations as functions of time.}
    %     %%%%%%%%%%%%%
    % \end{figure}
    % ----------------------------------------
    % ---------------- FIGURE ----------------
    \begin{figure}[h]
        \centering
        \includegraphics[width=\linewidth]{Findings/eps_diff_sims_combi.pdf}
        \caption{The absolute difference between the wall position from calculations and simulations as functions of time.}
        \label{fig:results:epsilon:eps_diff_sims_combi}
        %%%%%%%%%%%%%
    \end{figure}
    % ----------------------------------------
    In~\cref{fig:results:epsilon:eps_diff_sims_combi} we show the difference between the NG prediction, which for $e=\varepsilon/\varepsilon_\ast$ vs. $t_\omega=\omega (s-1)$ is the same for each simulation, and the field theory. Simulations with similar perturbation setup (cf.~\simnum{1},~\simnum{2},~\simnum{4},~\simnum{7}) but different box parameters gives the similar wall evolution. Note that there is a bump in the simulations with larger scalar field oscillations (\simnum{4}~\&~\simnum{7}). %
    Larger initial amplitude (\simnum{2}) increases the difference, as does larger scale parameter (\simnum{5}). %
    This might indicate that the deviation is due to higher-order effects \comment{and the linearised analysis fails.}

    % We acknowledge that there might be effects such as inter-kink and possibly intra-kink ($y$-direction) forces at play. \boxed{\textsf{ii}}



    % \comment{Possible explanation: Field evolution not independent in $y$-direction, at least when $\varepsilon_\ast$ is comparable to $1/p$}
    % \begin{bullets}
    %     \item Inter-kink forces, and possibly intra-kink forces.
    % \end{bullets}


    % \pensive{The Wiener process is scale-invariant\dots}


    \subsection{Adjusting the equation of motion}
        We take a closer look at the wall evolution in one particular simulation. We discussed the effect of changing the damping term in the equation of motion for \blahblah
        \comment{Study effect of changing damping term.}


        \begin{figure}[h]
            \centering
            \includegraphics[width=\linewidth]{dummy_normal1.8.png}
            %%%%%%%%%%
            \caption{The wall position as depicted by \dots.}
            \label{fig:results:epsilon:epsilon_sim1}
        \end{figure}


    \subsection{\tmptitle{Discussion}}
        
        Minimising oscillations does not seem to affect the wall evolution particularly. This can be seen by comparing simulations \simnum{1}, \simnum{3}, \simnum{4} and \simnum{7} which all have the same relative initial amplitude, but different levels of oscillations. However, changing the \emph{curvature} of the wall, seems to change the overall behaviour of the wall. In particular, increasing the parameter $\Upsilon^{\AC}_\ast$ from 16 to 18 (sim. \simnum{3}) or 24 (sim. \simnum{5}). We use this as a naive quantification of the badness of the eom for $\varepsilon$; the larger amplitude, the more likely we are to see higher-order effects, and the larger wavenumber, the farther we are from the wall normal coordinate $n\^\mu \propto \deltaup\^{\mu z}$. I suppose it is also fair to assume some inter-kink forces or perhaps intra-kink forces could contribute to the equation of motion. It would have been interesting to solve the actual eom for the wall normal coordinate and see if we could come closer to the simulated result. \comment{Repeated argument...}

        We notice that higher levels of scalar field fluctuations (\simnum{4}~\&~\simnum{7}) corresponds to simulations where the wall evolution is \cringe{discontinuous} in the very beginning. This is apparently not affecting the motion later. The two simulations in question are also characterised by initialisation closer to SSB, a feature that can technically be source of \emph{both} of these phenomena.


    % \subsubsection{Why not?}
    %     In this section, we want to provide answers to the most likely questions the reader might have. Near on any question starting with ``why did you not ...'' may be answered ``because of temporal and computational limitations.'' 
    %     \paragraph{Perturbation amplitude.} %
    %     Why was it not increased to better resolve the motion in space? Recall that there needs to be \emph{two} walls present, and the kink profile should really not affect the antikink profile. In the quasi-static limit, the wall's thickness goes as $\sim {(a\chi_+)}^{-1}$, i.e. from infinitely large at symmetry break. 





