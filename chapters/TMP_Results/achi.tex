% |||||||||||||||||||||||||||||||||
% |||||| 7.X Symmetron field ||||||
% |||||||||||||||||||||||||||||||||

% ------------------------------------------------
% labels: \label[type]{[type]:results:achi:[name]}
% ------------------------------------------------

%%%%%%%%%%%%%%%%%%%%%%%%%%%%%%%%%%%%%%%%%%%%%%%%%%
% \newcommand{\lbl}[1]{\textsf{\textbf{#1}}}
\newcommand{\brphi}{\breve{\phi}}
\newcommand{\brchi}{\breve{\chi}}
%%%%%%%%%%%%%%%%%%%%%%%%%%%%%%%%%%%%%%%%%%%%%%%%%%





The symmetron field $\chi$ ($\mathtt{achi}$ in the code) will at SSB roll into either minima, depending on the sign of the field right before it happens. The strength of the oscillations around the true minima depend on both the initial field value and its time derivative. 

As we saw in \nc{section XX}, the solution for 



\subsection{Technical note}
    % \noindent\rule{\textwidth}{2pt}
    %
    We have the equations
    \begin{subequations}
        \begin{equation}
            \ddot{\phi} + 2\dot{a}/a \,\dot{\phi} - \vec{\nabla}^2 \phi = -a^2 \dv*{V\ped{eff}}{\phi}
        \end{equation}
        and
        \begin{equation}
            \ddot{\brphi} + 2\dot{a}/a \,\dot{\brphi} = -a^2 \dv*{V\ped{eff}}{\phi}
        \end{equation}
    \end{subequations}
    % \begin{subequations}
    %     \begin{equation}
    %         \ddot{\chi} + 2\dot{a}/a \,\dot{\chi} - \vec{\nabla}^2 \chi = -a^2 \dv*{V\ped{eff}}{\chi}
    %     \end{equation}
    %     and
    %     \begin{equation}
    %         \ddot{\brchi} + 2\dot{a}/a \,\dot{\brchi} = -a^2 \phi_\infty^{-2} \dv*{V\ped{eff}}{\chi}.
    %     \end{equation}
    % \end{subequations}
    The latter is in focus first. We compare solutions obtained with different methods (\lbl{A}/\lbl{N}/\lbl{S}), variants (\lbl{--}/\lbl{0}/\lbl{1}/\lbl{X}) and initial conditions (\lbl{a}/\lbl{b}/\lbl{c}).
    %
    % Types:
    % \begin{description}
    %     \item[A] Completely analytical solution.
    %     \item[N] Numerical solution (\texttt{odeint}).
    %     \item[S] Simulated result.
    % \end{description}
    %
    Subtype:
    \begin{description}
        \item[0] Minima $\chi_\pm=\pm\sqrt{1-\upsilon}$
        \item[1] 
        \item[T] Equation 
        \item[X] $\chi$ evaluated at $z=0.8 L$
        \item[B] $\max{\chi}$ \comment{or $\sqrt{\chi\ped{avg}}$}  
    \end{description}
    %
    Initial conditions ($\tau_0$, $\brchi(\tau_0)$, $\dot{\brchi}(\tau_0)$) :
    \begin{description}
        \item[t] ($\tau_\ast$, $\chi_\ast$, $3\mathcal{C}$) 
        \item[b] 
    \end{description}

    % Example: \textsf{5$\to$}\textsf{\textchi:}\lbl{N1.a} \dots 


    % Format: \textsf{(sim. number)}\textbf{field:}\lbl{type.ICs}


    % \noindent\rule{\textwidth}{2pt}




\subsection{Some title}
    The solution to the
    \begin{figure}[h]\label[fig]{fig:results:achi:achi_sim1}
        \centering
        \includegraphics[width=\linewidth]{Findings/achi_sim1.pdf}
        %%%%%%%%%%
        \caption{$\brchi$ as function of conformal time $s=\tau/\tau_\ast$.}
        %%%%%%%%%%%%%
    \end{figure}



    These results, and simply the fact that they are \blahblah
    

    The dimensionless time variable $t_p = (\tau-\tau\ped{init}) \cdot p$ or $t_p = (\tau-\tau_\ast) \cdot p= (s-1)u$

    