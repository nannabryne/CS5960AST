
% ||||||||||||||||||||||||||||||||||||||
% |||||| 7.X Gravitational waves ... ||||||
% ||||||||||||||||||||||||||||||||||||||

% -----------------------------------------------
% labels: \label[type]{[type]:results:h11:[name]}
% -----------------------------------------------

%%%%%%%%%%%%%%%%%%%%%%%%%%%%%%%%%%%%%%%%%%%%%%%%%%
% \newcommand{\lbl}[1]{\textsf{\textbf{#1}}}
% \newcommand{\brphi}{\breve{\phi}}
% \newcommand{\brchi}{\breve{\chi}}
%%%%%%%%%%%%%%%%%%%%%%%%%%%%%%%%%%%%%%%%%%%%%%%%%%




The tensor field is in principle more complex (literally) than the previously discussed scalar field. It is therefore advantageous that our proposed field only has one degree of freedom since---in theory---$h_{\times} = 0$. 

% \begin{equation}
%     h\_{ij} = h\_+ e^+_{ij} + h\_\times e^+_{ij}
% \end{equation}

As we saw in the previous section, the wall evolution differs somewhat in the two models. In this section, we will investigate if and how this difference affects the gravitational wave modes, and we will \blahblah





\paragraph{About the output from the code ...}
    We quickly see from calculations that $h\_{ij}(\tau, \vec{k})\in \Real$. The code will have it differently, however, and consistently produces non-negligible imaginary components. Likely, this has to do with the different Fourier conventions used by hand and by code. We have not been able to resolve this completely (i.e.~find a suitable mapping), and so we present only the magnitude of the strain.


\paragraph{Computing the semi-analytical expression ...}
    To find $H_{1,2}(\tau, \vec{k})$ we need to use a numerical solver, and for this we chose \textit{Numpy}'s  \texttt{cumtrapz}; a method for integrating cumulatively with the trapezoidal rule. 
    
    %\texttt{odeint} from the \texttt{integrate}-module from the \texttt{Python} library \texttt{Scipy}. 





% \section{Mode by mode}
    We extract the relevant output from \texttt{gwasevolution} to compare with the analytical calculations. Nothing is assumed about the temporal part of the wall normal coordinate, so we may insert any function as $\epsilon_q(\tau)$ into \nc{Eq. XXX}[main expr.]. This is a huge advantage since the results from~\cref{chap:results} are not perfect. 
    
    An even bigger advantage would be to have the code output the wall position as a near-continuous function of time, but we only have the profile extracted from $\mathtt{achi}$ animation outputs, giving it a function with \blahblah


    \paragraph{A few take-aways.} %
    There are some results that need be mentioned, but not necessarily presented plots. 
    \subparagraph{Periodicity in $y$-mode.} %
    The outputted \texttt{hijFT} shows significantly smaller strains for $K_Y\neq n m_Y$ than for $K_Y = n m_Y$, something we interpret as a corroboration to the Dirac--delta factor in \nc{Eq. XXX}[main expr.]. It being non-zero may be a result of numerical error, but it is likely also related to the issue with \nc{the wavenumber ambiguity}. 
    \subparagraph{$K_Y=0$ is non-zero.} %
    For whatever reason, the code insists there are significant tensor perturbation propagating in the $Z$-direction. This is not what we expected from calculations, where $\vec{k}=(0,0,k\_{z})$ corresponds to zero strain.

    \comment{Looks like $k_z < k_y$ does not carry much information, but I believe that makes sense. They should not be ``kinky'' here, but free.}
    \comment{Real component looks generally more messy, might be due to not perfect spatial part?}

\subsection{Technical note}



\subsection{Idk}
    



\begin{bullets}
    \item Adding ingredients to expression for $h\_{ij}$ $\leadsto$ Separate effects %
    \begin{itemize}
        \item Changing $\epsilon$: Changes the ``phase'' 
        \item Changing $\sigma$: Introduces more small oscillations
        \item Changing $l$ (thickness): Even more of the small oscillations
    \end{itemize}
    \item 
\end{bullets}