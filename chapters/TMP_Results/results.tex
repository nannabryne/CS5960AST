%%%%%%%%%%%%%%%%%%%%%%%%%%%%%%%%%%%%%%%%%%%%%%
%%%%%%% Ch. X: Results %%%%%%%
%%%%%%%%%%%%%%%%%%%%%%%%%%%%%%%%%%%%%%%%%%%%%%


% -------------------------------------------
% labels: \label[type]{[type]:results:[name]}
% -------------------------------------------

%%%%%%%%%%%%%%%%%%%%%%%%%%%%%%%%%%%%%%%%%%%%%
\newcommand{\lbl}[1]{\textsf{\textbf{#1}}}
\newcommand{\completelbl}[4]{%
\textbf{#1)}%
\textbf{#2:}%
\lbl{#3.#4}%
}
% \newcommand{\completelbl}[4]{%
% \textbf{#1:}%
% % \textbf{#2:}%
% \lbl{#3.#4}%
% $\to${#2}
% }
%%%%%%%%%%%%%%%%%%%%%%%%%%%%%%%%%%%%%%%%%%%%%





% ////////////////// intro //////////////////


This chapter both presents and discusses the results concerning the wall normal coordinate $\epsilon$ from \nc{some section}[about analytic work] and the domain wall scalar field $\chi$. We compare results from analytical calculations and numerical simulations. 

% This chapter presents and comments results 

% To emphasise \blahblah, we refer to simulation output by their variable name, e.g.~$\mathtt{achi}$ corresponds to the (a)symmetron field $\chi$.


% ///////////////////////////////////////////





In order to keep track of the various results, or rather how they were obtained, we create a syntax that should make this task simple. When relevant, we will use the simulation labels from \nc{XXX}. In this section we make use of keys \lbl{A}, \lbl{N} and \lbl{S}, described below.
%
\begin{description}
    \item[A] Completely analytical solution.
    \item[N] Numerical solution (\texttt{odeint}).
    \item[S] Simulated result.
\end{description}
%
In general, the complete label of one result (perhaps a graph) might be \completelbl{3}{$\epsilon$}{N1}{b}, referring to setup 3, numerical solution (variant 1) with initial conditions b. It will become clear what this means in particular in the following sections.





\section{Symmetron field}\label[sec]{sec:results:achi}
    {\subimport{./}{achi.tex}}


\section{Domain wall dynamics}\label[sec]{sec:results:epsilon}
    {\subimport{./}{epsilon.tex}}

\section{Gravitational waves from perturbed symmetron domain walls}\label[sec]{sec:results:h11}
    {\subimport{./}{h11.tex}}