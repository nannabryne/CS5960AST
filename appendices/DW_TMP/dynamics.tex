% |||||||||||||||||||||||||||||||||||
% |||||| C.2 Dynamics ||||||
% |||||||||||||||||||||||||||||||||||

% --------------------------------------------
% labels: \label{[type]:walls:dynamics:[name]}
% --------------------------------------------




As in~\cref{sec:pertwalls:thinwall}, we vary the Nambu--Goto action (\cref{eq:walls:covariant_action,eq:pertwalls:thinwall:Nambu_Goto_action_dw_FLRW}), only now we include $\sigma$ inside the integral: $S\nped{NG}= -\integ[3]{\xi\sqrt{-\gamma}}[\hypsurf] \sigma$. We use $\hat{\gamma}\_{ab} = a^{-2}\gamma\_{ab}$ and find that the determinant is given by
\begin{equation}
    \sqrt{-\hat{\gamma}}= \sqrt{1 + \eta\_{ab} \partial\^a \epsilon \partial\_b \epsilon} 
    = 1 +  \frac{1}{2} \eta\_{ab} \partial\^a \epsilon \partial\_b \epsilon  + \mathscr{O}(\epsilon^3),
\end{equation}
where $\epsilon$ is the wall displacement variable. %
We vary the action with respect to small changes in $\epsilon$, giving
\begin{align}
    -\Fdv{S\nped{NG}}{\epsilon} \delta \epsilon &= \integ[3]{\xi}[\hypsurf]  \beta^3 \frac{1}{2}  
    \bclosed{ \eta\_{ab} \partial\^a \epsilon \partial\_{b}(\delta \epsilon) }
    \cdot 2  \nonumber \\ 
    &= \integ[3]{\xi} 
    \bclosed {- \beta^3 \eta\_{ab} \partial\^a \partial\_b \epsilon + \beta^3 3 (\partial_0 \beta / \beta) \partial_0 \epsilon } \delta\epsilon,
\end{align}
where $\beta^3 \equiv \sigma a^3$. %
In the last line, we used $\partial\_{\underaccent{\bar}{a}} \epsilon \partial\_{\underaccent{\bar}{a}}(\delta \epsilon)= \partial\_{\underaccent{\bar}{a}} (\partial\_{\underaccent{\bar}{a}} \epsilon\cdot \delta \epsilon)- \partial\_{\underaccent{\bar}{a}} \partial\_{\underaccent{\bar}{a}} \epsilon  \cdot \delta \epsilon$, where the under-bar signifies that it is not a summation, and cancelled vanishing surface terms. We get
\begin{equation}\label{eq:walls:dynamics:eom_beta}
    \partial_\tau^2 \epsilon + 3 ( \partial_\tau \beta / \beta) \partial_\tau \epsilon -  (\partial_x^2 + \partial_y^2) \epsilon = 0.
    % \partial_0^2 \epsilon + 3 ( \partial_0 \beta / \beta) \partial_0 \epsilon -  (\partial_1^2 + \partial_2^2) \epsilon = 0.
\end{equation}


% \begin{equation}
%     \hat{\gamma}\_{ab} = \eta\_{\mu\nu} \partial\_a X\^\mu \partial\_b X\^\nu
% \end{equation}







% Following~\cref{sec:pertwalls:thinwall}, we have the thin domain wall in four-dimensional expanding, flat spacetime $\Manifold$, represented by the hypersurface $\hypsurf$ located at $x\^\mu = X\^\mu(\xi\^a)$, given by
% \begin{equation}
%     X\^\mu(\xi\^a) = \Krondelta*{^{\mu}_a}\xi\^a + \Krondelta*{^{\mu}_3} X_\bot,%(\xi\^a),
% \end{equation}
% where Greek indices take values $0,1,2,3$ and Latin $a, b, c$ take $0,1,2$. Note that Latin indices $i,j,k, \dots$ still represent indices in the spatial sector.


% We go about the variation in a slightly less rigorous, yet more intuitive manner. Consider the Nambu--Goto action for the domain wall in expanding spacetime
% \begin{equation}
%     S\nped{NG} = -  \integ[3]{\xi\sqrt{-\gamma}}[\varSigma] \sigma,
% \end{equation}
% where we keep the surface tension generally dependent on time, and the induced metric is
% % With the embedding $X\^\mu(\xi\^a)$, we have
% \begin{equation}
%     \gamma\_{ab} = g\_{\mu\nu} \partial\_{a} X\^\mu \partial\_{b} X\^\nu,
% \end{equation}
% with determinant
% \begin{equation}
%     \gamma \equiv \det(\gamma) = \tilde{\epsilonup}\^{abc}\gamma\_{a0}\gamma\_{b1}\gamma\_{c2},
% \end{equation}
% where $\tilde{\epsilonup}$ is the Levi--Civita symbol. 
% % for the induced metric. 
% For $X_\bot \to \pert{X}_\bot = X_\bot +  \epsilon$, 
% % For $\xi^a = x\^a$ and $X\^\mu = (\tau, x, y, z_0 + \epsilon)$, 
% where $\epsilon$ is a small perturbation, we get
% % \begin{equation}\label{eq:derivations:NG_action:det_gamma}
% %     \gamma =  -a^6 \cclosed{1 + \eta\_{ab} \partial\^a \epsilon \partial\_b \epsilon},
% % \end{equation}
% % which gives 
% \begin{equation}
%     \sqrt{-\gamma}= a^3\sqrt{1 + \eta\_{ab} \partial\^a \epsilon \partial\_b \epsilon} 
%     = a^3 \cclosed{ 1 +  \frac{1}{2} \eta\_{ab} \partial\^a \epsilon \partial\_b \epsilon } + \mathscr{O}(\epsilon^3).
% \end{equation}
% We vary the action with respect to small changes in $\epsilon$, giving
% \begin{align}
%     \Fdv{S\nped{NG}}{\epsilon} \delta \epsilon &=- \integ[3]{\xi}[\hypsurf]  \sigma a^3 \frac{1}{2}  
%     \bclosed{ \eta\_{ab} \partial\^a \epsilon \partial\_{b}(\delta \epsilon) }
%     \cdot 2 + \mathscr{O}\big((\delta \epsilon)^2 \big)
% \end{align}
% With $\partial\_{\underaccent{\bar}{a}} \epsilon \partial\_{\underaccent{\bar}{a}}(\delta \epsilon)= \partial\_{\underaccent{\bar}{a}} (\partial\_{\underaccent{\bar}{a}} \epsilon\cdot \delta \epsilon)- \partial\_{\underaccent{\bar}{a}} \partial\_{\underaccent{\bar}{a}} \epsilon  \cdot \delta \epsilon$, where the under-bar signifies that it is not a summation, and vanishing surface terms, we get
% \begin{align}
%     -\Fdv{S\nped{NG}}{\epsilon} \delta \epsilon = \integ[3]{\xi} 
%     \bclosed {- \beta^3 \eta\_{ab} \partial\^a \partial\_b \epsilon + \beta^3 3 (\partial_0 \beta / \beta) \partial_0 \epsilon } \delta\epsilon,
% \end{align}
% where $\beta^3 \equiv \sigma a^3$. The equation of motion becomes
% \begin{equation}
%     \partial_0^2 \epsilon + 3 ( \partial_0 \beta / \beta) \partial_0 \epsilon -  (\partial_1^2 + \partial_2^2) \epsilon = 0.
% \end{equation}


% % For $\xi^a = x\^a$ and $X\^\mu = (\tau, x, y, z_0 + \epsilon)$, where $\epsilon$ is a small perturbation, we get
% % \begin{equation}
% %     \begin{split}
% %         \sqrt{-\gamma} &= a^3\sqrt{1-(\partial_\tau \epsilon)^2 + (\partial_x \epsilon)^2 + (\partial_y \epsilon)^2 } \\
% %         &= a^3 \cclosed{ 1 + \frac{1}{2} \bclosed{-(\partial_\tau \epsilon)^2 + (\partial_x \epsilon)^2 + (\partial_y \epsilon)^2} + \mathscr{O}(\epsilon^3)} 
% %     \end{split}
% % \end{equation}
% % We vary the action with respect to small changes in $\epsilon$ \blahblah
% % \begin{align}
% %     \Fdv{S\nped{NG}}{\epsilon} \delta \epsilon &= - \integ[3]{\xi}[\hypsurf]  \sigma a^3 \frac{1}{2}  \bclosed{ \partial_x \epsilon \partial_x (\delta \epsilon) + \partial_y \epsilon \partial_y (\delta \epsilon)  - \partial_\tau \epsilon \partial_\tau (\delta \epsilon) }\cdot 2 + \mathscr{O}((\delta \epsilon)^2)
% % \end{align}
% % With $\partial\_a \epsilon \partial\_a(\delta \epsilon)= \partial\_a (\partial\_a \epsilon \delta \epsilon)- \partial\_a \partial\_a \epsilon \delta \epsilon$ and vanishing surface terms, we get
% % \begin{align}
% %     \Fdv{S\nped{NG}}{\epsilon} \delta \epsilon &= 
% % \end{align}






\subsection{Full solution}\label{app:walls:dynamics:full_A_B}
    In matter-dominated universe,~\cref{eq:walls:dynamics:eom_beta} as solutions that are a superposition of $ \varepsilon(\tau) \sppt(x,y)$ where
    \begin{equation}\label{eq:walls:dynamics:eom_eps_s_MD}
        \varepsilon'' + \left( \frac{6}{s}  +\frac{9}{2s\left(s^6-1\right)} \right) \varepsilon' + \omega^2 \varepsilon = 0
    \end{equation}
    and $\tau_\ast^2(\partial_x^2 + \partial_y^2)\sppt= -\omega^2 $. %
    In~\cref{sec:pertwalls:mywalls} we obtained a solution in two regimes, one of these in terms of coefficients $A(\omega)$ and $B(\omega)$. %
    For reference, we give the coefficients $A$ and $B$ in~\cref{eq:pertwalls:mywalls:eps_s_II_MD}, obtained by matching solutions at $s=s\ped{sow}=1+\omega^{-1}$. 
    %

    \begin{subequations}\label{eq:walls:dynamics:full_A_B}
        First, we define
        \begin{multline}
            D\triangleq \sqrt{2/\ppi}\pclosed{ \omega^9 + 4\omega^8 + 6\omega^7 + 4\omega^6 + \omega^5} \sqrt{\omega+1 }  \\
            \times \bclosed{ \frac{ 6\omega^5 + 15\omega^4 + 20\omega^3 + 15\omega^2 + 6\omega + 1 }{\omega^6} }^{(1/4)}.
        \end{multline}
        With 
        \begin{multline}
            P_1 \triangleq  336\omega^9 + 1608\omega^8 + 1051\omega^7 - 6407\omega^6 -17850\omega^5  \\-22050\omega^4 -15162\omega^3 - 5766\omega^2 - 1036\omega - 46
        \end{multline}
        and
        \begin{multline}
            P_2 \triangleq  84\omega^9 + 1641\omega^8 + 7036\omega^7 +12708\omega^6 +11202\omega^5 \\+ 3120\omega^4 - 2490\omega^3 - 2292\omega^2 -  604\omega - 30,
        \end{multline}
        we have
        \begin{align}
            A(\omega ) &= - \frac{ P_1\cos{(\omega + 1)} - P_2\sin{(\omega + 1)} }{D}, \\
            B(\omega ) &= - \frac{ P_2\cos{(\omega + 1)} + P_1\sin({\omega + 1}) }{D}.
        \end{align}
        %
    \end{subequations}
    Now,~\cref{eq:pertwalls:mywalls:eps_s_complete_MD} with~\cref{eq:walls:dynamics:full_A_B} gives a completely analytical solution to~\cref{eq:pertwalls:mywalls:eom_eps_s_MD}.

