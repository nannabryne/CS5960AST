% |||||||||||||||||||||||||||||||||
% |||||| C.2 Stress--energy ||||||
% |||||||||||||||||||||||||||||||||

% ---------------------------------------------------
% labels: \label{[type]:walls:SE_tensor:[name]}
% ---------------------------------------------------




% \hl{FIX! OR DROP.} %
The variation of~\cref{eq:walls:covariant_action} gives~\cref{eq:pertwalls:eom_wall:SE_tensor_NG}, for which we require
% We rewrite the variation
\begin{equation}
    \Fdv{\sqrt{-\gamma}}{g\_{\mu\nu}} =\frac{-1}{2\sqrt{-\gamma}} \Fdv{\gamma}{g\_{\mu\nu}} = \frac{-a^4}{2\sqrt{-\gamma}} \Fdv{\hat{\gamma}}{\eta\_{\mu\nu}} .
\end{equation}
% We compute the 
To leading order in $\epsilon$, we get
% \begin{equation}
%     \Fdv{\gamma}{g\_{ab}}  = -a^4 \eta\_{ab}, \quad
%     \Fdv{\gamma}{g\_{a3}}  = -a^4 \partial\_a \epsilon, \quad
%     \Fdv{\gamma}{g\_{33}}  =  0.
% \end{equation}
\begin{equation}\label{eq:walls:SE_tensor:variation_of_det_gamma}
    \Fdv{\hat{\gamma}}{\eta\_{ab}}  = -\eta\^{ab}, \quad
    \Fdv{\hat{\gamma}}{\eta\_{a3}}  = -\partial\^a \epsilon, \quad
    \Fdv{\hat{\gamma}}{\eta\_{33}}  =  0.
\end{equation}
% \begin{equation}
%     \Fdv{\gamma}{g\_{\mu\nu}} = -a^4 \left(\begin{array}{cccc}
%         -1 & 0 & 0 & \partial_0 \epsilon  \\
%         0 & 1 & 0 & \partial_1 \epsilon \\
%         0 & 0 & 1 & \partial_2 \epsilon \\
%         \partial_0 \epsilon & \partial_1 \epsilon & \partial_2 \epsilon& 0  \\
%     \end{array}\right) + \mathscr{O}(\epsilon^2)
% \end{equation}
Inserted into~\cref{eq:pertwalls:eom_wall:SE_tensor_NG}, we get $W\^{\mu\nu}= T\^{\mu\nu}\rvert\nped{NG} + \mathscr{O}(\epsilon^2)$, i.e.
\begin{equation}
    W\^{\mu\nu}(\tau, \vec{x}) = \frac{\sigma \Diracdelta(z-z\ped{w})}{a^3}\Fdv{\hat{\gamma}}{\eta\_{\mu\nu}}.
\end{equation}
Note that this holds for any first-order perturbation $\epsilon(\tau,x,y)$. 
% \begin{equation}
%     \begin{split}
%         T\^{\mu\nu}
%     \end{split}
% \end{equation}





\subsection{Fourier-space SE tensor}\label{app:walls:SE_tensor:FT_SE}   
    We let $\epsilon=\epsilon(\tau,y)$. 
    % We let $\epsilon = \varepsilon(\tau) \sppt(y)$ and $\sppt(y)=\sin{py}$ in this subsection.
    Following~\cref{sec:pertwalls:gws:Fourier_SE_tensor}, we look at~\cref{eq:pertwalls:gws:def_I_s_and_a} for $\sppt(y)=\sin{py}$ in $\epsilon= \varepsilon(\tau) \sppt(y)$. 
    The trick is to identify the Jacobi--Anger expansion,~\cref{eq:cylinder:prop:JacobiAnger_sin},
    % \footnote{
    %     The analogous relation for cosine is obtained by inserting $bx \to bx + \ppi/2$ to get an extra factor $\im^n$ inside the sum.
    % }
    \begin{equation}
    f(x) \triangleq \eu[\im a \sin{bx}] = \sum_{n=-\infty}^{\infty} \Bessel[n](a) \eu[\im n bx].
    \end{equation}
    We postulate that the Fourier transform $\tilde{f}(\omega)\triangleq\integ{x} \sum_{n} F_n(x;\omega)$ is 
    \begin{align}
    \tilde{f}(\omega) &= %\integ{x} f(x)\eu[\im \omega x] =
        \sum_{n} \Bessel[n](a) \integ{x}\eu[\im (\omega+ nb)x]\nonumber \\
        &= 2\ppi  \sum_{n} \Bessel[n](a) \Diracdelta(\omega + nb).
    \end{align}
    We assumed that the %integration over $x$ and summation over $n$ 
    integration and summation operators commute, which is known to be true for $\integ{x}\sum_{n}\abs{F_n(x)} < \infty$ (Fubini's theorem). \lcomment{Comment?} 
    % \speak{Why is this not the case, again?} \blahblah
    For %$g(x)\equiv cb \cos{bx} \cdot f(x) = \frac{1}{2}cb{( \eu[\im bx] + \eu[-\im bx]) } f(x)$, we find
    \begin{equation}
    g(x)\triangleq cb \cos{bx} \cdot f(x) = \frac{1}{2}cb \pclosed{\eu[\im bx] + \eu[-\im bx] } f(x),
    \end{equation}
    we find:
    \begin{align}
    \tilde{g}(\omega) &= 
    \frac{cb}{2}\sum_{n} \Bessel[n](a) \integ{x} \bclosed{\eu[\im (\omega+ (n+1)b)x] +\eu[\im (\omega+ (n-1)b)x] }\nonumber \\
    &= \ppi cb \sum_{n} \bclosed{ \Bessel[n+1](a) +  \Bessel[n-1](a)} \Diracdelta(\omega + nb ) \nonumber \\
    &= \frac{2\ppi cb}{a} \sum_n n \Bessel[n](a) \Diracdelta(\omega + nb ),
    \end{align}
    where in the last line we uses~\cref{eq:derivations:cylinder:recurr_Z_plus}. 

    The non-vanishing SE-tensor components in~\cref{eq:pertwalls:gws:Ft_SE_tensor_general} read
    \begin{equation}\label{eq:walls:SE_tensor:W_ij}
        \begin{split}
            \tilde{W}\_{ab} &= -  a \sigma\ped{w} \eta\_{ab}\cdot 2\ppi \Diracdelta(k_x)\mathscr{D}\ped{w} I\ped{s},  \\
            \tilde{W}\_{iz} &= -  a \sigma\ped{w} \Krondelta{_{iy}} \cdot 2\ppi \Diracdelta(k_x)\mathscr{D}\ped{w} I\ped{a},
        \end{split}
    \end{equation}
    where the expressions in~\cref{eq:pertwalls:gws:def_I_s_and_a} are
    % We use this to find the relevant expressions in~\cref{sec:pertwalls:gws:Fourier_SE_tensor}:
    \begin{subequations}\label{eq:derivations:jacobianger:I_s_and_a}
    \begin{align}
        I\ped{s} %&= \integ{y} \sum_{n} \Bessel[n](k_z \varepsilon)  \eu[\im np y  ] \eu[\im k_y y ]\nonumber \\
        & = 2\ppi \sum_{n}  \Bessel[n](k_z \varepsilon) \Diracdelta(k_y + np),\label{eq:derivations:jacobianger:I_s_only} \\
        I\ped{a} %&= \integ{y} \sum_{n} \Bessel[n](k_z \varepsilon)  \eu[\im np y  ] \eu[\im k_y y ]\nonumber \\
        & = 2\ppi \frac{p}{k_z} \sum_{n} n \Bessel[n](k_z \varepsilon) \Diracdelta(k_y + np) = - \frac{k_y}{k_z}I\ped{s}.\label{eq:derivations:jacobianger:I_a_only}
    \end{align}
    \end{subequations} 



\subsection{Spin-2 projection}\label{app:walls:SE_tensor:spin2}
    We consider a symmetric $3\cross 3$ tensor ${S}\_{ij}$ with momentum $\vec{k}=(0,n_2, n_3)k$. Projected onto the TT gauge, the $11$-component reads
    \begin{align}\label{eq:walls:SE_tensor:TT_projection_11}
        S\ap{TT}_{11} = \ProjectionLambda{11}{ij}S\_{ij} = \frac{1}{2}\bclosed{ S\_{11} + 2n_2 n_3 S\_{23} - n_3^2 S\_{22} - n_2^2 S\_{33}},
    \end{align}
    where we used the projector in~\cref{eq:notation:projection_tensor} and $n_2^2 + n_3^2 = 1$. %
    We find that the linear polarisation basis is defined through the polarisation tensors
    \begin{equation}\label{eq:walls:SE_tensor:linear_pol_basis}
        e\indices*{^+_{ij}}= %
        {\left(\begin{array}{ccc}
            1 & 0 & 0 \\
            0 & -n_3^2 & n_2 n_3 \\
            0 & n_2 n_3 & -n_2^2 
        \end{array}\right)},%
        \quad %\land \quad %
        e\indices*{^{\times}_{ij}} =
        {\left(\begin{array}{ccc}
            0 & n_3 & -n_2 \\
            -n_3 &0 & 0 \\
            n_2 & 0& 0
        \end{array}\right)}, %; \quad \vec{k}=(0,k_y,k_z).%
    \end{equation}
    such that $S\iud*{\mathrm{TT}}{ij} = \sum_{P=+, \times} S_P e\indices*{^{P}_{ij}}$. This gives $S_+ = S\iud*{\mathrm{TT}}{11}$.

    Following~\cref{sec:pertwalls:gws}, we have $\tilde{W}\_{11} =  \tilde{W}\_{22} $ for the SE tensor in~\cref{eq:walls:SE_tensor:W_ij}, which by using \cref{eq:walls:SE_tensor:TT_projection_11} yields
    \begin{equation}\label{eq:walls:SE_tensor:W_plus_TT}
        \tilde{W}_+ = \frac{1}{2} \bclosed{ n_2^2 \tilde{W}\_{11} + 2 n_2 n_3 \tilde{W}\_{23} }
    \end{equation}
    With~\cref{eq:derivations:jacobianger:I_s_and_a} we get the plus-polarised SE tensor
    \begin{equation}\label{eq:walls:SE_tensor:W_plus_final}
        \tilde{W}_+  = - \frac{n_2^2 }{2}  \tilde{W}\_{11}.
    \end{equation}







