% |||||||||||||||||||||||||||||||||||||||||||||
% |||||| A.X Computation of Fourier SE tensor ||||||
% |||||||||||||||||||||||||||||||||||||||||||||


% -----------------------------------------------------
% labels: \label{[type]:derivations:jacobianger:[name]}
% -----------------------------------------------------


% ¨¨¨¨¨¨¨¨¨¨¨¨¨¨¨¨¨¨¨¨¨¨¨¨¨¨¨¨¨¨¨¨¨¨¨¨¨¨¨¨¨¨¨¨¨¨¨¨¨
% \newcommand{\pert}[1]{\accentset{\circ}{#1}}
\newcommand{\sppt}{\ALIASsppt}
% ¨¨¨¨¨¨¨¨¨¨¨¨¨¨¨¨¨¨¨¨¨¨¨¨¨¨¨¨¨¨¨¨¨¨¨¨¨¨¨¨¨¨¨¨¨¨¨¨¨

\important{Relevant for~\cref{sec:pertwalls:gw_production}}


We look at $\sppt(y)=\sin{py}$ in $\epsilon= \varepsilon(\tau) \sppt(y)$

The trick is to identify the Jacobi--Anger expansion,\footnote{
    The analogous relation for cosine is obtained by inserting $\theta = \theta' + \ppi/2$ to get an extra factor $\im^n$ inside the sum.
}
\begin{equation}
    \eu[\im x \sin{\theta}] = \sum_{n=-\infty}^{\infty} \Bessel[n](x) \eu[\im n \theta].
\end{equation}
