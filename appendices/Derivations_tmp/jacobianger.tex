% |||||||||||||||||||||||||||||||||||||||||||||
% |||||| A.X Computation of Fourier SE tensor ||||||
% |||||||||||||||||||||||||||||||||||||||||||||


% -----------------------------------------------------
% labels: \label{[type]:derivations:jacobianger:[name]}
% -----------------------------------------------------


% ¨¨¨¨¨¨¨¨¨¨¨¨¨¨¨¨¨¨¨¨¨¨¨¨¨¨¨¨¨¨¨¨¨¨¨¨¨¨¨¨¨¨¨¨¨¨¨¨¨
% \newcommand{\pert}[1]{\accentset{\circ}{#1}}
\newcommand{\sppt}{\ALIASsppt}
% ¨¨¨¨¨¨¨¨¨¨¨¨¨¨¨¨¨¨¨¨¨¨¨¨¨¨¨¨¨¨¨¨¨¨¨¨¨¨¨¨¨¨¨¨¨¨¨¨¨

% \important{Relevant for~\cref{sec:pertwalls:gws:Fourier_SE_tensor}} 


Following~\cref{sec:pertwalls:gws:Fourier_SE_tensor}, we look at~\cref{eq:pertwalls:gws:def_I_s_and_a} for $\sppt(y)=\sin{py}$ in $\epsilon= \varepsilon(\tau) \sppt(y)$. 
The trick is to identify the Jacobi--Anger expansion,~\cref{eq:cylinder:prop:JacobiAnger_sin},
% \footnote{
%     The analogous relation for cosine is obtained by inserting $bx \to bx + \ppi/2$ to get an extra factor $\im^n$ inside the sum.
% }
\begin{equation}
    f(x) \equiv\eu[\im a \sin{bx}] = \sum_{n=-\infty}^{\infty} \Bessel[n](a) \eu[\im n bx].
\end{equation}
We postulate that the Fourier transform $\tilde{f}(\omega)\equiv\integ{x} \sum_{n} F_n(x;\omega)$ is 
\begin{align}
    \tilde{f}(\omega) &= %\integ{x} f(x)\eu[\im \omega x] =
        \sum_{n} \Bessel[n](a) \integ{x}\eu[\im (\omega+ nb)x]\nonumber \\
        &= 2\ppi  \sum_{n} \Bessel[n](a) \Diracdelta(\omega + nb).
\end{align}
We assumed that the %integration over $x$ and summation over $n$ 
integration and summation operators commute, which is known to be true for $\integ{x}\sum_{n}\abs{F_n(x)} < \infty$ (Fubini's theorem). \rcomment{Comment?} 
% \speak{Why is this not the case, again?} \blahblah
For %$g(x)\equiv cb \cos{bx} \cdot f(x) = \frac{1}{2}cb{( \eu[\im bx] + \eu[-\im bx]) } f(x)$, we find
\begin{equation}
    g(x)\equiv cb \cos{bx} \cdot f(x) = \frac{1}{2}cb \pclosed{\eu[\im bx] + \eu[-\im bx] } f(x),
\end{equation}
we find:
\begin{align}
    \tilde{g}(\omega) &= 
    \frac{cb}{2}\sum_{n} \Bessel[n](a) \integ{x} \bclosed{\eu[\im (\omega+ (n+1)b)x] +\eu[\im (\omega+ (n-1)b)x] }\nonumber \\
    &= \ppi cb \sum_{n} \bclosed{ \Bessel[n+1](a) +  \Bessel[n-1](a)} \Diracdelta(\omega + nb ) \nonumber \\
    &= \frac{2\ppi cb}{a} \sum_n n \Bessel[n](a) \Diracdelta(\omega + nb ),
\end{align}
where in the last line we uses~\cref{eq:derivations:cylinder:recurr_Z_plus}. 

Now, the expressions in~\cref{eq:pertwalls:gws:def_I_s_and_a} are
% We use this to find the relevant expressions in~\cref{sec:pertwalls:gws:Fourier_SE_tensor}:
\begin{subequations}\label{eq:derivations:jacobianger:I_s_and_a}
    \begin{align}
        I\ped{s} %&= \integ{y} \sum_{n} \Bessel[n](k_z \varepsilon)  \eu[\im np y  ] \eu[\im k_y y ]\nonumber \\
        & = 2\ppi \sum_{n}  \Bessel[n](k_z \varepsilon) \Diracdelta(k_y + np),\label{eq:derivations:jacobianger:I_s_only} \\
        I\ped{a} %&= \integ{y} \sum_{n} \Bessel[n](k_z \varepsilon)  \eu[\im np y  ] \eu[\im k_y y ]\nonumber \\
        & = 2\ppi \frac{p}{k_z} \sum_{n} n \Bessel[n](k_z \varepsilon) \Diracdelta(k_y + np) = - \frac{k_y}{k_z}I\ped{s}.\label{eq:derivations:jacobianger:I_a_only}
    \end{align}
\end{subequations}


