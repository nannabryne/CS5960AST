






 
% \begin{bullets}
%     \item effect on $\sigma$, $\delta$
%     \item also initialisation of $q$
% \end{bullets}



We rewrite~\cref{eq:PT:symm_dws:eom_asym_chi_s} in terms of the time coordinate $\chi_+ =\sqrt{1-\upsilon}$,
\begin{equation}%\label[eq]{eq:stablesym:eom_asym_chi_chiplus}
    \dv[2]{\brchi}{\chi_+} - \frac{1}{\chi_+ \pclosed{1-\chi_+^2}} \dv{\brchi}{\chi_+} + m_\ast^2 \frac{\chi_+^2 \pclosed{\brchi^2-\chi_+^2}}{\pclosed{1-\chi_+^2}^3}\brchi = 0,
\end{equation}
where
\begin{equation}\label{eq:stablesym:m_ast_def}
    m_\ast= \frac{2\mu}{3\mathcal{H}_\ast \pclosed{1+\redshift_\ast}} = \frac{\sqrt{2}a_\ast^{3/2}}{3\xi_\ast}.
\end{equation}
The idea is to use this solution as boundary conditions for $\chi$:
% \begin{equation}
%     \chi(\tau, z\to \pm \infty) =\pm \brchi(\tau) \quad \land \quad \dot{\chi}(\tau, z\to \pm \infty) =\pm \dot{\brchi}(\tau)
% \end{equation}
\begin{equation}
    \chi|_{z\to \pm\infty} =\pm \brchi \quad \land \quad \dot{\chi}|_{z\to \pm\infty} =\pm \dot{\brchi}.
\end{equation}
% This generally alters the \blahblah
% \comment{Maybe rest in appendix?}
% We solve in two regimes, each solution expanded around \eqregimenum{I} $\chi_+=0$ and \eqregimenum{II} $\chi_+=1$. We get 
% \begin{subequations}
%     \begin{equation}
%         \brchi \simeq\begin{cases}
%             \brchi^{\text{\eqregimenum{I}}}  & \chi_+ \leq \chi_+^{\mathrm{match}} \\
%             \brchi^{\text{\eqregimenum{II}}}  & \chi_+ \geq \chi_+^{\mathrm{match}} \\
%         \end{cases}
%     \end{equation}
%     where
%     \begin{align}
%         %\brchi &\stackrel{\chi_+\sim 0}{\simeq} 
%         \brchi^{\text{\eqregimenum{I}}}  &= \chi_\ast + \frac{\mathcal{C}}{2}\chi_+^2 + \frac{\mathcal{C}-\chi_\ast^3 m_\ast^2}{8} \chi_+^4 \\ %& \equiv \brchi^b
%         %\brchi &\stackrel{\chi_+\sim 1}{\simeq} 
%         \brchi^{\text{\eqregimenum{II}}} &= \chi_+ + \frac{8(3-m_\ast^2)}{m_\ast^4} \pclosed{\chi_+ -1}^3 + \frac{1440 - 636m^2 + 41m^4}{2m^6} \pclosed{\chi_+ -1}^4 %\\%& \equiv \brchi^a\\%+ \BigO{ \pclosed{\chi_+ -1}^5} \\
%     \end{align}
% \end{subequations}

We solve in two regimes, each solution expanded around \eqregimenum{I} $\chi_+=0$ and \eqregimenum{II} $\chi_+=1$:
\begin{subequations}
    \begin{align}
        %\brchi &\stackrel{\chi_+\sim 0}{\simeq} 
        \brchi^{\text{\eqregimenum{I}}}  &= \chi_\ast + \frac{\mathcal{C}}{2}\chi_+^2 + \frac{\mathcal{C}-\chi_\ast^3 m_\ast^2}{8} \chi_+^4 ,\\ %& \equiv \brchi^b
        %\brchi &\stackrel{\chi_+\sim 1}{\simeq} 
        \brchi^{\text{\eqregimenum{II}}} &= \chi_+ + \frac{8(3-m_\ast^2)}{m_\ast^4} \pclosed{\chi_+ -1}^3 + \frac{1440 - 636 m_\ast^2 + 41 m_\ast^4}{2 m_\ast^6} \pclosed{\chi_+ -1}^4. %\\%& \equiv \brchi^a\\%+ \BigO{ \pclosed{\chi_+ -1}^5} \\
    \end{align}
\end{subequations}
We determine $\chi_\ast$ and $\mathcal{C}$ by matching these expressions at $\chi_+\ap{match}$, 
i.e. solving
\begin{equation}
    \begin{split}
        \brchi^{\text{\eqregimenum{I}}}\big|_{\chi_+=\chi_+\ap{match}} &= \brchi^{\text{\eqregimenum{II}}}\big|_{\chi_+=\chi_+\ap{match}} \\
        \dv{\brchi^{\text{\eqregimenum{I}}}}{\chi_+}\Big|_{\chi_+=\chi_+\ap{match}} &= \dv{\brchi^{\text{\eqregimenum{II}}}}{\chi_+}\Big|_{\chi_+=\chi_+\ap{match}}
    \end{split}
\end{equation}
% By determining $\chi_\ast$, $\mathcal{C}$ and $\chi_+\ap{match}$ 
To find a suitable $\chi_+\ap{match}$, we actually also match the solutions for the second derivative. The initial conditions best suited for the smallest oscillations possible are given by 
\begin{equation}\label{eq:stablesym:optimal_chi_breve}
    \brchi = \brchi\ap{ideal}\equiv \begin{cases}
        \brchi^{\text{\eqregimenum{I}}} &\text{if } \chi_+ \leq\chi_+\ap{match}, \\
        \brchi^{\text{\eqregimenum{II}}} &\text{if } \chi_+ \geq\chi_+\ap{match}.
    \end{cases}
\end{equation}

With the help of \textit{SageMath}~\citep{sagemath}, we find these constants for any given $m_\ast$. 
Solving this for the fiducial symmetron parameters (\cref{tab:PT:sims:sim_setups}, e.g.~simulation~\texttt{1}) %$(a_\ast, \xi_\ast)=(0.33, 3.33\times 10^{-4})$
$m_\ast \simeq 268.36$, we get $\chi_\ast \simeq 0.09656$, $\mathcal{C}\simeq 5.837$ and $\chi_+\ap{match}\simeq 0.2568$. This is demonstrated in~\cref{fig:stablesym:computation:stablesym_demo}. 



We update the domain wall profile
\begin{equation}\label{eq:stablesym:chi_w_quasistatic_FLRW_updated}
    \chi\ped{w} = \brchi \tanh{\pclosed{ \frac{a\brchi(z-z\ped{w})}{2L\nped{C}} }}
\end{equation}
and in turn the surface tension and thickness
\begin{equation}
    \sigma\ped{w} = \sigma_\infty \frac{1}{2} \pclosed{ 3\chi_+^2- \brchi^2 }\brchi \quad\text{and}\quad a\delta\ped{w} = \frac{\delta_\infty}{\brchi} ,
\end{equation}
where the latter is shown~\cref{app:walls:surface_tension:new_limits}. %
Note that both of these affect the thin-wall analysis in~\cref{chap:pertwalls}. We propose $\brchi= \brchi\rvert\ped{opt.}$.
% \comment{Maybe put a lot of these equations in appendix.}




\subsubsection{Optimal vs. idealised path}
    We distinguish between what we call the idealised path and the optimal path. The former, $\brchi\ap{ideal}$, is given by~\cref{eq:stablesym:optimal_chi_breve}, and the latter is the numerical solution to the asymptotic equation~\cref{eq:PT:symm_dws:eom_asym_chi_s} with initial conditions $\brchi\ap{ideal}$. They are generally marginally different, except at SSB, when $\chi_+ = 0$. \Cref{fig:stablesym:computation:stablesym_demo} demonstrates this for simulations~\simnum{1} and~\simnum{7} in~\cref{tab:PT:sims:sim_setups}, that were initiated at different times.
    % \subsection{Demonstration}
    % We demonstrate in~\cref{fig:stablesym:computation:stablesym_demo}
    % ------------------------------
    % ----------- FIGURE -----------
    \begin{figure}[h!]
        \centering
        \includegraphics[width=\linewidth]{Appendices/stablesym_demo.png}
        \caption{The asymptotic symmetron field $\breve{\chi}$ and the derivative $\breve{q}=a^2\dot{\breve{\chi}}$ with different initial conditions. The horizontal axis is $\chi_+=\sqrt{1-\upsilon}$. The thin, solid lines show the solutions with initial conditions used in simulations ($\chi_+$ and $q_+= a^2 \dot{\chi}_+$) and the dashed lines show solutions with optised initial conditions ($\breve{\chi}\ap{ideal}$ and $\breve{q}\ap{ideal} =a^2 \dot{\breve{\chi}}\ap{ideal}$).}
        \label{fig:stablesym:computation:stablesym_demo}
    \end{figure}%
    % ------------------------------

