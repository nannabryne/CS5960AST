%%%%%%%%%%%%%%%%%%%%%%%%%%%%%%%%%%%%%%%%%
%%%%%%% App. D: Misc. %%%%%%%
%%%%%%%%%%%%%%%%%%%%%%%%%%%%%%%%%%%%%%%%%


% ---------------------------------------
% labels: \label{[type]:misc:[name]}
% ---------------------------------------



% ¨¨¨¨¨¨¨¨¨¨¨¨¨¨¨¨¨¨¨¨¨¨¨¨¨¨¨¨¨¨¨¨¨¨¨¨
% LOCAL MACROS:
\newcommand\Ft{\ALIASFt}
\newcommand\ah{\ALIASah}
\newcommand\lcoord{\ALIASlcoord}
\newcommand\lcoordx{\ALIASlcoordx}
\newcommand\lcoordk{\ALIASlcoordk}
\newcommand\hpA{\ALIAShpA}
\newcommand\hpB{\ALIAShpB}
\newcommand\hpC{\ALIAShpC}
\newcommand\hpAB{\ALIAShpAB}
\newcommand\hpCR{\ALIAShpCR}
\newcommand\hpCI{\ALIAShpCI}
\newcommand\epsA{\ALIASepsA}
\newcommand\epsB{\ALIASepsB}
\newcommand\epsC{\ALIASepsC}
% ¨¨¨¨¨¨¨¨¨¨¨¨¨¨¨¨¨¨¨¨¨¨¨¨¨¨¨¨¨¨¨¨¨¨¨¨






\section{Simulation~\simnum{6}}\label{app:misc:sim6}
    Simulation~\simnum{6} is identical to simulation~\simnum{1}, except for a change in the scaled Compton wavelength from $\xi_\ast=3.33\times 10^{-4}$ to $\xi_\ast = 10^{-4}$ (see~\cref{tab:PT:sims:sim_setups}). %
    The Compton wavelength now is $L\nped{C} = \xi_\ast/H_0 \approx 0.30~\Mpch$ and the wall width $\delta\ped{w}\approx 0.42~\Mpch$, both way smaller than the spatial resolution $\Delta_\# \simeq 1.3~\Mpch$. %  
    % ------------------------------
    % ----------- FIGURE -----------
    \begin{figure}[hb]
        \centering
        \includegraphics[width=\linewidth]{Appendices/sim6_err.pdf}
        \caption{Simulation results from simulations~\simnum{1},~\simnum{7} and~\simnum{6}, as functions of scale factor $a$. \figpanel{Top panel}~The wall position $\epsB$. \figpanel{Bottom panels}~Spatially averaged values of the scalar fields $\chi$ and $q=a^2 \dot{\chi}$.}
        \label{fig:misc:sim6:sim6_err}
    \end{figure}%
    % ------------------------------
    \Cref{fig:misc:sim6:sim6_err} demonstrates how simulation~\simnum{6} ``fails'' after a time period corresponding to $\Delta \tau/L_\#= 1/2$, i.e. conformal time equal half the box length. Results from simulations~\simnum{1} and~\simnum{7} are shown for comparison. %
    Massless particles propagating in the $z$-direction will have travelled the distance corresponding to the separation between the walls between $a\ped{i}$ and the vertical dashed line in~\cref{fig:misc:sim6:sim6_err}. The fact that $\avg{q}$ blow up here is most likely not a coincidence, but rather an unfortunate collision of information at any of the walls.  




\section{On the continuous/discrete Fourier-comparison}
    One thing we simply did not have the time or space to do, was to go into detail about how the gravitational waves were computed in~\asgrd, and how this might affect the way we interpret the components of $\hpC$. 
    % \comment{Write short about this problem!} 
    In doing this analysis properly, we might have gained insight on the fact that $\hpCI \sim \hpB$ and $\hpCR \not\sim \hpB$, following the notation from~\cref{sec:PT:code,sec:PT:gwas}. On the other hand, we have~\citep{adamekGevolutionCosmologicalNbody2016}
    \begin{equation}
        \hpC[ij]\ap{new}(\lcoordk) = \eu[-\im \ppi (\lcoordk_i + \lcoordk_j)/N_\# ] \hpC[ij](\lcoordk),
    \end{equation}
    which for $\lcoordk_1 =\lcoord{u}= 0$ gives 
    \begin{equation}
        \hpC[11]\ap{new}(0,\lcoord{v},\lcoord{w}) = \hpC[11](0,\lcoord{v},\lcoord{w}).
    \end{equation}
    % for the $xx$-component 
    We only considered this component in our analysis, so it should not be affected by this.

    Still, definitions of lattice derivatives, momenta and TT-projection operator. We can test if
    \begin{equation}
        \hpC[ij](\lcoordk) =  \ProjectionLambda{ij}{kl}(\lcoord{k}k_\#) \hpC[kl](\lcoordk),
    \end{equation}
    which is true for any TT-projected tensor. From~\cref{eq:walls:SE_tensor:TT_projection_11} we have
    \begin{equation}
        \hpC[11](0,\lcoord{v},\lcoord{w}) = \frac{1}{2\tilde{\lcoord{n}}^2}  \bclosed{\tilde{\lcoord{n}}^2 \hpC[11] + 2 \lcoord{v}\lcoord{w}\hpC[23] - \lcoord{w}^2 \hpC[22] - \lcoord{v}^2 \hpC[33]  }(0,\lcoord{v},\lcoord{w}) , \quad \tilde{\lcoord{n}}^2 = \lcoord{v}^2 + \lcoord{w}^2;
    \end{equation}
    which naively implies
    \begin{align}
        \hpC[11]\ap{new}&= \frac{1}{2\tilde{\lcoord{n}}^2}  \bclosed{\tilde{\lcoord{n}}^2 \hpC[11]\ap{new} + 2 \lcoord{v}\lcoord{w}\hpC[23]\ap{new} - \lcoord{w}^2 \hpC[22]\ap{new} - \lcoord{v}^2 \hpC[33]\ap{new}  }\nonumber \\
        & = \frac{1}{2\tilde{\lcoord{n}}^2} \bclosed{\tilde{\lcoord{n}}^2 \hpC[11]+ 2 \eu[-\im \ppi (\lcoord{v}+\lcoord{w})/N_\#] \lcoord{v}\lcoord{w}\hpC[23] - \eu[-\im \ppi 2\lcoord{w}/N_\#] \lcoord{w}^2 \hpC[22] - \eu[-\im \ppi 2\lcoord{v}/N_\#]\lcoord{v}^2 \hpC[33]\ap{new}  }
    \end{align}
    for $\lcoordk = (0,\lcoord{v},\lcoord{w})$. \comment{Either find this result ($\sim$ 15 min), or remove this last part.}
    
    % \speak{For the record, since \[\hpC[ij]\ap{new}(\lcoordk) = \eu[-\im \ppi (\lcoordk_i + \lcoordk_j)/N_\# ] \hpC[ij](\lcoordk),  \] we get \[ \hpC[xx]\ap{new}(0,\lcoord{v},\lcoord{w}) = \hpC[xx](0,\lcoord{v},\lcoord{w}) \] our analysis should be unaffected by this}



\speak{\section*{IF TIME:}}

\speak{IF TIME: * Old simulations with $\varepsilon\sim L_\#/4$ *  ``Adjusting the equation of motion'' * Describe plateau in beginning, and the reasoning for discarding those results. Explain how this plateau showed up again in the thin-wall limit when changing the surface tension. *}

\iftime{classification of topological defects }
