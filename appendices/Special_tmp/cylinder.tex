%%%%%%%%%%%%%%%%%%%%%%%%%%%%%%%%%%%%%%%%%%
%%%%%% App. X: Cylinder Functions  %%%%%%%
%%%%%%%%%%%%%%%%%%%%%%%%%%%%%%%%%%%%%%%%%%


% -------------------------------------
% labels: \label{[type]:special:[name]}
% -------------------------------------

% ¨¨¨¨¨¨¨¨¨¨¨¨¨¨¨¨¨¨¨¨¨¨¨¨¨¨¨¨¨¨¨¨¨¨¨¨¨¨¨¨
% LOCAL MACROS:
\newcommand{\myemph}{\textbf}
\newcommandx\Zv[1][1=\nu]{\Cylindrical[#1][Z]}
\newcommandx\zn[1][1=n]{\Cylindrical[#1][z]}
\newcommandx\Rn[1][1=n]{\Cylindrical[#1][R]}
% ¨¨¨¨¨¨¨¨¨¨¨¨¨¨¨¨¨¨¨¨¨¨¨¨¨¨¨¨¨¨¨¨¨¨¨¨¨¨¨¨






\speak{Rewrite this to simpler version!} %

% Cylinder functions, also generally known as \newconcept{Bessel functions}, are 
Consider the ODE
\begin{equation}
    x^2 y'' + x y' + (x^2 -\nu^2 )y = 0
\end{equation}
where $'\equiv \dv*{}{x}$ and $x\in \Complex$. Solutions to this equation, $y(x)=\Zv(x)$, are {cylinder functions}, also generally known as {Bessel functions}. % where $\Zv[] =  \sum_{i=1,2} c_i \Zv[]^{(i)} =c_1 \Bessel[] + c_2 \Neumann[]$ and %
% $\nu\in\Complex$ is then the order of the Bessel functions, and 
$\Zv[]$ represents a linear combination of the regular Bessel functions of the first and second kind, $\Zv[] =  \sum_{i=1,2} c_i \Zv[]^{(i)}$, known as the \myemph{Bessel} and \myemph{Neumann functions} $\Bessel[]$ and $\Neumann[]$, respectively.



In the following, we will define and \blahblah 
\begin{itemize}
    \item $\Zv[]$ rep
    \item $\zn[]$
    \item $\Rn[]$ 
\end{itemize}

\section{IDK}


The solution to% Bessel's equation
% \begin{equation}
%     y'' + \frac{1-2a}{x}y'+ \bclosed{ {(bcx^{c-1})}^2 + \frac{a^2-\nu^2 c^2}{x^2} }y = 0
% \end{equation}
\begin{equation}
    x^2y'' + x{(1-2a)}y'+ \bclosed{ {(bcx^{c})}^2 + {(a^2-\nu^2 c^2)} }y = 0
\end{equation}
is
\begin{equation}
    y(x) = x^a \Cylindrical[\nu](bx^c).
\end{equation}
% where $\Cylindrical[\nu]$ is a linear combination of the Bessel functions of the first and second kind, $\Bessel$ and $\Neumann$.

% Let $\nu\in\Real$ and $n\in \Integer$. The \myemph{Bessel} and \myemph{Neumann functions}
% \begin{subequations}\label{eq:special:normal_Bessel}
%     \begin{align}
%         \Bessel[\nu](x) &= ,\\
%         \Neumann[\nu](x) &= .
%     \end{align}
% \end{subequations}


\subsection{Half-integer order}
    Let $n\in \Integer$. The spherical Bessel's equation
    \begin{equation}
        x^2 y'' + 2x y' + \pclosed{ x^2 - n(n+1) }y = 0
    \end{equation}
    has the general solution $y(x)=\zn[n](x)$, where $\zn[]=\sum_{i=1,2}c_i \zn[]^{(i)}$,
    \begin{equation}
        \zn[n]^{(i)} = \sqrt{\frac{\ppi}{2x}} \Zv[n+1/2]^{(i)}(x),
    \end{equation}
    represent the \myemph{spherical Bessel} (first kind) and \myemph{Neumann} (second kind) functions, given by
    % For $\nu=n+1/2$ we can express the functions in terms of \myemph{spherical Bessel} (first kind) and \myemph{Neumann} (second kind) functions
    \begin{subequations}\label{eq:special:spherical_Bessel}
        % \begin{align}
        %     \sphBessel[n](x) &= \sqrt{\frac{\ppi}{2x}} \Bessel[n+1/2](x)= +x^n \pclosed{-\frac{1}{x} \dv{}{x}}^n \pclosed{ \frac{\sin{x}}{x}}  ,\\
        %     \sphNeumann[n](x) &= \sqrt{\frac{\ppi}{2x}} \Neumann[n+1/2](x)=- x^n \pclosed{-\frac{1}{x} \dv{}{x}}^n \pclosed{ \frac{\cos{x}}{x}}.
        % \end{align}
        \begin{align}
            \sphBessel[n](x) &= +x^n \pclosed{-\frac{1}{x} \dv{}{x}}^n \pclosed{ \frac{\sin{x}}{x}}  ,\\
            \sphNeumann[n](x) &= - x^n \pclosed{-\frac{1}{x} \dv{}{x}}^n \pclosed{ \frac{\cos{x}}{x}}.
        \end{align}
        % These are linearly independent solutions to 
        % \begin{equation}
        %     x^2 y'' + 2x y' + \pclosed{ x^2 - n(n+1) }y = 0.
        % \end{equation}
    \end{subequations}
    % As a collective term, we use $\Cylindrical[n][z]$ for the spherical Bessel functions of first and second kind. 
    These are related to the \myemph{Riccati--Bessel} (first kind) and \myemph{--Neumann functions} (second kind) by
    \begin{subequations}\label{eq:special:Riccati_Bessel}
        \begin{align}
            \RiccatiBessel[n](x) &= +x\sphBessel[n](x),\\
            \RiccatiNeumann[n](x) &= -x\sphNeumann[n](x),
        \end{align}
        that satisfy
        \begin{equation}
            x^2 y'' + \pclosed{ x^2 - n(n+1) }y = 0.
        \end{equation}
    \end{subequations}
    We will use $\Cylindrical[n][R]$ as a reference to either of these special functions. 
    Note that $\RiccatiBessel[0](x)=\sin{x}$ and $\RiccatiNeumann[0](x)=\cos{x}$. 

    \begin{align}
        \zn^{(i)}(x) &= \sqrt{\frac{\ppi}{2x}} \Zv[n+1/2]^{(i)}(x) \\
        \Rn^{(i)}(x) &= (-1)^{i-1} x\zn^{(i)}(x)
    \end{align}



% \section{Useful identities}


\section{Properties}

\subsection{Some notable identities}
    Some recurrence relations \blahblah
    \begin{subequations}
        \begin{align}
           \frac{2\nu}{x}\Zv(x) &= \Zv[\nu-1](x) + \Zv[\nu+1](x), \\
            2\Zv'(x) &= \Zv[\nu-1](x) - \Zv[\nu+1](x), \\
            \Zv[\nu]'(x) &= \Zv[\nu-1](x) - \frac{\nu}{x}\Zv[\nu](x), \\
            % \Zv[\nu]' &= \Zv[\nu-1]' - \frac{\nu}{x}\Zv[\nu] \\
            \pclosed{x^\nu \Zv[\nu](x)}' &= x^\nu \Zv[\nu-1] (x), \\
            \zn'(x) &= \frac{n}{x}\zn(x) -\zn[n+1](x),
        \end{align}
    \end{subequations}


    The \newconcept{Jacobi--Anger expansion} reads
    \begin{subequations}
        \begin{align}
            \eu[\im a \sin{bx}]  &= \sum_{n=-\infty}^{\infty} \Bessel[n](a) \eu[\im n bx], \\
            \eu[\im a \cos{bx}]  &= \sum_{n=-\infty}^{\infty} \im^{n}\Bessel[n](a) \eu[\im n bx].
        \end{align}
    \end{subequations}
    


\subsection{Asymptotic behaviour}

    \begin{subequations}\label{eq:cylinder:prop:JacobiAnger}
        \begin{align}
            % \Bessel(x) &\stackrel{x \to \infty}{\simeq}  
            \lim_{x\to \infty} \Bessel(x) &=
            \sqrt{\frac{2}{\ppi x}} \cos{(x - \nu \ppi/2 - \ppi/4)}, \label{eq:cylinder:prop:JacobiAnger_sin}  \\
            % \Neumann(x) &\stackrel{x \to \infty}{\simeq} 
            \lim_{x\to \infty} \Neumann(x) &=
            \sqrt{\frac{2}{\ppi x}} \sin{(x - \nu \ppi/2 - \ppi/4)},\label{eq:cylinder:prop:JacobiAnger_cos} 
        \end{align}
    \end{subequations}


